\documentclass[11pt]{article}

\usepackage[top=0.6in,bottom=0.6in,right=1in,left=1in]{geometry} % Margins of the page
\usepackage{amsfonts, amssymb, amsmath, amsthm, enumitem} % math symbols
\usepackage{mathtools}
\usepackage{braket} % symbols for sets
\usepackage{nicefrac} % one-line fractions with nice format

%Conjuntos de números
\newcommand{\N}{\mathbb{N}}
\newcommand{\Z}{\mathbb{Z}}
\newcommand{\Q}{\mathbb{Q}}
\newcommand{\R}{\mathbb{R}}

%Shorter comands
\let\epsilon\varepsilon
\let\oldemptyset\emptyset
\let\emptyset\varnothing
\let\set\Set

%bold all lists$
\setlist[enumerate]{font=\bfseries}

\setlength{\parindent}{0pt} %no indent for the document
\setlength{\parskip}{1em} %add space between paragraphs
\pagestyle{empty}

\begin{document}

\title{\vspace{-2cm}Cálculo diferencial e Integral I \\ Semestre 2023-1 \\ Grupo 4031}
\author{Problemas de: inducción \\ \text{Torres Brito David Israel}}
\date{\today}
\maketitle
\thispagestyle{empty}

\begin{enumerate}
 \item Pruebe que \[ \sum_{i=1}^n i(i+1)=1\cdot2 + 2\cdot 3+3\cdot 4+\cdots + i\cdot (i+1) = \frac{n(n+1)(n+2)}{3}
  \] para toda $n\in \N$.
 
 \textbf{Demostración:} Procederemos por inducción sobre $n$. \begin{enumerate}[label=\roman*)]
  \item Se verifica para $n=1$: \begin{align*}
   (1)(1+1) &= \frac{1(1+1)(1+2)}{3}\\
   (1)(2) &= \frac{1(2)(3)}{3}\\
   2 &= \frac{6}{3}\\
   2 &= 2
  \end{align*}
  \item Supongamos que la fórmula se cumple para $n=k$, es decir, supongamos que \[ \sum_{i=1}^k i(i+1)=\frac{k(k+1)(k+2)}{3}
  \]
  \item Demostraremos, a partir de (ii), que la fórmula se cumple también para $n=k+1$. Es decir, probaremos que \begin{align*}
   \sum_{i=1}^{k+1} i(i+1) &= \frac{(k+1)\bigl((k+1)+1\bigr)\bigl((k+1)+2\bigr)}{3}\\
   &= \frac{(k+1)(k+2)(k+3)}{3}
  \end{align*}
  En efecto, notemos que \begin{align*}
   \sum_{i=1}^{k+1} i(i+1) &=\sum_{i=1}^k i(i+1) + (k+1)\bigl((k+1)+1\bigr)\\
   &=\sum_{i=1}^k i(i+1) + (k+1)(k+2)\\
   &=\frac{k(k+1)(k+2)}{3} + (k+1)\cdot (k+2) && \text{Por hipótesis	de inducción}\\
  &= \frac{k(k+1)(k+2)}{3} + \frac{3(k+1)\cdot (k+2)}{3}\\
  &= k\cdot \frac{(k+1)(k+2)}{3} + 3\cdot \frac{(k+1)\cdot (k+2)}{3}\\
  &= \frac{(k+1)(k+2)}{3} \cdot (k+3)\\
  &= \frac{(k+1)(k+2)}{3} \cdot \frac{3(k+3)}{3}\\
  &= \frac{(k+1)(k+2)\cdot 3(k+3)}{3\cdot 3}\\
  &= \frac{(k+1)(k+2)(k+3)}{3}
 \end{align*}
 \end{enumerate}
 Por tanto, \[ \sum_{k=1}^n k(k+1)=1\cdot2 + 2\cdot 3+3\cdot 4+\cdots + n\cdot (n+1) = \frac{n(n+1)(n+2)}{3}
 \] para toda $n\in \N$. \qed

 \item Pruebe que \[ \sum_{i=1}^n \frac{1}{(2i-1)(2i+1)} = \frac{1}{1\cdot 3}+\frac{1}{3\cdot 5}+\frac{1}{5\cdot 7} + \cdots + \frac{1}{(2i-1)(2i+1)} = \frac{n}{2n+1}
  \] Para toda $n\in \N$.

  \textbf{Demostración:} Procederemos por inducción sobre $n$. \begin{enumerate}[label=\roman*)]
   \item Se verifica para $n=1$: \begin{align*}
   \frac{1}{\bigl(2(1)-1\bigr)\bigl(2(1)+1\bigr)} &= \frac{1}{2(1)+1}\\
   \frac{1}{(2-1)(2+1)} &=  \frac{1}{2+1}\\
   \frac{1}{(1)(3)} &= \frac{1}{3} \\
   \frac{1}{3} &= \frac{1}{3}
   \end{align*}
   \item Supongamos que la fórmula se cumple para $n=k$, es decir, supongamos que \[
    \sum_{i=1}^k \frac{1}{(2i-1)(2i+1)} = \frac{k}{2k+1}
    \]
   \item Demostraremos, a partir de (ii), que la fórmula se cumple también para $n=k+1$. Es decir, probaremos que \begin{align*}
    \sum_{i=1}^{k+1}\frac{1}{(2i-1)(2i+1)} &= \frac{k+1}{2(k+1)+1}\\
    &= \frac{k+1}{2k+2+1}\\
    &= \frac{k+1}{2k+3}
   \end{align*}
   En efecto, notemos que \begin{align*}
    \sum_{i=1}^{k+1}\frac{1}{(2i-1)(2i+1)} &= \sum_{i=1}^k \frac{1}{(2i-1)(2i+1)} + \frac{1}{\bigl(2(k+1)-1\bigr)\bigl(2(k+1)+1\bigr)}\\
    &= \sum_{i=1}^k \frac{1}{(2i-1)(2i+1)} + \frac{1}{(2k+2-1)(2k+2+1)} \\
    &= \sum_{i=1}^k \frac{1}{(2i-1)(2i+1)} + \frac{1}{(2k+1)(2k+3)} \\
    &= \frac{k}{2k+1} + \frac{1}{(2k+1)(2k+3)} && \text{Hip. Ind.} \\
    &= \frac{k(2k+3)}{(2k+1)(2k+3)} + \frac{1}{(2k+1)(2k+3)} \\
    &= \frac{k(2k+3)+1}{(2k+1)(2k+3)} \\
    &= \frac{(2k^2+3k)+1}{(2k+1)(2k+3)} \\
    &= \frac{(2k^2+k+2k)+1}{(2k+1)(2k+3)} \\
    &= \frac{(2k^2+k)+(2k+1)}{(2k+1)(2k+3)} \\
    &= \frac{k(2k+1)+(2k+1)}{(2k+1)(2k+3)} \\
    &= \frac{(2k+1)(k+1)}{(2k+1)(2k+3)} \\
    &= \frac{k+1}{2k+3}
   \end{align*}
   Por tanto, \[ \sum_{i=1}^n \frac{1}{(2i-1)(2i+1)} = \frac{1}{1\cdot 3}+\frac{1}{3\cdot 5}+\frac{1}{5\cdot 7} + \cdots + \frac{1}{(2i-1)(2i+1)} = \frac{n}{2n+1}
  \] Para toda $n\in \N$. \qed
  \end{enumerate}

  \pagebreak

  \item Pruebe que $4^n-1$ es un múltiplo de $3$ para toda $n\in \N$.
  
  \textbf{Demostración:} Procedemos por inducción sobre $n$. \begin{enumerate}[label=\roman*)]
   \item Se verifica para $n=1$. \begin{align*}
    4^{(1)}-1 &= 4-1 \\
    &= 3
   \end{align*}
   \item Supongamos que la proposición es válida para $n=k$, es decir, suponemos que $\exists p\in \N$ tal que \[4^k-1=3p\]
   \item Probaremos que la proposición es válida para $n=k+1$, es decir, probaremos que $\exists q \in \N$ tal que $4^{k+1}-1=3q$.
   
   Recordemos que \begin{align*}
    a^k-b^k &= (a-b)\sum_{i=0}^{k-1} a^{k-1-i}y^{i} && \text{Demostrado en clase}
   \end{align*}
   Por lo que,
   \begin{align*}
    4^{k+1}-1 &= 4^{k+1}-1^{k+1}\\
    &= (4-1)\sum_{i=0}^{k} 4^{k-i} \cdot 1^{i}\\
    &= 3 \sum_{i=0}^{k} 4^{k-i}\\
    &= 3 q, q \in \N && \text{Por hipótesis	de inducción}
   \end{align*}
   Por tanto, $4^n-1$ es un múltiplo de $3$ para toda $n\in \N$. \qed
  \end{enumerate}

 \item Pruebe que $5^n-3^n$ es un número par para tods $n\in \N$.
 
 \textbf{Demostración:} Procedemos por inducción sobre $n$. \begin{enumerate}[label=\roman*)]
  \item Se verifica para $n=1$. \begin{align*}
   5^{(1)}-3^{(1)} &= 5-3 \\
   &= 2
  \end{align*}
  \item Supongamos que la proposición es válida para $n=k$, es decir, suponemos que $\exists p\in \N$ tal que \[5^k-3^k=2p\]
  \item Probaremos que la proposición es válida para $n=k+1$, es decir, probaremos que $\exists q \in \N$ tal que $5^{k+1}-3^{k+1}=2p$.
  
  Recordemos que \begin{align*}
   a^k-b^k &= (a-b)\sum_{i=0}^{k-1} a^{k-1-i}y^{i} && \text{Demostrado en clase}
  \end{align*}
  Por lo que,
  \begin{align*}
   5^{k+1}-3^{k+1} &=(5-3)\sum_{i=0}^{k} 4^{k-i} \cdot 3^{i}\\
   &= 2 \sum_{i=0}^{k} 4^{k-i} \cdot 3^{i}\\
   &= 2 q, q \in \N
  \end{align*}
 \end{enumerate}
 Por tanto, $5^n-3^n$ es un número par para tods $n\in \N$. \qed

 \item Pruebe que todo número natural $n\geq 7$ es igual a la suma de dos números; uno múltiplo de 3 y el otro múltiplo de 4.
 
 \textbf{Demostración:} Procedemos por inducción sobre $n$. \begin{enumerate}[label=\roman*)]
  \item Verificamos que se cumple para $n=7$. \begin{align*}
   7 &= 3 + 4
  \end{align*}

  \item Supongamos que la proposición es válida para $n=k>7$, es decir, suponemos que $\exists p, q\in\N$ tales que $k=3p+4q$.
  \item Probaremos que la proposición es válida para $n=k+1>7$, es decir, suponemos que $\exists s,t\in \N$ tales que $k+1=3s+4t$. En efecto, por hipótesis	de inducción, \begin{align*}
   k + 1 &= 3q+4p + 1 \\
   &= 3q + 4p + \frac{4p}{4p}\\
   &= 3q + \frac{4(p^2)}{p} + \frac{4p}{4p}\\
   &= 3q + 4 \biggl(\frac{p^2}{p} + \frac{p}{4p}\biggr)\\
   &= 3q + 4 \biggl(p+ \frac{1}{4}\biggr)\\
   &= 3t + 4s \\
  \end{align*}
 \end{enumerate}

 \item Prube que si $a\in \R$ es tal que $a\geq -1$, entonces $(1+a)^n\geq 1+na$ para toda $n\in \N$. (Esta desigualdad es conocida como \textit{desigualdad de Bernoulli}).
 
 \textbf{Demostración:} Por inducción sobre $n$. \begin{enumerate}[label=\roman*)]
  \item Verificamos que la desigualdad se cumple para $n=1$. \begin{align*}
   (1+a)^1 &\geq 1+(1)a\\
   1+a &\geq 1+a
  \end{align*}
  \item Supongamos que se cumple para $n=k$. Es decir, supongamos que \begin{align*}
   (1+a)^k &\geq 1+ ka
  \end{align*}
  \item Demostraremos a partir de (ii) que \[(1+a)^{k+1}\geq 1 + (k+1)a\]
  Notemos que \begin{align*}
   (1+a)^{k+1} &= (1+a)^k \cdot (1+a)\\
   &\geq (1+ka) \cdot (1+a) && \text{Hipótesis de inducción}\\
   &= (1+a)+ka(1+a)\\
   &= 1+a+ka+ka^2\\
   &= 1+(1+k)a+ka^2
  \end{align*} Debido a que $k\in \N$ y $a^2 \geq 0$, sigue que $ka\geq 0$, entonces, de la igualdad anterior sigue que $1+(1+k)a+ka^2 \geq 1+ (k+1)a$, y por transitividad, $(1+a)^{k+1} \geq 1+ (k+1)a$. \qed
 \end{enumerate}

 \item Pruebe que \[\sum_{i=1}^n \frac{1}{i^2} = \frac{1}{1^2}+\frac{1}{2^2}+\frac{1}{3^2}+\cdots + \frac{1}{i^2} < 2-\frac{1}{2n}\]
 \textbf{Demostración:} Procederemos por inducción sobre $n$. \begin{enumerate}[label=\roman*)]
  \item Verificamos que la desigualdad se cumple para $n=1$. \begin{align*}
   \frac{1}{1^2} &< 2 - \frac{1}{2(1)}\\
   \frac{1}{1} &< 2-\frac{1}{2}\\
   %\frac{2}{2}\cdot \frac{1}{1} &< \frac{2}{2}\cdot 2-\frac{1}{2}\\
   %\frac{2}{2} &< \frac{4}{2}-\frac{1}{2}\\
   1 &< \frac{3}{2}
  \end{align*}
  \item Supongamos que la desigualdad es válida para $n=k$, es decir, suponemos que \begin{align*}
   \sum_{i=1}^k \frac{1}{i^2} < 2-\frac{1}{2k}
  \end{align*}
  \item Demostraremos a partir de (ii) que la desigualdad se cumple para $n=k+1$, osea \begin{align*}
   \sum_{i=1}^{k+1} \frac{1}{i^2} < 2-\frac{1}{2(k+1)}
  \end{align*} Notemos que \begin{align*}
   \sum_{i=1}^{k+1} \frac{1}{i^2} &= \sum_{i=1}^k \frac{1}{i^2} + \frac{1}{(k+1)^2}\\
   &< 2-\frac{1}{2k} + \frac{1}{(k+1)^2} && \text{Hipótesis	de inducción}\\
  \end{align*}
 \end{enumerate}

\end{enumerate}

\end{document}