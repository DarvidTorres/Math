\documentclass[11pt]{article}

\usepackage[top=0.6in,bottom=0.6in,right=0.6in,left=0.6in]{geometry} % Margins of the page
\usepackage{amsfonts, amssymb, amsmath, amsthm, enumitem} % math symbols
\usepackage{mathtools}
\usepackage{braket} % symbols for sets
\usepackage{nicefrac} % one-line fractions with nice format

%Conjuntos de números
\newcommand{\N}{\mathbb{N}}
\newcommand{\Z}{\mathbb{Z}}
\newcommand{\Q}{\mathbb{Q}}
\newcommand{\R}{\mathbb{R}}

%Shorter comands
\let\epsilon\varepsilon
\let\oldemptyset\emptyset
\let\emptyset\varnothing
\let\set\Set

%bold all lists$
\setlist[enumerate]{font=\bfseries}

\setlength{\parindent}{0pt} %no indent for the document
\setlength{\parskip}{1em} %add space between paragraphs
\pagestyle{empty}

\begin{document}

\title{\vspace{-2cm}Cálculo diferencial e Integral I \\ Semestre 2023-1 \\ Grupo 4031}
\author{Examen parcial 1 \\ \text{Darvid}}
\date{\today}
\maketitle
\thispagestyle{empty}

\begin{enumerate}
 \item Sean $a,b\in \R$ distintos de $0$. Usando sólo las propiedades básicas de los números reales, pruebe que si $ab^{-1}=ba^{-1}$, entonces $a^2=b^2$.
 
 Primero demostraremos la ley de la cancelación para la multiplicación. Si $ab=cb$ y $b\neq 0$, entonces $a=c$.
 
 \textbf{Demostración:} Notemos que, por hipótesis, $b\neq 0$, por lo que $\exists b^{-1}\in \R$. Luego, \begin{align*}
  a &= a \cdot 1 && \text{Neutro multiplicativo}\\
  &= a \cdot (b\cdot b^{-1}) && \text{Inverso multiplicativo}\\
  &= (ab) b^{-1} && \text{Asociatividad}\\
  &= (cb) b^{-1} && \text{Hipótesis	}\\
  &= c\cdot (bb^{-1}) && \text{Asociatividad}\\
  &= c \cdot 1 && \text{Inverso multiplicativo}\\
  &= c && \text{Neutro multiplicativo}
 \end{align*}

Ahora, procedemos a resolver la proposición 1. Por hipótesis tenemos que $ab^{-1}=ba^{-1}$ y como $b\neq 0$, podemos usar lo demostrado arriba, así \begin{align*}
 ab^{-1} \cdot b &= ba^{-1} \cdot b && \text{Ley de la cancelación}\\
 ab^{-1} \cdot b &= a^{-1} b \cdot b && \text{Conmutatividad}\\
 a \cdot 1 &= a^{-1} b \cdot b && \text{Inverso multiplicativo}\\
 a &= a^{-1} b \cdot b && \text{Neutro multiplicativo}\\
 a \cdot a &= a^{-1} b \cdot b \cdot a && \text{Ley de la cancelación}\\
 a \cdot a &= b \cdot b \cdot a^{-1} \cdot a && \text{Conmutatividad}\\
 a \cdot a &= b \cdot b \cdot 1 && \text{Inverso multiplicativo}\\
 a \cdot a &= b \cdot b && \text{Neutro multiplicativo}\\
 a^2 &= b^2 && \text{Definición}
\end{align*} \qed

\item Encuentre todos los números $x\in \R$ que satisfacen la siguiente desigualdad: \begin{align*}
 \frac{1+|x|}{1-|x+1|} &< 0
\end{align*}

Sabemos que si $a\geq 0$ y $b \geq 0$, entonces $ab\geq 0$, y si $a<0$ y $b<0$, $ab>0$, por lo que en la desigualdad propuesta, los factores deben ser, uno positivo y el otro negativo, por lo que consideraremos los casos posibles: \begin{enumerate}[label=Caso (\arabic*)]
 \item Si $0<1+|x|$ y $(1-|x+1|)^{-1}<0$. \begin{enumerate}[label=\alph*)]
  \item De $0<1+|x|$, sigue que $-1<|x|$, luego, \begin{enumerate}[label=\roman*)]
   \item Si $x\geq 0$, entonces $|x|=x>-1$.
   \item Si $x<0$, entonces $|x|=-x>-1$, osea, $x<1$
  \end{enumerate} Así $-1<x$ o $x<1$.
  \item De $(1-|x+1|)^{-1}<0$, sigue que $\frac{1}{1-|x+1|}<0$, por notación. Sabemos que $1>0$, por lo que, el factor $(1-|x+1|)$ debe ser menor a $0$, entonces \begin{enumerate}[label=\roman*)]
   \item Si $|x+1|\geq 0 $, entonces $|x+1|=x+1$, por lo que $1-|x+1|=1-x+1<0$, de donde obtenemos que $2<x$.
   \item Si $|x+1|<0$, entonces $|x+1|=-x-1$, por lo que $1-|x+1|=1-x-1<0$, de donde obtenemos que $0<x$.
  \end{enumerate} Así $2<x$ o $0<x$.
 \end{enumerate}
 \item Si $1+|x|<0$ y $0<(1-|x+1|)^{-1}$. \begin{enumerate}[label=\alph*)]
  \item De $1+|x|<0$, obtenemos que $1<|x|$, luego \begin{enumerate}[label=\roman*)]
   \item Si $x\geq 0$, entonces $|x|=x>1$.
   \item Si $x<0$, entonces $|x|=-x<1$, por lo que $-1<x$.
  \end{enumerate} Así $x>1$ o $-1<x$.
   \item De $0<(1-|x+1|)^{-1}$, sigue que $\frac{1}{1-|x+1|}>0$, por notación, y como $1>0$, necesariamente el factor $1-|x+1|>0$, luego \begin{enumerate}[label=\roman*)]
    \item Si $x+1\geq 0$, entonces $|x+1|=x+1$, por lo que $1-|x+1|=1-x-1>0$, de donde obtenemos que $0>x$.
    \item Si $x<0$, entonces $|x+1|=-x-1$, por lo que $1-|x+1|=1+x+1>0$, de donde obtenemos que $-2<x$.
   \end{enumerate} Así $x>0$ o $-2<x$.
 \end{enumerate}
 Finalmente, debemos considerar que de la desigualdad propuesta, tenemos que $1-|x+1|\neq 0$. De donde sigue que $1\neq |x+1|$, luego, \begin{enumerate}
  \item Si $x+1\geq 0$, entonces $|x+1|=x+1\neq 1$, de donde obtenemos que $x\neq 0$.
  \item Si $x+1<0$, entonces $|x+1|=-x-1 \neq 1$, de donde obtenemos que $x\neq 2$.
 \end{enumerate}
 Con lo anterior, hemos considerado todos los posibles valores para $x$ en la desigualdad propuesta.
\end{enumerate}

\item Pruebe que $\big| |a+b-c|-|a+b|-|c| \big| = |a+b|+|c|-|a+b-c|$.

\textbf{Demostración:} Debemos considerar los casos para $\big| |a+b-c|-|a+b|-|c| \big|$. \begin{enumerate}[label=Caso (\arabic*)]
 \item Si $|a+b-c|-|a+b|-|c|\geq 0$, entonces \begin{align*}
  \big| |a+b-c|-|a+b|-|c| \big| &= |a+b-c|-|a+b|-|c| \geq 0
 \end{align*} Tomando el lado derecho de esta igualdad, tenemos que \begin{align*}
  |a+b-c|-|a+b|-|c| &\geq 0
 \end{align*} Luego, como $2\neq 0$, podemos aplicar la ley de la multiplicación, osea \begin{align*}
  2 \bigl(|a+b-c|-|a+b|-|c|\bigr) &\geq 2\cdot 0
 \end{align*} Sabemos que del lado derecho de esta desigualdad tenemos $2\cdot 0 =0$, y del lado izquierdo que \begin{align*}
  2|a+b-c|-2|a+b|-2|c| &\geq 0 && \text{P. Distributiva}\\
  |a+b-c|+|a+b-c|-|a+b|-|a+b|-|c|-|c| &\geq 0 && \text{Definición}\\
 \end{align*} Ahora, sumando en ambos lados de la desigualdad el inverso aditivo de $|a+b-c|$, $-|a+b|$ y $-|c|$, obtenemos \begin{align*}
  |a+b-c|-|a+b|-|c|&\geq |a+b|+|c|-|a+b-c|
 \end{align*} De la desigualdad anterior tenemos dos casos, si se cumple con igualdad, la demostración está concluida, pero si se cumple que \begin{align*}
  |a+b-c|-|a+b|-|c|&> |a+b|+|c|-|a+b-c|
 \end{align*} tendríamos una contradicción, pues \begin{align*}
  |a+b-c|-|a+b|-|c|&> |a+b|+|c|-|a+b-c|\\
  0 &> |a+b|+|c|-|a+b-c|-|a+b-c|+|a+b|+|c| && \text{Sumando inversos aditivos}\\
  0 &> 2|a+b-c|-2|a+b|-2|c| && \text{Definición}\\
  0 &> 2 \bigl(|a+b-c|-|a+b|-|c|\bigr)&& \text{Propiedad distributiva}\\
  0 &> |a+b-c|-|a+b|-|c| && \text{Multiplicando por el inv. mult. de 2}
 \end{align*}
 Pero lo anterior contradice nuestro supuesto inicial. Por tanto, tenemos una igualdad, es decir, $\big| |a+b-c|-|a+b|-|c| \big| = |a+b|+|c|-|a+b-c|$.

 \item Si $|a+b-c|-|a+b|-|c|< 0$, entonces \begin{align*}
  \big| |a+b-c|-|a+b|-|c| \big| &= |a+b-c|-|a+b|-|c| < 0
 \end{align*} Tomando el lado derecho de esta desigualdad, tenemos que \begin{align*}
  |a+b-c|-|a+b|-|c| &< 0\\
  0 &< -|a+b-c|+|a+b|+|c| && \text{Sumando inversos aditivos}\\
  0 &< |a+b|+|c|-|a+b-c| && \text{Conmutatividad}
 \end{align*} Pero la desigualdad anterior contradice nuestro supuesto, por lo que descartamos este caso.
 
 Por tanto,  $\big| |a+b-c|-|a+b|-|c| \big| = |a+b|+|c|-|a+b-c|$. \qed
\end{enumerate}

\item Pruebe por inducción que, si $a\leq -1$, entonces \begin{align*}
 1+\frac{n}{a} &\leq \biggl(1+\frac{1}{a}\biggr)^n
\end{align*} para toda $n\in \N$.

Procedemos por inducción sobre $n$. \begin{enumerate}[label=\roman*)]
 \item Verificamos que se cumple para $n=1$ \begin{align*}
  1+\frac{(1)}{a} &\leq \biggl(1+\frac{1}{a}\biggr)^{(1)} \\
  1+\frac{1}{a} &\leq 1+\frac{1}{a} \\
 \end{align*}
 \item Suponemos que se cumple para $n=k$, es decir, suponemos que \begin{align*}
  1+\frac{k}{a} &\leq \biggl(1+\frac{1}{a}\biggr)^k
 \end{align*}
 \item Probaremos que se cumple parar $n=k+1$, es decir, buscamos demostrar que \begin{align*}
  1+\frac{k+1}{a} &\leq \biggl(1+\frac{1}{a}\biggr)^{k+1}
 \end{align*}
 Primero demostraremos que \begin{align*}
  0 \leq 1+ \frac{1}{a}
 \end{align*} Retomando nuestra hipótesis, $a\leq -1$. Además, sabemos que $-1<0$, así, al multiplicar en ambos lados de nuestra hipótesis	por $-1$ el orden cambia, esto es \begin{align*}
  (-1)(-1) &\leq (-1) a \\
  1 &\leq -a && \text{Ley de los signos}
 \end{align*} También, sabemos que $0<1$, así, por transitividad tenemos que $0<-a$. De esto, sigue que $\frac{1}{-a}>0$, como se demostró en clase, el inverso multiplicativo de un número positivo es mayor a 0. Luego, por notación $-\frac{1}{a}>0$. Ahora, retomando de nueva cuenta nuestra hipótesis	\begin{align*}
  a\leq -1
 \end{align*} Al multiplicar a ambos lados por $-\frac{1}{a}$, la desigualdad se mantiene, como se demostró en clase, esto implica que \begin{align*}
  a \cdot \biggl(- \frac{1}{a}\biggr) &\leq -1 \cdot \biggl(- \frac{1}{a}\biggr)\\
  -1 &\leq \frac{1}{a} && \text{Ley de los signos}
 \end{align*} Luego, recordemos que $-1<0$, por lo que al tomar la desigualdad anterior y multiplicar por $-1$ en ambos lados, el orden cambia, es decir, \begin{align*}
  -1 &\leq \frac{1}{a} \\
  \frac{1}{a} \cdot (-1) &\leq (-1)(-1)\\
  -\frac{1}{a} &\leq 1 && \text{Le de los signos}
 \end{align*} De este modo, al sumar $\frac{1}{a}$ en ambos lados tenemos \begin{align*}
  -\frac{1}{a}+\frac{1}{a} &\leq 1+\frac{1}{a} && \text{Ley de la cancelación}\\
  0 &\leq 1+\frac{1}{a} && \text{Inverso aditivo}
 \end{align*} Ahora, debido a que este número $1+\frac{1}{a}$ es mayor o igual a $0$, al multiplicarlo en ambos de una desigualdad, mantiene el orden. Entonces, volvamos al problema original, retomando nuestra hipótesis de inducción tenemos\begin{align*}
  1+\frac{k}{a} &\leq \biggl(1+\frac{1}{a}\biggr)^k && \text{Hipótesis inductiva}\\
  \biggl(1+\frac{k}{a}\biggr) \cdot \biggl(1+\frac{1}{a}\biggr) &\leq  \biggl(1+\frac{1}{a}\biggr)^k \cdot \biggl(1+\frac{1}{a}\biggr) && \text{Demostrado arriba}\\
  \biggl(1+\frac{k}{a}\biggr) \cdot \biggl(1+\frac{1}{a}\biggr) &\leq  \biggl(1+\frac{1}{a}\biggr)^{k+1} && \text{Definición}
 \end{align*} Ahora, consideremos el lado izquierdo de esta desigualdad,
 
 \pagebreak

 \begin{align*}
  \biggl(1+\frac{k}{a}\biggr) \cdot \biggl(1+\frac{1}{a}\biggr) &= \biggl(1+\frac{k}{a}\biggr) \cdot 1 + \biggl(1+\frac{k}{a}\biggr) \cdot \frac{1}{a} && \text{P. Distributiva}\\
  &= \biggl(1+\frac{k}{a}\biggr)+ \biggl(1+\frac{k}{a}\biggr) \cdot \frac{1}{a} && \text{Neutro multiplicativo}\\
  &= \biggl(1+\frac{k}{a}\biggr)+ 1 \cdot \frac{1}{a}+\frac{k}{a} \cdot \frac{1}{a} && \text{P. Distributiva}\\
  &= \biggl(1+\frac{k}{a}\biggr)+ \frac{1}{a}+\frac{k}{a} \cdot \frac{1}{a} && \text{Neutro multiplicativo}\\
  &= \biggl(1+\frac{k}{a}\biggr)+ \frac{1}{a}+\frac{k}{a^2} && \text{Producto de fracciones}\\
 \end{align*} Entonces, la desigualdad queda como \begin{align*}
  \biggl(1+\frac{k}{a}\biggr)+ \frac{1}{a}+\frac{k}{a^2} &\leq \biggl(1+\frac{1}{a}\biggr)^{k+1} && \text{(*)}
 \end{align*} 
 
 Finalmente, recordemos que $k\in \N$, lo que implica que $k\geq 1$. Así mismo, $a^2\geq 0$, de donde sigue que $\frac{1}{a^2}>0$. Por propiedad de los positivos, el producto de dos positivos es positivo, entonces \begin{align*}
  k \cdot \frac{1}{a^2} = \frac{k}{a^2} &\geq 0
 \end{align*} De esta desigualdad sigue que \begin{align*}
  0 &\leq \frac{k}{a^2} \\
  1+\frac{k}{a}+ \frac{1}{a} &\leq 1+\frac{k}{a}+ \frac{1}{a} + \frac{k}{a^2} \\
  1+\frac{k+1}{a} &\leq 1+\frac{k}{a}+ \frac{1}{a} + \frac{k}{a^2} && \text{Suma de fracciones}
 \end{align*} Luego, observemos que el lado derecho de esta desigualdad, es el lado izquierdo de (*), así por transitividad, \begin{align*}
  1+\frac{k+1}{a} &\leq \biggl(1+\frac{1}{a}\biggr)^{k+1}
 \end{align*}
\end{enumerate} \qed
\end{enumerate} 



\end{document}