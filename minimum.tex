\documentclass[11pt]{article}

\usepackage[top=0.6in,bottom=0.6in,right=1in,left=1in]{geometry}
\usepackage{amsfonts, amssymb, amsmath, amsthm, enumitem}
\usepackage{mathtools}
\usepackage{braket}
\usepackage{adjustbox}
\usepackage{nicefrac} % for Elegant fractions in one line https://tex.stackexchange.com/questions/128496/elegant-fractions-in-one-line/128498

%Conjuntos de números
\newcommand{\N}{\mathbb{N}}
\newcommand{\Z}{\mathbb{Z}}
\newcommand{\Q}{\mathbb{Q}}
\newcommand{\I}{\mathbb{I}}
\newcommand{\R}{\mathbb{R}}

%Shorter comands
\let\epsilon\varepsilon
\let\oldemptyset\emptyset
\let\emptyset\varnothing
\let\set\Set

%bold all lists$
\setlist[enumerate]{font=\bfseries}

\setlength{\parindent}{0pt} %no indent for the document
\setlength{\parskip}{1em} %add space between paragraphs
\pagestyle{empty}

\begin{document}


\title{\vspace{-2cm}Cálculo I}
\author{Darvid \\ \texttt{darvid.torres@gmail.com}}
\date{\today}
\maketitle
\thispagestyle{empty}

\textbf{Definición:} Sea $a$ un número real, definimos el valor absoluto de $a$, denotado por $|a|$ como sigue: 
    \[
    |a| = 
    \left \{
        \begin{aligned}
        a &,\ \text{si} \ a \geq 0\\
        -a &,\ \text{si} \ a < 0 \\
        \end{aligned}
    \right .
    \]

\textbf{Observación.} $|a|\geq 0, \ \forall a\in \R$.

\subsection*{Lista de Ejercicios 4 (LE4)}

Sean $a$, $b$, $c$ números reales, demuestre lo siguiente:

\begin{enumerate}[label=\alph*)]
    \item $|a| \geq \pm a$. %A
    \item $|ab|=|a||b|$. %B
    \item $|a|=|-a|$. %C
    \item $|a+b|\leq |a|+|b|$. Desigualdad del triángulo. %D
    \item Si $b\neq 0$, entonces $\left| \frac{a}{b} \right| = \frac{|a|}{|b|}$. %E
    \item $|a|<b$ si y solo si $-c<b<c$. %F
    \item $ \big| |a|-|b| \big| \leq |a-b|$ %G
    \item $|a|^2=a^2$.
\end{enumerate}

\subsubsection*{Demostración}

\begin{enumerate}[label=\alph*)]

    %A

    \item
        \begin{enumerate}[label=\roman*)]
            \item Si $a \geq 0$, entonces $|a|=a$, así, $|a| \geq a$. Luego, $-a \leq 0$, de donde sigue que $a \geq -a$. Finalmente, $|a| \geq -a$.
            \item Si $a<0$, entonces $|a|=-a$, así, $|a| \geq -a$. Luego, $-a>0$, de donde sigue que $-a>a$. Finalmente, $|a| \geq a$.
        \end{enumerate}
        En cualquier caso, $|a| \geq \pm a$.

    %B

    \item 
        \begin{enumerate}[label=\roman*)]
            \item Si $a>0$ y $b>0$, entonces $|a|=a$ y $|b|=b$. Luego, $ab>0$ por lo que $|ab|=ab$. De este modo, $|ab| =|a||b|$.
            \item Si $a>0$ y $b<0$, entonces $|a|=a$ y $|b|=-b$. Luego, $ab<0$ por lo que $|ab|=-ab$. De este modo, $|  ab|=|a||b|$.
            \item Si $a<0$ y $b<0$, entonces $|a|=-a$ y $|b|=-b$. Luego, $ab>0$ por lo que $|ab|=ab$. De este modo, $|  ab|=|a||b|$.
        \end{enumerate}

    %C

    \item 
        \begin{enumerate}[label=\roman*)]
            \item Si $a \geq 0$, entonces $|a|=a$. Luego, $-a \leq 0$. Si $-a<0$, $|-a|=a$ y si $-a=0$, $|-a|=a$. De este modo, $|a|=|-a|$.
            \item Si $a<0$, entonces $|a|=-a$. Luego, $-a>0$ por lo que $|-a|=-a$. De este modo, $|a|=|-a|$.
        \end{enumerate}

    %D
    
    \item 
        \begin{enumerate}[label=\roman*)]
            \item Si $0 \leq a+b$, entonces $|a+b|=a+b$. Además, $a \leq |a|$ y $b \leq |b|$. Luego, $a+b \leq |a|+|b|$. Así, $|a+b| \leq |a|+|b|$.
            \item Si $0 > a+b$, entonces $|a+b|=-a-b$. Además, $-a \leq |a|$ y $-b \leq |b|$. Luego, $-a-b \leq |a|+|b|$. Así, $|a+b| \leq |a|+|b|$.
        \end{enumerate}
    
    %E

    \item 
        \begin{enumerate}[label=\roman*)]
            \item Si $a \geq 0$ y $b>0$, entonces $|a|=a$ y $|b|=b$. Además, $\frac{1}{b} >0$, de donde sigue que $\frac{a}{b} \geq 0$ por lo que $\big| \frac{a}{b} \big| = \frac{a}{b}$. De este modo, $ \big| \frac{a}{b} \big| = \frac{|a|}{|b|}$.
            \item Si $a \geq 0$ y $b<0$, entonces $|a|=a$ y $|b|=-b$. Además, $\frac{1}{b} <0$, de donde sigue que $\frac{a}{b} \leq 0$, por lo que $\big| \frac{a}{b} \big| =- \frac{a}{b}$. De este modo, $ \big| \frac{a}{b} \big| = \frac{|a|}{|b|}$.
            \item Si $a<0$ y $b>0$, entonces $|a|=-a$ y $|b|=b$. Además, $\frac{1}{b} >0$, de donde sigue que $\frac{a}{b} < 0$, por lo que $\big| \frac{a}{b} \big| =- \frac{a}{b}$. De este modo, $ \big| \frac{a}{b} \big| = \frac{|a|}{|b|}$.
            \item Si $a<0$ y $b<0$, entonces $|a|=-a$ y $|b|=-b$. Además, $\frac{1}{b} <0$, de donde sigue que $\frac{a}{b} > 0$ por lo que $\big| \frac{a}{b} \big| = \frac{a}{b}$. De este modo, $ \big| \frac{a}{b} \big| = \frac{|a|}{|b|}$.
        \end{enumerate}

    %F
        
    \item
        \begin{enumerate}[label=\roman*)]
            \item Supongamos que $|b|<c$. Por (a) de LE4, $ \pm b \leq |b|$, de donde sigue que $-b<c$ y $b<c$. Luego, $-c<b$. De este modo, $-c<b<c$.
            \item Supongamos que $-c<b<c$. Luego,
                \begin{enumerate}[label=\arabic*)]
                    \item Si $b \geq 0$, entonces $|b|=b$. Por lo que $|b|<c$.
                    \item Si $b < 0$, entonces $|b|=-b$. Por hipótesis, $-c<b$, por lo que $-b<c$. Así $|b|<c$.
                \end{enumerate}
        \end{enumerate}
    
    %G

    
    \item Por la desigualdad del triángulo,
        \begin{align*}
            |(a-b)+b| &\leq |a-b|+|b| \\
            |a| &\leq |a-b|+|b| \\
            |a|-|b| &\leq |a-b| && \text{(1)}
        \end{align*}
        Similarmente, 
        \begin{align*}
            |(b-a)+a| &\leq |b-a|+|a| \\
            |b| &\leq |b-a|+|a| \\
            |b|-|a| &\leq |b-a| \\
            -|b-a| &\leq |a|-|b| && \text{(2)}
        \end{align*}
        Luego, aplicando (f) de LE4 en (1) y (2), $\big| |a| - |b| \big| \leq |a-b|$.

    %H

    \item Por (o) de LE3, $a^2\geq 0$, por lo que \begin{align*}
        a^2 &= |a^2|\\
        &= |a\cdot a|\\
        &= |a| \cdot |a| && \text{Por (b) de LE4}\\
        &= |a|^2
    \end{align*} \qed
\end{enumerate} 

\end{document}