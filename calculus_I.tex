\documentclass[11pt]{article}

\usepackage[top=0.5in,bottom=0.5in,right=0.5in,left=0.5in]{geometry}
%\usepackage[T1]{fontenc}
%\usepackage[spanish]{babel}
\usepackage{amsfonts, amssymb, amsmath, amsthm, enumitem}
\usepackage{mathtools} %\coloneqq command
\usepackage{braket}
%\usepackage[spanish]{babel}
\usepackage{adjustbox}
\usepackage{nicefrac} % for Elegant fractions in one line https://tex.stackexchange.com/questions/128496/elegant-fractions-in-one-line/128498
\usepackage{physics} % for nice brackets \qty(...) \qty[...] \qty{...} https://www.reddit.com/r/LaTeX/comments/7qkbfv/redefine_left_right/
\usepackage{tcolorbox} % Frame box
\usepackage{array,multirow}

\usepackage{xcolor} % package to color brackets. https://latex-tutorial.com/color-latex/

\usepackage{relsize} %% to reduce size of contradition symbol

\renewcommand*{\proofname}{\textbf{Demostración:}}

%Conjuntos de números
\newcommand{\N}{\mathbb{N}}
\newcommand{\Z}{\mathbb{Z}}
\newcommand{\Q}{\mathbb{Q}}
\newcommand{\I}{\mathbb{I}}
\newcommand{\R}{\mathbb{R}}

%Shorter comands
\newcommand{\contradiction}{\mathsmaller{\stackrel{\mathlarger{\nabla}}{\boldsymbol{\boldsymbol{\circ}}}}}
\newcommand{\defined}{\coloneqq}
\newcommand{\bfit}[1]{\textbf{\textit{#1}}}
\let\epsilon\varepsilon
\let\oldemptyset\emptyset
\let\emptyset\varnothing
\let\set\Set
\let\union\cup
\let\propersubset\subset
\let\subset\subseteq
\let\intersection\cap

%the code below manipulates space around \align environment https://tex.stackexchange.com/questions/47400/remove-vertical-space-around-align
%\usepackage{etoolbox}
%\newcommand{\zerodisplayskips}{%
%\setlength{\abovedisplayskip}{0em}}%
%%\setlength{\belowdisplayskip}{-0em}%
%%\setlength{\abovedisplayshortskip}{0em}}%
%%\setlength{\belowdisplayshortskip}{-0em}}
%\appto{\normalsize}{\zerodisplayskips}
%\appto{\small}{\zerodisplayskips}
%\appto{\footnotesize}{\zerodisplayskips}

%bold all lists$
\setlist[enumerate]{font=\bfseries}
%\setlist{leftmargin=0em}
\setlist[itemize]{topsep=0pt}%{noitemsep, topsep=0pt}

\setlength{\parindent}{0pt} %no indent for the document
\setlength{\parskip}{1em} %add space between paragraphs
\pagestyle{empty}

\begin{document}

\title{Cálculo I}
\author{Darvid \\ \texttt{darvid.torres@gmail.com}}
\date{\today}
\maketitle
\thispagestyle{empty}

\part*{Axiomas de campo}

Existe un conjunto llamado conjunto de los números reales, denotado por $\R$. A los elementos de este conjunto los llamaremos números reales. Este conjunto está dotado con dos operaciones binarias: $+$ (suma) y $\cdot$ (multiplicación). 
%\begin{enumerate}[label=\roman*)]
% \item Suma: $+$
% \item Multiplicación: $\cdot$
%\end{enumerate}
%\vspace{-0.5cm}
%\begin{center}
%\noindent\begin{minipage}[r]{5.5cm}
%\begin{align*}
% \text{Suma} \ + : \R \times \R &\to \R\\
% (m,n) &\mapsto m+n
%\end{align*}
%\end{minipage}%
%\begin{minipage}[l]{6.5cm}
%\begin{align*}
% \text{y} \qquad \text{Multiplicación} \ \cdot : \R \times \R &\to \R\\
% (m,n) &\mapsto m \cdot n
%\end{align*}
%\end{minipage}
%\end{center}
%
%La notación anterior denota la cerradura de estas operaciones, es decir, que para cuales quiera dos números reales $(m,n)$, la suma y multiplicación son números reales, $(m+n)\in \R$ y $(m\cdot n)\in \R$.
%
%Las suma y multiplicación de números reales satisfacen los siguientes \textbf{axiomas}:
%
\section*{Axiomas de la suma}
La suma satisface las siguientes propiedades:
\begin{enumerate}%[label=S\arabic*., start=0]
 \item Cerradura (de la suma): Si $x,y\in \R$, entonces $x+y \in \R$.
 %La suma es cerrada. Esto significa que si $x,y\in \R$, se verifica que $x+y \in \R$.

 \item Conmutatividad (de la suma): Si $x, y\in \R$, entonces $x+y =y+x$.
 %La suma es conmutativa. Esto significa que si $x, y\in \R$, se verifica que $x+y =y+x$.

 \item Asociatividad (de la suma): Si $x, y, z\in \R$, entonces $(x+y)+z = x+(y+z)$.
 %La suma es asociativa. Esto significa que si $x, y, z\in \R$, se verifica que $(x+y)+z = x+(y+z)$.
 %
 \item Neutro aditivo (o cero): $\exists 0\in \R$ tal que si $x\in \R$, entonces $x+0=x$.
 %Existe un número real llamado neutro aditivo (o cero), denotado por $0$, tal que si $x\in \R$, entonces $x+0=x$.
 %Existe un número real llamado elemento neutro para la suma o cero, denotado por $0$, el cual satisface la siguiente condición: $ x+0=x,\forall x \in \R$
 \item Inverso aditivo: Si $x\in \R$, entonces $\exists (-x)\in \R$ tal que $x+(-x)=0$.
 %Si $x\in \R$ y $x\neq 0$, entonces existe un número real llamado inverso aditivo de $x$, denotado por $-x$, tal que $x+(-x)=0$.
 %Para cada número real $x$ existe un número real llamado inverso aditivo de $x$, denotado por $-x$ (menos $x$); la propiedad que caracteriza a este elemento es $x + (-x) = 0$.
\end{enumerate}

\begin{tcolorbox}
  \subsection*{Necesidad de justificar}
  
  \textit{Proposición}: Si $a$, $b$ y $c$ son números reales tales que $a+c=b+c$, entonces $a=b$. El siguiente es un esbozo de la prueba propuesta por un estudiante: \begin{align*}
   a+c &= b+c\\
   a &= b+c-c\\
   a &= b \end{align*}
  Aunque el resultado anterior no es incorrecto, debemos justificar cada igualdad a partir de las propiedades conocidas con el fin de preservar rigurosiad, al menos en la primera parte de este curso. Esto ayudará a que el lector se famirialice con el uso de las propiedades básicas de los números reales, antes de proceder a realizar pruebas más elaboradas.
\end{tcolorbox}



\subsection*{Lista de Ejercicios 1}

Sean $a$, $b$, y $c$ números reales, demuestre lo siguiente:

\begin{enumerate}[label=\alph*)]
 \item Si $a+b=a$, entonces $b=0$. (Unicidad del neutro aditivo).
 \begin{proof}
  \begin{align*}
  b &= b + 0 && \text{Neutro aditivo}\\
  &= b + \bigl(a+(-a)\bigr) && \text{Inverso aditivo}\\
  &= (b+a) + (-a) && \text{Asociatividad}\\
  &= (a+b) + (-a) && \text{Conmutatividad}\\
  &= a + (-a) && \text{Hipótesis}\\
  &= 0 && \text{Neutro aditivo} \qedhere
  \end{align*}
 \end{proof}\vspace{-1em}
 %\item Demuestre que el elemento neutro para la suma es único. (Unicidad del neutro aditivo).
 %\begin{proof} 
 %Supongamos que existen 0 y $\tilde{0}$ números reales tales que $a+0 = a$ y $a+\tilde{0} = a$. Luego, \begin{align*}
 % 0 &=a+(-a) && \text{Inverso aditivo}\\
 % &=\left( a+\tilde{0} \right)+\left(-a\right) && \text{Hipótesis}\\
 % &=\left( \tilde{0}+a \right)+\left(-a\right) && \text{Conmutatividad}\\
 % &=\tilde{0} + \bigl( a + \left(-a \right)\bigr) && \text{Asociatividad}\\
 % &=\tilde{0} + 0 && \text{Inverso aditivo}\\
 % &=\tilde{0} && \text{Neutro aditivo}\qedhere
 % \end{align*} 
 %\end{proof}
 
 \item Si $a+b=0$, entonces $b=-a$. (Unicidad del inverso aditivo).
 \begin{proof}
 \begin{align*}
   b &= b + 0 && \text{Neutro aditivo}\\
   &= b + \bigl(a+(-a)\bigr) && \text{Inverso aditivo}\\
   &= (b+a) + (-a) && \text{Asociatividad}\\
   &= (a+b) + (-a) && \text{Conmutatividad}\\
   &= 0 + (-a) && \text{Hipótesis}\\
   &= (-a) + 0 && \text{Conmutatividad}\\
   &= -a && \text{Neutro aditivo} \qedhere
  \end{align*}
 \end{proof}
 %\item Demuestre que el inverso aditivo de cada número real es único. (Unicidad del inverso aditivo).
 %\begin{proof} 
 % Sea $a\in \R$ arbitrario pero fijo. Supongamos que existen $-a$ y $-\tilde{a}$ números reales tales que $a + \left(-a\right) = 0$ y $a + \left(- \tilde{a}\right) = 0$. Notemos que:
 %\begin{align*}
 %-a &= -a+0 && \text{Neutro aditivo}\\
 %&= 0+\left(-a\right) && \text{Conmutatividad}\\
 %&= \bigl(a+\left(-\tilde{a} \right)\bigr)+\left(-a\right) && \text{Hipótesis}\\
 %&= \bigl(\left(-\tilde{a} \right)+a\bigr)+\left(-a\right) && \text{Conmutatividad}\\
 %&= \left(-\tilde{a} \right)+\bigl(a+\left(-a\right)\bigr) && \text{Asociatividad}\\
 %&= \left(-\tilde{a} \right)+0 && \text{Inverso aditivo}\\
 %&= -\tilde{a} && \text{Neutro aditivo} \qedhere
 %\end{align*} 
 %\end{proof}
 \textbf{Nota:} Demostrar proposiciones para números reales arbitrarios (cualesquiera elementos de $\R$), nos permite reutilizar las \textit{formas} como esquema para otras pruebas. Por ejemplo, la \textit{forma} de la unicidad del inverso aditivo, $x+y=0 \Longrightarrow y=-x$, nos permite sustituir $x$ y $y$ por cuales quiera números reales, como en el ejemplo que sigue:
 
 \bfit{Corolario:} $-(-a)=a$. (Inverso aditivo del inverso aditivo).
 \begin{proof}
 \begin{align*}
  0 &= a + (-a) && \text{Inverso aditivo}\\
  &= (-a) + a && \text{Conmutatividad}
 \end{align*} Por la unicidad del inverso aditivo sigue que $a=-(-a)$.
 \end{proof}
 \textbf{Nota:} En esta demostración, al emplear la \textit{forma} de la unicidad del inverso aditivo, $x+y=0 \Longrightarrow y=-x$, hemos tomado $x=(-a)$ y $y=a$.

 \item $-0 = 0$. (Cero es igual a su inverso aditivo).
 \begin{proof}
  \begin{align*}
   0 &= 0 + (-0) && \text{Inverso aditivo}\\
   &= (-0) + 0 && \text{Conmutatividad}\\
   &= -0 && \text{Neutro aditivo} \qedhere
  \end{align*}
 \end{proof}
 %\begin{proof} 
 % Por la propiedad del neutro aditivo tenemos que $0+0=0$. Además, el inverso aditivo de $0$ satisface que $0 + (-0) = 0$. Debido a que el inverso aditivo de cada número real es único, de la igualdad anterior sigue que $-0 = 0$. \qedhere 
 %\end{proof}
%
 \item Si $a\neq 0$, entonces $-a\neq 0$.
 \begin{proof} 
  Si $-a=0$, se verifica que  \begin{align*}
   a &= a + 0 && \text{Neutro aditivo}\\
   &= a + (-a) && \text{Hipótesis}\\
   &= 0 && \text{Inverso aditivo}
  \end{align*} Por contraposición, si $a\neq 0$, entonces $-a\neq 0$.
  %Otra forma:
  %Sea $a$ un número real distinto de cero tal que $-a=0$. El inverso aditivo satisface que $a+(-a)=0$, y de la hipótesis, $a+0=0$. De esta igualdad sigue que $a=0$, lo que contradice nuestro supuesto inicial, por lo que que $-a\neq 0$.
 \end{proof}

 \item $-(a+b)=(-a)+(-b)$. (Distribución del signo).

 \begin{proof} 
  \begin{align*}
   0 &= 0 + 0 && \text{Neutro aditivo} \\
   &= \bigl(a+(-a)\bigr) + \bigl(b + (-b)\bigr) && \text{Inverso aditivo} \\
   &= a + \Bigl((-a)+ \bigl(b + (-b)\bigr)\Bigr) && \text{Asociatividad} \\
   &= a + \Bigl( \bigl((-a)+b\bigr) +(-b)\Bigr) && \text{Asociatividad} \\
   &= a + \Bigl( \bigl(b+(-a)\bigr) +(-b)\Bigr) && \text{Conmutatividad} \\
   &= a + \Bigl( b + \bigl( (-a)+(-b) \bigr) \Bigr) && \text{Asociatividad} \\
   &= (a+b) + \bigl((-a)+ (-b)\bigr) && \text{Asociatividad}
   \end{align*}
   Por la unicidad del inverso aditivo, $(-a)+ (-b)=-(a+b)$.  
 \end{proof}

 \textbf{Nota:} En esta demostración, al emplear la \textit{forma} de la unicidad del inverso aditivo, $x+y=0 \Longrightarrow y=-x$, hemos tomado $x=(a+b)$ y $y=(-a)+ (-b)$.

 \bfit{Corolario:} $-\bigl(a+(-b)\bigr)=b+(-a)$. \begin{proof} 
  \begin{align*}
  -\bigl(a+(-b)\bigr)&= (-a) + \bigl(-(-b)\bigr) &&\text{Distribución del signo} \\
  &= (-a) + b &&\text{Inverso aditivo del inverso aditivo} \\
  &= b +(-a) &&\text{Conmutatividad} \qedhere
  \end{align*} 
 \end{proof}

  \textbf{Nota:} En esta demostración, al emplear la \textit{forma} de la distribución del signo, $-(x+y)=(-x)+(-y)$, hemos tomado $x=a$ y $y=(-b)$.

 \item Si $a+c=b+c$, entonces $a=b$. (Ley de cancelación de la suma).
 \begin{proof} 
 \begin{align*}
  a &= a+0 && \text{Neutro aditivo}\\
  &= a+ \bigl(c+(-c)\bigr) && \text{Inverso aditivo}\\
  &= (a+c) + (-c) && \text{Asociatividad}\\
  &= (b+c) + (-c) && \text{Hipótesis}\\
  &= b + \bigl(c+(-c)\bigr) && \text{Asociatividad}\\
  &= b + 0 && \text{Inverso aditivo}\\
  &= b &&\text{Neutro aditivo} \qedhere
 \end{align*} 
\end{proof}

\textbf{Observación:} En el segundo paso de la demostración, podíamos sustituir $0$ por $a+(-a)$ o por $b+(-b)$ (o por cualquier suma igual a $0$). sin embargo, no en todos los casos resultaría útil. Observamos pues que para demostrar proposiciones matemáticas no basta con conocer las propiedades que satisfacen los \textit{objetos} (en este caso números reales) con los que trabajamos; también requerimos intuir su uso apropiado. La experiencia indica que esta intuición se adquiere con la práctica. El lector debería verificar qué ocurre si sustituimos $0$ por $a+(-a)$ o $b+(-b)$ en el segundo paso de esta prueba.

\textbf{Nota:} Si el contexto es claro, enunciaremos esta proposición como ley de cancelación.
\end{enumerate}



\section*{Axiomas de la multiplicación}
La multiplicación $\cdot$ satisface las siguientes propiedades:

\begin{enumerate}[start=6]%[label=M\arabic*., start=0]
 \item Cerradura (de la multiplicación): Si $x,y\in \R$, entonces $x\cdot y\in \R$.
 %La multiplicación es cerrada. Esto significa que para cualesquiera números reales $m$ y $n$ se verifica que $m\cdot n\in \R$.

 \item Conmutatividad (de la multiplicación): Si $x,y\in \R$, entonces $x\cdot y = y\cdot x$.
 %La multiplicación es conmutativa. Esto significa que para cualesquiera números reales $m$ y $n$ se verifica que: $ m \cdot n = n \cdot m $.

 \item Asociatividad (de la multiplicación): Si $x,y,z\in \R$, entonces $(x\cdot y) \cdot z = x \cdot (y\cdot z)$.
 %La multiplicación es asociativa. Esto significa que para cualesquiera números reales $m$, $n$ y $l$ se verifica que: $ m \cdot (n \cdot l) = (m \cdot n) \cdot l $.

 \item Neutro multiplicativo (o uno): $\exists 1\in \R$ y $1\neq 0$ tal que si $x\in \R$, entonces $x\cdot 1 = x$.
 %Existe un número real distinto de cero, llamado elemento identidad para la multiplicación o uno, denotado por $1$, que satisface la siguiente condición: $ m \cdot 1 = m,\forall m \in \R $.

 \item Inverso multiplicativo: Si $x\in \R$ y $x\neq 0$, entonces $\exists x^{-1}\in \R$ tal que $x\cdot x^{-1}=1$.
 %Para cada número real $m$ distinto de cero existe un número real llamado inverso multiplicativo de $m$, denotado por $m^{-1}$, este elemento tiene la siguiente propiedad: $m \cdot m^{-1} = 1$.
\end{enumerate}

\subsection*{Lista de Ejercicios 2}

Sean $a$, $b$, y $c$ números reales, demuestre lo siguiente:
 \begin{enumerate}[label=\alph*)]
 \item Si $a\neq 0$ y $a\cdot b = a$, entonces $b=1$. (Unicidad del neutro multiplicativo).
 \begin{proof} 
 \begin{align*}
  b &= b \cdot 1 && \text{Neutro multiplicativo}\\
  &= b \cdot (a\cdot a^{-1}) && \text{Inverso multiplicativo}\\
  &= (b\cdot a) \cdot a^{-1} && \text{Asociatividad}\\
  &= (a\cdot b) \cdot a^{-1} && \text{Conmutatividad}\\
  &= a \cdot a^{-1} && \text{Hipótesis}\\
  &= 1 && \text{Inverso multiplicativo} \qedhere
 \end{align*} 
 \end{proof}
 \textbf{Nota:} La prueba requiere que $a\neq 0$, pues de otro modo (si $a=0$), no podemos garantizar que $b=1$. Veremos la prueba de este hecho más adelante.
 %Demuestre que el elemento identidad para la multiplicación es único. (Unicidad del neutro multiplicativo).
 %
 %\begin{proof} 
 % Supongamos que existen $1$ y $\tilde{1}$ números reales tales que $a\cdot 1=a$ y $a\cdot\tilde{1}=a$. Luego,
 %\begin{align*}
 %1 &= a \cdot a^{-1} && \text{Inverso multiplicativo}\\
 %&= \left( a \cdot \tilde{1} \right) \cdot a^{-1} && \text{Por hipótesis}\\
 %&= \left( \tilde{1} \cdot a \right) \cdot a^{-1} && \text{Conmutatividad}\\
 %&= \tilde{1} \cdot \left( a \cdot a^{-1} \right) && \text{Asociatividad}\\
 %&= \tilde{1} \cdot 1 && \text{Inverso multiplicativo}\\
 %&= \tilde{1} && \text{Neutro multiplicativo}\qedhere
 %\end{align*}  
 %\end{proof}
 
 \item Si $a\neq 0$ y $a\cdot b =1$, entonces $b=a^{-1}$. (Unicidad del inverso multiplicativo).
 \begin{proof} 
 \begin{align*}
  b &= b \cdot 1 && \text{Neutro multiplicativo}\\
  &= b \cdot (a\cdot a^{-1}) && \text{Inverso multiplicativo}\\
  &= (b\cdot a) \cdot a^{-1} && \text{Asociatividad}\\
  &= a^{-1} \cdot (a\cdot b) && \text{Conmutatividad}\\
  &= a^{-1} \cdot 1 && \text{Hipótesis}\\
  &= a^{-1} && \text{Neutro multiplicativo} \qedhere
 \end{align*} 
 \end{proof}
 \textbf{Nota:} La prueba requiere que $a\neq 0$, pues de otro modo no podemos garantizar la existencia de su inverso multiplicativo.
 %Demuestre que el inverso multiplicativo de cada número real distinto de cero es único. (Unicidad del inverso multiplicativo).
 %
 %\begin{proof} Supongamos que existen $a^{-1}$ y $\tilde{a}^{-1}$ números reales, distintos de cero, tales que $a \cdot a^{-1} = 1$ y $a \cdot \tilde{a}^{-1} = 1$. Luego,
 % \begin{align*}
 %  a^{-1} &= a^{-1} \cdot 1 && \text{Neutro multiplicativo} \\
 %  &= a^{-1} \cdot \left(a \cdot \tilde{a}^{-1} \right) && \text{Por hipótesis} \\
 %  &= \left( a^{-1} \cdot a \right) \cdot \tilde{a} ^{-1} && \text{Asociatividad} \\
 %  &= \left(a \cdot a^{-1} \right) \cdot \tilde{a}^{-1} && \text{Conmutatividad} \\
 %  &= 1 \cdot \tilde{a}^{-1} && \text{Inverso multiplicativo} \\
 %  &= \tilde{a}^{-1} \cdot 1 && \text{Conmutatividad} \\
 %  &= \tilde{a}^{-1} && \text{Neutro multiplicativo}\qedhere
 %  \end{align*}  
 %\end{proof}
 
 \item $1=1^{-1}$. (Uno es inverso multiplicativo).
 \begin{proof} 
 \begin{align*}
  1 &= 1 \cdot 1^{-1} && \text{Inverso multiplicativo}\\
  &= 1^{-1} \cdot 1 && \text{Conmutatividad}\\
  &= 1^{-1} && \text{Neutro multiplicativo} \qedhere
 \end{align*} 
 \end{proof}
 \textbf{Nota:} Por el axioma del neutro multiplicativo sabemos que $1\neq 0$, por lo que existe su inverso multiplicativo.
 %\begin{proof}
 % Por axioma del neutro multiplicativo tenemos que $1\cdot 1 = 1$ y $1\neq 0$, por lo que existe $1^{-1}$ tal que $1 \cdot 1^{-1}=1$. Por unicidad del inverso multiplicativo, de la igualdad anterior sigue que $1=1^{-1}$.
 %\end{proof}


 
 \item Si $c\neq 0$ y $a\cdot c=b\cdot c$, entonces $a=b$. (Ley de cancelación de la multiplicación).
 \begin{proof} 
  \begin{align*}
   a &= a\cdot 1 && \text{Neutro multiplicativo}\\
   &= a \cdot \bigl(c\cdot c^{-1}\bigr) && \text{Inverso multiplicativo}\\
   &= (a \cdot c) \cdot c^{-1} && \text{Asociatividad}\\
   &= (b \cdot c) \cdot c^{-1} && \text{Hipótesis}\\
   &= b \cdot \bigl(c\cdot c^{-1}\bigr) && \text{Asociatividad}\\
   &= b \cdot 1 && \text{Inverso multiplicativo}\\
   &= b && \text{Neutro multiplicativo}\qedhere
   \end{align*} 
 \end{proof}
 \textbf{Observación:} La prueba requiere que $c\neq 0$, pues de otro modo no podemos garantizar la existencia de su inverso multiplicativo.
 %Para esta proposición requerimos que $c\neq 0$, ya que de caber la posibilidad de que $c=0$, no tendríamos garantía de que $a=b$. El lector debería verificar este hecho. (Ej. $2\cdot 0 = 1 \cdot 0$).

 \textbf{Nota:} Si el contexto es claro, enunciaremos esta proposición como ley de cancelación.
 \end{enumerate}
 
\section*{Propiedad distributiva}

Introducimos la propiedad que nos permite relacionar las operaciones de suma $+$ y multiplicación $\cdot$

\begin{enumerate}[start=11]%[label=P.D.]
 \item Distribución (de la multiplicación sobre la suma): Si $x,y,z\in \R$, entonces $x\cdot (y+z)=x\cdot y+ x\cdot z$.
 %Distribución de la multiplicación sobre la suma. Para cualesquiera números reales $m$, $n$ y $l$ se verifica que: $ m \cdot (n+l)=m \cdot n+m \cdot l $.
\end{enumerate}

\begin{tcolorbox}
  \subsection*{Ejemplo de argumento circular}
%
%En este apartado utilizaremos un ejemplo para puntualizar la falta de rigor en la que pueden caer los estudiantes; dichas puntualizaciones pueden parecer exageradas y el lector podría considerar que el autor está siendo \textit{pedante} en el uso sintáctico de los axiomas, pero la idea es proveer al estudiante de un uso riguroso de la formalización matemática.
%
Proposición: $b\cdot 0 = 0$. El siguiente es un esbozo de la prueba propuesta por un estudiante:
% \begin{align*}
% b \cdot 0 &= b\cdot \bigl(a+(-a)\bigr) && \text{Inverso aditivo}\\
% &= ab + (-ab) && \text{Distribución}\\
% &= 0 && \text{Inverso aditivo}
%\end{align*}
%De inmediato podemos señalar la \textbf{ambigüedad} en el uso de la notación en el segundo paso. ¿Qué debemos entender por $-ab$? Por como fue enunciado, se pretendía que $-ab$ fuera el inverso aditivo de $ab$, es decir, se intuye que $-ab=-(ab)$, pero al enunciar la propiedad distributiva se asumió que $-ab=(-a)\cdot b$, es decir, se supone la igualdad $-(ab)=(-a)\cdot b$, pero esta requiere demostración.
%
%Asimismo, al utilizar la propiedad distributiva se está empleando conmutatividad no enunciada, pues la síntaxis de la distribución indica que el número a la izquierda de la multiplicación debe ser \textit{distribuido} a la izquierda de los componentes de la suma, es decir, $b\cdot \bigl(a+(-a)\bigr)=b\cdot a+ b\cdot (-a)$ por la propiedad distributiva, y luego $b\cdot a+ b\cdot (-a)=a\cdot b+ (-a)\cdot b$, por conmutatividad.
%
%Al considerar el comentario anterior, el estudiante reescribe el esbozo como sigue:
 \begin{align*}
 b\cdot 0 &= b\cdot \bigl(a+ (-a)\bigr) && \text{Inverso aditivo}\\
 &= b\cdot a + b\cdot (-a) && \text{Distribución}\\
 &= a\cdot b + (-a) \cdot b && \text{Conmutatividad}\\
 &= 0 && \text{¿?}
\end{align*}
Pero se requiere probar que $a\cdot b+(-a)\cdot b=0$. Observemos ahora el siguiente esbozo para esta prueba:  \begin{align*}
 a\cdot b+ (-a) \cdot b &= b\cdot a + b\cdot (-a) && \text{Conmutatividad}\\
 &= b\cdot \bigl(a+(-a)\bigr) && \text{Distribución}\\
 &= b\cdot 0 && \text{Inverso aditivo}\\
 &= 0 && \text{¿?}
\end{align*}
No obstante, se ha propuesto un \textbf{argumento circular}, por lo que no es posible verificar ninguna de las proposiciones anteriores. Requerimos pues, depender únicamente de axiomas o proposiciones previamente probadas para continuar.
\end{tcolorbox}



\subsection*{Lista de Ejercicios 3 (LE3)}

Sean $a$ y $b$ números reales, demuestre lo siguiente:

\begin{enumerate}[label=\alph*)]

 \item $a \cdot 0 = 0$. (Multiplicación por $0$).
 \begin{proof} 
  \begin{align*}
   a\cdot0&=a\cdot0+0 && \text{Neutro aditivo}\\
   &=a\cdot0+\bigl(a+\left(-a\right)\bigr) && \text{Inverso aditivo}\\
   &=a\cdot0+\bigl(a\cdot1+\left(-a\right)\bigr) && \text{Neutro multiplicativo}\\
   &=\left(a\cdot0+a\cdot1\right)+\left(-a\right) && \text{Asociatividad}\\
   &=\bigl(a\cdot\left(0+1\right)\bigr)+\left(-a\right) && \text{Distribución}\\
   &=a\cdot1+\left(-a\right) && \text{Neutro aditivo}\\
   &=a+\left(-a\right) && \text{Neutro multiplicativo}\\
   &=0 && \text{Inverso aditivo} \qedhere
  \end{align*} 
 \end{proof}

 \bfit{Corolario:} Si $a\neq 0$, entonces $a^{-1}\neq 0$. (Cero no es inverso multiplicativo).
 \begin{proof} 
  Sea $a\neq 0$. Si $a^{-1}=0$, se verifica que \begin{align*}
   1 &= a\cdot a^{-1} && \text{Inverso multiplicativo}\\
   &= a\cdot 0 && \text{Hipótesis}\\
   &= 0 && \text{Multiplicación por $0$}
  \end{align*} Pero esto contradice la propiedad del neutro multiplicativo. Por tanto, si $a\neq 0$, entonces $a^{-1}\neq 0$.
 \end{proof}

 \textbf{Nota:} El axioma del neutro multiplicativo no implica directamente que $0$ no pueda ser inverso multiplicativo de algún número real, únicamente indica que si $x\in \R$ y $x\neq 0$, entonces $\exists x^{-1}$. El axioma tampoco especifica que para $0$ el inverso multiplicativo no existe, sin embargo, si suponemos su existencia, es decir, si $\exists 0^{-1}\in \R$ tal que $0\cdot 0^{-1}=1$, tenemos por la multiplicación por $0$ que $0=1$, lo que es una contradicción.

 \item Si $ a \cdot b = 0 $, entonces $a=0$ o $b=0$ (disyunción).
 \begin{proof} Demostraremos primero que si $a\neq 0$ y $b\neq 0$, entonces $a\cdot b\neq 0$.
 
 Sea $a\neq 0$ y $b\neq 0$. Notemos que \begin{align*}
  a &= a\cdot 1 && \text{Neutro multiplicativo}\\
  &= a\cdot \bigl(b\cdot b^{-1}\bigr) && \text{Inverso multiplicativo}\\
  &= (a\cdot b) \cdot b^{-1} && \text{Asociatividad}
 \end{align*}
 Por hipótesis $a\neq 0$, por lo que $0\neq (a\cdot b) \cdot b^{-1}$. Además, $b^{-1}\neq 0$, pues cero no es inverso multiplicativo.
 
 Si $a\cdot b=0$, por la multiplicación por cero, $(a\cdot b) \cdot b^{-1}=0$, lo que es una contradicción. Por tanto, si $a\neq 0$ y $b\neq 0$, entonces $a\cdot b\neq 0$. Finalmente, por contraposición, si $a\cdot b =0$, entonces $a=0$ o $b=0$.
 \end{proof}
 %\begin{proof} 
 %Supongamos que $a$ es distinto de $0$.
 %\begin{align*}
 %b &= b \cdot 1	&& \text{Neutro multiplicativo} \\
 %&= b \cdot  \left(a \cdot a^{-1}  \right) 	&& \text{Inverso multiplicativo} \\
 %&= \left(b\cdot a\right)  \cdot a^{-1}	&& \text{Asociatividad} \\
 %&= \left(a\cdot b\right)  \cdot a^{-1}	&& \text{Conmutatividad} \\
 %&= 0 \cdot a^{-1}	&& \text{Por hipótesis}\\
 %&= a^{-1} \cdot 0	&& \text{Conmutatividad}\\
 %&= 0 && \text{Multiplicación por $0$} \qedhere
 %\end{align*} 
 %\end{proof}
 %\textbf{Nota:} Al utilizar la expresión $A$ o $B$ nos referimos a la disyunción (lógica).%, es decir, la proposición es verdadera si únicamente $a=0$, únicamente $b=0$ o ambos $a$ y $b$ son iguales a cero.

 \item Si $a\neq 0$ y $b\neq 0$, entonces $(a \cdot b)^{-1}=a^{-1} \cdot b^{-1}$. (Multiplicación de inversos multiplicativos).
 \begin{proof} 
 \begin{align*}
  1 &= b\cdot b^{-1} && \text{Inverso multiplicativo}\\
  &= (b\cdot 1) \cdot b^{-1} && \text{Neutro multiplicativo}\\
  &= \bigl(b\cdot (a\cdot a^{-1})\bigr) \cdot b^{-1} && \text{Inverso multiplicativo}\\
  &= (b\cdot a) \cdot \bigl(a^{-1} \cdot b^{-1}\bigr) && \text{Asociatividad}\\
  &= (a\cdot b) \cdot \bigl(a^{-1} \cdot b^{-1}\bigr) && \text{Conmutatividad}
 \end{align*} Por la unicidad del inverso multiplicativo $a^{-1} \cdot b^{-1}=(a\cdot b)^{-1}$.
 \end{proof}
 \textbf{Nota:} En esta demostración está implícito que $\exists (a\cdot b)^{-1}\in \R$, lo cual es válido pues hemos probado que si $a\neq 0$ y $b\neq 0$, entonces $a\cdot b\neq 0$, por lo que existe su inverso multiplicativo.
 %\begin{proof} 
 % \begin{align*}
 % && \quad \left(a \cdot b \right) \cdot  \left(a^{-1} \cdot b^{-1}  \right)	&=	 \left( \left(a \cdot b\right) \cdot a^{-1}  \right) \cdot b^{-1}  	&& \text{Asociatividad}\\
 % && \quad &=	 \left( \left(b\cdot a \right) \cdot a^{-1}  \right) \cdot b^{-1}  	&& \text{Conmutatividad}\\
 % && \quad &=	 \Bigl(b\cdot  \left(a \cdot a^{-1}\right) \Bigr) \cdot b^{-1}	&& \text{Asociatividad}\\
 % && \quad &=	 \left(b\cdot 1 \right) \cdot b^{-1}	&& \text{Inverso multiplicativo}\\
 % && \quad &=	b\cdot b^{-1}	&& \text{Neutro multiplicativo}\\
 % && \quad &=	1	&& \text{Inverso multiplicativo}
 % \end{align*}
 % Sigue que $\left(a^{-1} \cdot b^{-1} \right)$ es inverso multiplicativo de $\left( a \cdot b\right)$, y por la unicidad del inverso multiplicativo, sigue que $\left(a^{-1} \cdot b^{-1} \right) = \left( a \cdot b\right)^{-1}$. 
 %\end{proof}



 \item Si $a\neq 0$, entonces $\left( a^{-1} \right)^{-1}=a$.
 \begin{proof} 
  \begin{align*}
   1 &= a\cdot a^{-1} && \text{Inverso multiplicativo}\\
   &= a^{-1} \cdot a && \text{Conmutatividad}
  \end{align*} Por la unicidad del inverso multiplicativo sigue que $a=\bigl(a^{-1}\bigr)^{-1}$.
 \end{proof}
 \textbf{Nota:} En esta demostración está implícito que $\exists\left( a^{-1} \right)^{-1}\in \R$, lo cual es válido pues cero no es inverso multiplicativo, es decir, tenemos $a^{-1}\neq 0$, por lo que existe su inverso multiplicativo.
 
 Al emplear la \textit{forma} de la unicidad del inverso multiplicativo, $x\neq 0 \land x\cdot y=1 \Longrightarrow y = x^{-1}$, hemos tomado $x=a^{-1}$ y $y=a$.
 %\begin{proof} 
 % El inverso multiplicativo de $a$ satisface que $a\cdot a^{-1}=1$, y por conmutatividad, $a^{-1} \cdot a=1$, de esta igualdad se sigue que $a$ es inverso multiplicativo de $a^{-1}$. Similarmente, el inverso multiplicativo de $a^{-1}$ satisface que $ a^{-1} \cdot \left( a^{-1} \right)^{-1} =1$, y por la unicidad del inverso multiplicativo, sigue que $\left( a^{-1} \right)^{-1}=a$.  
 %\end{proof}

 \item $(-1)=(-1)^{-1}$. (Menos uno es inverso multiplicativo).
 
 \begin{proof} Primero probaremos la existencia de $(-1)^{-1}$.
  
 Si $-1=0$, tenemos que $1+(-1)=1+0$, y por neutro aditivo $1+(-1)=1$, pero el inverso aditivo satisface que $1+(-1)=1$, de donde sigue que $1=0$, lo que contradice la propiedad del neutro multiplicativo. Por tanto, $-1\neq 0$, por lo que $\exists (-1)^{-1}\in \R$. Luego,

 \vspace{-1em} \begin{align*}
  0 &= 1 + (-1) && \text{Inverso aditivo}\\
  &= (-1)\cdot (-1)^{-1} + (-1) && \text{Inverso multiplicativo}\\
  &= (-1)\cdot (-1)^{-1} + (-1) \cdot 1 && \text{Neutro multiplicativo}\\
  &= (-1) \cdot \Bigl((-1)^{-1} + 1\Bigr) && \text{Distribución}
 \end{align*} Como $-1\neq 0$, sigue que $(-1)^{-1} + 1=0$, y por conmutatividad $1+(-1)^{-1}=0$. Finalmente, por unicidad del inverso aditivo, $(-1)^{-1}=-1$.
 \end{proof}
 \textbf{Nota:} En esta demostración, al emplear la \textit{forma} de la unicidad del inverso aditivo, $x+y=0 \Longrightarrow y=-x$, hemos tomado $x=1$ y $y=(-1)^{-1}$.

 \item $(-a) \cdot b = -(a \cdot b) = a \cdot (-b)$. (Multiplicación por inverso aditivo).% (Ley de los signos: menos por más/más por menos es menos).
 \begin{proof} \leavevmode
 \begin{center}\vspace{-2em}
 \begin{minipage}[t]{.5\linewidth}
  \begin{align*}
   0 &= b\cdot 0 && \text{Multiplicación por $0$}\\
   &= b \cdot \bigl(a+(-a)\bigr) && \text{Inverso aditivo}\\
   &= b\cdot a + b\cdot (-a) && \text{Distribución}\\
   &= a\cdot b + (-a) \cdot b && \text{Conmutatividad}
  \end{align*}
 \end{minipage}%
 \begin{minipage}[t]{.5\linewidth}
  \begin{align*}
   0 &= a\cdot 0 && \text{Multiplicación por $0$}\\
   &= a \cdot \bigl(b+(-b)\bigr) && \text{Inverso aditivo}\\
   &= a\cdot b + a\cdot (-b) && \text{Distribución}
  \end{align*}
 \end{minipage}
 \end{center} Por unicidad del inverso aditivo, se verifica que $(-a)\cdot b = -(a\cdot b)=a\cdot (-b)$.
 \end{proof}%
 \textbf{Nota:} En esta demostración, al emplear la \textit{forma} de la unicidad del inverso aditivo, $x+y=0 \Longrightarrow y=-x$, hemos tomado, $x=a\cdot b$ y $y=(-a)\cdot b$, por una parte y $y=a\cdot (-b)$, por la otra.
 %Otra forma:
 %\begin{proof} 
 %\begin{align*}
 % (-a) \cdot b &= \bigl( \left(-1 \right) \cdot a \bigr) \cdot b && \text{Multiplicación por ($-1$)}\\
 % &= (-1) \cdot (a \cdot b) && \text{Asociatividad}\\
 % &= -(a \cdot b) && \text{Multiplicación por ($-1$)} && \text{(*)}\\
 % &= -(b \cdot a) && \text{Conmutatividad}\\
 % &= (-1) \cdot (b\cdot a) && \text{Multiplicación por ($-1$)}\\
 % &= \bigl((-1)\cdot b\bigr) \cdot a && \text{Asociatividad}\\
 % &= a\cdot \bigl((-1)\cdot b\bigr) && \text{Conmutatividad}\\
 % &= a \cdot (-b) && \text{Multiplicación por ($-1$)} && \text{(**)}
 % \end{align*} Por (*) y (**) tenemos que $ (-a) \cdot b = -(a \cdot b) = a \cdot (-b)$.
 %\end{proof}
 %\textbf{Nota:} A partir de esta demostración evitamos la ambigüedad que se mencionó en el apartado \textit{Una nota sobre rigurosidad}.

 \bfit{Corolario:}
 \begin{enumerate}[label=\roman*)]
  \item $(-a)\cdot(-b)=a\cdot b$.% (Ley de los signos: menos por menos es más).
  %\vspace{-1em}
  \begin{proof} 
   \begin{align*}
    \qquad (-a)\cdot (-b) &= a \cdot \bigl(-(-b)\bigr) && \text{Multiplicación por inverso aditivo}\\
    \qquad &= a \cdot b && \text{Inverso aditivo del inverso aditivo} \qedhere
   \end{align*} 
  \end{proof}
  \textbf{Nota:} Al emplear la \textit{forma} de la multiplicación por inverso aditivo, $(-x)\cdot y = x \cdot (-y)$, hemos tomado $x=a$ y $y=(-b)$.
  %Otra demostración
  %\begin{proof} 
  % \begin{align*}
  %  (-a) \cdot (-b) &= (-a) \cdot \bigl( (-1) \cdot b \bigr) && \text{Multiplicación por ($-1$)}\\
  %  &= \bigl( (-a) \cdot (-1) \bigr) \cdot b && \text{Asociatividad}\\
  %  &= \bigl( (-1) \cdot (-a) \bigr) \cdot b && \text{Conmutatividad}\\
  %  &= -(-a) \cdot b && \text{Multiplicación por ($-1$)}\\
  %  &= a \cdot b && \text{Unicidad del inverso aditivo} \qedhere
  % \end{align*} 
  %\end{proof}
  %\textbf{Nota:} A las proposiciones $ (-m) \cdot n = -(m \cdot n) $ y $ (-m) \cdot (-n) = m \cdot n $, las enunciaremos como \textbf{leyes de los signos}.
%
  %\item $(-1) \cdot a =-a $. (Multiplicación por -1).
%
  %Por el teorema, $(-1)\cdot a = -(1\cdot a) = -a$.
  %
  %\begin{proof} 
  % \begin{align*}
  % 0 &= a \cdot 0 && \text{Multiplicación por $0$}\\
  % &= a \cdot \bigl(1+(-1)\bigr) && \text{Inverso aditivo}\\
  % &= a \cdot 1 + a \cdot (-1)  && \text{Distribución}\\
  % &= a + a \cdot (-1)  && \text{Neutro multiplicativo}\\
  % &= a + (-1) \cdot a  && \text{Conmutatividad}
  % \end{align*} Sigue que $(-1) \cdot a$ es inverso aditivo de $a$, el cual es único, por lo que $(-1) \cdot a = -a$.
  %\end{proof}
  %\textbf{Observación:} Al multiplicar cualquier número real por ($-1$) obtenemos el inverso multiplicativo de ese número real.
  %
  %Otra forma de demostrar este hecho es la siguiente: \begin{align*}
  % -a&=-a+0 && \text{Neutro aditivo} \\
  % &=-a+a \cdot 0 && \text{Multiplicación por $0$} \\
  % &=-a+a \cdot  \bigl( 1+ \left( -1 \right)  \bigr) && \text{Inverso aditivo} \\
  % &=-a+ \bigl( a \cdot 1+a \cdot  \left( -1 \right)  \bigr) && \text{Distribución} \\
  % &=-a+ \bigl( a+a \cdot  \left( -1 \right)  \bigr) && \text{Inverso aultiplicativo} \\
  % &= \left( -a+a \right) +a \cdot  \left( -1 \right) && \text{Asociatividad} \\
  % &=a+ \left( -a \right) +a \cdot  \left( -1 \right) && \text{Conmutatividad} \\
  % &=0+a \cdot  \left( -1 \right) && \text{Inverso aditivo} \\
  % &=a \cdot  \left( -1 \right) + 0 && \text{Conmutatividad} \\
  % &=a \cdot  \left( -1 \right) && \text{Neutro aditivo} \\
  % &= \left( -1 \right)  \cdot a && \text{Conmutatividad}
  %\end{align*} \qed



  \item $-(a^{-1})=(-a)^{-1}=(-1)\cdot a^{-1}$. (Inverso aditivo del inverso multiplicativo).
  %\vspace{-1em}
  \begin{proof} 
  \begin{align*}
  \qquad (-1)\cdot a^{-1} &= -(1\cdot a^{-1}) && \text{Multiplicación por inverso aditivo}\\
  \qquad &= -\bigl(a^{-1}\bigr) && \text{Neutro multiplicativo}
  \end{align*}
  Similarmente,
  \begin{align*}
   -(a^{-1}) &= \bigl(-(a^{-1})\bigr) \cdot 1 && \text{Neutro multiplicativo}\\
   &= -\Bigl(\qty(a^{-1})\cdot 1\Bigr) && \text{Multiplicación por inverso aditivo}\\
   &= -\Bigl(a^{-1}\Bigr) && \text{Neutro multiplicativo} \qedhere
  \end{align*}
  \end{proof}
  \textbf{Nota:} Al emplear la \textit{forma} de la multiplicación por inverso aditivo, $(-x)\cdot y = -(x\cdot y)$, hemos tomado $x=1$ y $y=a^{-1}$, por una parte, y $x=(a^{-1})$ y $y=1$, por la otra.
 \end{enumerate}% casos especiales (-a)b=-ab

\end{enumerate}



\textbf{Notación}
%Esta sección tiene el propósito de introducir al lector al uso de notación por cambio de \textit{etiqueta}, esto es, asignar distintos símbolos a \textit{objetos} con los que hemos trabajado, con el fin de agilizar la demostración de teoremas.
\begin{itemize}
\item Si $x$ y $y$ son números reales, representaremos con el símbolo $x-y$ a la suma $x+(-y)$.
\item Si $x,y\in \R$ y $y\neq 0$, representaremos con el símbolo $ \frac{x}{y}$ al número $x \cdot y^{-1}$.

Es inmediato que si $w\neq 0$, entonces $\frac{w}{w} = w\cdot w^{-1} = 1$.

\item Si $x$ y $y$ son números reales, representaremos con el símbolo $xy$ a la multiplicación $x\cdot y$.
\end{itemize}

\subsection*{Lista de ejercicios 4 (LE4)}

Sean $a$, $b$, $c$ y $d$ números reales, demuestre lo siguiente:

\begin{enumerate}[label=\alph*)]
 \item $\frac{a}{b}=a\cdot \frac{1}{b}$, si $b\neq 0$.% (Definición de la división).
 \begin{proof} 
 \begin{align*}
  \frac{a}{b} &= a\cdot b^{-1} && \text{Notación}\\
  &= (a\cdot 1) \cdot b^{-1} && \text{Neutro multiplicativo}\\
  &= a\cdot \bigl(1\cdot b^{-1}\bigr) && \text{Asociatividad}\\
  &= a\cdot \frac{1}{b} && \text{Notación} \qedhere
 \end{align*}
 \end{proof}
 %\begin{proof} \begin{align*} \qquad
 % a\cdot \frac{1}{b} &= (a\cdot 1) \cdot \frac{1}{b} && \text{Neutro multiplicativo}\\
 % &= \bigl(a\cdot 1^{-1}\bigr) \cdot \frac{1}{b} && \text{Uno es inverso multiplicativo}\\
 % &= \frac{a}{1} \cdot \frac{1}{b} && \text{Notación}\\
 % &= \frac{a\cdot 1}{1\cdot b} && \text{Multiplicación de fracciones}\\
 % &= \frac{a}{b} && \text{Neutro multiplicativo} \qedhere
 %\end{align*}
 %\end{proof}
 %\begin{align*}
 % \frac{a}{b} &= a\cdot b^{-1} && \text{Notación}\\
 % &= a\cdot 1 \cdot 1\cdot b^{-1}&& \text{Neutro multiplicativo}\\
 % &= a \cdot 1^{-1} \cdot 1\cdot b^{-1} && \text{Uno es inverso multiplicativo}\\
 % &= \frac{a}{1} \cdot \frac{1}{b} && \text{Notación}\\
 % &= \frac{a\cdot 1}{1\cdot b} && \text{Multiplicación de fracciones}\\
 % &= \frac{a}{b} && \text{Neutro multiplicativo} \qedhere
 %\end{align*}

 \item $a \cdot \frac{c}{b} = \frac{ac}{b}$, si $b \neq 0$.
 \begin{proof} 
 \begin{align*}
  a\cdot \frac{c}{b} &= a\cdot \bigl(c\cdot b^{-1}\bigr) && \text{Notación}\\
  &= (ac) \cdot b^{-1} && \text{Asociatividad}\\
  &= \frac{ac}{b} && \text{Notación} \qedhere
 \end{align*}
 \end{proof}
 %\vspace{-1em}
 %\begin{proof} 
 %\begin{align*}
 % a\cdot \frac{c}{b} &= a\cdot 1 \cdot \frac{c}{b} && \text{Neutro multiplicativo}\\
 % &= a\cdot 1^{-1} \cdot \frac{c}{b} && \text{Uno es inverso multiplicativo}\\
 % &= \frac{a}{1} \cdot \frac{c}{b} && \text{Notación}\\
 % &= \frac{ac}{1\cdot b} && \text{Multiplicación de fracciones}\\
 % &= \frac{a}{b} && \text{Neutro multiplicativo} \qedhere
 %\end{align*}
 %\end{proof}
 %\begin{align*}
 % a\cdot \frac{c}{b} &= a\cdot c \cdot \frac{1}{b} && \text{Definición de la división}\\
 % &= a\cdot c \cdot 1 \cdot \frac{1}{b} && \text{Neutro multiplicativo}\\
 % &= a\cdot c\cdot 1^{-1} \cdot \frac{1}{b} && \text{Uno es inverso multiplicativo}\\
 % &= \frac{ac}{1} \cdot \frac{1}{b} && \text{Notación}\\
 % &= \frac{ac\cdot 1}{1\cdot b} && \text{Multiplicación de fracciones}\\
 % &= \frac{ac}{b} && \text{Neutro multiplicativo}
 %\end{align*}
 %De la división por $1$, tenemos que $a\cdot \frac{c}{b}=\frac{a}{1}\cdot \frac{c}{b}$, y por este teorema $\frac{a}{1}\cdot \frac{c}{b}=\frac{ac}{b\cdot 1}$, osea $a \cdot \frac{c}{b} = \frac{ac}{b}$.
 %Otra forma de demostrar esta proposición es la siguiente:
 %\begin{align*}
 % a \cdot \frac{c}{b} &= a \cdot \left( c \cdot b^{-1} \right) && \text{Notación}\\
 % &= \left( a \cdot c \right) \cdot b^{-1} && \text{Asociatividad}\\
 % &= \frac{ac}{b} && \text{Notación}
 %\end{align*}

 \item $\frac{a}{b} \cdot \frac{c}{d} = \frac{ac}{bd}$, si $b, d \neq 0$. (Multiplicación de fracciones).
 %\vspace{-1em} \begin{proof} \begin{align*}
 % \quad \frac{a}{b} \cdot \frac{c}{d} &= ab^{-1} \cdot cd^{-1} && \text{Notación}\\
 % &= ac\cdot b^{-1}d^{-1} && \text{Conmutatividad}\\
 % &= ac\cdot (bd)^{-1} && \text{Multiplicación de inversos multiplicativos}\\
 % &= \frac{ac}{bd} && \text{Notación} \qedhere
 %\end{align*}
 %\end{proof}
 \begin{proof}
 \begin{align*}\qquad \ \qquad
 \frac{a}{b} \cdot \frac{c}{d} &= \left( a \cdot b^{-1} \right) \cdot \left( c \cdot d^{-1} \right) && \text{Notación}\\
 &= a \cdot \Bigl( b^{-1} \cdot \left( c \cdot d^{-1} \right) \Bigr) && \text{Asociatividad}\\
 &= a\cdot \Bigl(\bigl(b^{-1}\cdot c\bigr) \cdot d^{-1}\Bigr) && \text{Asociatividad}\\
 &= a\cdot \Bigl(\bigl(c\cdot b^{-1}\bigr) \cdot d^{-1}\Bigr) && \text{Conmutatividad}\\
 &= a\cdot \Bigl(c\cdot \bigl(b^{-1}\cdot d^{-1}\bigr)\Bigr) && \text{Conmutatividad}\\
 &= a\cdot \Bigl(c\cdot \bigl(b\cdot d\bigr)^{-1}\Bigr) && \text{Multiplicación de inversos multiplicativos}\\
 &= (a\cdot c) \cdot (b\cdot d)^{-1} && \text{Asociatividad}\\
 &= \frac{ac}{bd} && \text{Notación} \qedhere
 \end{align*} 
 \end{proof}
 
 \item $\frac{a}{b} = \frac{ac}{bc}$, si $b,c \neq 0$. (Cancelación de factores en común).
 \begin{proof} 
 \begin{align*}
 \frac{a}{b}&=\frac{a}{b}\cdot 1 && \text{Neutro multiplicativo}\\
 &= \frac{a}{b}\cdot \bigl(c \cdot c^{-1}\bigr) && \text{Inverso multiplicativo}\\
 &= \frac{a}{b} \cdot \frac{c}{c} && \text{Notación}\\
 &= \frac{ac}{b\cdot c} && \text{Multiplicación de fracciones} \qedhere
 \end{align*}
 \end{proof}
 %Otra forma de demostrar esta proposición es la siguiente:
 %\begin{align*}
 % \frac{ac}{bc} &= a \cdot c \cdot \left( bc \right)^{-1} && \text{Notación}\\
 % &= a \cdot c \cdot b^{-1} \cdot c^{-1} && \text{Multiplicación de inversos multiplicativos}\\
 % &= a \cdot b^{-1} \cdot c \cdot c^{-1} && \text{Conmutatividad}\\
 % &= a \cdot b^{-1} \cdot 1 && \text{Inverso multiplicativo}\\
 % &= a \cdot b^{-1} && \text{Neutro multiplicativo}\\
 % &= \frac{a}{b} && \text{Notación}
 %\end{align*}
 
 \item $\frac{\frac{a}{b}}{\frac{c}{d}} = \frac{ad}{bc}$, si $b, c, d \neq 0$. (Regla del sandwich).
 \begin{proof} 
 \begin{align*} \qquad \ \qquad
 \frac{\frac{a}{b}}{\frac{c}{d}} &= \frac{\left( a \cdot b^{-1} \right)}{\left( c \cdot d^{-1} \right)} && \text{Notación}\\
 &= \left( a \cdot b^{-1} \right) \cdot \left( c \cdot d^{-1} \right)^{-1} && \text{Notación}\\
 &= \left( a \cdot b^{-1} \right) \cdot \left( c^{-1} \cdot \left( d^{-1} \right) ^{-1} \right) && \text{Multiplicación de inversos multiplicativos}\\
 &= \left( a \cdot b^{-1} \right) \cdot \left( c^{-1} \cdot d \right) && \text{Unicidad del inverso multiplicativo}\\
 &= \left( a \cdot b^{-1} \right) \cdot \left( d \cdot c^{-1} \right) && \text{Conmutatividad}\\
 &= \frac{a}{b} \cdot \frac{d}{c} && \text{Notación}\\
 &= \frac{ad}{bc} && \text{Multiplicación de fracciones} \qedhere
 \end{align*}  
 \end{proof}

 \bfit{Corolario:} $\left(\frac{a}{b}\right)^{-1} = \frac{b}{a}$ si $a,b \neq 0$.
 \begin{proof}
  \begin{align*}
    \left(\frac{a}{b}\right)^{-1} &= \frac{1}{\frac{a}{b}} && \text{Notación}\\
    &= \frac{1^{-1}}{\frac{a}{b}} && \text{Uno es inverso multiplicativo}\\
    &= \frac{\frac{1}{1}}{\frac{a}{b}} && \text{Notación}\\
    &= \frac{1\cdot b}{1\cdot a} && \text{Teorema}\\
    &= \frac{b}{a} && \text{Neutro multiplicativo} \qedhere
  \end{align*}
 \end{proof}

 \item $\frac{a}{c} \pm  \frac{b}{c}=\frac{a\pm b}{c}$, si $c\neq 0$. (Suma de fracciones con denominador conmún).
 \begin{proof}
 \begin{align*}
 \frac{a}{c} \pm  \frac{b}{c} &= \bigl(a\cdot c^{-1}\bigr) \pm  \bigl(b\cdot c^{-1}\bigr) && \text{Notación}\\
 &= \bigl(c^{-1} \cdot a\bigr) \pm  \bigl(c^{-1} \cdot b\bigr) && \text{Conmutatividad}\\
 &= c^{-1} \cdot (a\pm b) && \text{Distribución}\\
 &= (a\pm b) \cdot c^{-1} && \text{Conmutatividad}\\
 &= \frac{a\pm b}{c} && \text{Notación} \qedhere
 \end{align*}
 \end{proof}

 \item $\frac{a}{b} \pm \frac{c}{d} = \frac{ad \pm bc}{bd} $, si $b, d \neq 0$. (Suma de fracciones).
 \begin{proof} 
 \begin{align*} \qquad \ \qquad
 \frac{a}{b} \pm \frac{c}{d} &= \frac{ad}{bd} \pm \frac{cb}{db} && \text{Cancelación de factores en común}\\
 &= \frac{ad}{bd} \pm \frac{cb}{bd} && \text{Conmutatividad}\\
 &= \frac{ad \pm cb}{bd} && \text{Suma de fracciones con denominador común} \qedhere
 \end{align*}
 \end{proof}

 \item $\frac{a}{-b} = -\frac{a}{b}=\frac{-a}{b}$, si $b\neq 0$.
 \begin{proof} \leavevmode
  \begin{center}\vspace{-2em}
   \begin{minipage}[t]{.5\linewidth}
    \begin{align*}
     \frac{-a}{b} &= (-a)\cdot b^{-1} && \text{Notación}\\
     &= -\bigl(ab^{-1}\bigr) && \text{Multiplicación por inverso aditivo}\\
     &= -\frac{a}{b} && \text{Notación}
    \end{align*}
   \end{minipage}%
   \begin{minipage}[t]{.5\linewidth}
    \begin{align*}
     \frac{a}{-b} &= a \cdot (-b)^{-1} && \text{Notación}\\
   &= -\bigl(ab^{-1}\bigr) && \text{Multiplicación por inverso aditivo}\\
   &= -\frac{a}{b} && \text{Notación} \qedhere
    \end{align*}
   \end{minipage}
   \end{center} 
 \end{proof}
 \textbf{Nota:} En esta prueba está implícito que $\exists (-b)^{-1}\in \R$, lo cual es válido, pues $b\neq 0$, por lo que $-b\neq 0$.
 %\begin{proof} \begin{align*}
 % \frac{a}{b} \pm \frac{c}{d} &= ab^{-1} \pm cd^{-1} && \text{Notación}\\
 % &= a\cdot 1 \cdot b^{-1} \pm c\cdot 1 \cdot d^{-1} && \text{Neutro multiplicativo}\\
 % &= ad\cdot d^{-1}b^{-1} \pm cb\cdot b^{-1}d^{-1} && \text{Inverso multiplicativo}\\
 % &= ad \cdot \bigl(db\bigr)^{-1} \pm cd \cdot \bigl(bd\bigr)^{-1} && \text{Multiplicación de inversos multiplicativos}\\
 % &= (bd)^{-1} \bigl(ad\pm cd\bigr) && \text{Distribución}\\
 % &= \frac{ad \pm bc}{bd} && \text{Notación} \qedhere
 %\end{align*}
 %\end{proof}
 %\begin{proof} 
 %\begin{align*}
 %\frac{a}{b} \pm \frac{c}{d}  &=	a \cdot b^{-1} \pm c \cdot d^{-1} && \text{Notación}\\
 %&=	\left( a \cdot 1 \right)   \cdot b^{-1} \pm \left( c \cdot 1 \right) \cdot d^{-1} && \text{Neutro multiplicativo}\\
 %&=	\Bigl( a \cdot  \left( d \cdot d^{-1} \right) \Bigr) \cdot b^{-1} \pm \Bigl( c \cdot  \left( b \cdot b^{-1}  \right)  \Bigr)  \cdot d^{-1} && \text{Inverso multiplicativo}\\
 %&=	\Bigl(  \left( a \cdot d \right) \cdot d^{-1} \Bigr) \cdot b^{-1} \pm \Bigl(  \left( c \cdot b \right) \cdot b^{-1} \Bigr) \cdot d^{-1} && \text{Asociatividad}\\
 %&=	\left( a \cdot d \right)  \cdot  \left( d^{-1} \cdot b^{-1}  \right) \pm \left( c \cdot b \right)  \cdot  \left( b^{-1} \cdot d^{-1}  \right) && \text{Asociatividad}\\
 %&=	\left( a \cdot d \right)  \cdot  \left( b^{-1} \cdot d^{-1}  \right) \pm \left( c \cdot b \right)  \cdot  \left( b^{-1} \cdot d^{-1}  \right) && \text{Conmutatividad}\\
 %&=	\left( b^{-1} \cdot d^{-1}  \right) \cdot \left( a \cdot d \pm c \cdot b \right) && \text{Distribución}\\
 %&=	\left( a \cdot d \pm c \cdot b \right) \cdot  \left( b^{-1} \cdot d^{-1} \right) && \text{Conmutatividad}\\
 %&=	\left( a \cdot d \pm c \cdot b \right) \cdot  \left( b \cdot d \right)^{-1} && \text{Multiplicación de inversos multiplicativos}\\
 %&=	\left( a \cdot d \pm b \cdot c \right) \cdot \left( b \cdot d \right)^{-1} && \text{Conmutatividad}\\
 %&=	\frac{ad \pm bc}{bd} && \text{Notación} \qedhere
 %\end{align*} 
 %\end{proof}
 %\item $\frac{-a}{-b}=\frac{a}{b}$, si $b\neq 0$.
 % \begin{align*}
 %  \frac{-a}{-b} &= (-a)\cdot (-b)^{-1} && \text{Notación}\\
 %  &= (-a)\cdot (-b^{-1}) && \text{Multiplicación por inverso aditivo}\\
 %  &= -(a\cdot b^{-1}) && \text{Distribución del signo}\\
 %  &= - \frac{a}{b} && \text{Notación}
 % \end{align*}
 % %\begin{align*}
 % % \frac{-a}{-b} &= \frac{(-1)\cdot a}{(-1)\cdot b} && \text{Multiplicación por ($-1$)}\\
 % % &=\frac{-1}{-1} \cdot \frac{a}{b} && \text{Teorema}\\
 % % &= (-1)\cdot (-1)^{-1}\cdot \frac{a}{b} && \text{Notación}\\
 % % &= 1\cdot \frac{a}{b} && \text{Inverso multiplicativo}\\
 % % &= \frac{a}{b} && \text{Neutro multiplicativo}
 % % \end{align*}
 % \textbf{Nota:} En esta prueba está implícito que $-b$ tiene inverso multiplicativo, lo cual es válido, pues\begin{align*}
 %  -b^{-1}&=-\bigl(b^{-1}\bigr) && \text{Notación}\\
 %  &= -\qty\big(1\cdot b^{-1}) && \text{Neutro multiplicativo}\\
 %  &= - \qty(\frac{1}{b}) && \text{Notación}
 % \end{align*}
 

 %
 %$a=\frac{a}{1}$. (División por $1$).
 %\begin{proof} 
 % El neutro multiplicativo satisface que $a=a\cdot 1=a\cdot 1^{-1}$, y por notación $a=\frac{a}{1}$.
 %\end{proof}

%\item Si $ab^{-1}=ba^{-1}$, entonces $a^2=b^2$.
%\begin{proof} 
%\begin{align*}
% a^2 &= a\cdot a && \text{Notación}\\
% &= (a\cdot a) \cdot 1 && \text{Neutro multiplicativo}\\
% &= (a\cdot a)\cdot\bigl( b\cdot b^{-1}\bigr) && \text{Inverso multiplicativo}\\
% &= a\cdot \Bigl(a\cdot \bigl( b\cdot b^{-1}\bigr)\Bigr) && \text{Asociatividad}\\
% &= a\cdot \Bigl(a\cdot \bigl(  b^{-1}\cdot b\bigr)\Bigr) && \text{Conmutatividad}\\
% &= a\cdot \Bigl(\bigl(a\cdot b^{-1}\bigr)\cdot b \Bigr) && \text{Asociatividad}\\
% &= a\cdot \Bigl(\bigl(b\cdot a^{-1}\bigr)\cdot b \Bigr) && \text{Hipótesis}\\
% &= a\cdot \Bigl(\bigl(a^{-1}\cdot b\bigr)\cdot b \Bigr) && \text{Conmutatividad}\\
% &= a\cdot \Bigl(a^{-1}\cdot (b\cdot b) \Bigr) && \text{Asociatividad}\\
% &= \bigl(a\cdot a^{-1}\bigr) \cdot (b\cdot b)&& \text{Asociatividad}\\
% &= 1 \cdot (b\cdot b)&& \text{Inverso multiplicativo}\\
% &= (b\cdot b) \cdot 1 && \text{Conmutatividad}\\
% &= b\cdot b && \text{Neutro multiplicativo}\\
% &= b^2 && \text{Notación} \qedhere
%\end{align*}
%\end{proof}
%
%\item Si $a^2=b^2$, entonces $a=b$ o $a=-b$.
%\begin{proof} 
% \begin{align*}
%  0 &= b^2 - b^2 && \text{Inverso aditivo}\\
%  &= a^2 - b^2 && \text{Hipótesis}\\
%  &= a^2 - b^2 + 0&& \text{Neutro aditivo}\\
%  &= a^2 - b^2 + ab-ab&& \text{Inverso aditivo}\\
%  &= a^2 + a b - b^2 -ab && \text{Conmutatividad}\\
%  &= (a^2 + a b) + (- b^2)+(-a b)&& \text{Asociatividad}\\
%  &= (a a + a b) + (- b b)+(-a b)&& \text{Notación}\\
%  &= (a a + a b) -(b b + a b)&& \text{Distribución del signo}\\
%  &= a (a+b) - b(b+a) && \text{Distribución}\\
%  &= (a+b)  (a-b) && \text{Distribución}
%  \end{align*} De la multiplicativo por $0$, sigue que $a+b=0$ o $a-b=0$. Sumando inverso aditivo de $b$ en ambas igualdades tenemos que $a=-b$ o $a=b$.
%\end{proof}
\end{enumerate}

\begin{tcolorbox}
\subsection*{Una nota sobre notación}

Las siguientes son todas las \textit{formas} en que podríamos sumar/multiplicar tres números reales $a$, $b$ y $c$.
\begin{center}
 \begin{minipage}[c]{.2\linewidth}
  \begin{enumerate}[label=\roman*.]
   \item $(a \ \nicefrac{+}{\cdot} \ b) \ \nicefrac{+}{\cdot} \ c$
   \item $a \ \nicefrac{+}{\cdot} \ (b \ \nicefrac{+}{\cdot} \ c)$
   \item $a \ \nicefrac{+}{\cdot} \ (c \ \nicefrac{+}{\cdot} \ b)$
  \end{enumerate}
  \end{minipage}%
  \begin{minipage}[c]{.2\linewidth}
   \begin{enumerate}[start=4,label=\roman*.]
    \item $(a \ \nicefrac{+}{\cdot} \ c) \ \nicefrac{+}{\cdot} \ b$
    \item $(c \ \nicefrac{+}{\cdot} \ a) \ \nicefrac{+}{\cdot} \ b$
    \item $c \ \nicefrac{+}{\cdot} \ (a \ \nicefrac{+}{\cdot} \ b)$
   \end{enumerate}
   \end{minipage}%
  \begin{minipage}[c]{.2\linewidth}
   \begin{enumerate}[start=7,label=\roman*.]
   \item $c \ \nicefrac{+}{\cdot} \ (b \ \nicefrac{+}{\cdot} \ a)$
   \item $(c \ \nicefrac{+}{\cdot} \ b) \ \nicefrac{+}{\cdot} \ a$
   \item $(b \ \nicefrac{+}{\cdot} \ c) \ \nicefrac{+}{\cdot} \ a$
  \end{enumerate}
  \end{minipage}
  \begin{minipage}[c]{.2\linewidth}
   \begin{enumerate}[start=10,label=\roman*.]
   \item $b \ \nicefrac{+}{\cdot} \ (c \ \nicefrac{+}{\cdot} \ a)$
   \item $b \ \nicefrac{+}{\cdot} \ (a \ \nicefrac{+}{\cdot} \ c)$
   \item $(b \ \nicefrac{+}{\cdot} \ a) \ \nicefrac{+}{\cdot} \ c$
  \end{enumerate}
  \end{minipage}
\end{center}

Podemos probar igualdad de todas ellas a partir de las propiedades de la suma/multiplicación:
\begin{align*}
 (a \ \nicefrac{+}{\cdot} \ b) \ \nicefrac{+}{\cdot} \ c &= a \ \nicefrac{+}{\cdot} \ (b \ \nicefrac{+}{\cdot} \ c) && \text{Asociatividad} && \text{Formas (i) y (ii)}\\
 &= a \ \nicefrac{+}{\cdot} \ (c \ \nicefrac{+}{\cdot} \ b) && \text{Conmutatividad} && \text{Forma (iii)}\\
 &= (a \ \nicefrac{+}{\cdot} \ c) \ \nicefrac{+}{\cdot} \ b && \text{Asociatividad} && \text{Forma (iv)}\\
 &= (c \ \nicefrac{+}{\cdot} \ a) \ \nicefrac{+}{\cdot} \ b && \text{Conmutatividad}&& \text{Forma (v)}\\
 &= c \ \nicefrac{+}{\cdot} \ (a \ \nicefrac{+}{\cdot} \ b) && \text{Asociatividad}&& \text{Forma (vi)}\\
 &= c \ \nicefrac{+}{\cdot} \ (b \ \nicefrac{+}{\cdot} \ a) && \text{Conmutatividad}&& \text{Forma (vii)}\\
 &= (c \ \nicefrac{+}{\cdot} \ b) \ \nicefrac{+}{\cdot} \ a && \text{Asociatividad}&& \text{Forma (viii)}\\
 &= (b \ \nicefrac{+}{\cdot} \ c) \ \nicefrac{+}{\cdot} \ a && \text{Conmutatividad}&& \text{Forma (ix)}\\
 &= b \ \nicefrac{+}{\cdot} \ (c \ \nicefrac{+}{\cdot} \ a) && \text{Asociatividad}&& \text{Forma (x)}\\
 &= b \ \nicefrac{+}{\cdot} \ (a \ \nicefrac{+}{\cdot} \ c) && \text{Conmutatividad}&& \text{Forma (xi)}\\
 &= (b \ \nicefrac{+}{\cdot} \ a) \ \nicefrac{+}{\cdot} \ c && \text{Asociatividad}&& \text{Forma (xii)}
\end{align*}

A partir de esta igualdad (y otras probadas anteriormente) introducimos la siguiente \textbf{notación:}

\begin{itemize}%[label=\roman*)]
\item Si $x$, $y$ y $z$ son números reales, representaremos con el símbolo $x+y+z$ a la suma de estos.
%
%Podemos usar esta notación sin ambigüedad ya que hemos probado que todas las formas de sumar tres números reales son equivalentes.
%
%\item Si $x$ y $y$ son números reales, representaremos con el símbolo $xy$ a la multiplicación $x\cdot y$; a esta multiplicación la llamaremos el producto de $x$ y $y$.
%
%Podemos usar esta notación sin ambigüedad pues $x\cdot y = y\cdot x$ por la conmutatividad de la multiplicación.
%%
%Ya que hemos probado que $(-x)\cdot y = -(x \cdot y)=x\cdot (-y)$, por esta notación podemos reescribir $(-x)y=-(xy)=x(-y)$.

\item Si $x$, $y$ y $z$ son números reales, representaremos con el símbolo $xyz$ a la multiplicación de estos.
%cualquiera de los símbolos \begin{itemize}%[label=\roman*)]
% \item $x\cdot y\cdot z$
% \item $x\cdot z\cdot y$
% \item $y\cdot x\cdot z$
% \item $y\cdot z\cdot x$
% \item $z\cdot x\cdot y$
% \item $x\cdot y\cdot z$
%\end{itemize}
%A la multiplicación de $x$, $y$ y $z$.
%
%Podemos usar esta notación sin ambigüedad ya que hemos probado que todas las formas de multiplicar tres números reales son equivalentes.
%
%Las siguientes son todas las \textit{formas} en que podríamos sumar tres números reales $a$, $b$ y $c$.
%\begin{center}
% \begin{minipage}[c]{.2\linewidth}
%  \begin{enumerate}[label=\roman*.]
%   \item $(a+b)+c$
%   \item $a+(b+c)$
%   \item $a+(c+b)$
%  \end{enumerate}
%  \end{minipage}%
%  \begin{minipage}[c]{.2\linewidth}
%   \begin{enumerate}[start=4,label=\roman*.]
%    \item $(a+b)+c$
%    \item $a+(b+c)$
%    \item $a+(c+b)$
%   \end{enumerate}
%   \end{minipage}%
%  \begin{minipage}[c]{.2\linewidth}
%   \begin{enumerate}[start=7,label=\roman*.]
%   \item $c+(b+a)$
%   \item $(c+b)+a$
%   \item $(b+c)+a$
%  \end{enumerate}
%  \end{minipage}
%  \begin{minipage}[c]{.2\linewidth}
%   \begin{enumerate}[start=10,label=\roman*.]
%   \item $b+(c+a)$
%   \item $b+(a+c)$
%   \item $(b+a)+c$
%  \end{enumerate}
%  \end{minipage}
%\end{center}
%Podemos probar igualdad de todas ellas a partir de las propiedades de la suma.
%%las anteriores pueden obtenerse utilizando los axiomas de conmutatividad y asociatividad de la suma, con lo que garantizamos la igualdad de todas ellas, y descartamos la necesidad de enumerarlas todas como axiomas.
%%Notemos que para sumar tres números diferentes $(a,b,c)$, siempre requerimos ejecutar, primero, la suma de dos de ellos $(a+b)$, y luego tomar este resultado para sumarlo al tercer número $(a+b)+c$, a esto alude el lado izquierdo de la igualdad de la asociatividad de la suma (\textbf{S2}). También podemos cambiar el orden, realizando la suma de $a$ y $b$, luego tomar el número $c$ y sumarle a este último el resultado que habíamos obtenido con anterioridad: $c+(a+b)$. Estos resultados (\textbf{i} y \textbf{vii}) satisfacen igualdad debido a la conmutatividad de la suma (\textbf{S1}), $(a+b)+c=c+(a+b)$.
%%Notemos que las formas (\textbf{i} y \textbf{vii}) satisfacen igualdad debido a la conmutatividad de la suma (\textbf{S1}), $(a+b)+c=c+(a+b)$. Asimismo, por asociatividad de la suma (\textbf{S2}), las formas \textbf{(i)} y \textbf{(ii)} satisfacen igualdad, por lo que tenemos $(a+b)+c=a+(b+c)$. De esto el lector puede inferir cuál es el uso de estos axiomas y cómo demostrar la igualdad de todas las formas de sumar tres números reales.
%\begin{align*}
% (a+b)+c &= a+(b+c) && \text{Asociatividad} && \text{Formas (i) y (ii)}\\
% &= a+(c+b) && \text{Conmutatividad} && \text{Forma (iii)}\\
% &= (a+c)+b && \text{Asociatividad} && \text{Forma (iv)}\\
% &= (c+a)+b && \text{Conmutatividad}&& \text{Forma (v)}\\
% &= c+(a+b) && \text{Asociatividad}&& \text{Forma (vi)}\\
% &= c+(b+a) && \text{Conmutatividad}&& \text{Forma (vii)}\\
% &= (c+b)+a && \text{Asociatividad}&& \text{Forma (viii)}\\
% &= (b+c)+a && \text{Conmutatividad}&& \text{Forma (ix)}\\
% &= b+(c+a) && \text{Asociatividad}&& \text{Forma (x)}\\
% &= b+(a+c) && \text{Conmutatividad}&& \text{Forma (xi)}\\
% &= (b+a)+c && \text{Asociatividad}&& \text{Forma (xii)}
%\end{align*}
%Similarmente, las siguientes son todas las \textit{formas} en que podríamos multiplicar tres números reales $a$, $b$ y $c$.
%\begin{center}
% \begin{minipage}[c]{.2\linewidth}
%  \begin{enumerate}[label=\roman*.]
%   \item $(a\cdot b)\cdot c$
%   \item $a\cdot (b\cdot c)$
%   \item $a\cdot (c\cdot b)$
%  \end{enumerate}
%  \end{minipage}%
%  \begin{minipage}[c]{.2\linewidth}
%   \begin{enumerate}[start=4,label=\roman*.]
%    \item $(a\cdot b)\cdot c$
%    \item $a\cdot (b\cdot c)$
%    \item $a\cdot (c\cdot b)$
%   \end{enumerate}
%   \end{minipage}%
%  \begin{minipage}[c]{.2\linewidth}
%   \begin{enumerate}[start=7,label=\roman*.]
%   \item $c\cdot (b\cdot a)$
%   \item $(c\cdot b)\cdot a$
%   \item $(b\cdot c)\cdot a$
%  \end{enumerate}
%  \end{minipage}
%  \begin{minipage}[c]{.2\linewidth}
%   \begin{enumerate}[start=10,label=\roman*.]
%   \item $b\cdot (c\cdot a)$
%   \item $b\cdot (a\cdot c)$
%   \item $(b\cdot a)\cdot c$
%  \end{enumerate}
%  \end{minipage}
%\end{center}
%
%Análogamente, podemos probar la igualdad de todas ellas a partir de las propiedades de la multiplicación.
%\vspace{-1em}\begin{align*}
% (a\cdot b)\cdot c &= a\cdot (b\cdot c) && \text{Asociatividad} && \text{Formas (i) y (ii)}\\
% &= a\cdot (c\cdot b) && \text{Conmutatividad} && \text{Forma (iii)}\\
% &= (a\cdot c)\cdot b && \text{Asociatividad} && \text{Forma (iv)}\\
% &= (c\cdot a)\cdot b && \text{Conmutatividad}&& \text{Forma (v)}\\
% &= c\cdot (a\cdot b) && \text{Asociatividad}&& \text{Forma (vi)}\\
% &= c\cdot (b\cdot a) && \text{Conmutatividad}&& \text{Forma (vii)}\\
% &= (c\cdot b)\cdot a && \text{Asociatividad}&& \text{Forma (viii)}\\
% &= (b\cdot c)\cdot a && \text{Conmutatividad}&& \text{Forma (ix)}\\
% &= b\cdot (c\cdot a) && \text{Asociatividad}&& \text{Forma (x)}\\
% &= b\cdot (a\cdot c) && \text{Conmutatividad}&& \text{Forma (xi)}\\
% &= (b\cdot a)\cdot c && \text{Asociatividad}&& \text{Forma (xii)}
%\end{align*}

%Así como reutilizar las \textit{formas} de proposiciones probadas nos permite agilizar la escritura de demostraciones; establecer notación tiene el mismo propósito. No obstante, cada vez que acordemos el uso de notación esta debe ser justificada para evitar ambigüedad; en ocasiones la justificación es tan simple como utilizar diferentes \textit{etiquetas} para los mismos \textit{objetos}, como ya ha sido empleada, pero en otras, la notación requiere de mayor explicación para su uso adecuado.% y evitar \textit{abuso de la notación}.

%\textbf{Notación:}
\item Si $x$ y $y$ son números reales, representaremos con el símbolo $-xy$ a cualquiera de $(-x)\cdot y$, $-(x \cdot y)$ o $x\cdot (-y)$.

Podemos usar esta notación sin ambigüedad ya que hemos probado que $(-x)\cdot y = -(x \cdot y)=x\cdot (-y)$.
%
%Por la notación (ii) podemos rees%cribir esta igualdad como $(-x)y=-(xy)=x(-y)$, sin embargo, de esta no se obtiene el símbolo $-xy$.
%
\item Si $x\in \R$, representaremos con el símbolo $-x^{-1}$ al inverso multiplicativo de $-x$ o al inverso aditivo de $x^{-1}$.
 
Podemos usar esta notación sin ambigüedad ya que hemos probado que $-(x^{-1})=(-x)^{-1}$.
%
%\item Si $x$ y $y$ son números reales, representaremos con el símbolo $x-y$ a la suma $x+(-y)$.
%
%De esta notación obtenemos que $z-xy$ representa a cualquiera de las sumas $z+(-x)b$, $z+\bigl(-(xy)\bigr)$ y $z+x(-y)$ Podemos usar esta notación sin ambigüedad ya que hemos probado que $(-x)y=-(xy)=x(-y)$.
%Observemos que sería un error reescribir la multiplicación $x\cdot (-y)$ como $a-b$, pues reservamos esta notación para la suma $a+(-y)$, por lo que $x\cdot (-y)$ debería reescribirse como $x(-y)$.
%

\item Al número $1+1$ lo denotaremos con el símbolo $2$. Al número $2+1$ lo denotaremos con el símbolo $3$...
\end{itemize}

%\textbf{Ejercicio:} Reescriba las demostraciones de las listas de ejercicios 1,2 y 3, utilizando esta notación.

\textbf{Nota:} El uso de notación es opcional y en ocasiones prescindimos de ella.%prescindiremos de ella para procurar claridad.% en las demostraciones.

 %\textbf{Demostración:} \begin{align*}
 % \frac{a}{-b} &= \frac{-(-a)}{-b} && \text{Unicidad del inverso aditivo}\\
 % &= \bigl(-(-a)\bigr) \cdot (-b)^{-1} && \text{Notación}\\
 % &=(-1)\cdot (-a) \cdot (-b)^{-1} && \text{Multiplicación por ($-1$)}\\
 % &=(-1) \cdot \frac{-a}{-b} && \text{Notación}\\
 % &= (-1) \cdot \frac{a}{b} && \text{Multiplicación de fracciones}\\
 % &= -\frac{a}{b} && \text{Multiplicación por ($-1$)} && \text{(*)}\\
 % &= (-1) \cdot \frac{a}{b} && \text{Multiplicación por ($-1$)}\\
 % &= \frac{(-1)\cdot a}{b} && \text{(a) de LE4}\\
 % &= \frac{-a}{b} && \text{Multiplicación por ($-1$)} && \text{(**)}
 %\end{align*} De las igualdades (*) y (**) tenemos que $\frac{a}{-b} = -\frac{a}{b}=\frac{-a}{b}$.


\end{tcolorbox}
\section*{Un campo finito}

Si $a,b\in \R$ son tales que $a-b=b-a$, entonces $a=b$. El siguiente es un esbozo de la prueba:
\begin{align*}
 2a %&= (1+1)a && \text{Notación}\\
 %&= a(1+1) && \text{Conmutatividad}\\
 %&= a\cdot 1 + a\cdot 1 && \text{Distribución}\\
 %&= a + a && \text{Neutro multiplicativo}\\
 &= a + a && \text{Notación}\\
 %&= a +a + 0&& \text{Neutro aditivo}\\
 &= a+a+b-b && \text{Inverso aditivo}\\
 &= a-b+a+b && \text{Conmutatividad}\\
 &= b-a +a +b && \text{Hipótesis}\\
 %&= b+0 + b && \text{Inverso aditivo}\\
 %&= b+b && \text{Neutro aditivo}\\
 &= b+b && \text{Inverso aditivo}\\
 &= 2b && \text{Notación}
\end{align*}
%\begin{align*}
%a+(-b)&=b+(-a) && \text{Hipótesis} \\
%\bigl(a+(-b)\bigr)+b&=\bigl(b+(-a)\bigr)+b&&\text{Ley de cancelación} \\
%a + \bigl((-b)+b\bigr)&= b+\bigl((-a)+b\bigr)&&\text{Asociatividad} \\
%a + \bigl(b+(-b)\bigr)&= b+\bigl(b+(-a)\bigr)&&\text{Conmutatividad} \\
%a + 0&= b+\bigl(b+(-a)\bigr)&&\text{Inverso aditivo} \\
%a + 0&= (b+b)+(-a)&&\text{Asociatividad} \\
%(a + 0) + a&= \bigl((b+b)+(-a)\bigr)+a&&\text{Ley de cancelación} \\
%(a + 0) + a&= (b+b)+\bigl((-a)+a\bigr)&&\text{Asociatividad} \\
%(a + 0) + a&= (b+b)+\bigl(a+(-a)\bigr)&&\text{Conmutatividad} \\
%(a + 0) + a&= (b+b)+0&&\text{Inverso aditivo} \\
%a+ a&= b+b&&\text{Neutro aditivo} \\
%a\cdot 1+ a\cdot 1&= b\cdot 1+ b\cdot 1&&\text{Neutro multiplicativo} \\
%a\cdot (1+1)&= b\cdot (1+1)&&\text{Distribución} \\
%a\cdot (2) &= b\cdot (2) &&\text{Notación} \\
%a &= b && \text{¿Ley de cancelación?}
%\end{align*}
A pesar de que se verifica la igualdad $2a=2b$, aún necesitamos justificar que $a=b$. Podríamos apelar a la ley de cancelación de la multiplicación, pero para su uso requerimos que $2\neq 0$, el cual es un hecho que hasta ahora no ha sido demostrado. No obstante, los axiomas que hemos listado y los resultados que hemos obtenido de ellos no son suficientes para probar este hecho, el lector debería indagar en las implicaciones de definir que $2=0$ y decidir si este hecho es contradictorio. Para clarificar este punto, consideremos el siguiente conjunto: %, pues si suponemos que $2=0$, tendríamos consistencia con todos los axiomas listados hasta ahora, es decir, no habría motivo para suponer que $2\neq 0$.
%Supongamos que $2=0$.
%
%\begin{enumerate}[label=\roman*)]
% \item Cerradura de la suma: Por axioma de neutro multiplicativo, $1\in \R$, y por la cerradura de la suma, $1+1\in \R$, por notación $2\in \R$. También, por axioma de neutro aditivo $0\in \R$. Vemos que la hipótesis $2=0$ es consistente.
% \item Conmutatividad de la suma. Sabemos que $a+b=b+a$, en el caso $a=b$ tenemos $a+a=b+b$. Tomando $a=1$ y $b=0$, tenemos $1+1=0+0$, osea $2=0$, lo cual es consistente con la hipótesis.
% \item Asociatividad de la suma. Sabemos que $(a+b)+c=a+(b+c)$, tomando $a=b=c=1$ tenemos $(1+1)+1=1+(1+1)$ y por notación
%\end{enumerate}

%Sea $\Omega\defined \set{0,1}$ un conjunto dotado con las operaciones de suma y multiplicación.
Sea $\Omega$ un conjunto dotado con las operaciones suma $+$ y multipllicación $\cdot$ que satisfacen las siguientes propiedades:
\begin{enumerate}
\item Cerradura (de la suma): Si $x,y\in \Omega$, entonces $x+y\in \Omega$.
\item Conmutatividad (de la suma): Si $x,y\in \Omega$, entonces $x+y=y+x$.
\item Asociatividad (de la suma): Si $x,y,z \in \Omega$, entonces $x+(y+z)=(x+y)+z$.
\item Neutro aditivo: $\exists 0 \in \Omega$ tal que si $x\in \Omega$, entonces $x+0=x$.
\item Inverso aditivo: para cada $x\in \Omega$, $\exists (-x)\in \Omega$ tal que $x+(-x)=0$.
\item Cerradura (de la multiplicación): Si $x,y\in \Omega$, entonces $x\cdot y \in \Omega$.
\item Conmutatividad (de la multiplicación): Si $x,y\in \Omega$, entonces $x\cdot y = y\cdot x$.
\item Asociatividad (de la multiplicación): Si $x,y,z \in \Omega$, entonces $x\cdot (y\cdot z)=(x\cdot y)\cdot z$.
\item Neutro multiplicativo: $\exists 1\in \Omega$ tal que si $x\in \Omega$, entonces $x\cdot 1=x$.
\item Inverso multiplicativo: si $x\in \Omega$ tal que $x\neq 0$, entonces $\exists x^{-1}$ tal que $x\cdot x^{-1}=1$.
\item Distribución (de la multiplicación sobre la suma): Si $x,y,z\in \R$, entonces $x\cdot(y+z)=x\cdot y+x\cdot z$.
\end{enumerate}

¿Qué elementos pertenecen a $\Omega$?

Sabemos que $0$ y $1$ son elementos de $\Omega$, en virtud de los axiomas (4) y (9). Asimismo, el axioma (5) garantiza la existencia de $-1$ y $-0$. De la misma manera, por el axioma (10) podemos afirmar que $1^{-1}$ es un miembro de $\Omega$. Sin embargo, los axiomas de conmutatividad (2) y (7), de asociatividad (3) y (8), y el axioma de distribución (11), no son axiomas de existencia y para su uso requerimos elementos de $\Omega$, es decir, no podemos \textit{conocer} elementos adicionales de $\Omega$ apartir de estos.

Con estas consideraciones, sabemos que $\set{0,1,-0,-1,1^{-1}}\subset \Omega$. Sin embargo, hemos probado que $0=-0$ y $1=1^{-1}$, por lo que hasta ahora, solo podemos afirmar que $0,1,-1$ son miembros de $\Omega$.

Por otra parte, por el axioma de cerradura de la multiplicación (6), se verifica lo siguiente:

\begin{enumerate}[label=\roman*)]
 \item $0\cdot 0\in \Omega$, pero como $0\cdot 0=0$, no encontramos un miembro distinto a los conocidos.
 \item $0 \cdot 1 \in \Omega$, pero como $0\cdot 1=0$, no encontramos un miembro distinto a los conocidos. 
 \item $0 \cdot (-1) \in \Omega$, pero como $0\cdot (-1)=0$, no encontramos un miembro distinto a los conocidos.
 \item $1\cdot (-1) \in \Omega$, pero como $1\cdot (-1)=-1$, no encontramos un miembro distinto a los conocidos.
 \item $1\cdot 1\in \Omega$, pero como $1\cdot 1=1$, no encontramos un miembro distinto a los conocidos.
\end{enumerate}
Finalmente, por el axioma de cerradura (1) se verifica lo siguiente:
\begin{enumerate}[label=\roman*)]
 \item $0+0 \in \Omega$, pero como $0+0=0$, no encontramos un miembro distinto a los conocidos.
 \item $0+1 \in \Omega$, pero como $0+1=1$, no encontramos un miembro distinto a los conocidos.
 \item $0+(-1) \in \Omega$, pero como $0+(-1)=-1$, no encontramos un miembro distinto a los conocidos.
 \item $1+(-1)\in \Omega$, pero como $1+(-1)=0$, no encontramos un miembro distinto a los conocidos.
 \item$1+1\in \Omega$, el cual es un elemento del que no podemos afirmar sea distinto a los conocidos.
\end{enumerate}
Si definimos que bajo $\Omega$, $2=0$, es decir, que $1+1=0$, entonces $1+1$ no sería un miembro distinto a los conocidos. Además, por unicidad del inverso aditivo, si $1+1=0$, sigue que $1=-1$. De este modo, $\Omega$ cumpliría con todos los axiomas de campo consistentemente y su extensión sería $\Omega \defined \set{0,1}$.

Por lo anterior, para expandir el conjunto de los números reales, requerimos establecer propiedades adicionales.
%
%\textbf{Nota:} Hasta este punto al probar proposiciones el autor ha procurado enunciar cada propiedad que se utiliza, sin embargo, se espera que el lector sea capáz de inferir el uso de estas y de la notación en lo que resta del curso. Por ello dejaremos de enunciar cada propiedad o proposición que sea empleada excepto en aquellos casos en los que el autor considere que se requiera para preservar claridad.% Sin embargo, el dejar de enunciar propiedades empleadas lo realizaremos paulatinamente, de modo que el lector pueda acostumbrarse a esta forma de escritura.



\part*{Axiomas de orden}

%\subsection*{Números positivos}
%
%\begin{enumerate}[label=O\arabic*)]
% \item Se cumple una y sólo una de las siguientes condiciones (Tricotomía):
%  \begin{enumerate}[label=\roman*)]
%  \item $0<x$.% En este caso decimos que $x$ es un número real positivo.
%  \item $x=0$.
%  \item $x<0$.% En este caso decimos que $x$ es un número real negativo.
%  \end{enumerate}
% \item Si $0<x$ y $0<y$, entonces $0<x+y$.
% \item Si $0<x$ y $0<y$, entonces $0<x\cdot y$.
%\end{enumerate}
Existe un subconjunto del conjunto de los números reales llamado conjunto de los números reales positivos, denotado con el símbolo $\R^+$, el cual satisface las siguientes propiedades:%. Describimos las propiedades de orden de $\R$ en términos de $\R^+$.% El conjunto $\R^+$ satisface los siguientes \textbf{axiomas (de orden)}:
\vspace{-1em} \begin{enumerate}[start=12]%[label=O\arabic*)]
\item Si $x, y \in \R^+$, entonces $x + y \in \R^+$. (Cerradura de la suma en $\R^+$).
\item Si $x, y \in \R^+$, entonces $x \cdot y \in \R^+$. (Cerradura de la multiplicación en $\R^+$).
\item Para cada $x\in \R$, se verifica una y solo una de las siguientes condiciones (Tricotomía):
\begin{enumerate}[label=\roman*)]
\item $x \in \R^+$.
\item $x = 0$.
\item $-x \in \R^+$.
\end{enumerate}
\end{enumerate}

\subsection*{Lista de Ejercicios}
\begin{enumerate}[label=\alph*)]
 \item Demuestre que $0\notin \R^+$.
 \begin{proof} 
 Si $0\in \R^+$ se contradice el axioma de tricotomía.
 \end{proof}

 \item Demuestre que $1\in \R^+$. (Uno es positivo).
 \begin{proof} Consideremos los casos: \begin{enumerate}[label=\roman*)]
 \item $1=0$, pero este caso se descarta por axioma del neutro multiplicativo.
 \item Si $-1\in \R^+$, por cerradura de la multiplicación en $\R^+$, $(-1) \cdot (-1)=1\in \R^+$, pero esto contradice la propiedad de tricotomía.
 \end{enumerate} Por tanto, $1\in \R^+$.
 \end{proof}
\end{enumerate}

\textbf{Observación:} para cualesquiera números reales $a$ y $b$ tenemos que $a=b$ o $a\neq b$. A su vez, \begin{enumerate}[label=\roman*)]
\item Si $a=b$, entonces $a-b=0$.
\item Si $a\neq b$, por tricotomía tenemos dos casos (excluyentes):\begin{itemize}
 \item $a-b\in \R^+$.%, lo que denotamos como $b<a$ (o $a>b$).
 \item $-(a-b)=b-a\in \R^+$.%, lo que denotamos como $a<b$ (o $b>a$).
\end{itemize}%  o .%$-(x-y)=y-x\in \R^+$.% (o equivalentemente $x-y\in \R^-$).
\end{enumerate}

A partir de esta observación introducimos la siguiente \bfit{definición:}

Sean $x,y\in \R$, \begin{itemize}
 \item Si $x-y\in \R^+$, escribimos $y<x$ (o $x>y$).
\end{itemize}

De esta definición sigue que dados $x,y\in \R$, por tricotomía se verifica una y solo una de las siguientes condiciones\vspace{-1em} \begin{enumerate}[label=\roman*)]
 \item $y<x$.
 \item $x=y$.
 \item $x<y$.
\end{enumerate}

\textbf{Notación:} Dados $x,y,z\in \R$, utilizaremos la notación \begin{itemize}
 \item $y \leq x$ (o $x\geq y$) para indicar que $y<x$ o $x=y$.
 \item $x<y<z$ para indicar que $x<y$ y $y<z$.
\end{itemize}
%
%Hasta este punto sabemos que $x-y$ es un número real, pero aún no contamos con las herramientas para verificar la pertenencia de $x-y$ respecto de $\R^+$. Sin embargo, cuando dicha pertenencia ocurra, usaremos la siguiente \textbf{definición}:
%%Si $x,y\in \R$, por la cerradura de la suma en $\R$, se verifica que $x-y\in \R$, pero aún 
%\begin{itemize}
%\item Si $x-y\in \R^+$, lo denotamos como $y<x$ (o $x>y$).% o $x>y$. %diremos que $x$ es mayor que $y$ y
%\item Si $x-y\in \R^-$, lo denotamos como $x<y$ (o $y>x$).% o $y>x$. %diremos que $x$ es menor que $y$ y
%\end{itemize}

%\bfit{Definición:}  El conjunto de los números reales no negativos es el conjunto: \[
%  \R^+\cup \set{0}
%  \]

\section*{Números reales negativos}

\bfit{Definición:} Llamaremos al conjunto $\R^-\defined \R\setminus\bigl(\R^+\union \set{0}\bigr)$ conjunto de los números reales negativos.
%\begin{itemize}
% \item Llamaremos al conjunto $\R^-\defined \R\setminus\bigl(\R^+\union \set{0}\bigr)$ conjunto de los números reales negativos.
% \item Llamaremos al conjunto $\R^+\union \set{0}$ conjunto de los números reales no negativos.
%\end{itemize}

\subsection*{Lista de ejercicios 6 (LE6)}

\begin{enumerate}[label=\alph*)]
 \item Demuestre que $\R^+$ y $\R^-$ son conjuntos disjuntos.
 \begin{proof}%$ $\newline 
 Si $\R^+ \intersection \Bigl(\R\setminus\bigl(\R^+\union \set{0}\bigr)\Bigr) \neq \emptyset$, entonces $\exists x\in \R$ tal que \begin{itemize}%[label=\roman*)]
 \item[(*)] $x\in \R^+$, y 
 \item[(**)] $x\in \R\setminus\bigl(\R^+\union \set{0}\bigr)$.
 \end{itemize}  De (**) sigue que $x\notin \R^+\union \set{0}$, es decir, $x\notin \set{0}$ y $x\notin \R^+$, pero esto contradice a (*).

 Por tanto, $\R^+ \intersection \Bigl(\R\setminus\bigl(\R^+\union \set{0}\bigr)\Bigr) = \emptyset$, es decir, $\R^+$ y $\R^-$ son conjuntos disjuntos.
 \end{proof}

 \item Demuestre que $x\in \R^+$ si y solo si $-x\in \R^-$.
 \begin{proof}\leavevmode
 \begin{itemize}
 \item[$\Rightarrow)$] Sea $x\in \R^+$. Por tricotomía, $-x\notin \R^+$ y $x\neq 0$ (y por esto $-x\neq 0$). Sigue que $-x\in \R\setminus \bigl(\R^+\union \set{0}\bigr)$, y por definción, $-x\in \R^-$.
 \item[$\Leftarrow)$] Sea $-x\in \R^-$. Por definición, $-x\in \R\setminus\bigl(\R^+\union \set{0}\bigr)$, por lo que, $-x\notin \R^+\union \set{0}$, es decir, $-x\notin \R^+$ y $-x\notin \set{0}$ (y por tanto $-x\neq 0$). Entonces, por tricotomía, $-(-x)=x\in \R^+$. \qedhere
 \end{itemize}
 \end{proof}

 \item Demuestre que si $x,y \in \R^-$, entonces $x+y\in \R^-$. (Cerradura de la suma en $\R^-$).
 \begin{proof}
 Sea $x,y\in \R^-$. Sigue que $(-x), (-y)\in \R^+$, y por la cerradura de la suma en $\R^+$, $-x-y=-(x+y)\in \R^+$, por lo que $x+y\in \R^-$.
 \end{proof}

 \item Demuestre que $x\in \R^+$ si y solo si $0<x$. (Caracterización de $\R^+$).
 \begin{proof}\leavevmode
 \begin{itemize}
 \item[$\Rightarrow)$] Sea $x\in \R^+$, notemos que $x=x-0\in \R^+$, lo que denotamos como $0<x$.
 \item[$\Leftarrow)$] Sea $0<x$, por definición $x-0=x\in \R^+$. \qedhere
 \end{itemize}
 \end{proof}

 \item Demuestre que $x\in \R^-$ si y solo si $x<0$. (Caracterización de $\R^-$).
 \begin{proof}\leavevmode
 \begin{itemize}
 \item[$\Rightarrow)$] Sea $x\in \R^-$. Notemos que $-x=0-x\in \R^+$, por lo que $x<0$.
 \item[$\Leftarrow)$] Sea $x<0$. Por definición, $0-x=-x\in \R^+$, por lo que $x\in \R^-$. \qedhere
 \end{itemize}
 \end{proof}
\end{enumerate}

\subsection*{Lista de Ejercicios 5 (LE5)}
Sean $a$, $b$, $c$ y $d$ números reales, demuestre lo siguiente:
\begin{enumerate}[label=\alph*)]
%\item $a \in \R^+$ si y solo si $a>0$. (Definición de número real positivo).
%\begin{proof} \leavevmode
% \begin{itemize}
%  \item[$\Rightarrow)$] El cero satisface que $a=a+0=a-0$. Si $a \in \R^+$, entonces $a-0 \in \R^+$, es decir, $0<a$.
%  \item[$\Leftarrow)$] Si $0<a$, entonces $a-0 \in \R^+$, y sabemos que el cero satisface que $a-0=a+0=a$, por lo que $a \in \R^+$. \qedhere
% \end{itemize}
%\end{proof}
%\bfit{Definición:}  Si $x$ es un número real tal que $0<x$ diremos que $x$ es un númeo real positivo.
 %La definición de número real positivo es equivalente a número real mayor a cero.
 %Todo número real positivo es mayor a cero, y todo número real mayor a cero es un número real positivo.
 %
 %\textbf{Nota:} La definición de un número real positivo no está asociada al signo que acompaña al número.%, es decir, la proposición es válida para $-n\in \R^+ \iff -m>0$. El lector debería verificar este hecho.
 %Esta demostración, cuya forma es $m\in \R^+ \iff m>0, \forall m\in \R$, nos permite reparar en el hecho de que la definición de un número real positivo no está asociada al signo que acompaña al número, es decir, la proposición es válida para $-a>0$, es decir, $-a\in \R^+ \iff -a>0, \forall -a\in \R$. El lector debería verificar este hecho.
% \item $0<a$ si y solo si $-a<0$. (Definición de número real negativo).
%  \begin{itemize}
%   \item[$\Rightarrow)$] Si $0<a$, por definición $a-0\in \R^+$, sigue que $a-0=a+0=a\in \R^+$.
%   \item[$\Leftarrow)$] Si $-a<0$, por definición $0-(-a)\in \R^+$ y  por unicidad del inverso aditivo $0-(-a)=a\in \R^+$.
%  \end{itemize}
%  \bfit{Definición:}  Si $x$ es un número real tal que $x<0$, diremos que $x$ es un número real negativo.
%  %\begin{center}\vspace{-1em}
  %\begin{minipage}[l]{.5\linewidth}
  % \begin{align*}
  %  \Rightarrow) \qquad \qquad
  %  0 &< a && \text{Hipótesis}\\
  %  0 -a &< a- a && \text{Teorema}\\
  %  -a &< 0 && \text{Inverso aditivo}
  % \end{align*}
  %\end{minipage}%
  %\begin{minipage}[r]{.5\linewidth}
  % \begin{align*}
  %  \Leftarrow) \qquad \qquad
  %  -a &< 0 && \text{Hipótesis}\\
  %  -a + a &< 0 +a && \text{Teorema}\\
  %  0 &< a && \text{Inverso aditivo}
  % \end{align*}
  %\end{minipage}
  %\end{center}
  %Otra forma de demostrar este hecho es la siguiente:
  %\begin{proof}\leavevmode
  % \begin{enumerate}[label=\roman*)]
  %  \item Supongamos que $0<a$. Notemos que: \begin{align*}
  %   0 + (-a) &< a + (-a) && \text{Ley de cancelación}\\
  %   0 + (-a) &< 0 && \text{Inverso aditivo}\\
  %   -a &< 0 && \text{Neutro aditivo}
  %  \end{align*}
  %  \item Supongamos que $-a<0$. Notemos que:
  %  \begin{align*}
  %   -a + a &< 0 + a && \text{Ley de cancelación}\\
  %   0 &< 0 + a && \text{Inverso aditivo}\\
  %   0 &< a && \text{Neutro aditivo} \qedhere
  %  \end{align*}
  % \end{enumerate} 
  %\end{proof}
  %\textbf{Observación:} El inverso aditivo de cualquier número real positivo es menor a cero.
  % El conjunto de los números reales negativos se representa con el símbolo $\R^-$.
  %\bfit{Definición:}  El conjunto de los números reales negativos se representa con el símbolo $\R^-$ y se define como los elementos en los números reales cuyos inversos aditivos pertenecen al conjunto de los números reales positivos:
 %\[
 % \R^-\coloneqq \set{m\in \R: -a\in \R^+}
 % \]
%
 %\item $0<a$ si y solo si $-a<0$.
 %\vspace{-1em}
 %\begin{proof}
 % \begin{itemize}
 %  \item[$\Rightarrow)$] Sea $0<a$. Por tricotomía tenemos tres casos \begin{enumerate}[label=\roman*)]
 %   \item Si $0<-a$, por la cerradura de la suma en $\R^+$, $0<-a+a=0$, lo que es una contradicción.
 %   \item El inverso aditivo satisface $a-a=0$. Si $-a=0$, sigue que $a-a=a+0=a$, es decir, $a=0$, lo que es una contradicción.
 %  \end{enumerate} Por tanto si $0<a$, entonces $-a<0$.
 %  \item[$\Leftarrow)$] Sea $-a<0$. Por tricotomía tenemos tres casos \begin{enumerate}[label=\roman*)]
 %   \item El inverso aditivo satisface que $a-a=0$. Si $a<0$, por la cerradura de la suma en $\R^+$,
 %  \end{enumerate}
 % \end{itemize}
 %\end{proof}
% \item Si $a<b$, entonces $\exists! x\in \R^+$ tal que $a+x=b$.
% \begin{proof}\vspace{-1em} \leavevmode
% \begin{enumerate}[label=\roman*)]
% \item Primero probaremos su existencia. Sea $a<b$. Por definición, $b-a\in \R^+$. Tomando $x=b-a$, tenemos que $a+x=a+b-a=b$.
% \item Ahora probaremos la unicidad. Sean $x,y\in \R^+$ tales que $a+x=b$ y $a+y=b$, notemos que $x=b-a$ y $y=b-a$, es decir, $x=y$. \qedhere
% \end{enumerate}
% \end{proof}
 \item Si $a<b$ y $b<c$, entonces $a<c$. (Transitividad).
 \begin{proof} 
  Por definición $(b-a) , (c-b) \in \R^+$. Por cerradura de la suma en $\R^+$, $(b-a) + (c-b) \in \R^+$. De donde sigue que $(b-a)+(c-b)=b-a+c-b=c-a\in \R^+$, es decir, $a<c$.
 \end{proof}
  %\begin{align*}
  %(b-a) + (c-b) &= b - a + c -b && \text{Notación}\\
  %&= b-a -b+c && \text{Conmutatividad}\\
  %&= b-b -a+c && \text{Conmutatividad}\\
  %&= 0 - a +c && \text{Inverso aditivo}\\
  %&= -a +c && \text{Neutro aditivo}\\
  %&= c-a && \text{Conmutatividad}
  %\end{align*} Entonces $c-a \in \R^+$, es decir, $a<c$.

 \item $a<b$ si y solo si $a+c<b+c$. (Ley de cancelación de la suma en desigualdades).
 
 \begin{proof} \leavevmode
 \begin{itemize}
  \item[$\Rightarrow)$] Si $a<b$, por definición, $b-a \in \R^+$. Luego, $b-a=b-a+c-c=b+c-a-c=b+c-(a+c)$. De este modo, $b+c-(a+c)\in \R^+$, es decir, $a+c<b+c$.
  \item[$\Leftarrow)$] Si $a+c<b+c$, por definición $b+a-(a+c)\in \R^+$. Luego, $b+c-(a+c)=b+c-a-c=b-a$. De este modo, $b-a\in \R^+$, es decir, $a<b$. \qedhere
  \end{itemize} 
 \end{proof}
 \textbf{Nota:} Si el contexto es claro, enunciaremos esta proposición como ley de cancelación.

 \bfit{Corolario:}
 \begin{enumerate}[label=\roman*)]
  \item $0<a$ si y solo si $-a<0$.
  \begin{center}\vspace{-1em}
   \begin{minipage}[l]{.5\linewidth}
    \begin{align*}
     \Rightarrow) \qquad \qquad
     0 &< a && \text{Hipótesis}\\
     0 + (-a) &< a + (-a) && \text{Ley de cancelación}\\
     %0 + (-a) &< 0 && \text{Inverso aditivo}\\
     -a &< 0% && \text{Neutro aditivo}
    \end{align*}
  \end{minipage}%
  \begin{minipage}[r]{.5\linewidth}
   \begin{align*}
    \Leftarrow) \qquad \qquad
    -a &< 0 && \text{Hipótesis}\\
    -a + a &< 0 + a && \text{Ley de cancelación}\\
    %0 &< 0 + a && \text{Inverso aditivo}\\
    0 &< a%&& \text{Neutro aditivo} \qedhere
   \end{align*}
  \end{minipage}
  \end{center}

  \item $a<a+b$ si y solo si $0<b$.
  \begin{center}\vspace{-1em}
  \begin{minipage}[l]{.5\linewidth}
   \begin{align*}
    \Rightarrow) \qquad \qquad
    a &< a + b && \text{Hipótesis}\\
    a -a &< a + b - a && \text{Ley de cancelación}\\
    0 &< b% && \text{Inverso aditivo}
   \end{align*}
  \end{minipage}%
  \begin{minipage}[r]{.5\linewidth}
   \begin{align*}
    \Leftarrow) \qquad \qquad
    0 &< b && \text{Hipótesis}\\
    0 + a &< a + b && \text{Ley de cancelación}\\
    a &< a + b% && \text{Neutro aditivo}
   \end{align*}
  \end{minipage}
  \end{center}
  \item $a+b<a$ si y solo si $b<0$.
  \begin{center}\vspace{-1em}
  \begin{minipage}[l]{.5\linewidth}
   \begin{align*} \Rightarrow) \qquad \qquad
    a+b &< a && \text{Hipótesis}\\
    a+b-a &< a-a && \text{Ley de cancelación}\\
    b &< 0% && \text{Inverso aditivo}
   \end{align*}
  \end{minipage}%
  \begin{minipage}[r]{.5\linewidth}
   \begin{align*} \Leftarrow) \qquad \qquad
    b &< 0 && \text{Hipótesis}\\
    b + a &< 0 + a && \text{Ley de cancelación}\\
    b+ a &< a% && \text{Neutro aditivo}
   \end{align*}
  \end{minipage}
  \end{center}
  \item $-a<b$ si y solo si $-b<a$.
  \begin{center}\vspace{-1em}
  \begin{minipage}[l]{.5\linewidth}
   \begin{align*}
    \Rightarrow) \qquad \qquad -a &< b && \text{Hipótesis}\\
    -a+a-b &< b +a-b && \text{Ley de cancelación}\\
    %a-a-b &< b-b+a && \text{Conmutatividad}\\
    %0-b &< 0 + a && \text{Inverso aditivo}\\
    -b &< a% && \text{Inverso aditivo}
   \end{align*}
  \end{minipage}%
  \begin{minipage}[r]{.5\linewidth}
   \begin{align*}
    \Leftarrow) \qquad \qquad   -b &< a && \text{Hipótesis}\\
    -b + b-a &< a + b-a && \text{Ley de cancelación}\\
    %b-b-a &< b+a-a && \text{Conmutatividad}\\
    %0 -a &< b + 0 && \text{Inverso aditivo}\\
    -a &< b% && \text{Inverso aditivo}
   \end{align*}
  \end{minipage}
  \end{center}
  \item $a<-b$ si y solo si $b<-a$.
  \begin{center}\vspace{-1em}
  \begin{minipage}[l]{.5\linewidth}
   \begin{align*}
    \Rightarrow) \qquad \qquad
    a &< -b && \text{Hipótesis}\\
    a - a + b &< -b -a +b && \text{Ley de cancelación}\\
    b &< -a% && \text{Inverso aditivo}
   \end{align*}
  \end{minipage}%
  \begin{minipage}[r]{.5\linewidth}
   \begin{align*}
    \Leftarrow) \qquad \qquad
    b &< -a && \text{Hipótesis}\\
    b -b +a &< -a + -b +a && \text{Ley de cancelación}\\
    a &< -b% && \text{Inverso aditivo}
   \end{align*}
  \end{minipage}
  \end{center}
  \item $a<b$ si y solo si $-b<-a$.
  \begin{center}\vspace{-1em}
  \begin{minipage}[l]{.5\linewidth}
   \begin{align*}
    \Rightarrow) \qquad \qquad
    a &< b && \text{Hipótesis}\\
    a -a -b &< b -b -a && \text{Ley de cancelación}\\
    -b &< -a% && \text{Inverso aditivo}
   \end{align*}
  \end{minipage}%
  \begin{minipage}[r]{.5\linewidth}
   \begin{align*}
    \Leftarrow) \qquad \qquad
    -b &< -a && \text{Hipótesis}\\
    -b + b +a &< -a +b -a && \text{Ley de cancelación}\\
    a &< b &&% \text{Inverso aditivo}
   \end{align*}
  \end{minipage}
  \end{center}
  
  \item Si $a<b<c$, entonces $-c<-b<-a$.
  
  Notemos que $a < b \Rightarrow -b < -a$ y $b < c\Rightarrow -c < -b$. Es decir, $-c<-b<-a$.

 \end{enumerate} % Corolario

 \item Si $a<b$ y $c < d$, entonces $a+c<b+d$. (Suma \textit{vertical} de desigualdades).
 \begin{proof} 
 Por definición $b-a\in \R^+$ y $d-c\in \R^+$. Por cerradura de la suma en $\R^+$ se verifica que $(b-a)+(d-c)\in \R^+$. Luego, $(b-a)+(d-c)=b+d-a-c=b+d-(a+c)$. Por lo que $b+d-(a+c)=\in \R^+$, es decir, $a+c<b+d$.
 \end{proof}
 \textbf{Observación:} La suma \textit{vertical} de desigualdades preserva el orden.
 %\textbf{Nota:} No hay doble implicación, es decir, si $a+c<b+d$, no es posible demostrar —a partir de este hecho únicamente, que $a<b$ y $c<d$. Ej. $a=4, c=0, b=3, d=4$, se cumple que $a+c=4+0=4<7=3+4=b+d$, pero $c=0<4=d$ y $a=4<3=b$ implica una contradicción.
 %Esta proposición difiere de la ley de la cancelación (de la suma en desigualdades) ya que no se satisface una doble implicación, es decir, si $a+c<b+d$, no es posible demostrar —a partir de esta hipótesis únicamente, que $a<b$. Ej. $a=4, c=0, b=3, d=4$, cumplen la hipótesis $a+c=4+0=4<7=3+4=b+d$, pero $a=4<3=b$ es una contradicción.

 \bfit{Corolario:}\begin{enumerate}[label=\roman*)]
  \item Si $0<a$ y $0<b$, entonces $0<a+b$.
  
  Por este teorema, $0+0=0<a+b$.
%
  \item Si $a<0$ y $b<0$, entonces $a+b<0$.% (Cerradura de la suma en $\R^-$).
  
  Por este teorema, $a+b<0=0+0$.
 \end{enumerate}

 \item Sea $a<b$. Encuentre las condiciones que deben cumplirse para que $a<b<a+b$, $a<a+b<b$, o $a+b<a<b$.
 \begin{enumerate}[label=\roman*)]
  \item Si $0<a$, por ley de cancelación, $0+b<a+b$, por lo que $b<a+b$, y así $a<b<a+b$.
  \item Si $a<0$ y $0<b$, por ley de cancelación,
  \begin{center}\vspace{-1em}
  \begin{minipage}[l]{.3\linewidth}
   \begin{align*}
      a &< 0\\
      a + b &< 0 +b\\
      a+b &< b
   \end{align*}
  \end{minipage}%
  \begin{minipage}[r]{.3\linewidth}
   \begin{align*}
      0 &< b\\
      0 + a &< b +a\\
      a &< b +a
   \end{align*}
  \end{minipage}
  \end{center}
  Por tanto, $a<a+b<b$.
  \item Si $b<0$. por ley de cancelación, $b+a<0+a$, es decir, $a+b<a$, por lo que $a+b<a<b$.
 \end{enumerate}

 \bfit{Conclusión:} Si $a<b$ y
 \begin{itemize}
  \item $0<a$, entonces $a<b<a+b$.
  \item $a<0$, entonces $a<a+b<b$.
  \item $b<0$, entonces $a+b<a<b$.
 \end{itemize}

 \item Sea $a<b$ y $c\in \R$. Encuentre las condiciones que deben cumplirse para que $ac<bc$ o $bc<ac$.
 \begin{enumerate}[label=\roman*)]
  \item Sea $c \in\R^+$. Por definición $b-a \in\R^+$. Por cerradura de la multiplicación en $\R^+$ se verifica que $c(b-a) \in\R^+$. Sigue que $c(b-a)=cb-ca=bc-ac \in\R^+$, es decir, $ac<bc$. 
  %
   %\textbf{Observación:} La multiplicación por números reales positivos preserva el orden de la desigualdad.
   %\textbf{Nota:} De este resultado sigue que si $m<0$ y $0<n$, entonces $mn<0=0\cdot n$ —multiplicación por cero.
  \item Sea $-c\in\R^+$. Como $b-a\in\R^+$, por cerradura de la multiplicación en $\R^+$, $-c(b-a) \in\R^+$. Sigue que $-c(b-a)= -c \Bigl( b + (-a) \Bigr)=(-c) \cdot b + (-c) \cdot (-a)=-cb +ca $. Finalmente, $-cb +ca=ac - bc \in\R^+$, es decir, $bc<ac$.
  %\textbf{Observación:} La multiplicación por números reales negativos cambia el orden de la desigualdad.
 %\textbf{Nota:} De esta demostración sigue que: \begin{enumerate}[label=\roman*)]
 % \item Si $0<m$ y $n<0$, entonces $mn<0$. (Positivo por negativo/negativo por positivo es negativo).
 % \item Si $m<0$ y $n<0$, entonces $0<mn$. (Negativo por negativo es positivo).
 %\end{enumerate}
 \end{enumerate}

 \bfit{Conclusión:}
 \begin{itemize}
  \item Si $a<b$ y $0<c$, entonces $ac<bc$.% (Multiplicación por positivo).
  \item Si $a<b$ y $c<0$, entonces $bc<ac$.% (Multiplicación por negativo).
 \end{itemize}

 \item Sea $a,b\in \R$. Encuentre las condiciones que deben cumplirse para que $ab<0$ o $0<ab$. (Ley de los signos).
 
 Si $a$ o $b$ son cero, tenemos que $ab=0$, por lo que descartamos esta posiblidad. Por tricotomía, $0<a$ o $a<0$ y $0<b$ o $b<0$, entonces observemos los casos:
 \begin{enumerate}[label=\roman*)]
  \item Si $0<a$ y $0<b$, por la cerradura de la multiplicacón en $\R^+$, tenemos que $0<ab$.
  \item Sin pérdida de generalidad, si $0<a$ y $b<0$, tenemos que $0<-b$, y por la cerradura de la multiplicación en $\R^+$, $0<-ab$, por lo que $ab<0$.
  \item Si $a<0$ y $b<0$, entonces $0<-a$ y $0<-b$, por lo que $0<(-a)(-b)=ab$.
 \end{enumerate}

 Conclusión:
 \begin{itemize}
  \item Por (i) y (ii), para verificar $ab<0$, un componente debe ser mayor a cero y el otro menor a cero.%un componente del producto debe ser positivo y el otro negativo.
  \item Por (iii), para verificar $0<ab$, ambos componentes deben ser mayores o ambos menores a cero.%del producto deben ser positivos o ambos negativos.
 \end{itemize}
%
% \textbf{Nota:} Nos referiremos a (1) y (2) como ley de los signos. 
 %\item $0<ab$ si y solo si $0<a$ y $0<b$ o $a<0$ y $b<0$. (Producto positivo).
 %\textbf{Demostración:}
 %\begin{itemize}
 % \item[$\Rightarrow$)] Sea $0<ab$. Sin pérdida de generalidad, observemos la tricotomía de $a$. Si $a=0$, tenemos que $ab=0$, lo que contradice la hipótesis, por lo que tenemos dos casos:
 % \begin{enumerate}[label=\roman*)]
 %  \item si $a>0$, el inverso multiplicativo es positivo, $a^{-1}>0$. Entonces, \begin{align*}
 %   0 &< ab && \text{Hipótesis}\\
 %   0 \cdot a^{1} &< ab \cdot a^{-1} && \text{Multiplicación por positivo}\\
 %   0 &< ab \cdot a^{1} && \text{Multiplicación por cero}\\
 %   0 &< ba \cdot a^{1} && \text{Conmutatividad}\\
 %   0 &< b && \text{Inverso multiplicativo}
 %  \end{align*}
 %  \item si $a<0$, el inverso multiplicativo es negativo, $a^{-1}<0$. Entonces, \begin{align*}
 %   0 &< ab && \text{Hipótesis}\\
 %   ab \cdot a^{-1} &< 0 \cdot ab && \text{Multiplicación por negativo}\\
 %   ab \cdot a^{1} &< 0&& \text{Multiplicación por cero}\\
 %   ba \cdot a^{1} &< 0&& \text{Conmutatividad}\\
 %   b &< 0&& \text{Inverso multiplicativo}
 %  \end{align*}
 % \end{enumerate}
 % \item[$\Leftarrow$)] Por tricotomía, de la disyunción tenemos casos excluyentes:
 % \begin{enumerate}[label=\roman*)]
 %  \item Si $0<a$ y $0<b$, por la cerradura de la multiplicación en $\R^+$, $0<ab$.
 %  \item Si $a<0$ y $b<0$, \begin{align*}
 %   a &< 0 && \text{Hipótesis}\\
 %   0 \cdot b &< a\cdot b && \text{Multiplicación por negativo}\\
 %   0 &< ab && \text{Multiplicación por cero}
 %  \end{align*}
 % \end{enumerate}
 %\end{itemize} \qed

 \item Si $a<b$ demuestre que $a<\frac{a+b}{2}<b$. (Punto medio).
 \begin{proof} \leavevmode% \vspace{-1em}
  \begin{center}\vspace{-2em}
  \begin{minipage}[r]{.4\linewidth}
  \begin{align*}
  a &< b \\
  a + a &< a+b \\
  2a &< a+b \\
  %2a \cdot \frac{1}{2} &< (a+b) \cdot \frac{1}{2} \\
  %\frac{2a}{2} &< \frac{a+b}{2} \\
  a &< \frac{a+b}{2}
  \end{align*}
  \end{minipage}%
  \begin{minipage}[l]{.4\linewidth}
  \begin{align*}
  a &< b \\
  a + b &< b+b \\
  a +b &< 2b \\
  %(a+b) \cdot \frac{1}{2} &< 2b \cdot \frac{1}{2} \\
  %\frac{a+b}{2} &< \frac{2b}{2} \\
  \frac{a+b}{2} &< b \qedhere
  \end{align*}
  \end{minipage}%
  \end{center}% Por notación, $a < \frac{a+b}{2} < b$.
 \end{proof}
 
 \bfit{Definición:}  Al número $\frac{a+b}{2}$ lo llamaremos el punto medio entre $a$ y $b$.

 \bfit{Observación:} $b-\frac{a+b}{2} = \frac{a+b}{2}-a$ (la \textit{distancia desde} $a$ y \textit{desde} $b$ al punto medio es la misma).
 \[b-\frac{a+b}{2} = \frac{2b}{2}-\frac{a+b}{2} = \frac{2b-a-b}{2}= \frac{b-a}{2}= \frac{b-2a+a}{2}= \frac{a+b-2a}{2}= \frac{a+b}{2}-\frac{2a}{2} = \frac{a+b}{2}-a\]

 \item Sea $a\in \R$. Encuentre las condiciones que deben cumplirse para que $a^{-1}<a$ o $a<a^{-1}$.
 
 Para que $\exists a^{-1}$, requerimos $a\neq 0$. También, sabemos que $a\neq 1$ y $a\neq -1$ pues $1=1^{-1}$ y $-1=(-1)^{-1}$, pero buscamos desigualdad. Entonces, observemos los casos: \begin{enumerate}[label=\roman*)]
  \item Si $a<-1$, entonces $1<-a$, y por transitividad $0<-a$, por lo que $0<-a^{-1}$, luego, \begin{align*}
   1 &< -a \\
   1\cdot \bigl(-a^{-1}\bigr) &< (-a) \cdot \bigl(-a^{-1}\bigr) \\
   -a^{-1} &< 1\\
   -1 &< a^{-1}
  \end{align*} Por transitividad, $a < a^{-1}$.
  \item Si $-1<a<0$, por notación $-1<a$ y $a<0$, de donde sigue que $-a<1$ y $0<-a$, por lo que $0<-a^{-1}$. \begin{align*}
   -a &< 1 \\
   (-a) \cdot (-a^{-1}) &< 1\cdot (-a^{-1})\\
   1 &< -a^{-1}\\
   a^{-1} &< -1
  \end{align*} Por transitividad, $a^{-1}<a$.
  \item Si $0<a<1$, por notación $0<a$ y $a<1$, de donde sigue que $0<a^{-1}$. Luego \begin{align*}
   a &< 1 \\
   a\cdot a^{-1} &< 1 \cdot a^{-1}\\
   1 &< a^{-1}
  \end{align*} Por transitividad $a<a^{-1}$.
  \item Si $1<a$, por transitividad $0<a$, por lo que $0<a^{-1}$. Luego, \begin{align*}
   1 <& a\\
   1 \cdot a^{-1} &< a \cdot a^{-1}\\
   a^{-1} &< 1
  \end{align*} Por transitividad $a^{-1}<a$.
 \end{enumerate}
 Conclusión: \begin{itemize}
  \item Por (i) y (iii), $a<a^{-1}$, si $a<-1$ o $0<a<1$.
  \item Por (ii) y (iv), $a^{-1}<a$, si $-1<a<0$ o $1<a$.
 \end{itemize}

 \item Sea $a\in \R$. Encuentre las condiciones que deben cumplirse para que $0<a^{-1}$ o $a^{-1}<0$.
 
 \begin{enumerate}[label=\roman*)]
  \item Sea $0<a$. Supongamos que $a^{-1}<0$. Como $0<a$, al multiplicar en desigualdades preserva el orden, por lo que $a^{-1} \cdot a<0 \cdot a$. Por un lado, tenemos el inverso multiplicativo $a^{-1} \cdot a = 1$, y por el otro, tenemos una multiplicación por cero, $0\cdot a=0$, con lo que tenemos que $1<0$, pero esto es una contradicción. Sabemos que $a^{-1}\neq 0$, ya que $0$ no es inverso multiplicativo. Por tricotomía, $a^{-1}>0$.
  \item Sea $a<0$. Sigue que $0<-a$, por lo que $-a^{-1}>0$, de donde sigue que $a^{-1}<0$.
 \end{enumerate}

 \bfit{Conclusión:}
 \begin{itemize}
  \item Si $0<a$, entonces $0<a^{-1}$.% (Inverso multiplicativo positivo).
  \item Si $a<0$, entonces $a^{-1}<0$.
 \end{itemize}
 %\text{Nota:} Análogamente, Si $a<0$, entonces $a^{-1}<0$. (Inverso multiplicativo negativo).

 \item Sea $a\in \R$. Encuentre las condiciones que deben cumplirse para que $1<a^{-1}$, $0<a^{-1}<1$, $-1<a^{-1}<0$, o $a^{-1}<1$.
 \begin{enumerate}[label=\roman*)]
  
  \item Sea $0<a<1$. Notemos que $0<a \Rightarrow 0<a^{-1}$. Luego,
  \begin{align*}
    a &< 1\\
    a\cdot a^{-1} &< 1\cdot a^{-1}\\
    1 &< a^{-1}
   \end{align*}

   \item Sea $1<a$. Notemos que $1 < a \Rightarrow 0 < a \Rightarrow 0 < a^{-1}$. Luego,
   \begin{align*}
    1 &< a\\
    1\cdot a^{-1} &< a\cdot a^{-1}\\
    0< a^{-1} &< 1
   \end{align*}

   \item Sea $-1<a<0$. Notemos que $a<0 \Rightarrow 0<-a$ y $-1 < a \Rightarrow -a < 1$, por lo que $0< -a < 1$. Luego,
   \begin{align*}
    1 &< -a^{-1}\\
    a^{-1} &< -1
   \end{align*}

   \item Sea $a<-1$. Notemos que $a<-1 \Rightarrow 1<-a$. Luego,
   \begin{align*}
    0<-a^{-1} &<1\\
    -1 < a^{-1} &< 0
   \end{align*}
  \end{enumerate}

  \bfit{Conclusión:}
  \begin{itemize}
    \item Si $0<a<1$, entonces $1<a^{-1}$.
    \item Si $1<a$, entonces $0<a^{-1}<1$.
    \item Si $-1<a<0$, entonces $a^{-1}<-1$.
    \item Si $a<-1$, entonces $-1<a^{-1}<0$.
  \end{itemize}

  \item Sea $a<b$. Encuentre las condiciones que deben cumplirse para que $\frac{1}{a}<\frac{1}{b}$ o $\frac{1}{b}<\frac{1}{a}$.
 
 Sabemos que $a\neq 0$ y $b\neq 0$, pues requerimos la existencia de su inverso multiplicativo. Luego, por tricotomía, $0<a$ o $a<0$ y $0<b$ o $b<0$, entonces observemos los casos: \begin{enumerate}[label=\roman*)]
  \item Si $0<a$ y $0<b$, por ley de los signos, $0<ab$, por lo que $0<\frac{1}{ab}$. Entonces, $a\cdot \frac{1}{ab} = \frac{1}{b} < \frac{1}{a} = b \cdot \frac{1}{ab}$.
  \item Si $a<0$ y $0<b$, entonces $\frac{1}{a}<0$ y $0<\frac{1}{b}$. Por transitividad, $\frac{1}{a}<\frac{1}{b}$.
  % por ley de los signos, $ab<0$, por lo que $\frac{1}{ab}<0$. De donde sigue que $b \cdot \frac{1}{ab} = \frac{1}{a} < \frac{1}{b} = a\cdot \frac{1}{ab}$.
  %\item Si $b<0$ y $0<a$, por transitividad, $b<a$, lo que contradice el supuesto inicial (descartamos este caso).
  \item Si $a<0$ y $b<0$, por ley de los signos $0<a a$, $0<b b$ y $0<ab$. Luego,\begin{align*}
   a &< b && \text{Supuesto inicial}\\
   a \cdot (ab) &< b \cdot (ab) && \text{(*)}
   %a^2b &< ab^2 && \text{Notación} && \text{(*)}
  \end{align*} Por ley de los signos, $aa\cdot bb>0$, de donde sigue que $\frac{1}{aa\cdot bb}>0$. De (*) obtenemos que \begin{align*}
   \bigl(a \cdot (ab)\bigr) \cdot \frac{1}{aa\cdot bb} &< \bigl(b \cdot (ab)\bigr)\cdot \frac{1}{aa\cdot bb}\\
   \frac{1}{b} &< \frac{1}{a}
  \end{align*}
 \end{enumerate}
 \bfit{Conclusión:} Si $a<b$ y\begin{itemize}
  \item $0<a$ y $0<b$, o $a<0$ y $b<0$, entonces, $\frac{1}{b}<\frac{1}{a}$. %si ambos componentes son mayores o ambos menores a cero, los inversos multiplicativos invierten el orden.%positivos o ambos negativos, entonces los inversos multiplicativos invierten el orden.
  \item $a<0$ y $0<b$, entonces $\frac{1}{a}<\frac{1}{b}$.%Por (ii), si el componente menor es menor a cero y el mayor es mayor a cero, entonces los inversos multiplicativos conservan el orden.
 \end{itemize}
 %
 %\item Si $0 \leq a<b$ y $0 \leq c<d$, entonces $ac<bd$.
 %
 %\begin{enumerate}[label=\roman*)]
 % \item Si $a=0$ o $c=0$, por (g) de LE1 se verifica que $ac=0$. Luego, por (j) de LE3, se verifica que $0<b$ y $0<d$. Así, $ac<bd$.
 % \item Si $a>0$ y $c>0$. Por hipótesis, $a<b$, y por (e) de LE3, sigue que $ac<bc$. También, tenemos que $c<d$, y por (e) de LE3, sigue que $bc<db$. Finalmente, por (j) de LE3, se verifica que $ac<bd$. \qed
 %\end{enumerate}
 %
 %\item Si $a<b$ y $0<ab$, entonces $b^{-1}<a^{-1}$.
 %
 %Notemos que:
 %\begin{align*}
 %a &< b && \text{Por hipótesis} \\
 %a-a &< b-a && \text{Por (d) de LE3} \\
 %0 &< b-a && \text{Inverso aditivo} \\
 %0 \cdot \frac{1}{ab} &< (b-a) \cdot \frac{1}{ab} && \text{Por (h) y (e) de LE3}\\
 %0 &< \frac{b-a}{ab} && \text{Multiplicación por $0$ y (a) de LE2}\\
 %0 &< \frac{1}{a} - \frac{1}{b} && \text{Por (c) de LE2}\\
 %\frac{1}{b} &< \frac{1}{a} && \text{Por (d) de LE3}
 %\end{align*} \qed

 \item Sea $a,b\in\R$. Encuentre las condiciones que deben cumplirse para que $1<\frac{a}{b}$,$0<\frac{a}{b}<1$, $-1<\frac{a}{b}<0$, o $\frac{a}{b}<-1$.
 \begin{enumerate}[label=\roman*)]
  \item Si $0<b<a$, se tiene que $0<\frac{1}{b}$. Luego,
  \begin{align*}
    b &< a\\
    b \cdot \frac{1}{b}&<\cdot \frac{1}{b} a\\
    1 &< \frac{a}{b}
  \end{align*}

  \item Si $0<a<b$, se tiene que $0<\frac{1}{b}$. Luego,
  \begin{align*}
    0 < a &< b\\
    0 \cdot \frac{1}{b} < a\cdot \frac{1}{b} &< b\cdot \frac{1}{b}\\
    0 < \frac{a}{b}&<1
  \end{align*}

  \item Si $a<0<b$, se tiene que $0<\frac{1}{b}$. Luego,
  \begin{align*}
    a &< 0\\
    a \cdot \frac{1}{b}&< 0\cdot \frac{1}{b}\\
    \frac{a}{b} &< 0 && \text{(*)}\\
    0 &< -\frac{a}{b}
  \end{align*}
  Del mismo modo,
  \begin{align*}
    -1
  \end{align*}
  Por (*) y (**) se verifica que $-1<\frac{a}{b}<0$.

 \end{enumerate}

 \item Sea $a,b\in \R$. Encuentre las condiciones que deben cumplirse para que $1<ab$, $0<ab<1$, $-1<ab<0$, o $ab<-1$.
  \begin{enumerate}[label=\roman*)]
    \item Sea $1<a$ y $1<b$. Por ley de cancelación, $0<a-1$ y $0<b-1$. Luego,
    \begin{align*}
      0 &< (b-1)(a-1)\\
      0 &< \bigl(b+(-1)\bigr)\bigl(a+(-1)\bigr)\\
      0 &< a\bigl(b+(-1)\bigr) + (-1)\bigl(b+(-1)\bigr)\\
      0 &< ab + (-1)a + (-1)b + (-1)(-1)\\
      0 &< ab + (-1)a + (-1)b + 1\\
      0 &< ab + (-1)(a+b) + 1\\
      0 &< ab -(a+b) + 1\\
      a+b &< ab + 1
    \end{align*}
    Por la suma vertical de desigualdades, $1+1<a+b$, y por transitividad, $1+1<ab+1$, de donde $1<ab$.

    \item Sea $0<a<1$ y $1<b$. Notemos que $1<a^{-1}$, por lo que 

    \item Sea $0<a<1$ y $0<b<1$. Notemos que $ab>0$. Como $a<1$, sigue que $ab<b$, y a su vez, $b<1$, por lo que $0<ab<1$.
    
   \item Sea $-1<a<0$ y $0<b<1$. Notemos que $ab<0$. Como $b<1$, sigue que $a<ab$, y a su vez $-1<a$, por lo que $-1<ab<0$.
   
   \item Si $-1<a<0$ y $-1<b<0$. Se tiene que $-1<a<0 \Rightarrow a^{-1}<-1 \Rightarrow 1<-a^{-1}$ y $-1<b<0 \Rightarrow b^{-1}<-1 \Rightarrow 1<-b^{-1} \Rightarrow 0<-b^{-1}$.
   \begin{align*}
    1 &< -a^{-1}\\
    1\cdot (-b^{-1}) &< (-a^{-1})\cdot (-b^{-1})\\
    -b^{-1} &< a^{-1}b^{-1}\\
    1 &< a^{-1}b^{-1}&& \text{$1<-b^{-1}$}\\
    1 &< (ab)^{-1}
   \end{align*}
   Por lo que $0<\bigl((ab)^{-1}\bigr)^{-1} < 1$, es decir, $0<ab<1$.
  \end{enumerate}

  \bfit{Conclusión:}
  \begin{itemize}
    \item Si $1<a$ y $1<b$, entonces $1<ab$.
    \item Si $0<a<1$ y $0<b<1$, entonces $0<ab<1$.
    \item Si $-1<a<0$ y $0<b<1$, entonces $-1<ab<0$.
    \item Si $-1<a<0$ y $-1<b<0$, entonces $0<ab<1$.
  \end{itemize}

 \item Sea $a<b$ y $c<d$, encuentre las condiciones que deben cumplirse para que $ac<bd$ o $bd<ac$.
  
 \begin{enumerate}[label=\Roman*)]
   \item Sea $0<a<b$.
   \begin{enumerate}[label=\roman*)]
    \item Si $0<c<d$, entonces $a<b \Rightarrow ac<bc$ y $c<d \Rightarrow bc<bd$. Por transitividad, $ac<bd$.
    \item Si $c<0<d$, entonces $c<d \Rightarrow ac < ad$ y $a<b \Rightarrow ad<bd$. Por transitividad, $ac<bd$.
    \item Si $c<d<0$, entonces $a<b \Rightarrow bc<ac$ y $c<d \Rightarrow ac<ad$. Por transitividad, $bc<ad$.
   \end{enumerate}
   \item Sea $a<0<b$.
   \begin{enumerate}[label=\roman*)]
    \item Si $0<c<d$, entonces $a<b \Rightarrow ac<bc$ y $c<d \Rightarrow bc<bd$. Por transitividad, $ac<bd$.
    \item Si $c<0<d$, entonces 
   \end{enumerate}
 \end{enumerate}

 \item Sea $\frac{a}{b}<\frac{c}{d}$. Encuentre las condiciones que deben cumplirse para que $\frac{a}{b}<\frac{a+c}{b+d}<\frac{c}{d}$. (Mediante).

 Sabemos que $b\neq 0$ y $d\neq 0$. También, por definición, $\frac{c}{d}-\frac{a}{b}>0$, es decir, $(bc-ad)/(bd)>0$. Como $b$ y $d$ son distintos de cero, tenemos que $bd\neq 0$. Asimismo, $bc-ad\neq 0$.

 Buscamos que $\frac{a}{b}<\frac{a+c}{b+d}<\frac{c}{d}$, para lo que es necesario que
 \begin{center}%\vspace{-1em}
 \begin{minipage}[l]{.5\linewidth}
 \begin{align*}
  \frac{a}{b} &< \frac{a+c}{b+d}\\
  0 &< \frac{a+c}{b+d} - \frac{a}{b}\\
  &= \frac{b(a+c)-\bigl(a(b+d)\bigr)}{b(b+d)}\\
  &= \frac{ab+bc-ab-ad}{b(b+d)}\\
  &= \frac{bc-ad}{b(b+d)}\\
  &= \frac{bc-ad}{bd} \cdot \frac{d}{b+d}
 \end{align*}
 \end{minipage}%
 \begin{minipage}[r]{.5\linewidth}
 \begin{align*}
  \frac{a+c}{b+d} &< \frac{c}{d}\\
  0 &< \frac{c}{d} -\frac{a+c}{b+d}\\
  &= \frac{c(b+d)-\bigl(d(a+c)\bigr)}{d(b+d)}\\
  &= \frac{bc+cd-ad-cd}{d(b+d)}\\
  &= \frac{bc-ad}{d(b+d)}\\
  &= \frac{bc-ad}{bd} \cdot \frac{b}{b+d}
 \end{align*}
 \end{minipage}
 \end{center}
 Como $(bc-ad)/(bd)>0$, entonces $\frac{d}{b+d}>0$ y $\frac{b}{b+d}>0$. Por esto, $b+d\neq 0$. Finalmente, tenemos dos casos: \begin{enumerate}[label=\roman*)]
  \item Si $b>0$, entonces $b+d>0$ y $d>0$.
  \item Si $b<0$, entonces $b+d<0$ y $d<0$.
 \end{enumerate}

 Por tanto, debe cumplirse que $b$ y $d$ deben ser ambos mayores a cero o ambos menores a cero.

\end{enumerate}


\part*{Inducción matemática}

\bfit{Definición:}  Sea $A\subset \R$, decimos que $A$ es un conjunto inductivo si se cumplen las siguientes condiciones:
 \begin{enumerate}[label=\roman*)]
  \item $1 \in A$.
  \item Si $n \in A$ entonces se verifica que $n+1 \in A$.
 \end{enumerate}

\subsection*{Lista de Ejercicios 7 (LE7)}

\begin{enumerate}[label=\arabic*)]
 \item ¿El conjunto de los números reales es un conjunto inductivo?
 
 \textbf{Respuesta:} Sí, ya que $1 \in \R$, y si $n\in \R$, entonces $n+1 \in \R$ por la cerradura de la suma en $\R$.


 \item ¿$\R^+$ es un conjunto inductivo?
 
 \textbf{Respuesta:} Sí, pues $1\in \R^+$, y si $n\in \R^+$, entonces $n+1 \in \R^+$ por la cerradura de la suma en $\R^+$.

 \item Sea $\mathcal{F}= \set{A\subset \R: \text{$A$ es un conjunto inductivo}}$.
 \begin{enumerate}[label=\alph*)]
 \item Demuestre que $\mathcal{F}$ es no vacío.
 %\vspace{-1em}
 \begin{proof}
  Como $\R \subset \R$, y $\R$ es un conjunto inductivo, entonces $\R \in \mathcal{F}$, por lo que $\mathcal{F} \neq \emptyset$.
 \end{proof}% \vspace{-1em}
 \item Demuestre que $\bigcap \mathcal{F}$ es un conjunto inductivo.% (Definición del conjunto de los números naturales).
 %\vspace{-1em}
 \begin{proof}
 Por definición, $1\in A, \forall A\in \mathcal{F}$, por lo que $1\in \bigcap \mathcal{F}$. Luego, si $n\in A, \forall A\in \mathcal{F}$, como cada $A$ es un conjunto inductivo, $n+1\in A, \forall A\in \mathcal{F}$, por lo que Si $n \in \bigcap \mathcal{F}$, entonces $n+1 \in \bigcap \mathcal{F}$. Por tanto, $\bigcap \mathcal{F}$ es un conjunto inductivo.
 \end{proof}% \vspace{-1em}
 \end{enumerate}

 %\item Sea $A\coloneqq \{B \subseteq \R: B \text{ es un conjunto inductivo}\}$. Demuestre que $A\neq \emptyset$ y que $C=\bigcap B$ es un conjunto inductivo.
 %
 %\begin{proof} 
 %Claramente $A \neq \emptyset$, pues $\R, \R^+ \subseteq A$.
 %
 %Luego, por hipótesis, $\forall B \in A$ tenemos que $B\subseteq \R $ por lo que $C\subseteq \R$. Además, $\forall B\in A$, se verifica que $1\in B$. Consecuentemente, $1\in C$. Por otra parte, si $n\in B$ para todo $B\in A$, tendremos que $n+1\in B$, por lo que $n+1 \in C$. Por tanto, $C$ es un conjunto inductivo.
 %\end{proof}
\end{enumerate}
%
%\bfit{Definición:}  Al conjunto $C$ de (3) de LE6 lo llamaremos conjunto de los números naturales y lo %denotaremos con el símbolo $\N$.

\bfit{Definición:}

Sea $\mathcal{F}= \set{A\subset \R: \text{$A$ es un conjunto inductivo}}$. Llamaremos al conjunto $\N = \bigcap \mathcal{F}$ conjunto de los números naturales.

\subsection*{Lista de ejercicios 8 (LE8)}

Demuestre lo siguiente:

\begin{enumerate}[label=\alph*)]
 \item Si $m,n\in \N$, entonces $m+n \in \N$. (Cerradura de la suma en $\N$).
 \begin{proof}$ $\newline
  Sea $m\in \N$ arbitrario pero fijo. Definimos $A=\{ n\in \N : m+n \in \N \}$. Por definición, $1\in \N$ y $m+1\in \N$, entonces $1\in A$, es decir, $A\neq \emptyset$. \\[5pt] Por otra parte, si $n\in A$ debe ser el caso que $n\in \N$ y $m+n\in \N$. Como $\N$ es un conjunto inductivo, $n+1 \in \N$ y $(m+n)+1 \in \N$, luego, por la asociatividad de la suma, $m+(n+1)\in \N$. Por la condición de $A$, se cumple que $n+1\in A$, por lo que $A$ es un conjunto inductivo. De esto se concluye que $\N\subseteq A$ y como $A\subseteq \N$, $A=\N$. En otras palabras, la suma de números naturales es un número natural. 
 \end{proof}



 \item Si $m,n\in \N$, entonces $m\cdot n \in \N$. (Cerradura de la multiplicación en $\N$).
 \begin{proof}$ $\newline
  Sea $m\in \N$ arbitrario pero fijo. Definimos $A=\{n\in \N: m\cdot n \in \N\}$. Por definición, $1 \in \N$. Adenás, $m\cdot 1 \in \N$, entonces $1 \in A$, es decir $A \neq \emptyset$.
 
  Luego, si $n \in A$ debe ser el caso que $n\in \N$ y $m \cdot n \in \N$. Por cerradura de la suma en $\N$ se verifica que $m\cdot n + m \in \N$. Notemos que $m\cdot n + m=m\cdot n + m\cdot 1 = m\cdot (n+1)$, por lo que $m \cdot (n+1) \in \N$. Como $\N$ es un conjunto inductivo, tenemos que $n+1\in \N$. De este modo, $n+1\in A$. Lo que implica que $A$ es un conjunto inductivo. De esto se concluye que $\N \subseteq A$ y como $A\subseteq \N$, $A=\N$. En otras palabras, la multiplicación de números naturales es un número natural.
 \end{proof}

 \item $1\leq n, \forall n\in \N$. (Elemento mínimo de $\N$).
 \begin{proof} 
  Sea $A\coloneqq \{n\in \N: n\geq 1\}$. Como $1\in \N$ y $1\geq 1$, tenemos que $1\in A$.

  Si $n\in A$ debe ser el caso que $n\in \N$ y $1\leq n$. Además, por la cerradura de la suma en $\N$, $n+1\in \N$. Luego, $0 \leq 1$ de donde sigue que $n \leq n+1$. Por transitividad, $1\leq n+1$, por lo que $n+1\in A$, lo que implica que $A$ es un conjunto inductivo, es decir, $\N\subseteq A$ y como $A\subseteq \N$, $A=N$. En otras palabras, $n\geq 1, \forall n\in\N$.
 \end{proof}

 \bfit{Definición:}  Sea $A\subseteq \R$ con $A \neq \emptyset$, decimos que $m$ es elemento mínimo de $A$ si $m\in A$ y $m\leq a, \forall a\in A$.

 \item Para todo $n\in \N$ con $n>1$ se verifica que $n-1\in \N$.
 \begin{proof} 
  Sea $A \coloneqq \set{n\in \N | n>1, n-1\in \N}\union \set{1}$. Sea $m\in A$ con $m>1$, tenemos que $m\in \N$, y como $\N$ es un conjunto inductivo, se verifica que $m+1\in\N$. Luego, $(m+1)-1=m$, por lo que $(m+1)-1\in \N$. Como $m>1$, por transitividad, $m>0$, de donde sigue que $m+1>1$, por lo que $m+1\in A$. De este modo, $A$ es un conjunto inductivo, con lo que $\N \subseteq A$, y como $A\subseteq \N$, $A=\N$. Por tanto $\forall n\in \N$ con $n>1$ se verifica que $n-1\in \N$.
 \end{proof}

 \item Sean $m$ y $n$ números naturales. Si $n<m$, entonces $m-n\in\N$. 
 \begin{proof} 
  Sea $n\in \N$ arbitrario pero fijo. Definimos $A \coloneqq \set{m\in \N| n<m, \ m-n\in\N}\union \set{1}$. Si $m_0\in A$ con $m_0>1$, tenemos que $n<m_0$ y $m_0-n\in \N$ con $m_0\in \N$. Como $\N$ es un conjunto inductivo, sigue que $m_0+1\in \N$. Además, $m_0<m_0+1$, y por transitividad $n<m_0+1$. Luego, por la cerradura de la suma en $\N$ tenemos que $(m_0-n)+1 =(m_0+1)-n \in \N$, por lo que $m_0+1\in A$. De este modo, $A$ es un conjunto inductivo, con lo que $\N \subseteq A$, y como $A\subseteq \N$, se cumple que $A=\N$.
 \end{proof}

 \bfit{Corolario:} \begin{enumerate}[label=\roman*)]
  \item Sea $x\in \R$. Si $n\in \N$ y $n<x<n+1$, entonces $x$ no es un número natural. (La \textit{distancia} entre un número natural y su \textit{sucesor} es $1$).
  
  \begin{proof}\leavevmode
    Por hipótesis, $n<x$, de donde sigue que $n+(-x+1)<x+(-x+1)$, osea, $n-x+1<1$. Como $\N$ es un conjunto inductivo, $n+1\in \N$. Ahora, supongamos que $x\in \N$, de la hipótesis $x<n+1$ sigue que $n+1-x\in \N$, por este teorema, y como $1$ es elemento mínimo de $\N$, tenemos que $1\leq n+1-x$. Esto implica que $1\leq n+1-x < 1$, lo que es una contradicción. Por tanto, $x$ no es un número natural.
  \end{proof}

  \textbf{Nota:} Otra forma de plantear esta proposición es la siguiente (ii):

  \item Sea $x\in \R$. Si $m\in \N$ y $m-1<x<m$, entonces $x$ no es un número natural. (La \textit{distancia} entre un número natural y su \textit{antecesor} es $1$).
  
  \begin{proof}\leavevmode
    Sea $x\in \N$. Por hipótesis $x<m$ y $m-1<x$, de donde obtenemos:
  \begin{center}\vspace{-1em}
   \begin{minipage}[l]{.4\linewidth}
   \begin{align*}
    x +(-m+1) &< m +(-m+1)\\
    %x -m &< m-m\\
    %x-m &< 0\\
    x+1-m &< 1
   \end{align*}
   \end{minipage}%
   \begin{minipage}[r]{.4\linewidth}
   \begin{align*}
    (m-1) +1 &< (x)+1\\
    %m-1-(m-1) &< x - (m-1)\\
    %0 &< x-m+1\\
    m &< x+1
   \end{align*}
   \end{minipage}
  \end{center}
  Por hipótesis $x\in \N$, y como $\N$ es un conjunto inductivo, sigue que $x+1\in \N$. Como $m<x+1$, con $m\in \N$, por este teorema se verifica que $x+1-m \in \N$, y como $1$ es elemento mínimo de $\N$, sigue que $1\leq x+1-m$. Esto implica que $1\leq x+1-m<1$, lo que es una contradicción. Por tanto, $x$ no es un número natural.
  \end{proof}

  \item Todo subconjunto no vacío de $\N$ tiene elemento mínimo. (Principio del buen orden).
  
  \begin{proof}\leavevmode
    Sea $A\subset \N$ con $A\neq \emptyset$. Supongamos que $A$ no tiene elemento mínimo.
  
  Como $A\neq \emptyset$, se tiene que $\exists x\in A$, y como $A\subset \N$, entonces $x\in \N$. Sabemos que $1$ es elemento mínimo de $\N$, por lo que, en particular $1\leq x$. Como $A$ no tiene elemento mínimo, no puede ser el caso que $x=1$, pues $1\leq x,\forall x\in A$. De esto sigue que $1<x$, y por este teorema, se verifica que $x-1\in \N$, y sabemos que $x-1<x$. Como $\N$ es un conjunto inductivo, $x+1\in \N$, y sabemos que $x<x+1$. De este modo, tenemos que $x-1<x<x+1$, pero por este teorema, esta desigualdad implica que $x$ no es un número natural, lo que es una contradicción. Por tanto, $A$ tiene elemento mínimo.
  \end{proof}
  %\begin{proof} 
  %Sea $A\subset \N$ con $A \neq \emptyset$. Supongamos que $A$ no tiene elemento mínimo.%, es decir, supongamos que si $c\leq a, \forall a\in A$, entonces $c\notin A$.
  %
  %Sea $S\defined \set{n\in \N| n<a, \forall a\in A}$.
  %%
  %%\textbf{Nota:} Nuestra intención es mostrar que $S$ es inductivo, con lo que llegaremos a una contradicción a partir del supuesto (de que $A$ no tiene elemento mínimo), por ello procedemos como sigue:
  %
  %Si $1\notin S$, entonces $\exists a_0\in A$ tal que $a_0 \leq 1$. Como $a_0\in \N$, sabemos que $1\leq a_0$, por lo que $1\leq a_0 \leq 1$, lo que implica que $a_0=1\in A$, pero $1\leq a, \forall a\in A$, pues los elementos de $A$ son números naturales. Por tanto, $1$ es elemento mínimo de $A$, pero esto contradice nuestro supuesto inicial. Por tanto, $1\in S$.
  %%
  %%\textbf{Nota:} Si suponemos que $\exists n\in S$ y esto implica que $n+1\in S$ estaríamos probamos que $S$ es un conjunto inductivo.
  %
  %Si $n\in S$, tenemos que $n\in \N$ (*) y $n<a, \forall a\in A$ (**).
  %
  %Luego, si $n+1\notin S$, entonces $\exists a_0\in A$ tal que $a_0 \leq n+1$. De (*) se sigue que $n+1\in \N$, pues $\N$ es un cojunto inductivo. A su vez, de (**) se tiene que, en particular, $n<a_0$, por lo que $n<a_0\leq n+1$ (***). Como $n$, $a_0$ y $n+1$ son números naturales, (***) no puede cumplirse con desigualdad estricta, así que $a_0=n+1\in A$. Por nuestro supuesto inicial $A$ no tiene elemento mínimo, esto es, $\exists m\in A$ tal que $m<n+1$, y así $n<m<n+1$, lo que es una contradicción. Por tanto, $n+1\in S$, es decir, $S$ es un conjunto inductivo, y por definición $S\subset \N$ y $\N\subset S$, por lo que $S=\N$.
  %
  %Finalmente, dado que $A\neq \emptyset$, $\exists a_0\in A$ tal que $n<a_0, \forall n\in \N$, por lo que podemos elegir $n=a_0<a_0$, lo que es una contradicción. Por tanto, $A$ debe tener elemento mínimo.
  %%notemos que $A\intersection S=\emptyset$, y dado que $A\subseteq \N$ y $S=\N$, sigue que $A=\emptyset$, pero esto es una contradicción. Por tanto, si $A\subseteq \N$ con $A\neq \emptyset$, entonces $A$ tiene elemento mínimo.
  %\end{proof}
  %
  %Todo subconjunto no vacío del conjunto de los números naturales tiene elemento mínimo. Esto significa que si $A\subseteq \N$ y $A \neq \emptyset$, entonces existe un elemento $c\in A$ tal que $c\leq a, \forall a\in A$.
  
  %\textbf{Observación:}
  %
  %Sabemos —por (c) de LE7— que cualquier subconjunto no vacío de $\N$ está acotado inferiormente. El principio del buen orden nos garantiza que cualquier subconjunto no vacío de $\N$ contiene una de sus cotas inferiores, a la que llamamos elemento mínimo.
  %
  %Notemos que si suponemos la existencia de un subconjunto no vacío de $\N$ tal que ninguna de sus cotas inferiores esté contenida en el conjunto, estaríamos negando el principio del buen orden. Es así cómo procedemos a probar el teorema.
  
  %Sea $A\subseteq \N$ con $A\neq \emptyset$. Supongamos que $A$ no contiene ninguna de sus cotas inferiores, es decir, supongamos que si $c\leq a, \forall a\in A$, entonces $c\notin A$.
  %
  %Definimos el conjunto $L\coloneqq \set{n\in \N: n\leq a, \forall a\in A}$. Es claro que $1\in L$. Veamos que si $n\in L$, tendríamos que $n\leq a, \forall a\in A$. Luego, si $n+1\notin L$, entonces $\exists a_0\in A$ tal que $n+1>a_0$, por lo que $n\leq a_0<n+1$, y —por (h) de LE7— no puede ser el caso que $n<a_0$, de donde sigue que $n=a_0$, pero esto contradice nuestro supuesto inicial, entonces, debe ser el caso que $n+1\in L$. Consecuentemente, $L$ es un conjunto inductivo, y —por definición— $\N\subseteq L$ y $L\subseteq \N$, lo que implica que $L=\N$.
  %
  %Finalmente, notemos que $A$ y $L$ son disjuntos, y dado que $A\subseteq \N$ y $L=\N$, sigue que $A=\emptyset$, pero esto es una contradicción. Por tanto, si $A\subseteq \N$ con $A\neq \emptyset$, entonces $\exists c\in A$ tal que $c\leq a, \forall a\in A$. \qed
  %%
  %\textbf{\textit{Teorema.}} Si $A\subseteq \N$ y $A\neq \emptyset$ y $A$ está acotado superiormente, entonces $A$ tiene elemento máximo, esto es existe un elemento $c\in A$ tal que $a\leq c, \forall a\in A$.
  %
  %\textbf{Demostración:}
  %
  %Sea $A\subseteq\N$ con $A\neq \emptyset$ y $A$ acotado superiormente. Supongamos que si $c\geq a, \forall a\in A$, entonces $c\notin A$.
  %
  %Definimos el conjunto $-A\coloneqq \set{-a: a\in A}$. Como $A$ está acotado superiormente, $\exists c\in \R$ tal que $c\geq a, \forall a\in A$. Notemos que $-a\geq -c, \forall -a\in -A$, lo que implica que $-A$ está acotado inferiormente, y por esto, $A$ tiene elemento mínimo. Sea $m$ el elemento mínimo de $-A$. Veamos que $m\leq -a, \forall -a\in -A$ de donde sigue que $a\leq -m, \forall a\in A$, con $-m\in A$ pero esto contradice nuestro supuesto inicial. Por tanto, $A$ tiene elemento máximo. \qed
  %

  %\item Sea $x\in \R$. Si $n\in \N$ y $n-1<x<n$, entonces $x$ no es un número natural.
  %\begin{proof}
  %Supongamos que $x\in \N$. Por hipótesis $x<n$ y $n-1<x$, de donde obtenemos:
  %\begin{center}\vspace{-1em}
  % \begin{minipage}[l]{.4\linewidth}
  % \begin{align*}
  %  x &< n\\
  %  %x -n &< n-n\\
  %  x-n &< 0\\
  %  x-n +1 &< 1
  % \end{align*}
  % \end{minipage}%
  % \begin{minipage}[r]{.4\linewidth}
  % \begin{align*}
  %  n-1 &< x\\
  %  %n-1-(n-1) &< x - (n-1)\\
  %  %0 &< x-n+1\\
  %  n &< x+1
  % \end{align*}
  % \end{minipage}
  %\end{center}
  %Suponemos que $x\in \N$, y como $\N$ es un conjunto inductivo, sigue que $x+1\in \N$. Como $n<x+1$, con $n\in \N$, por el teorema se verifica que $x+1-n \in \N$, por lo que $1\leq x+1-n$. No obstante, tenemos que $x-n+1<1$, osea $1\leq x+1-n<1$, lo que es una contradicción. Por tanto, $x$ no es un número natural.
  %\end{proof}

  \item Sea $S\subseteq \N$ tal que $S$ es un conjunto inductivo, entonces $S=\N$. (Principio de inducción matemática).

  \begin{proof} 
  Sea $S\neq \N$. El conjunto $\N\setminus S$ es no vacío (ya que de serlo, tendríamos $S=\N$). Pr definición, $1\in S$ y por esto, $1\notin \N\setminus S$. Como $\N\setminus S\subset \N$, por el principio del buen orden, tiene elemento minimo. Sea $m$ el elemento mínimo de $\N\setminus S$, como $m\in \N$, sigue que $1 \leq m$. Como $m\in \N\setminus S$ y $1\notin \N\setminus S$ tenemos que $m\neq 1$, por lo que $m>1$, y por este teorema, $m-1\in \N$. Debido a que $m-1<m$ y $m$ es el elemento mínimo de $\N \setminus S$, tenemos que $m-1\notin \N\setminus S$, osea $m-1\in S$. Luego, dado que $S$ es un conjunto inductivo, se verifica que $(m-1)+1=m\in S$ lo que es una contradicción. Por tanto, $S=\N$. 
  \end{proof}

  \item Sea $x\in \R^+$. Si $n\in \N$ y $x+n\in \N$, entonces $x\in \N$.

  \begin{proof}\leavevmode
    Por definición, $x>0$, por lo que $n<n+x$. Por hipótesis, $x+n\in \N$ y $n\in \N$. Luego, por este teorema, $(x+n)-n \in \N$, osea, $x\in \N$.
  \end{proof}
 \end{enumerate} %Corolario

\end{enumerate}

\pagebreak

\subsection*{Una nota sobre inducción matemática}

\bfit{Proposición:} $0<\frac{1}{n}\leq 1, \forall n\in \N$. \begin{proof}
  Sea $A \defined \set{n|0<\frac{1}{n} \leq 1, n\in \N}$. Notemos que $1\in A$, pues $0<\frac{1}{(1)} \leq 1$. Luego, si $n\in A$, tenemos que $0<\frac{1}{n} \leq 1$. También, $n<n+1$, por lo que $\frac{1}{n+1} < \frac{1}{n}$. Finalmente, como $n+1>0$, sigue que $\frac{1}{n+1}>0$, y por transitividad, $0<\frac{1}{n+1}\leq 1$. Como $\N$ es un conjunto inductivo, entonces $n+1\in \N$, se verifica $n+1\in A$, y por el principio de inducción matemática, $A =\N$.
\end{proof}

El lector notará que la prueba consiste en: \begin{enumerate}
  \item Definir un subconjuto $A$ de números naturales (el cual satisface la propiedad objetivo).
  \item Demostrar que $A$ es un conjunto inductivo: \begin{itemize}
    \item Exhibir que $1$ pertenece a $A$ (caso base).
    \item Plantear que algún número natural $n$ pertenece a $A$ (hipótesis de inducción o paso inductivo).
    \item Probar que $n+1$ pertenece a $A$.
  \end{itemize}
  \item Enunciar el prinicipio de inducción matemática (que garantiza la propiedad para todo número natural).
\end{enumerate}

\textbf{Nota:} No se exige que $n+1\in A$ sea una consecuencia de que $n\in A$, sin embargo, este suele ser el caso; por lo que, si se prescinde del paso inductivo para probar que $n+1$ cumple la propiedad enunciada, es un buen hábito detenerse y comprobar el desarrollo, puede ser que se haya cometido un error o que en realidad no necesite inducción para la prueba.

Esta \textit{receta} nos permite probar proposiciones sobre los números naturales; no obstante, la tradición de los libros de texto es definir de manera implícita el conjunto con el que se trabaja y —si acaso— enunciar el principio de inducción matemática al inicio de la prueba. Por ejemplo:

\bfit{Proposición:} $0<\frac{1}{n}\leq 1, \forall n\in \N$.
\begin{proof}
  Procedemos por inducción. \begin{enumerate}[label=\roman*)]
    \item Es claro que $n=1$ satisface la desigualdad, pues $0<\frac{1}{1} \leq 1$.
    \item Supongamos que la desigualdad se cumple para $n=k$, es decir, supongamos que \begin{align*}
      0&<\frac{1}{k}\leq 1 && \text{(hipótesis de inducción)}
    \end{align*}
    \item Demostraremos, a partir de la hipótesis de inducción, que la desigualdad se cumple también para $n=k+1$. Es decir, probaremos que \begin{align*}
      0&<\frac{1}{k+1}\leq 1
    \end{align*}
    En efecto, notemos que $k<k+1$, de donde obtenemos que $\frac{1}{k+1} < \frac{1}{k}$. Además, como $k+1>0$, sigue que $\frac{1}{k+1}>0$. Y de la hipótesis de inducción tenemos que \[0 < \frac{1}{k+1} < \frac{1}{k} \leq 1\]
    es decir, \begin{align*}
      0<\frac{1}{k+1}\leq 1.
    \end{align*}
  \end{enumerate}
  Por tanto, $0<\frac{1}{n}\leq 1, \forall n\in \N$.
\end{proof}

Sin embargo, el lector debería ser cuidadoso de no considerar el uso de inducción matemática como la única estratégia para demostrar proposiciones sobre los elementos de $\N$, por ejemplo, la proposición también puede ser probada por casos:

\bfit{Proposición:} $0<\frac{1}{n}\leq 1, \forall n\in \N$.
 \begin{proof}
  Sea $n\in \N$ arbitrario pero fijo. Sabemos que $n\geq 1$, por lo que tenemos dos casos: \begin{enumerate}[label=\roman*)]
   \item Si $n=1$, tenemos que $\frac{1}{n}=\frac{1}{1}=1$. Por lo que $0<\frac{1}{n}\leq 1$.
   \item Si $n>1$, por transitividad $n>0$, lo que implica que $\frac{1}{n}>0$. Retomando la hipótesis, \begin{align*}
    n &> 1\\
    n \cdot \frac{1}{n} &> 1\cdot \frac{1}{n}\\
    1 &> \frac{1}{n}
   \end{align*} Por lo que $0<\frac{1}{n}\leq 1$. 
  \end{enumerate} Como $n$ es arbitrario, se verifica que $0<\frac{1}{n}\leq 1, \forall n\in \N$.
 \end{proof}

\section*{Potenciación}

\bfit{Definición:}  Sea $b\in \R$ y $n\in \N$,
 \[
  b^n = \left\{
 \begin{array}{@{}r@{\thinspace}l}
  b, &  \ \text{si}  \ n = 1\\
  b\cdot b^{n-1}, &  \ \text{si}  \ n > 1
  %\frac{1}{b^n}, &  \ \text{si}  \ n < 1\\
 \end{array} \right. \]
%
 %\textbf{Notación:} Sea $b\in \R$ y $n\in \N$. Si $b \neq 0$, denotaremos $\frac{1}{b^n}$ como $b^{-n}$.
%
\subsection*{Lista de Ejercicios 9 (LE9)}

Sean $a, b\in \R$ y $m,n\in \N$, demuestre lo siguiente:

\begin{enumerate}[label=\alph*)]
  \item $1^n=1, \forall n\in \N$.
  \vspace{-1em}
  \begin{proof} Sea $A=\set{n|1^n=1}$.
  \begin{enumerate}[label=\roman*)]
  \item Notemos que $1\in A$, pues $1^1=1$, por definición.
  \item Si $n\in A$, entonces $1^n=1$.
  \item Por definición, \begin{align*}
  1^{n+1} = \left\{
  \begin{array}{@{}r@{\thinspace}l}
    1, &  \ \text{si}  \ n+1 = 1\\
    1\cdot 1^{n}, &  \ \text{si}  \ n+1 > 1
    %\frac{1}{b^n}, &  \ \text{si}  \ n < 1\\
  \end{array} \right.
  \end{align*}
 Como $n>0$, se tiene $n+1>1$, de donde sigue que $1^{n+1}=1\cdot 1^n=1^n$, por (ii) se verifica que $1^n=1$, es decir, $1^{n+1}=1$, lo que implica que $n+1\in A$.
 \end{enumerate}

 Por el principio de inducción matemática, $A=\N$, por lo que $1^n=1, \forall n\in \N$.
 \end{proof}

 \item $0^n=0, \forall n\in \N$.
 \begin{proof}
 Es claro que $0^1=0$. Luego, si $0^n=0$, tenemos que $0^{n+1}=0 \cdot 0^n$, lo que es una multiplicación por $0$, es decir, $0^{n+1}=0$.
 \end{proof}
 
  \item $a^m \cdot a^n = a^{m+n}$.

 \begin{proof} Sea $m\in \N$ arbitrario pero fijo. Definimos $A=\set{n|a^{m+n}=a^m\cdot a^n}$.

 \begin{enumerate}[label=\roman*)]
 \item Por definición \begin{align*}
  a^{m+1} = \left\{
  \begin{array}{@{}r@{\thinspace}l}
    a, &  \ \text{si}  \ m+1 = 1\\
    a\cdot a^{m}, &  \ \text{si}  \ m+1 > 1
    %\frac{1}{b^n}, &  \ \text{si}  \ n < 1\\
  \end{array} \right.
  \end{align*}
  
  Como $m>0$, sigue que $m+1>1$, por lo que $a^{m+1}=a\cdot a^m=a^m\cdot a^1$, lo que implica que $1\in A$.
 \item Si $n\in A$, tenemos que $a^{m+n}=a^m\cdot a^n$.
 \item Por definición \begin{align*}
  a^{m+(n+1)} = \left\{
   \begin{array}{@{}r@{\thinspace}l}
    a, &  \ \text{si}  \ m+(n+1) = 1\\
    a\cdot a^{m+n}, &  \ \text{si}  \ m+(n+1) > 1
   \end{array} \right.
  \end{align*}

 Como $m+n>0$, sigue que $m+(n+1)>1$. Por tanto, \begin{align*}
  a^{m+(n+1)} &=a\cdot a^{m+n} && \text{Definición}\\
  &= a\cdot a^m \cdot a^n && \text{Por (ii)}\\
  &= a^m \cdot a^{n+1} && \text{Por (i)}
 \end{align*}
 \end{enumerate}

 Por el principio de inducción matemática, $A=\N$, por lo que $a^m \cdot a^n = a^{m+n}, \forall m,n\in \N$.
 \end{proof}

 \item Si $b\neq 0$, entonces $\frac{a^n}{b^n}= \qty(\frac{a}{b})^n$.
 \begin{proof}
 Sea $A=\set{n|\frac{a^n}{b^n}= \qty(\frac{a}{b})^n}$.
 \begin{enumerate}[label=\roman*)]
 \item Notemos que $\frac{a^1}{b^1}= \frac{a}{b}= \qty(\frac{a}{b})^1$. Por lo que $1\in A$.
 \item Si $n\in A$, tenemos $\frac{a^n}{b^n}= \qty(\frac{a}{b})^n$.
 \item Como $n>0$, se tiene $n+1>1$, por definición
  \begin{align*}
    \qty(\frac{a}{b})^{n+1} &= \frac{a}{b}\cdot \qty(\frac{a}{b})^n\\
    &=\frac{a}{b}\cdot \frac{a^n}{b^n} && \text{Por (ii)}\\
    &= \frac{a\cdot a^n}{b\cdot b^n}\\
    &= \frac{a^{n+1}}{b^{n+1}} && \text{Por tanto, $n+1\in A$.}
   \end{align*}\
 \end{enumerate}
 Por el principio de inducción matemática, $A=\N$, por lo que $\frac{a^n}{b^n}= \qty(\frac{a}{b})^n$.
 \end{proof}

 \item $(ab)^n = a^nb^n, \forall n\in \N$.
 \begin{proof}
  Procedemos por inducción. \begin{enumerate}[label=\roman*)]
    \item Se verifica para $n=1$, pues $(ab)^1=ab=a^1b^1=a^nb^n$.
    \item Supongamos que la igualdad se verifica para $n=k$, es decir, suponemos que \begin{align*}
      (ab)^k = a^kb^k
    \end{align*}
    \item Si $n=k+1$ sigue que \begin{align*}
      (ab)^{k+1} &= (ab)^k (ab) && \text{(c) de LE9}\\
      &= a^kb^k (ab) && \text{Hip. Inducción}\\
      &= a^{k+1} b^{k+1} && \text{(c) de LE9} \qedhere 
    \end{align*}
  \end{enumerate}
 \end{proof}

 \item $a^{mn} = (a^m)^n, \forall m,n\in \N$.
 \begin{proof}
  Sea $n\in \N$ arbitrario pero fijo. Definimos $A\defined \set{m|(a^m)^n = a^{mn}, m\in \N}$. Es claro que $1\in A$, pues $(a^1)^n = (a)^n = a^n = a^{1\cdot n} = a^{mn}$. Si $m_0\in A$, entonces $(a^{m_0})^n = a^{m_0n}$ con $m_0\in \N$. Como $\N$ es un conjunto inductivo, $m+1\in \N$, de donde sigue que $a^{m+1} = a^m\cdot a$, por (c) de LE9. Por ello $(a^{m+1})^n = \big(a^m\cdot a\big)^n$, y por (e) de LE9, obtenemos que $\big(a^m\cdot a\big)^n = a^{mn}\cdot a^n$, de donde $a^{mn}\cdot a^n = a^{mn+n} = a^{n(m+1)}$, y así, $m+1\in A$, por lo que $A$ es un conjunto inductivo. Como $A\subseteq \N$, y por el principio de inducción matemática $A=\N$. Por tanto $(a^m)^n = a^{mn}, \forall m,n\in \N$.
 \end{proof}

 \end{enumerate}

 \section*{Números enteros}

 \bfit{Proposición:} Sea $m,n\in \N$ con $m\neq n$. Si $a\neq 0$, entonces $\frac{a^m}{a^n} = a^{m-n}$.
 
 \begin{proof}
   Sea $n\in \N$ arbitrario pero fijo. Definimos $A=\set{m \in \N|\frac{a^m}{a^n}=a^{m-n}, \ m>n}\union\set{1}$. 
   Sea $m_0 \in A$ con $m_0>1$, entonces $\frac{a^{m_0}}{a^n} = a^{m_0-n}$ con $n<m_0$ y $m_0\in \N$. Como $\N$ es un conjunto inductivo, sigue que $m_0+1\in \N$, y $m_0<m_0+1$, por transitividad, $n<m_0+1$. Sigue que \begin{align*}
    \frac{a^{m_0+1}}{a^n} &= \frac{a^{m_0}\cdot a}{a^n}\\
    &= a\cdot \frac{a^{m_0}}{a^n}\\
    &= a\cdot a^{m_0-n}
   \end{align*}
   Como $m_0,n \in \N$ y $m_0>n$, se verifica, por (e) de LE8, que $m_0-n\in \N$, y por (c) de LE9, $a\cdot a^{m_0-n} = a^{1+(m_0-n)} = a^{(m_0+1)-n}$, lo que implica que $m_0+1 \in A$. Por el principio de inducción matemática, $A=\N$. Como $n$ es arbitrario, si $m>n$, entonces $\frac{a^m}{a^n} = a^{m-n}$.
 \end{proof}

 El lector notará que para que la proposición $\frac{a^m}{a^n} = a^{m-n}$ sea probada, además de que $a\neq 0$, requerimos que $m\neq n$, pues si tenemos el caso donde $m=n$, entonces \begin{align*}
  \frac{a^m}{a^n} &= \frac{a^n}{a^n}\\
  &= \qty(\frac{a}{a})^n && \text{(d) de LE9}\\
  &= \bigl(a\cdot a^{-1}\bigr)^n && \text{Notación}\\
  &= 1^n && \text{Inverso multiplicativo}\\
  &= 1 && \text{(a) de LE9}
\end{align*}
Por otro lado, $a^{m-n} = a^{n-n} = a^0$. Sin embargo, hasta este punto, no es posible probar que $a^0 = 1$, pues definimos las propiedades de los exponentes para númeos naturales, y tenemos que $0\notin \N$. Por tanto, introducimos el hecho de que $x^0 = 1, \forall x\in \R$ como una definición, pero al hacerlo, expandimos las propiedades de potenciación al conjunto de los números enteros.

  \bfit{Definción:} \begin{itemize}
    \item Al conjunto $\N \cup \set{0} \cup \set{-n: n\in \N}$ lo llamaremos conjunto de los números enteros y lo representaremos con el símbolo $\Z$.
    \item Al conjunto $\set{-n: n\in \N}$ lo llamaremos conjunto de los números enteros negativos y lo representaremos con el símbolo $\Z^-$.
    %\item Al conjunto $\N$ también lo llamaremos conjunto de los números enteros positivos y lo representaremos con el símbolo $\Z^+$.
   \end{itemize}
   
   %\textbf{Observación.} Los conjuntos $\N$, ${0}$, ${-n: n\in \N}$ son disjuntos por pares.

%\bfit{Definición:}  Sea $a\in \R$ y $m\in \Z$,
% \[
%  a^m = \left\{
% \begin{array}{@{}r@{\thinspace}l}
%  a\cdot a^{m-1}, &  \ \text{si}  \ m > 0\\
%  1, &  \ \text{si}  \ m = 0\\
%  \frac{1}{a^{-m}}, &  \ \text{si}  \ m < 0 \ \text{y} \ a\neq 0
% \end{array} \right. \]

\subsection*{Lista de Ejercicios 10 (LE10)}

\begin{enumerate}[label=\alph*)]
  \item ¿El conjunto de los números enteros es un campo (satisface los axiomas de campo)?
  
  \textbf{Respuesta:} No, pues el axioma del inverso multiplicativo solo se satisface para $1$ y $-1$.

  \item Sea $s\in \Z$, demuestre que si $s<j<s+1$, entonces $j\notin \Z$.
  \begin{proof} Supongamos que $j\in \Z$. Tenemos tres casos:
    \begin{enumerate}[label=\Roman*)]
      \item Si $j\in \set{0}$, tenemos que $0<s+1$, de donde sigue que $s+1\in \N$, por definición (de $\Z$), pero como $s<0$, obtenemos $s+1<1$, lo que es una contradicción, pues todo número natural es mayor o igual a 1.
      \textbf{Nota:} Se asume que la suma es cerrada en $\Z$, pues la hipótesis únicamente exige que $s\in \Z$, pero no que $s+1\in \Z$. El lector debería verificar este hecho.
      \item Si $j\in\N$, tenemos tres casos para $s$:
      \begin{enumerate}[label=\roman*)]
        \item Si $s=0$, sigue que $s=0<j<1=s+1$, pero esto es una contradicción, pues todo número natural es mayor o igual a 1.
        \item Si $s\in \N$, por el corolario (i) de (e) de LE8, es una contradicción.
        \item Si $-s\in \N$, se tiene que $-s>0$, por lo que $s<0$, de done sigue que $s+1<1$, pero $j<s+1$ y por transitividad, $j<1$, lo que es una contradicción.
      \end{enumerate}
      \item Si $-j\in \N$, sigue que $-j>0$, por lo que $j<0$, y por transitividad, $s<0$, de esto sigue que $s\neq 0$ y $s\notin \N$, por lo que $-s\in \N$. Como $s<j$ y $j<s+1$, sigue que $-j<-s$ y $-(s+1)=-s-1<-j$; de este modo, $-s-1<-j<-s$, con $-s, -j\in \N$, pero se contradice el corolario (ii) de (e) de LE8.
    \end{enumerate}
    En cualquier caso $j\notin \Z$.
  \end{proof}
\end{enumerate}

\bfit{Definición:} Sea $a\in \R$ y $n\in \N$, \begin{itemize}
  \item $a^0=1, \forall a\in \R$.
  \item $a^{-n}=\frac{1}{a^n}$, si $a\neq 0$.
\end{itemize}

\subsection*{Lista de Ejercicios 11 (LE11)}

 \begin{enumerate}[label=\alph*)]
  \item $1^m = 1, \forall m\in \Z$. \begin{proof}
    Si $m=0$, la prueba está terminada, y hemos probado que la proposición es verdadera si $m\in \N$. Luego, si $m<0$, tenemos que $1^m = \frac{1}{1^{-m}}$, donde $-m\in \N$, por lo que $1^m = \frac{1}{1} = 1$.
  \end{proof}

  \item Sea $a,b\neq 0$, entonces $\frac{a^n}{b^n} = \left(\frac{b}{a}\right)^{-n}, \forall n\in \Z$.
  \begin{proof}\leavevmode
    \begin{enumerate}[label=\roman*)]
      \item Si $n\in \N$, notemos que \begin{align*}
        \left(\frac{b}{a}\right)^{-n} &= \frac{1}{\left(\frac{b}{a}\right)^n} && \text{Definición}\\
        &= \frac{1}{\frac{b^n}{a^n}} && \text{(d) de LE9}\\
        &= \frac{a^n}{b^n} && \text{Regla del sandwich}\\
        &= \left(\frac{a}{b}\right)^n && \text{(d) de LE9}
      \end{align*}
      \item Si $n=0$, notemos que $\frac{a^n}{b^n} = \frac{a^0}{b^0} = \frac{1}{1} = 1 = \left(\frac{b}{a}\right)^0 = \left(\frac{b}{a}\right)^n$.
      \item Si $n<0$, notemos que \begin{align*}
        \frac{a^n}{b^n} &= a^n \cdot \frac{1}{b^n}\\
        &= \frac{1}{a^{-n}} \cdot \frac{1}{\frac{1}{b^{-n}}} && \text{Definición}\\
        &= \frac{1}{a^{-n}\cdot \frac{1}{b^{-n}}}\\
        &= \frac{1}{\frac{a^{-n}}{b^{-n}}}\\
        &= \frac{b^{-n}}{a^{-n}} && \text{Regla del sandwich}\\
        &= \left(\frac{b}{a}\right)^{-n} && \text{(d) de LE9} \qedhere
      \end{align*}
    \end{enumerate}
  \end{proof}

  \item $\frac{a^n}{b^n} = \left(\frac{a}{b}\right)^n, \forall n\in \Z$.
  \begin{proof}\leavevmode
    \begin{enumerate}[label=\roman*)]
      \item Ya probamos que la proposición es verdadera para $n\in \N$.
      \item Si $n=0$, tenemos que $\frac{a^n}{b^n} = \frac{a^0}{b^0} = \frac{1}{1} = 1 = \left(\frac{a}{b}\right)^0= \left(\frac{a}{b}\right)^n$.
      \item Sea $a\neq 0$ y $b\neq 0$, si $n<0$, sigue que \begin{align*}
        \frac{a^n}{b^n} &= a^n \cdot \frac{1}{b^n}\\
        &= \frac{1}{a^{-n}} \cdot \frac{1}{\frac{1}{b^{-n}}} && \text{Definición}\\
        &= \frac{1}{a^{-n}\cdot \frac{1}{b^{-n}}}\\
        &= \frac{1}{\frac{a^{-n}}{b^{-n}}} && \text{Con $-n\in \N$}\\
        &= \frac{1}{\left(\frac{a}{b}\right)^{-n}} && \text{(d) de LE9}\\
        &= \left(\frac{a}{b}\right)^n && \text{Definición} \qedhere
      \end{align*}
    \end{enumerate}
  \end{proof}

  \item $a^m\cdot a^n = a^{m+n}, \forall m\in \Z$.
  \begin{proof}\leavevmode
  \begin{enumerate}[label=\Roman*)]
    \item Sin pérdida de generalidad, si $n=0$, tenemos que $a^m\cdot a^0 = a^m \cdot 1 = a^m = a^{m+0}=a^{m+n}$.
    \item Ya probamos que la proposición es verdadera para $m,n\in \N$.
    \item Sea $a\neq 0$, y \begin{enumerate}[label=\roman*)]
      \item Sin pérdida de generalidad, si $n<0$ y $m>0$, tenemos $a^m\cdot a^n = \frac{a^m}{a^{-n}}$, donde $-n\in \N$; a su vez, \begin{itemize}
        \item Si $-n = m$, entonces \begin{align*}
          \frac{a^m}{a^{-n}} &= 1 \\
          &= a^0\\
          &= a^{m-m}\\
          &= a^{m-(-n)}\\
          &= a^{m+n}
        \end{align*}
        \item Si $-n\neq m$, por (e) de LE9, sigue que $\frac{a^m}{a^{-n}} = a^{m-(-n)} = a^{m+n}$.
      \end{itemize}
      \item Si $n<0$ y $m<0$, entonces \begin{align*}
        a^m\cdot a^n &= \frac{1}{a^{-m}} \cdot \frac{1}{a^{-m}}\\
        &= \frac{1}{a^{-m}a^{-n}}
      \end{align*}
      Como $-m, -n\in \N$, por (c) de LE9, \begin{align*}
        \frac{1}{a^{-m}a^{-n}} &= \frac{1}{a^{-m+(-n)}}\\
        &= \frac{1}{a^{-(m+n)}}\\
        &= a^{m+n} && \text{Definición} \qedhere
      \end{align*}
    \end{enumerate}
  \end{enumerate}
  \end{proof}

  \item $(ab)^m = a^mb^m, \forall m\in \Z$.
  \begin{proof}\leavevmode
    \begin{enumerate}
      \item Ya probamos que la proposición es verdadera para $m\in \N$.
      \item Si $m=0$, sigue que $(ab)^m = 1 = a^0b^0 = a^mb^m$.
      \item Si $m<0$, con $a\neq 0$ y $b\neq 0$, entonces \begin{align*}
        \bigl(ab\bigr)^m &= \frac{1}{(ab)^{-m}}\\
        &= \frac{1}{a^{-m}b^{-m}} && \text{(e) de LE9}\\
        &= a^mb^m && \text{Definición}
      \end{align*}
    \end{enumerate}
  \end{proof}

  \item $a^{mn} = (a^m)^n, \forall m,n \in \Z$.
  \begin{proof}\leavevmode
    \begin{enumerate}[label=\roman*)]
      \item Ya probamos que la proposición es verdadera para $m,n\in \N$.
      \item Notemos que \begin{itemize}
        \item Si $m=0$, sigue que $a^{mn} = a^0 = 1 = 1^n = (a^0)^n = (a^m)^n$.
        \item Si $n=0$, sigue que $a^{mn} = a^0 = 1 = (a^m)^0 = (a^m)^n$.
      \end{itemize}
      \item Finalmente, si $a\neq 0$ \begin{itemize}
        \item Si $n<0$ y $m\in \N$, sigue que \begin{align*}
          \bigl(a^m\bigr)^n &= \frac{1}{(a^m)^{-n}} && \text{Definición}\\
          &= \frac{1}{a^{-mn}} && \text{(f) de LE9}\\
          &= a^{mn} && \text{Definición}
        \end{align*}
        \item Si $m<0$ y $n\in \N$, sigue que \begin{align*}
          \bigl(a^m\bigr)^n &= \left(\frac{1}{a^{-m}}\right)^n && \text{Definición}\\
          &= \frac{1^n}{(a^{-m})^n} && \text{(d) de LE9}\\
          &= \frac{1}{a^{-mn}} && \text{(f) de LE9}\\
          &= a^{mn} && \text{Definición} && \qedhere
        \end{align*}
      \end{itemize}
    \end{enumerate}
  \end{proof}

  \item Sea $a<b$ y $n\in \Z$, encuentre las condiciones que deben cumplirse para que $a^n< b^n$ o $b^n< a^n$.
  \begin{enumerate}[label=\roman*)]
    \item Si $n=0$, no se cumple ninguna condición, pues $a^0=b^0$.
    \item Sea $n\in \N$,
    \begin{enumerate}[label=\roman*)]
      \item Si $n=1$,
      \begin{align*}
        a &< b && \text{Hip.}\\
        a^1 &< b^1
      \end{align*}
      \item Supongamos que $a^k<b^k$, con $k\in \N$.
      \item Notemos que
      \begin{align*}
        a^k &< b^k && \text{Hip. Ind.}\\
        a^k \cdot a &< b^k 
      \end{align*}
    \end{enumerate}
  \end{enumerate}
  \end{enumerate}

\pagebreak

\section*{Notación sigma}

\textbf{Definición:} Sea $a,b\in \R$ y $f: \R\to \R$, definimos a la sumatoria como sigue:

\[
    \sum_{{\color{teal}n}=a}^{b} f({\color{teal}n}) \defined \left\{
    \begin{array}{@{}l@{\thinspace}l}
    f(a) + \sum_{{\color{violet}n}=a+1}^{b} f({\color{violet}n}) &,  \ \text{si}  \ b\geq a\\
    0 &,  \ \text{si}  \ b<a.
    \end{array} \right. \]

Decimos que
\begin{itemize}
  \item $n$ es el índice,
  \item $a$ el límite inferior,
  \item $b$ el límite superior,
  \item $f(n)$ el elemento típico (o genérico)
\end{itemize}
de la sumatoria. También decimos que $n$ \textit{itera} desde $a$ hasta $b$.

\textbf{Ejemplos:}

\begin{enumerate}
  \item \begin{align*}
    \sum_{i=2}^{5} i^2 &= (2)^2 + \sum_{i=3}^{5} i^2\\
    &= 4 + (3)^2 + \sum_{i=4}^{5} i^2\\
    &= 4 + 9 + (4)^2 + \sum_{i={\color{magenta}5}}^{{\color{magenta}5}} i^2 && (*)\\
    &= 4 + 9 + 16 + ({\color{magenta}5})^2 + \sum_{i=6}^{5} i^2\\
    &= 4 + 9 + 16 + 25 + 0\\
    &= 54
  \end{align*}

  \item \begin{align*}
    \sum_{m=-3}^{-1} 2m  &= 2(-3) + \sum_{m=-2}^{-1} 2m\\
    &= -6 + 2(-2) + \sum_{m={\color{magenta}-1}}^{{\color{magenta}-1}} 2m  && (\dag)\\
    &= -6 + -4 + 2({\color{magenta}-1}) + \sum_{m=0}^{-1} 2m\\
    &= -6 + -4 + -2 + 0\\
    &= -12
  \end{align*}

  \item \begin{align*}
    \sum_{n=0}^{3} 2^n &= 2^0 + \sum_{n=1}^{3} 2^n\\
    &= 1 + 2^1 + \sum_{n=2}^{3} 2^n\\
    &= 1 + 2 + \sum_{n={\color{magenta}3}}^{{\color{magenta}3}} 2^n  && (\ddag)\\
    &= 1 + 2 + 2^{\color{magenta}3} + \sum_{n=4}^{3} 2^n\\
    &= 1 + 2 + 8 + 0\\
    &= 11
  \end{align*}

  \item \begin{align*}
    \sum_{j=0}^{-1} j &= 0
  \end{align*}

  \item \begin{align*}
    \sum_{n=1}^{3} \frac{k}{n+1} &= \frac{k}{(1)+1} + \sum_{n=2}^{3} \frac{k}{n+1}\\
    &= \frac{k}{2} + \frac{k}{(2)+1} + \sum_{n=3}^{3} \frac{k}{n+1}\\
    &= \frac{k}{2} + \frac{k}{3} + \frac{k}{(3)+1} + \sum_{n=4}^{3} \frac{k}{n+1}\\
    &= \frac{k}{2} + \frac{k}{3} + \frac{k}{4} + 0\\
    &= \frac{k}{2} + \frac{k}{3} + \frac{k}{4}\\
  \end{align*}
\end{enumerate}


  El lector notará que, en el caso en que los límites inferior y superior son iguales ($*, \dag, \ddag$), la imagen de la sumatoria es el elemento típico \textit{evaluado} en el índice, es decir,

  \textbf{Observación:} Si $a=b$, entonces
  \begin{align*}
    \sum_{n=a}^{b} f(n) &= f(n) && \text{(Índices iguales de la sumatoria)}
  \end{align*}

  \begin{proof}\leavevmode
    Sea $m\in \R$,
  \begin{align*}
    \sum_{{\color{magenta}n}=m}^{m} f({\color{magenta}n}) &= f(m={\color{magenta}n}) + \sum_{{\color{cyan}n}=m+1}^{m} f({\color{cyan}n}) && \text{Definición}\\
    &= f(n) + 0 && \text{Definición}\\
    &= f(n) && \qedhere
  \end{align*}
  \end{proof}

  \textbf{Nota:} En este caso, el índice \textit{itera} en un único valor.

  A partir de esto tenemos que:
  \[
    \sum_{n=a}^{b} f(n) = \left\{
    \begin{array}{@{}l@{\thinspace}l}
    f(a) + \sum_{n=a+1}^{b} f(n) &,  \ \text{si}  \ a<b.\\
    f(n) &,  \ \text{si}  \ a=b\\
    0 &,  \ \text{si}  \ a>b
    \end{array} \right. \]

    El lector notará también que la suma del primer termino hasta el ultimo es
    igual a la suma del ultimo hasta el primero, es decir,

    \textbf{Proposición:} Sea $(b-a)\in \N$ arbitrario pero fijo, entonces
    \begin{align*}
      \sum_{n=a}^{b} f(n) = f(b) + \sum_{n=a}^{b-1} f(n) && \text{(Sumatoria inversa)}
    \end{align*}

    \begin{proof}\leavevmode
      \begin{enumerate}[label=\Roman*)]
%        \item Si $b-a\in \set{0}$,
%        \begin{align*}
%          \sum_{n=a}^{b} f(n) &= f(b) && \text{Definición}\\
%          &= f(b) + 0\\
%          &= f(b) + \sum_{n=a}^{a-1} f(n) && \text{$a-1<a$}\\
%          &= f(b) + \sum_{n=a}^{b-1} f(n)
%        \end{align*}
          \item Se verifica para $(b-a)=1$,
          \begin{align*}
            \sum_{n=a}^{a+1} f(n) &= f(a) + \sum_{n=a+1}^{a+1} f(n) && \text{Definición}\\
            &= f(a) + \sum_{n=b}^{b} f(n) && \text{Hipótesis}\\
            &= f(a) + f(b)\\
            &= f(b) + \sum_{n=a}^{a} f(n)\\
            &= f(b) + \sum_{n=a}^{b-1} f(n) && \text{$b-a=1 \Rightarrow b-1=a$}
          \end{align*}

          \item Supongamos que si $b-a=k$, entonces
          \begin{align*}
            \sum_{n=a}^{b} f(n) &= f(b) + \sum_{n=a}^{b-1} f(n)
          \end{align*}

          \item Notemos que si $b-a=k+1$, se tiene que
          \begin{align*}
            \sum_{n=a}^{a+k+1} f(n) &= f(a) + \sum_{n=a+1}^{a+k+1} f(n) && \text{Definición}\\
            &= f(a) + f(a+k+1) + \sum_{n=a+1}^{a+k} f(n) && \text{Hip. Ind.}\\
            &= f(a+k+1) + f(a) + \sum_{n=a+1}^{a+k} f(n)\\
            &= f(a+k+1) + \sum_{n=a}^{a+k} f(n) && \text{Definición}\\
            &= f(b) + \sum_{n=a}^{b-1} f(n)
          \end{align*}
        \end{enumerate}
    \end{proof}

    \textbf{Observación:} Sea $a,b\in \R$.
    \begin{itemize}
      \item Si $b=a$, entonces $b-a\in \set{0}$, y por índices iguales de la sumatoria se tiene que $\sum_{n=a}^{b} f(n) = f(n)$.
      \item Por definición, si $b<a$ se tiene que $\sum_{n=a}^{b} f(n) =0$, que en particular se verifica si $b-a\in \set{-n:n\in \N}$.
    \end{itemize}
    A partir de esta observación y de la Sumatoria Inversa se tiene que

    \textbf{Definición:} Si $(b-a)\in \Z$, entonces

    \[\sum_{n=a}^{b} f(n) \defined \left\{
    \begin{array}{@{}l@{\thinspace}l}
      0 &,  \ \text{si}  \ a>b\\
      f(n) &,  \ \text{si}  \ a=b\\
      f(b) + \sum_{n=a}^{b-1} f(n) &,  \ \text{si}  \ a<b.
    \end{array} \right. \]

    De este modo, siempre que la \textit{distancia} entre los límites de la sumatoria sea un número entero, contaremos con una definición alternativa para la sumatoria. Dado que contamos con una definición que puede ser planteada de dos maneras, podemos utilizar cualquiera (de las dos) a conveniencia; por ejemplo:
\begin{center}
  \begin{minipage}[c]{.5\linewidth}
    \begin{align*}
      \sum_{n=-1}^{1} n^3 &= (-1)^3 + \sum_{n=0}^{1} n^3\\
      &= -1 + 0^3 + \sum_{n=1}^1 n^3\\
      &= -1 + 0 + 1^3\\
      &= 0
    \end{align*}
   \end{minipage}%
  \begin{minipage}[c]{.5\linewidth}
    \begin{align*}
      \sum_{n=-1}^{1} n^3 &= 1^3 + \sum_{n=-1}^{0} n^3\\
      &= 1 + 0^3 + \sum_{n=-1}^{-1} n^3\\
      &= 1 + 0 + (-1)^3\\
      &= 0
    \end{align*}
  \end{minipage}
  \end{center}

\subsection*{Lista de Ejercicios 11 (LE11)}

Sea $a,b,p,q,s,t\in \Z$, demuestre lo siguiente:

\begin{enumerate}[label=\alph*)]
  \item \[\sum_{n=p}^{q} g(n) + \sum_{n=s}^{t} h(n) = \sum_{n=s}^{t} h(n) + \sum_{n=p}^{q} g(n)\qquad \text{(Conmutatividad de la sumatoria)}\]
  \begin{proof}\leavevmode
    \begin{enumerate}[label=\Roman*.]
      \item Primero probaremos que $\Bigl(\sum_{n=a}^{b} f(n)\Bigr)\in \R,\forall n\in \Z$, es decir, que la imagen de la sumatoria siempre es un número real; la motivación es que, al estar definida \textit{recursivamente}, la función podría parecer asignar números reales a funciones, pero este no es el caso.
      
      Por definición, si $a>b$, entonces, $\Bigl(\sum_{n=a}^{b} f(n)\Bigr)=0\in \R$; si $a=b$, entonces $\Bigl(\sum_{n=a}^{b} f(n)\Bigr)=f(n)\in \R$. Para el caso $a<b$ procedemos por inducción:
      \begin{enumerate}[label=\roman*)]
        \item Se verifica para $b=a+1$,
        \begin{align*}
          \sum_{n=a}^{b} f(n) &= \sum_{n=a}^{a+1} f(n)\\
          &= f(n+1) + \sum_{n=a}^{a} f(n)\\
          &= f(n+1) + f(n)
        \end{align*}
        Como $f(n+1)\in \R$ y $f(n)\in \R$ y la suma es cerrada en $\R$ se tiene que $\Bigl(f(n+1) + f(n)\Bigr)\in \R$, osea, $\Bigl(\sum_{n=a}^{b} f(n)\Bigr)\in \R$.
        \item Supongamos que $\Bigl(\sum_{n=a}^{b} f(n)\Bigr)\in \R$, con $b=a+k$, para algún $k\in \N$.
      \item Si $b=a+k+1$,
      \begin{align*}
        \sum_{n=a}^{b} f(n) &= \sum_{n=a}^{a+k+1} f(n)\\
        &= f(n+k+1) + \sum_{n=a}^{a+k} f(n)\\
        &= f(n+k+1) + \sum_{n=a}^{a+k} f(n)
      \end{align*}
      Como $f(n+k+1)\in \R$ y $\sum_{n=a}^{a+k} f(n)\in \R$ (hip. ind.), se tiene que $\Bigl(f(n+k+1) + \sum_{n=a}^{a+k} f(n)\Bigr)\in \R$, es decir, $\Bigl(\sum_{n=a}^{b} f(n)\Bigr)\in \R$.
      \end{enumerate}
      En cualquier caso $\Bigl(\sum_{n=a}^{b} f(n)\Bigr)\in \R$.
    

    \item Finalmente demostramos la Conmutatividad de la sumatoria.
    
    Como $\Bigl(\sum_{n=p}^{q} g(n)\Bigr)\in \R$ y $\Bigl(\sum_{n=s}^{t} h(n)\Bigr)\in \R$, por conmutatividad de la suma en $\R$, sigue que \[\sum_{n=p}^{q} g(n) + \sum_{n=s}^{t} h(n) = \sum_{n=s}^{t} h(n) + \sum_{n=p}^{q} g(n)\]
    \end{enumerate}
  \end{proof}

  \item \[\sum_{n=a}^{b} f(n) + \sum_{n=a}^{b} g(n) = \sum_{n=a}^{b} \Bigl(f(n) + g(n)\Bigr) \qquad \text{(Asociatividad de la sumatoria)}\]
  \begin{proof}\leavevmode
    \begin{enumerate}[label=\Roman*)]
      \item Si $a>b$,
      \begin{align*}
        \sum_{n=a}^{b} f(n) + \sum_{n=a}^{b} g(n) = 0 = \sum_{n=a}^{b} \Bigl(f(n) + g(n)\Bigr)
      \end{align*}
      \item Si $a=b$,
      \begin{align*}
        \sum_{n=a}^{b} f(n) + \sum_{n=a}^{b} g(n) &= f(n) + g(n) = \sum_{n=a}^{b} \Bigl(f(n) + g(n)\Bigr)
      \end{align*}
      \item Si $b>a$,
      \begin{enumerate}[label=\roman*)]
        \item Se comprueba para $b=a+1$,
        \begin{align*}
          \sum_{n=a}^{a+1} \Bigl(f(n) + g(n)\Bigr) &= \Bigl(f(a) + g(a)\Bigr) + \sum_{n=a+1}^{a+1} \Bigl(f(n) + g(n)\Bigr)\\
          &= f(a) + g(a) + \Bigl(f(a+1) + g(a+1)\Bigr)\\
          &= \Bigl(f(a) + f(a+1)\Bigr) + \Bigl(g(a) + g(a+1)\Bigr) && \text{Asociatividad (de la suma)}\\
          &= \Biggl(f(a) + \sum_{n=a+1}^{a+1} f(n)\Biggr) + \Biggl(g(a) + \sum_{n=a+1}^{a+1} g(n)\Biggr)\\
          &= \sum_{n=a}^{a+1} f(n) + \sum_{n=a}^{a+1} g(n)
        \end{align*}
        \item Supongamos que se verifica para $b=a+k$, con $k\in \N$, es decir, suponemos que
        \begin{align*}
          \sum_{n=a}^{a+k} \Bigl(f(n) + g(n)\Bigr) &= \sum_{n=a}^{a+k} f(n) + \sum_{n=a}^{a+k} g(n)
        \end{align*}
        \item Notemos que
        \begin{align*}
          \sum_{n=a}^{a+k+1} \Bigl(f(n) + g(n)\Bigr) &= \Bigl(f(a+k+1) + g(a+k+1)\Bigr) + \sum_{n=a}^{a+k} \Bigl(f(n) + g(n)\Bigr)\\
          &= f(a+k+1) + g(a+k+1) + \sum_{n=a}^{a+k} f(n) + \sum_{n=a}^{a+k} g(n) && \text{Hip. Inducción}\\
          &= \Biggl(f(a+k+1) + \sum_{n=a}^{a+k} f(n)\Biggr) + \Biggl(g(a+k+1) + \sum_{n=a}^{a+k} g(n)\Biggr)\\
          &= \sum_{n=a}^{a+k+1} f(n) + \sum_{n=a}^{a+k+1} g(n)
        \end{align*}
      \end{enumerate}
    \end{enumerate}
  \end{proof}


  \item Sea $c\in \R$, \[c\cdot \sum_{n=a}^{b} f(n) = \sum_{n=a}^{b} \Bigl(c \cdot f(n)\Bigr) \qquad \text{(Distributividad de la sumatoria)}\]
    \begin{proof}\leavevmode
      \begin{enumerate}[label=\Roman*)]
        \item Si $a>b$, \begin{align*}
          c\cdot \sum_{n=a}^{b} f(n) = c\cdot 0 = \sum_{n=a}^{b} \Bigl(c \cdot f(n)\Bigr)
        \end{align*}
        \item Si $a=b$, \begin{align*}
          c\cdot \sum_{n=a}^{b} f(n) = c\cdot f(n) = \sum_{n=a}^{b} \Bigl(c \cdot f(n)\Bigr) 
        \end{align*}
        \item Si $b>a$,
          \begin{enumerate}[label=\roman*)]
          \item Se comprueba para $b=a+1$, \begin{align*}
            \sum_{n=a}^{a+1} \Bigl(c \cdot f(n)\Bigr) &= c\cdot f(a) + \sum_{n=a+1}^{a+1} c\cdot f(n)\\
            &= c\cdot f(a) + c\cdot f(a+1)\\
            &= c \cdot \Bigl(f(a) + f(a+1)\Bigr)\\
            &= c\cdot \Biggl(f(a) + \sum_{n=a+1}^{a+1} f(n)\Biggr)\\
            &= c\cdot \sum_{n=a}^{a+1} f(n)
          \end{align*}
          \item Supongamos que se verifica para $b=a+k$, con $k\in \N$, es decir, suponemos que \begin{align*}
            \sum_{n=a}^{a+k} \Bigl(c \cdot f(n)\Bigr) = c\cdot \sum_{n=a}^{a+k} f(n) 
          \end{align*}
          \item Notemos que \begin{align*}
            \sum_{n=a}^{a+k+1} \Bigl(c \cdot f(n)\Bigr) &= c\cdot f(a+k+1) + \sum_{n=a}^{a+k} \Bigl(c \cdot f(n)\Bigr)\\
            &= c\cdot f(a+k+1) + c\cdot \sum_{n=a}^{a+k} f(n) && \text{Hip. Inducción}\\
            &= c \cdot \Biggl(f(a+k+1)+ \sum_{n=a}^{a+k} f(n)\Biggr)\\
            &= c\cdot \sum_{n=a}^{a+k+1} f(n)
          \end{align*}
          \end{enumerate}
        \end{enumerate}
    \end{proof}
  
    \textbf{Corolario:} Sea $s,t\in \R$ \begin{enumerate}[label=\roman*)]  
      \item \[s\cdot \sum_{n=a}^{b} f(n) + t\cdot \sum_{n=a}^{b} g(n) = \sum_{n=a}^{b} \Bigl(s\cdot f(n) + t\cdot g(n)\Bigr)\]
      \begin{proof}
        \begin{align*}
          s\cdot \sum_{n=a}^{b} f(n) + t\cdot \sum_{n=a}^{b} g(n) &= \sum_{n=a}^{b} \Bigl(s\cdot f(n)\Bigr) + \sum_{n=a}^{b} \Bigl(t\cdot g(n)\Bigr) && \text{Distributividad de la sumatoria}\\
          &= \sum_{n=a}^{b} \Bigl(s\cdot f(n) + t\cdot g(n)\Bigr) && \text{Asociatividad} \qedhere
        \end{align*}
      \end{proof}
      \item \[\sum_{n=a}^{b} f(n) - \sum_{n=a}^{b} g(n) = \sum_{n=a}^{b} \Bigl(f(n) - g(n)\Bigr)\]
      \begin{proof}
        \begin{align*}
          \sum_{n=a}^{b} f(n) - \sum_{n=a}^{b} g(n) &= \sum_{n=a}^{b} f(n) + (-1)\sum_{n=a}^{b} g(n)\\
          &= \sum_{n=a}^{b} \Bigl(f(n) + (-1)\cdot g(n)\Bigr) && \text{Por (i) de este corolario}\\
          &= \sum_{n=a}^{b} \Bigl(f(n) - g(n)\Bigr)
        \end{align*}
      \end{proof}
    \end{enumerate}

  \item \[\sum_{n=a}^{b} \left(\sum_{m=s}^{t} f(n,m)\right) = \sum_{m=s}^{t} \left(\sum_{n=a}^{b} f(n,m)\right)\]
  
  \begin{enumerate}[label=\Roman*)]
    \item Si 
  \end{enumerate}
  
  \item \[\sum_{n=a}^{b} \left(\sum_{m=s}^{t} \Bigl(f(n) \cdot  g(m)\Bigr)\right) = \left(\sum_{n=a}^{b} f(n)\right)  \cdot \left(\sum_{m=s}^{t} g(m)\right)\]

  \item \[\sum_{n=a}^{b} \Bigl(f(n) \cdot  g(n)\Bigr) \neq \left(\sum_{n=a}^{b} f(n)\right)  \cdot \left(\sum_{n=a}^{b} g(n)\right)\]
  


  \item Sea $c\in \R$, si $a\leq b$, entonces \begin{align*}
    \sum_{n=a}^b c &= (b-a+1)c
  \end{align*}
  \begin{proof}\leavevmode
    \begin{enumerate}[label=\Roman*)]
      \item Se comprueba para $a=b$, \begin{align*}
        \sum_{n=a}^b c = c = 1\cdot c = (b-a+1)\cdot c
      \end{align*}
      \item Si $a<b$ se tiene que \begin{enumerate}[label=\roman*)]
        \item Se verifica para $b=a+1$, \begin{align*}
          \sum_{n=a}^{a+1} c &= c + \sum_{n=a+1}^{a+1} c = c + c = 2c = (2+a-a)c = (1+1+a-a)c = \Bigl((a+1)-a+1\Bigr) c
        \end{align*}
        \item Supongamos que se cumple para $b=a+k$, con $k\in \N$; es decir, suponemos que \begin{align*}
          \sum_{n=a}^{a+k} c &= \Bigl((a+k)-a+1\Bigr) c
        \end{align*}
        \item Notemos que \begin{align*}
          \sum_{n=a}^{a+k+1} c &= c + \sum_{n=a}^{a+k} c\\
          &= c + \Bigl((a+k)-a+1\Bigr) c && \text{Hip. Inducción}\\
          &= \Bigl(1 + \bigl((a+k)-a+1\bigr) \Bigr)c\\
          &= \Bigl((a+k+1)-a+1\Bigr)c
        \end{align*}
      \end{enumerate}
    \end{enumerate}
  \end{proof}

  \textbf{Nota:} En esta proposición se restringe que $a\leq b$, pues si $a>b$, se tiene que $\sum_{n=a}^{b} c = 0 \neq (b-a+1)c$; únicamente en el caso que $c=0$, se cumpliría la igualdad con $a>b$.

  \textbf{Definición:} $(b-a+1)$ es el número de \textit{iteraciones}, \textit{cíclos}, o \textit{sumandos} de la sumatoria $\sum_{n=a}^{b} f(n)$.

  \bfit{Corolario:} Si $c\in \R$ y $n\in \N$, entonces $\sum_{i=1}^n c = nc$.
  \begin{proof} $\sum_{i=1}^n c = \bigl((n-1)+1\bigr) c = nc$.
  \end{proof}

  \item Sea $\ell, m\in \R$ y $c\in \Z$, encuentre las condiciones que deben cumplirse para que
  
  \[\sum_{n=a}^b f(\ell+m\cdot n) = \sum_{n=a+c}^{b+c} f(\ell+m\cdot n-c)\]

  \begin{enumerate}[label=\Roman*)]
    \item Notemos que si $c=0$, la proposición es \textit{tautológica}; por lo que, en adelante, suponemos que $c\neq 0$.
    \item Si $a>b$, no importa qué valores tome $\ell$ o $m$, la proposición se verifica por definición:
    \begin{align*}
      \sum_{n=a}^b f(\ell+m\cdot n) = 0 = \sum_{n=a+c}^{b+c} f(\ell+m\cdot n-c)
    \end{align*}
    \item Si $a=b$ y $m=0$,
    \begin{align*}
      \sum_{n=a+c}^{b+c} f(\ell+m\cdot n-c) &= \sum_{n=a+c}^{a+c} f(\ell+m\cdot n-c)\\
      &= f\bigl(\ell+0\cdot (a+c)-c\bigr)\\
      &=f(\ell+0-c)\\
      &= f(\ell-c)\\
      &\neq f(\ell)\\
      &= f(\ell+0)\\
      &= f(\ell+0\cdot a)\\
      &= \sum_{n=a}^a f(\ell+m\cdot n)\\
      &= \sum_{n=a}^b f(\ell+m\cdot n)
    \end{align*}
    Por lo que descartamos este caso.

    \item Si $a=b$, $m\neq 0$ y $m\neq 1$,
    \begin{align*}
      \sum_{n=a+c}^{b+c} f(\ell+m\cdot n-c) &= \sum_{n=a+c}^{a+c} f(\ell+m\cdot n-c)\\
      &= f\bigl(\ell+m(a+c)-c\bigr)\\
      &= f\bigl(\ell+ma+mc-c\bigr)\\
      &\neq f\bigl(\ell+ma\bigr)\\
      &= \sum_{n=a}^a f(\ell+m\cdot n)\\
      &= \sum_{n=a}^b f(\ell+m\cdot n)
    \end{align*}
    Por lo que descartamos este caso.

    \item Si $a=b$ y $m=1$,
    \begin{align*}
      \sum_{n=a+c}^{b+c} f(\ell+m\cdot n-c) &= \sum_{n=a+c}^{a+c} f(\ell+m\cdot n-c)\\
      &= f\bigl(\ell+1\cdot (a+c)-c\bigr)\\
      &= f(\ell + a)\\
      &= f(\ell+1\cdot a)\\
      &= \sum_{n=a}^a f(\ell+m\cdot n)\\
      &= \sum_{n=a}^b f(\ell+m\cdot n)
    \end{align*}

    \item Si $a<b$ y $m=0$.
    
    Para $b=a+1$ se tiene
    \begin{align*}
      \sum_{n=a}^b f(\ell+m\cdot n) &=\sum_{n=a}^{a+1} f(\ell+m\cdot n)\\
      &= f\bigl(\ell+0\cdot (a+1)\bigr) + \sum_{n=a}^{a} f(\ell+0\cdot n)\\
      &= f\bigl(\ell+0\cdot (a+1)\bigr) + f(\ell+0\cdot a)\\
      &= f(\ell) + f(\ell)\\
      &= 2 f(\ell)\\
      &\neq 2 f(\ell - c) \\
      &= f(\ell-c) + f(\ell-c)\\
      &= f\bigl(\ell+0\cdot (a+c)-c\bigr) + f\Bigl(\ell +0\cdot (a+c+1)-c\Bigr)\\
      &= f\bigl(\ell+0\cdot (a+c)-c\bigr) + \sum_{n=a+c+1}^{a+c+1} f(\ell+m\cdot n-c)\\
      &= \sum_{n=a+c}^{(a+1)+c} f(\ell+m\cdot n-c)\\
      &= \sum_{n=a+c}^{b+c} f(\ell+m\cdot n-c)
    \end{align*}
    Por lo que descartamos este caso.

    \item Si $a<b$, $m\neq 0$ y $m\neq 1$,
    Para $b=a+1$ se tiene
    \begin{align*}
      \sum_{n=a}^b f(\ell+m\cdot n) &=\sum_{n=a}^{a+1} f(\ell+m\cdot n)\\
      &= f\bigl(\ell+m\cdot (a+1)\bigr) + \sum_{n=a}^{a} f(\ell+m\cdot n)\\
      &= f\bigl(\ell+m\cdot (a+1)\bigr) + f(\ell+m\cdot a)\\
      &= f(\ell+ma+m) + f(\ell+ma)\\
      &\neq f(\ell+ma+mc-c) + f(\ell+ma+mc+m-c)\\
      &= f\bigl(\ell+m\cdot (a+c)-c\bigr) + f\Bigl(\ell +m\cdot (a+c+1)-c\Bigr)\\
      &= f\bigl(\ell+m\cdot (a+c)-c\bigr) + \sum_{n=a+c+1}^{a+c+1} f(\ell+m\cdot n-c)\\
      &= \sum_{n=a+c}^{(a+1)+c} f(\ell+m\cdot n-c)\\
      &= \sum_{n=a+c}^{b+c} f(\ell+m\cdot n-c)
    \end{align*}
    Por lo que descartamos este caso.

    \item Si $a<b$, $m=1$,
    \begin{enumerate}[label=\roman*)]
      \item Si $b=a+1$,
      \begin{align*}
        \sum_{n=a}^b f(\ell+m\cdot n) &=\sum_{n=a}^{a+1} f(\ell+n)\\
        &= f\bigl(\ell+ (a+1)\bigr) + \sum_{n=a}^{a} f(\ell+ n)\\
        &= f(\ell+a+1) + f(\ell+a)\\
        &= f(\ell+a+1) + f(\ell +(a+c)-c)\\
        &= f\bigl(\ell+(a+1+c)-c\bigr) + \sum_{n=a+c}^{a+c} f(\ell+ n-c)\\
        &= \sum_{n=a+c}^{(a+1)+c} f(\ell+n-c)\\
        &= \sum_{n=a+c}^{b+c} f(\ell+m\cdot n-c)
      \end{align*}

      \item Supongamos que se verifica para $b=a+k$, con $k\in \N$; es decir, supenmos que
      \begin{align*}
        \sum_{n=a}^{a+k} f(\ell+n) &= \sum_{n=a+c}^{(a+k)+c} f(\ell+n-c)
      \end{align*}

      \item Notemos que
      \begin{align*}
        \sum_{n=a}^{a+k+1} f(\ell+n) &= f\Bigl(\ell+ (a+k+1)\Bigr) + \sum_{n=a}^{a} f(\ell+ n)\\
        &= f(\ell+a+k+1) + f(\ell+a)\\
        &= f(\ell+a+k+1) + f(\ell +(a+c)-c)\\
        &= f\bigl(\ell+(a+k+1+c)-c\bigr) + \sum_{n=a+c}^{a+c} f(\ell+ n-c)\\
        &= \sum_{n=a+c}^{(a+k+1)+c} f(\ell+n-c)
      \end{align*}
    \end{enumerate}


    Por lo que en general, planteamos la proposición como sigue:

    Si $c\in \Z$, $\ell \in \R$, entonces \[\sum_{n=a}^b f(\ell+n) = \sum_{n=a+c}^{b+c} f(\ell+n-c) \qquad \text{(Cambio de límites 1)}\]
    
  \end{enumerate}

  \item Sea $c\in \Z$, $m\in \R$, \[\sum_{n=a}^b f(m-n) = \sum_{n=a+c}^{b+c} f\bigl(m-(n-c)\bigr) \qquad \text{(Cambio de límites 2)}\]
  \begin{proof}\leavevmode
    \begin{enumerate}[label=\roman*)]
    \item Si $a>b$,
    \begin{align*}
      \sum_{n=a}^b f(m-n) = 0 = \sum_{n=a+c}^{b+c} f\bigl(m-(n-c)\bigr)
    \end{align*}

    \item Si $a=b$,
    \begin{align*}
      \sum_{n=a+c}^{b+c} f\bigl(m-(n-c)\bigr) &= \sum_{n=a+c}^{a+c} f\bigl(m-(n-c)\bigr)\\
      &= f\Bigl(m-\bigl((a+c)-c\bigr)\Bigr)\\
      &= f(m-a)\\
      &= \sum_{n=a}^a f(m-n)\\
      &= \sum_{n=a}^b f(m-n)
    \end{align*}

    \item Si $a<b$,
    \begin{enumerate}[label=\roman*)]
      \item Se verifica para $b=a+1$,
      \begin{align*}
        \sum_{n=a+c}^{(a+1)+c} f\bigl(m-(n-c)\bigr) &= f\Bigl(m-\bigl((a+c)-c\bigr)\Bigr) + \sum_{n=a+c+1}^{a+c+1} f\bigl(m-(n-c)\bigr)\\
        &= f(m-a) + f\Bigl(m-\bigl((a+c+1)-c\bigr)\Bigr)\\
        &= f(m-a) + f\bigl(m-(a+1)\bigr)\\
        &= f(m-a) + f(m-a-1)\\
        &= f(m-a) + f\bigl(m-(a+1)\bigr)\\
        &= f(m-a) + \sum_{n=a+1}^{a+1} f(m-n)\\
        &= \sum_{n=a}^{a+1} f(m-n)
      \end{align*}

      \item Supongamos qu ese verifica para $b=a+k$, con $k\in \N$; es decir, suponemos que
      \begin{align*}
        \sum_{n=a}^{a+k} f(m-n) = \sum_{n=a+c}^{(a+k)+c} f\bigl(m-(n-c)\bigr)
      \end{align*}

      \item Notemos que
      \begin{align*}
        \sum_{n=a+c}^{(a+k+1)+c} f\bigl(m-(n-c)\bigr) &= f\Bigl(m-\bigl((a+k+1+c)-c\bigr)\Bigr) + \sum_{n=a+c}^{a+k+c} f\bigl(m-(n-c)\bigr)\\
        &= f\bigl(m-(a+k+1)\bigr) + \sum_{n=a}^{a+k} f(m-n) && \text{Hip. Ind.}\\
        &= \sum_{n=a}^{a+k+1} f(m-n)
      \end{align*}
    \end{enumerate}
  \end{enumerate}
  \end{proof}

  \item Sea $m\in \R$ y $c\in \Z$, \[\sum_{n=a}^{b}f\bigl(m\pm n\bigr) = \sum_{n=a+c}^{b+c}f\bigl(m\pm (n- c)\bigr) \qquad \text{(Cambio de índice)}\]
  \begin{proof}\leavevmode
    \begin{enumerate}[label=\roman*)]
      \item Por el cambio de límites 1 se tiene que 
      \begin{align*}
        \sum_{n=a}^{b}f\bigl(m+ n\bigr) = \sum_{n=a+c}^{b+c}f\bigl(m+ (n- c)\bigr)
      \end{align*}
      \item Por el cambio de límites 2 se tiene que
      \begin{align*}
        \sum_{n=a}^{b}f\bigl(m-n\bigr) = \sum_{n=a+c}^{b+c}f\bigl(m-(n- c)\bigr)
      \end{align*}
    \end{enumerate}
  \end{proof}

  \textbf{Nota:} El lector encontrará que en ocasiones, en lugar de utilizar este teorema simplemente se trabaja con susbsituciones sobre el índice, por ejemplo:

  \begin{align*}
    \sum_{i=1}^{n} (i-1) &= \sum_{j=0}^{n-1} j && \text{donde $j=i-1$}
  \end{align*}

  \item Si $s\leq j\leq t$, \[\sum_{n=s}^{t}f(n) = \sum_{n=s}^{j}f(n) + \sum_{n=j+1}^{t}f(n) \qquad \text{(Partir la suma)}\]
  
  \textbf{Nota:} Alternativamente podemos escribir esta igualdad como sigue: Si $s\leq j\leq t$, entonces $\sum_{n=s}^{t}f(n) = \sum_{n=s}^{j-1}f(n) + \sum_{n=j}^{t}f(n)$. El lector debería verificar esta equivalencia

  \begin{proof}
    Consideremos los casos:
    \begin{enumerate}[label=\Roman*)]
      \item Si $s=j=t$,
      \begin{align*}
        \sum_{n=s}^{t}f(n) &= f(s)\\
        &= f(s) + 0\\
        &= \sum_{n=s}^{j}f(n) + \sum_{n=j+1}^{t}f(n)
      \end{align*}
      \item Si $s<j=t$,
      \begin{align*}
        \sum_{n=s}^{t}f(n) &= \sum_{n=s}^{j}f(n)\\
        &= \sum_{n=s}^{j}f(n) + 0\\
        &= \sum_{n=s}^{j}f(n) + \sum_{n=j+1}^{t}f(n)
      \end{align*}
      \item Si $s=j<t$,
      \begin{align*}
        \sum_{n=s}^{t}f(n) &= f(s) + \sum_{n=s+1}^{t}f(n)\\
        &= \sum_{n=s}^{j}f(n) + \sum_{n=j+1}^{t}f(n)
      \end{align*}
      \item Si $s<j<t$.
        Sea $j\in \Z$ arbitrario pero fijo,
        \begin{enumerate}[label=\roman*)]
          \item Si $t=j+1$,
          \begin{align*}
            \sum_{n=s}^{t}f(n) &=  \sum_{n=s}^{j+1}f(n)\\
            &= f(j+1) + \sum_{n=s}^{j}f(n)\\
            &= \sum_{n=s}^{j}f(n) + f(j+1) && \text{Conmutatividad}\\
            &= \sum_{n=s}^{j}f(n) + \sum_{n=j+1}^{t}f(n)
          \end{align*}
          \item Supongamos que $\sum_{n=s}^{j+k}f(n) = \sum_{n=s}^{j}f(n) + \sum_{n=j+1}^{j+k}f(n)$ para algún $k\in \N$.
          \item Si $t=j+k+1$, notemos que
          \begin{align*}
            \sum_{n=s}^{t}f(n) &= \sum_{n=s}^{j+k+1}f(n)\\
            &= f(j+k+1) + \sum_{n=s}^{j+k}f(n)\\
            &= f(j+k+1) + \sum_{n=s}^{j}f(n) + \sum_{n=j+1}^{j+k}f(n) && \text{Hip. Ind.}\\
            &= \sum_{n=s}^{j}f(n) + \sum_{n=j+1}^{j+k}f(n) +f(j+k+1)\\
            &= \sum_{n=s}^{j}f(n) + \sum_{n=j+1}^{j+k+1}f(n)\\
            &= \sum_{n=s}^{j}f(n) + \sum_{n=j+1}^{t}f(n)
          \end{align*}
        \end{enumerate}
    \end{enumerate}
  \end{proof}

  \textbf{Nota:} En esta proposición se restringe que $s\leq j\leq t$, pues la proposición no es válida para todo $s\geq j\geq t$:
  \begin{enumerate}[label=\Roman*)]
  \item Si $s>j>t$,
  \begin{align*}
    \sum_{n=s}^{t} f(n) = 0 = \sum_{n=s}^{j} f(n) + \sum_{n=j+1}^{t} f(n)
  \end{align*}

  \item Si $s>j=t$,
  \begin{align*}
    \sum_{n=s}^{t} f(n)= 0 = \sum_{n=s}^{j} f(n) + \sum_{n=j+1}^{t}
  \end{align*}

  \item Si $s=j>t$,
  \begin{enumerate}[label=\roman*)]
    \item $\sum_{n=s}^{j} f(n) + \sum_{n=j+1}^{t} f(n) = f(n) + 0 = f(n)$.
    \item $\sum_{n=s}^{t} f(n) = 0$.
    
    El lector notará que para partir la suma en este caso, debe cumplirse que $f(n)=0$, pero esto dependerá de cada función y de los íncides, por lo que en general, $f(n)\neq 0$, por ejemplo para cualquier sumatoria $\sum_{n=a}^{b} c$, donde $c\neq 0$.

  \end{enumerate}

  \bfit{Corolario:} \[\sum_{n=a}^{b}f(n) = \sum_{n=0}^{b} f(n) - \sum_{n=0}^{a-1} f(n)\]
  \begin{proof}\leavevmode
    \begin{align*}
      \sum_{n=0}^{b} f(n) - \sum_{n=0}^{a-1} f(n) &= \sum_{n=0}^{a} f(n) + \sum_{a+1}^{b} f(n) - \sum_{n=0}^{a-1}f(n) && \text{Partir la suma}\\
      &= \sum_{n=0}^{a-1} f(n) + \sum_{n=a}^{a} f(n) + \sum_{a+1}^{b} f(n) - \sum_{n=0}^{a-1}f(n) && \text{Partir la suma}\\
      &= \sum_{n=a}^{a} f(n) + \sum_{a+1}^{b} f(n)\\
      &= \sum_{n=a}^{b}f(n)
    \end{align*}
  \end{proof}

  \end{enumerate}

  \item \[\sum_{n=a}^{b}f(n) = \sum_{n=0}^{b-a} f(b-n)\]
  \begin{proof}\leavevmode
    \begin{enumerate}[label=\Roman*)]
    \item Si $b<a$, entonces $b-a<0$, por lo que
    \begin{align*}
      \sum_{n=0}^{b-a} f(b-n) = 0 = \sum_{n=a}^{b}f(n)
    \end{align*}
    
    \item Si $b=a$,
    \begin{align*}
      \sum_{n=0}^{b-a} f(b-n) &= \sum_{n=0}^{0} f(b-n)\\
      &= f(b-0)\\
      &= f(b)\\
      &= \sum_{n=a}^{b} f(n)
    \end{align*}
    

    \item Si $b>a$,
    \begin{enumerate}[label=\roman*)]
      \item Se verifica para $b=a+1$,
      \begin{align*}
        \sum_{n=0}^{b-a} f(b-n) &= \sum_{n=0}^{(a+1)-a} f(a+1-n)\\
        &= \sum_{n=0}^{1} f(a+1-n)\\
        &= f(a+1-0) + \sum_{n=1}^{1} f(a+1-n)\\
        &= f(a+1) + f(a+1-1)\\
        &= f(a+1) + f(a)\\
        &= f(a) + f(a+1)\\
        &= \sum_{n=a}^{a+1} f(n)\\
        &= \sum_{n=a}^{b} f(n)
      \end{align*}

      \item Supongamos que se verifica para $b=a+k$, con $k\in \N$, es decir, suponemos que
      \begin{align*}
        \sum_{n=a}^{a+k}f(n) = \sum_{n=0}^{(a+k)-a} f(a+k-n) = \sum_{n=0}^{k} f(a+k-n)
      \end{align*}

      \item Si $b=a+k+1$,
      \begin{align*}
        \sum_{n=a}^{a+k+1} f(n) &= f(a+k+1) + \sum_{n=a}^{a+k} f(n)\\
        &= f(a+k+1) + \sum_{n=0}^{k} f(a+k-n) && \text{Hip. Ind.}\\
        &= f\bigl(a+k-(-1)\bigr) + \sum_{n=0}^{k} f(a+k-n)\\
        &= \sum_{n=-1}^k f(a+k-n) && \text{Definición (de sumatoria)}\\
        &= \sum_{n=-1+(1)}^{k+(1)} f\bigl(a+k-(n-1)\bigr) && \text{Cambio de índice}\\
        &= \sum_{n=0}^{k+1} f(a+k+1-n)
      \end{align*}
      
    \end{enumerate}

    \end{enumerate}
  \end{proof}

  \bfit{Corolario:} \[\sum_{n=0}^{b}f(n) = \sum_{n=0}^{b} f(b-n)\]

  \begin{proof}\leavevmode
    La porposición se verifica por el teorema para $a=0$.
  \end{proof}

    
  \end{enumerate}

\subsection*{Una nota sobre la notación sigma}

\textbf{Abuso de la notación}

Considere el siguiente argumento planteado por un estudiante, y decida si está o no de acuerdo:

Para números reales $a$ y $b$ tales que $a<b$, hemos definido $\sum_{n=a}^{b} f(n) = f(a) + \sum_{n=a+1}^{b}f(n)$; el lector notará que del lado izquierdo de esta igualdad $n=a$ pero del lado derecho $n=a+1$, es decir, se emplea el mismo símbolo $n$ para números enteros distintos, lo que resulta útil en este caso ya que la definición de sumatoria está dada \textit{recursivamente}; sin embargo, de manera minuciosa (y pedante), se podría reescribir como $\sum_{n=a}^{b}f(n) = f(a) + \sum_{n+1=a+1}^{b}f(n+1)$, pero en cada iteración tendríamos que cambiar el índice (inferior) y, con ello, la variable a evaluar, lo que llegaría a ser laborioso, especialmente si se requiere de expresar múltiples iteraciones; por ello, se acude al abuso mencionado.

\textbf{Extensión}

Usualmente, el alcance de una suma se extiende hasta el primer símbolo de suma o resta que no está entre paréntesis o que no es parte de algún término más amplio (por ejemplo, en el numerador de una fracción), de manera que:

\begin{align*}
  \sum_{i=1}^{n} i^2 + 1 = \left(\sum_{i=1}^{n}i^2\right) + 1 = 1 + \sum_{i=1}^{n} i^2 \neq \sum_{i=1}^{n} \bigl(i^2+1\bigr)
\end{align*}

dado que esto puede resultar confuso, generalmente es más seguro encerrar el argumento de la sumatoria entre paréntesis (como en la segunda forma arriba) o mover los términos finales al principio (como en la tercera forma arriba). Una excepción (a la confusión) es cuando se suman dos sumas, como en

\begin{align*}
  \sum_{i=1}^{n} i^2 + \sum_{i=1}^{n^2} i = \left(\sum_{i=1}^{n}i^2\right) + \left(\sum_{i=1}^{n^2}i\right)
\end{align*}

\section*{Valor absoluto}

\bfit{Definición:}  Sea $a$ un número real, definimos el valor absoluto de $a$, denotado por $|a|$ como sigue:

 \[
  |a| = \left\{
 \begin{array}{@{}r@{\thinspace}l}
  a, &  \ \text{si}  \ a>0\\
  0, &  \ \text{si}  \ a=0\\
  -a, & \  \text{si} \  a<0
 \end{array} \right. \]
Notemos que $|a|\geq 0, \ \forall a\in \R$, y que la definición es equivalente a las siguientes:

\begin{center}
\begin{minipage}[c]{.3\linewidth}
 \[|a| = \left\{
  \begin{array}{@{}r@{\thinspace}l}
   a, & \ \text{si} \ a\geq 0\\
   -a, & \ \text{si} \ a<0
  \end{array} \right.\]
 \end{minipage}%
\begin{minipage}[c]{.3\linewidth}
 \[|a| = \left\{
  \begin{array}{@{}r@{\thinspace}l}
   a, & \ \text{si} \ a>0\\
   -a, & \ \text{si} \ a\leq 0
  \end{array} \right.\]
\end{minipage}
\end{center}

El lector devería verificar este hecho. (\textit{Hint}: $0=-0$).

\subsection*{Lista de Ejercicios 4 (LE4)}

Sean $a$, $b$, $c$ números reales, demuestre lo siguiente:

\begin{enumerate}[label=\alph*)]
\item $\pm a\leq |a|$.

\begin{proof}
 Por casos.
 \begin{enumerate}[label=\roman*)]
  \item Si $0 \leq a$, por definición, $|a|=a$, por lo que $a\leq |a|$. Luego, por la hipótesis tenemos que $-a \leq 0$, y por transitividad, $-a\leq |a|$.
  \item Si $a<0$, por definición, $|a|=-a$, por lo que $-a\leq |a|$. Luego, por la hipótesis tenemos que $0<-a$, y por transitividad, $a<|a|$.
 \end{enumerate}
 En cualquier caso, $\pm a\leq |a|$.
\end{proof} 

\item $|a|=|-a|$.
\begin{proof}
 Por casos.
 \begin{enumerate}[label=\roman*)]
  \item Si $0 \leq a$, por definición, $|a|=a$. Luego, por la hipótesis tenemos que $-a \leq 0$. Si $-a<0$, $|-a|=a$ y si $-a=0$, $|-a|=a$. De este modo, $|a|=|-a|$.
  \item Si $a<0$, por definición, $|a|=-a$. Luego, por la hipótesis tenemos que $0<-a$, por lo que $|-a|=-a$. De este modo, $|a|=|-a|$.
 \end{enumerate}
 En cualquier caso, $|a|=|-a|$.
\end{proof}

\item $|ab|=|a||b|$.

\begin{proof}
 Por casos.
 \begin{enumerate}[label=\roman*)]
  \item Si $a>0$ y $b>0$, por definición, $|a|=a$ y $|b|=b$. Luego, $ab>0$ por lo que $|ab|=ab$. Por tanto, $|ab| =|a||b|$.
  \item Si $a>0$ y $b<0$, por definición, $|a|=a$ y $|b|=-b$. Luego, $ab<0$ por lo que $|ab|=-ab$. Por tanto, $|  ab|=|a||b|$.
  \item Si $a<0$ y $b<0$, por definición, $|a|=-a$ y $|b|=-b$. Luego, $ab>0$ por lo que $|ab|=ab$. Por tanto, $|  ab|=|a||b|$.
 \end{enumerate}
 En cualquier caso, $|ab|=|a||b|$.
\end{proof}

\item $|a|^2=a^2$.
\begin{proof} 
 $0 \leq a^2 = |a^2|= |a\cdot a|=|a| \cdot |a|= |a|^2$. \qedhere
\end{proof}

\item $|a|<b$ si y solo si $-b<a<b$.

\begin{proof} \leavevmode
 \begin{itemize}
  \item[$\Rightarrow)$] Sea $|a|<b$.
  
  Sabemos que $\pm a \leq |a|$, y por transitividad $a<b$ y $-a<b$, por lo que $-b<a$. Por tanto, $-b<a<b$.
  \item[$\Leftarrow)$] Sea $-b<a<b$. Tenemos dos casos: \begin{enumerate}[label=\roman*)]
   \item Si $0\leq |a|$, por definición, $|a|=a$, y por la hipótesis, $|a|<b$.
   \item Si $a<0$, por definición, $|a|=-a$, y por la hipótesis, $|a|<b$.
  \end{enumerate} En cualquier caso, $|a|<b$. \qedhere 
 \end{itemize}
\end{proof}

\textbf{Nota:} Nos referiremos a esta proposición como teorema para eliminar el valor absoluto en algunas desigualdades.

 \item $|a+b|\leq |a|+|b|$. (Desigualdad del triángulo).

 \begin{proof} 
  Por casos.
  \begin{enumerate}[label=\roman*)]
   \item Si $0 \leq a+b$, por definición, $|a+b|=a+b$. Como, $a \leq |a|$ y $b \leq |b|$, entonces, $a+b \leq |a|+|b|$. Por tanto, $|a+b| \leq |a|+|b|$.
   \item Si $a+b<0$, por definición, $|a+b|=-\bigl(a+b\bigr)=-a-b$. Como, $-a \leq |a|$ y $-b \leq |b|$, entonces, $-a-b \leq |a|+|b|$. Por tanto, $|a+b| \leq |a|+|b|$. \qedhere
  \end{enumerate} 
 \end{proof}
%
 %\textbf{Nota:} En esta demostración se observaron los casos para la suma $a+b$. Sin embargo, resulta conveniente observar los casos para $a$ y $b$, pues esto nos permitirá notar cuando la desigualdad del triángulo es estricta y cuando se cumple con igualdad.   
 %\begin{enumerate}[label=\roman*)]
 % \item Supongamos que $|a+b|=|a|+|b|$, 
 %\end{enumerate}

 \item $\big| |a|-|b| \big| \leq |a-b|$. (Desigualdad del triángulo inversa).

 \begin{proof} \leavevmode
 \begin{center}\vspace{-2.5em}
 \begin{minipage}[t]{.5\linewidth}
 \begin{align*}
  |(b-a)+a| &\leq |b-a|+|a| && \text{Desg. del trig.} \\
  |b| &\leq |b-a|+|a| \\
  -|b-a| &\leq |a|-|b| \\
  -|a-b| &\leq |a|-|b| && \text{(*)}
 \end{align*}
 \end{minipage}%
 \begin{minipage}[t]{.5\linewidth}
 \begin{align*}
  |(a-b)+b| &\leq |a-b|+|b| && \text{Desg. del trig.} \\
  |a| &\leq |a-b|+|b| \\
  |a|-|b| &\leq |a-b| && \text{(**)}
 \end{align*}
 \end{minipage}
 \end{center}
 De las desigualdades (*) y (**) sigue que $\big| |a| - |b| \big| \leq |a-b|$.
 \end{proof}

 \bfit{Corolario:} $|a|-|b|\leq |a-b|$ y $|b|-|a|\leq |a-b|$.
 \begin{proof}
 Por la desigualdad del triángulo inversa, $\big| |a|-|b| \big| \leq |a-b|$, y notemos que $\pm \bigl(|a|-|b|\bigr)\leq \big| |a|-|b| \big|$, por transitividad sigue que $|a|-|b|\leq |a-b|$, también \begin{align*}
  -|a-b| &\leq |a|-|b|\\
  -\bigl(|a|-|b|\bigr) &\leq |a-b| \\
  |b|-|a| &\leq |a-b| && \qedhere
 \end{align*}
 \end{proof}

\item Si $b\neq 0$, entonces $\left| \frac{a}{b} \right| = \frac{|a|}{|b|}$.

\begin{proof} Por casos.
 \begin{enumerate}[label=\roman*)]
  \item Si $a \geq 0$ y $b>0$, entonces $|a|=a$ y $|b|=b$. Además, $\frac{1}{b} >0$, de donde sigue que $\frac{a}{b} \geq 0$ por lo que $\big| \frac{a}{b} \big| = \frac{a}{b}$. Por tanto, $ \big| \frac{a}{b} \big| = \frac{|a|}{|b|}$.
  \item Si $a \geq 0$ y $b<0$, entonces $|a|=a$ y $|b|=-b$. Además, $\frac{1}{b} <0$, de donde sigue que $\frac{a}{b} \leq 0$, por lo que $\big| \frac{a}{b} \big| =- \frac{a}{b}$. Por tanto, $ \big| \frac{a}{b} \big| = \frac{|a|}{|b|}$.
  \item Si $a<0$ y $b>0$, entonces $|a|=-a$ y $|b|=b$. Además, $\frac{1}{b} >0$, de donde sigue que $\frac{a}{b} < 0$, por lo que $\big| \frac{a}{b} \big| =- \frac{a}{b}$. Por tanto, $ \big| \frac{a}{b} \big| = \frac{|a|}{|b|}$.
  \item Si $a<0$ y $b<0$, entonces $|a|=-a$ y $|b|=-b$. Además, $\frac{1}{b} <0$, de donde sigue que $\frac{a}{b} > 0$ por lo que $\big| \frac{a}{b} \big| = \frac{a}{b}$. Por tanto, $ \big| \frac{a}{b} \big| = \frac{|a|}{|b|}$. \qedhere
 \end{enumerate} 
\end{proof}

\end{enumerate}

%\bfit{Definición:} \begin{itemize}
% \item Llamaremos al conjunto $\Z = \N \union \set{0} \union \set{-n | n\in \N}$ conjunto de los números enteros.
% \item Llamaremos al conjunto $\Z^- = \set{-n| n\in \N}$ conjunto de los números enteros negativos.
% \item Al conjunto $\N$ también lo llamaremos conjunto de los números enteros positivos y lo denotaremos con el símbolo $\Z^+$.
%\end{itemize}



\subsection*{Lista de Ejercicios 8 (LE8)}

Sean $a$ y $b$ números reales, demuestre lo siguiente:

\begin{enumerate}[label=\alph*)]
 \item $0 \leq a^{2n} \, \forall n\in \N$.
 \item Si $0\leq a$, entonces $ 0 \leq a^n, \, \forall n\in \N$.
 \item Si $0 \leq a <b$, entonces $a^n < b^n, \, \forall n\in \N$.
 \item Si $0 \leq a <b$, entonces $a^n \leq ab^n < b^n \, \forall n\in \N$.
 \item Si $0<a<1$, entonces $a^n<a \, \forall n\in \N$.
 \item Si $1<a$, entonces $a<a^n \, \forall n\in \N$.
\end{enumerate}

\subsubsection*{Demostración}

\begin{enumerate}[label=\alph*)]

 %A
 \item Pendiente

 %B
 \item Por inducción matemática. 

 \begin{enumerate}[label=\roman*)]
  \item Verificamos que se cumple para $n=1$. \begin{align*}
  0 &\leq a^1 \\
  0 &\leq a
  \end{align*}
  \item Suponemos que se cumple para $n=k$, para algún $k \in \N$. Es decir,  suponemos que \[0 \leq a^k\]
  \item Probaremos a partir de (ii) que $0 \leq a^{k+1}$. En efecto, por hipótesis de  inducción \begin{align*}
  0 &\leq a^k \\
  0 \cdot a &\leq a^k \cdot a \\
  0 &\leq a^{k+1}
  \end{align*}
 \end{enumerate}

 %C
 \item Por inducción matemática.
 \begin{enumerate}[label=\roman*)]
  \item Verificamos que se cumple para $n=1$. \begin{align*}
  a^1 &< b^1 \\
  a &< b
  \end{align*}
  \item Suponemos que se cumple para $n=k$, para algún $k\in \N$. Es decir, suponemos que \[a^k < b^k\]
  \item Probaremos, a partir de (ii) que $a^{k+1} < b^{k+1}$. En efecto, por (c) de LE5, garantizamos que $0 \leq a^k$, lo que nos permite, por (a) de LE5, afirmar que
  \begin{align*}
  a^k \cdot a &< b^k \cdot b \\
  a^{k+1} &< b^{k+1}
  \end{align*}
 \end{enumerate}

 %D
 \item Tenemos que $a<b$, como $0\leq a<b$, sigue que $0<b$, entonces $a\cdot b < b\cdot b$, osea $ab<b^2$. Luego, $a \cdot a \leq ab$. Finalmente, $a^2\leq ab < b^2$.
 
 %E
 \item Pendiente
 
 %F
 \item Pendiente

\end{enumerate}

\bfit{Definición:}  Sea $A$ un subconjunto no vacío de $\R$, decimos que $A$ está acotado:
\begin{itemize}
 \item superiormente si $\exists M\in \R$ tal que $a \leq M, \forall a\in A$. En este caso decimos que $M$ es cota superior de $A$.

 \item inferiormente si $\exists m\in \R$ tal que $m \leq a, \forall a\in A$. En este caso decimos que $m$ es cota inferior de $A$.

 \item si está acotado superior e inferiormente.
 %\item si $\exists n\in \R$ tal que $|a|\leq n,\forall a \in A$. En este caso decimos que $n$ es una cota de $A$.
\end{itemize}

\bfit{Definición:}  Sea $A\subset \R$ tal que $A$ es no vacío y está acotado superiormente, decimos que un número real $S$ es supremo de $A$ si $S$ satisface las siguientes condiciones:
\begin{itemize}
 \item $S$ es cota superior de $A$.
 \item Si $K$ es cota superior de $A$, entonces $S\leq K$.% ($S$ es la cota superior más pequeña de $A$).
\end{itemize}

En este caso escribimos $S=\sup(A)$.

\bfit{Definición:} . Sea $A$ un subconjunto no vacío del conjunto de los números reales, acotado inferiormente, decimos que un número real $L$ es ínfimo de $A$ si $L$ satisface las siguientes condiciones: \begin{itemize}
 \item $L$ es cota inferior de $A$.
 \item Si $K$ es cota inferior de $A$, entonces $K\leq L$, es decir, $L$ es la cota inferior más grande de $A$.
\end{itemize}

En este caso escribimos $M=\inf(A)$

\subsection*{Lista de ejercicios 8 (LE8)}

%Falso o verdadero: \begin{enumerate}[label=\arabic*.]
% \item Si $E$ es un subconjunto de $\R$ acotado superiormente, entonces $E$ es un conjunto acotado.
% Falso. Consideremos el conjunto $\R\backslash \R^+$, el cual es un subconjunto de $\R$, y es no vacío, pues $-1\in \R\backslash \R^+$. Además, $b\leq 0, \forall b\in \R\backslash\R^+$, por lo que el conjunto está acotado superiormente. Supongamos que el conjunto propuesto está acotado. Es decir, suponemos que $\exists m$ tal que $|b|\leq m, \forall b\in \R\backslash \R^+$. Por (f) de LE4, $-m \leq b$ y, por transitividad, $-m\leq 0$, de donde sigue que $-m-1\leq -1$, pero $-1<0$, entonces $-m-1<0$, lo que implica que $-m-1\in \R\backslash \R^+$, por lo que $|-m-1|\leq m$. Luego, notemos que $|-m-1|=-(-m-1)$, es decir, tenemos que $m+1\leq m$, pero de esto se concluye que $1\leq 0$, lo cual es una contradicción. Por tanto, aunque $\R\backslash \R^+$ está acotado superiormente, no está acotado.
% \item Si $E$ es un subconjunto acotado de $\R$, entonces $E$ está acotado superiormente e Inferiormente.
% Verdadero. Sea $E$ un subconjunto no vacío de $\R$. Si $E$ está acotado, entonces $\exists m$ tal que $|b|\leq m,\forall b \in E$. Por (f) de LE4, $-m\leq b \leq m$, por lo que el conjunto está acotado superiormente e inferiormente.
%\end{enumerate}
Demuestre lo siguiente:

\begin{enumerate}
 \item Sea $A$ un subconjunto no vacío de $\R$. $A$ está acotado si y solo si $A$ está acotado superior e inferiormente.
 
 \begin{proof} \leavevmode
  \begin{itemize}
   \item[$\Rightarrow)$] ads
   \item[$\Leftarrow)$] asdf
  \end{itemize}
 \end{proof}

 \item Sea $A$ un subconjunto no vacío de $\R$, si $A$ tiene supremo, este es único.
 
 \begin{proof} 
  Supongamos que $s_1$ y $s_2$ son supremos de $A$. Como $s_1$ es una cota superior de $A$ y $s_2$ es elemento supremo, entonces $s_2\leq s_1$. Similarmente, $s_1\leq s_2$. Por tanto, $s_1=s_2$. 
 \end{proof}

 \item Sea $A$ un subconjunto no vacío de $\R$, si $A$ tiene ínfimo, este es único.
 
 \begin{proof} 
  Supongamos que $m_1$ y $m_2$ son ínfimos de $A$. Como $m_1$ es una cota superior de $A$ y $m_2$ es elemento ínfimo, entonces $m_1\leq m_2$. Similarmente, $m_2\leq m_1$. Por tanto, $m_1=m_2$. 
 \end{proof}

 \item Una cota superior $M$ de un conjunto no vacío $S$ de $\R$ es el supremo de $S$ si y solo si para toda $\varepsilon>0$ existe $s_\varepsilon \in S$ tal que $M-\varepsilon<s_\varepsilon$.

 \begin{proof} 
  \begin{enumerate}[label=\roman*)]
   \item Sea $M$ una cota superior de $S$ tal que $\forall \epsilon>0, \exists s_{\epsilon}$ tal que $M-\epsilon<s_{\epsilon}$. Si $M$ no es el supremo de $S$, tendríamos que $\exists V$ tal que $s_@a \leq V < M$. Elegimos $\epsilon = M-V$, con lo que $V<s_{\epsilon}$, lo que contradice nuestra hipótesis. Por tanto, $M$ es el supremo de $S$.
   \item Sea $M$ el supremo de $S$ y $\epsilon>0$. Como $M<M+\epsilon$, entonces $M-\epsilon$ no es una cota superior de $S$, por lo que $\exists s_\epsilon$ tal que $s_\epsilon>M-\epsilon$. \qedhere
   \end{enumerate} 
 \end{proof}
\end{enumerate}

\subsection*{Axioma del supremo}

Todo subconjunto no vacío del conjunto de los números reales que sea acotado superiormente tiene supremo.

\textbf{\textit{Teorema.}} El conjunto de los números naturales no está acotado superiormente.

\textbf{Demostración:}

Supongamos que el conjunto de los números naturales está acotado superiormente. Entonces existe un número real $M$ tal que $n\leq M, \forall n\in \N$. Como el conjunto de los números naturales es no vacío, entonces, por el axioma del supremo, $\N$ tiene supremo.

Sea $L\coloneqq \sup{(\N)}$. Como $L-1$ no es cota superior de $\N$, ya que $L>L-1$ y $L$ es la cota superior más pequeña, existe un núero natural $n_0$ tal que $n_0>L-1$, lo cual implica que $n_0+1<L$, pero esto contradice la hipótesis	de que $L$ es supremo de $\N$. Por tanto, el conjunto de los números naturales no está acotado superiormente. \qed

\textbf{\textit{Teorema.}} Si $A\subseteq \R, A\neq \emptyset$ y $A$ está acotado inferiormente, entonces $A$ tiene ínfimo.

\textbf{Demostración:}

Sea $A\subseteq \R, A\neq \emptyset$ y $A$ está acotado inferiormente. El conjunto $-A \coloneqq \set{-a: a\in A}$ está acotado superiormente y, por el axioma del supremo, $-A$ tiene supremo. Sea $M\coloneqq \sup{(A)}$, entonces $M\geq -a, \forall -a\in -A$. Notemos que $-M\leq a, \forall a\in A$, esto es $-M$ es el ínfimo de $A$. \qed

\subsection*{Propiedad Arquimediana del conjunto de los números reales}

Para cada número real $x$ existe un número natural $n$ tal que $x<n$.

\textbf{Demostración:}

Supongamos que existe $x\in \R$ tal que $n\leq x, \forall n\in \N$. Notemos que $x$ es una cota superior de $\N$, pero esto contradice el teorema que establece que el conjunto de los números naturales no está acotado superiormente. Por tanto, se satisface la propiedad arquimediana del conjunto de los números reales. \qed

\subsection*{Lista de Ejercicios 9 (LE9)}

\begin{enumerate}[label=\alph*)]
 \item Si $S \coloneqq \set{\frac{1}{n}: n\in \N}$, entonces $\inf{S=0}$.
 \item Si $t>0$, entonces $\exists n\in \N$ tal que $0<\frac{1}{n}<t$.
 \item Si $y>0$, entonces $\exists n\in \N$ tal que $n-1\leq y< n$.
 \item Sea $x\in \R$, demuestre que $\exists! \, n\in \Z$ tal que $n\leq x<n+1$.
\end{enumerate}

\subsubsection*{Demostración}

\begin{enumerate}[label=\alph*)]
 \item Sabemos que $0<n^{-1}\leq 1, \forall n\in \N$, por lo que $S$ está acotado inferiormente por $0$; de esto sigue que $S$ tiene ínfimo. Sea $w\coloneqq \inf{S}$. Por definición, $\frac{1}{n}\geq w\geq 0, n\in \N$. Supongamos que $w>0$. Por la propiedad arquimediana $\exists n_0$ tal que $\frac{1}{w} < n_0$, de donde sigue que $w<\frac{1}{n_0}$ con $\frac{1}{n_0} \in S$, lo cual es una contradicción. Por tanto, $w=0$. \qed
 
 \item Por la propiedad arquimediana $\exists n$ tal que $\frac{1}{t}<n$. Como $n$ y $t$ son mayores que $0$, sigue que $0<\frac{1}{n}<t$. \qed
 
 \item Por la propiedad arquimediana, el conjunto $E\coloneqq \set{m\in \N: y<m}$ es no vacío. Además, por el principio del buen orden, $\exists n\in E$ tal que $n\leq m, \forall m\in E$. Notemos que $n-1<n$, por lo que $n-1\notin E$, lo que implica que $n-1\leq y<n$. \qed
 
 \item Definimos el conjunto $A\coloneqq \set{n\in \Z: x<n}$. Por la propiedad arquimediana $\exists n_0 \in \N$ tal que $x<n_0$, así $n_0\in A$, por lo que $A\neq \emptyset$. Sabemos también que $A$ está acotado inferiormente, de manera que $A$ tiene elemento mínimo. Sea $n$ el elemento mínimo de $A$. Notemos que $n-1<n$, de donde sigue que $n-1\leq x<n$. Luego, $n-1\in \Z$, al que definimos como $m=n-1$, por lo que $m\leq x<m+1$.
 
 Finalmente, supongamos que $\exists m, n\in \Z$ tales que $m\leq x<m+1$ y $n\leq x<n+1$. Si $m\neq n$, sin pérdida de generalidad, $m>n$. Por ello, \begin{align*}
  n < m &\leq x<n+1 \\
  n < m &<n+1 \\
  0 < m-n &<1
 \end{align*}  
 Lo que contradice la cerradura de la suma en $\Z$. Por tanto, $m=n$, es decir, el número entero que satisface $n\leq x<n+1$ es único. \qed
% 
 %\textbf{Demostración alternativa:}
%
 %Definimos el conjunto $A\coloneqq \set{n\in \Z: n\leq x}$. Por la propiedad arquimediana $\exists n_0\in \N$ tal que $n_0>x$. Observemos que \begin{enumerate}[label=\roman*)]
 %\item Si $x\geq 0$, $-x\leq 0$ y $-x<n_0$, de donde sigue que $-n_0<x$, por lo que $-n_0\in A$.
 %\item Si $x<0$, $x<n_0$, de donde sigue que $-n_0\leq x$, por lo que $-n_0\in A$.
 %\end{enumerate} Consecuentemente, $A$ es no vacío. También sabemos que $A$ está acotado superiormente, por el axioma del supremo, $A$ tiene supremo. Sea $m\coloneqq \sup(A)$. Por definición, $m\leq x$. Notemos que $m+1>m$. Luego, si $m+1\leq x$ tendríamos que $m+1\in A$ pero como $m$ es el supremo de $A$ seguiría que $m \geq m+1$, lo cual es una contradicción, entonces debe ser el caso que $m\leq x<m+1$.\qed
\end{enumerate}

\section*{Funciones}

\bfit{Definición:}  Sean $a$ y $b$ objetos cualesquiera, definimos la pareja ordenada $(a,b)$ como sigue: \[
 (a,b)\coloneqq \set{\set{a}, \set{a,b}}\]
Al objeto $a$ lo llamaremos primer componente de la pareja ordenada $(a,b)$ y al objeto $b$ lo llamaremos segundo componente de la pareja ordenada $(a,b)$.

\textbf{Teorema:} $(a,b)=(c,d)$ si y solo si $a=c$ y $b=d$.

\textbf{Demostración:} Pendiente



\section*{Entorno}

\bfit{Definición:}  Sea $a$, $b$ números reales, definimos el intervalo \begin{itemize}
 \item abierto, como $(a,b)\defined \set{x\in \R| a<x<b}$
 \item semicerrado-abierto, como $[a,b)\defined \set{x\in \R|a\leq x <b}$
 \item semiabierto-cerrado, como $(a,b]\defined \set{x\in \R|a<x\leq b}$
 \item cerrado, como $[a,b]\defined \set{x\in \R|a\leq x\leq b}$.
\end{itemize}

\textbf{Definición.} Sea $\ell \in \R$ y $\varepsilon>0$. El vecindario-$\varepsilon$ de $\ell$ es el conjunto $V_\varepsilon(\ell)\defined \{ x\in \R: |x-\ell|<\varepsilon\}$.

Notemos que por el teorema para eliminar valores absolutos en algunas desigualdades, \[|x-\ell| < \epsilon= -\epsilon < x-\ell < \epsilon= \ell-\epsilon < x < \ell+\epsilon\]
Por lo que el vecindario-$\epsilon$ de $\ell$ es equivalente al intervalo abierto: $(\ell-\epsilon, \ \ell+\epsilon)$.

\subsection*{Lista de Ejercicios 5 (LE5)}

Sean $a,b \in \R$. Demuestre lo siguiente:

\begin{enumerate}[label=\alph*)]
 \item Si $0 \leq a < \varepsilon$ para toda $\varepsilon > 0$, entonces $a=0$.
 
 \begin{proof} 
  Supongamos que $0<a$, sigue que $0<\frac{a}{2}<a$. En particular, $\epsilon=\frac{a}{2}$, entonces $\varepsilon<a$, pero esto contradice nuestra hipótesis de que $a< \varepsilon$ para toda $\varepsilon>0$. Por tanto, $a=0$. 
 \end{proof}

 \item Si $a \leq b + \varepsilon$ para toda $\varepsilon > 0$, entonces $a \leq b$.
 
 \begin{proof} 
  Sean $a$ y $b$ números reales tales que $a \leq b + \varepsilon$, $\forall \varepsilon > 0$. Supongamos que $a > b$. Luego, $a-b>0$. Notemos que $(a-b) \cdot \frac{1}{2} > 0 \cdot \frac{1}{2}$, es decir $\frac{(a-b)}{2} > 0$. Sea $\varepsilon = \frac{(a-b)}{2}$, sigue que $a=2\varepsilon+b$. Además, $2\varepsilon > \varepsilon$, de donde obtenemos $2 \varepsilon + b > \varepsilon + b$. De este modo, $a > b+\varepsilon$, pero esto contradice nuestra hipótesis. Por tanto, $a \leq b$. 
 \end{proof}

 \item Si $x\in V_\varepsilon(a)$ para toda $\varepsilon>0$, entonces $x=a$.
 \begin{proof} 
  Si $x\in V_\varepsilon(a)$ tenemos que $|x-a|<\varepsilon$. Además, $0\leq |x-a|$, por definición. Así, $0\leq |x-a|<\varepsilon$. Como esta desigualdad se cumple para toda $\varepsilon>0$, por (p) de LE3, sigue que $|x-a|=0$. De este modo, $|x-a|=x-a$ con $x-a=0$. Por tanto, $x=a$. 
 \end{proof}

 \item Sea $U:=\{x: 0<x<1\}$. Si $a\in U$, sea $\varepsilon$ el menor de los números $a$ y $1-a$. Demuestre que $V_\varepsilon(a) \subseteq U$.
 \begin{proof} \leavevmode
  \begin{enumerate}[label=\roman*)]
   \item Si $a>1-a$, tenememos $\varepsilon=1-a$. Sea $y\in V_\varepsilon(a)$, entonces $|y-a|<1-a$. De (f) de LE4 sigue que $a-1<y-a<1-a$ (*). Tomando el lado derecho de (*) obtenemos $y<1$. Luego, de la hipótesis sigue que $2a>1$, osea $2a-1>0$. Del lado izquierdo de la desigualdad (*), tenemos $2a-1<y$, por lo que $0<y$.
   \item Si $1-a>a$, tenemos $\varepsilon=a$. Sea $y\in V_\varepsilon(a)$, entonces $|y-a|<a$. De (f) de LE4 sigue que $-a<y-a<a$. Sumando $a$ en esta desigualdad obtenemos $0<y<2a$. Luego, de la hipótesis sigue que $1>2a$, entonces $0<y<1$.\end{enumerate}
   En cualquier caso, $0<y<1$, lo que implica que $V_\varepsilon(a) \subseteq U$. 
 \end{proof}

 \item Demuestre que si $a\neq b$, entonces existen $U_\varepsilon(a)$ y $V_\varepsilon(b)$ tales que $U\cap V =\emptyset$.
 \begin{proof} 
  Supongamos que para toda $U_\varepsilon(a)$ y $V_\varepsilon(b)$ se cumple que $U_\varepsilon(a) \cap V_\varepsilon(b) \neq \emptyset$. Entonces, existe $x$ tal que $x\in U_\varepsilon(a)$ y $x\in V_\varepsilon(b)$. Como en ambas vecindades tenemos $\epsilon>0$ arbitraria, por (a) de LE5, sigue que $x=a$ y $x=b$, pero esto contradice el supuesto de que $a\neq b$. Por tanto, deben existir $U_\varepsilon(a)$ y $V_\varepsilon(b)$ tales que $U\cap V =\emptyset$. 
 \end{proof}
\end{enumerate}

\section*{Sucesiones}

\bfit{Definición:}  Una sucesión es una función %$X: n\in \N \mapsto x_n \in \R$
\begin{align*}
 X: \ & \N \to \R \\
 \ &  n \mapsto x_n 
\end{align*}
%
Llamamos a $x_n$ el n-ésimo término. Otras etiquetas para la sucesión son $(x_n)$, $(x_n:n\in \N)$, que denotan orden y se diferencian del rango de la función $\{x_n:n\in \N\}\subseteq \R$.

\bfit{Definición:}  Una sucesión $(x_n)$ es convergente si $\exists \ell \in \R$ tal que para cada $\varepsilon>0$ existe un número natural $n_\varepsilon$ (que depende de $\varepsilon$) de modo que los términos $x_n$ con $n\geq n_\varepsilon$ satisfacen que $|x_n-\ell|<\varepsilon$.

Decimos que $(x_n)$ converge a $\ell \in \R$ y llamamos a $\ell$ el límite de la sucesión y escribimos $\lim (x_n) = \ell$.

%Notemos que por LE5(a) se cumple que $x_n=x$ con $n\geq m$. Esto es falso.
\bfit{Definición:}  Una sucesión es divergente si no es convergente.

\bfit{Definición:}  Una sucesión $(x_n)$ está acotada si $\exists M\in \R^+$ tal que $|x_n|\leq M, \forall n\in \N$.

\subsection*{Lista de Ejercicios 10 (LE10)}

Demuestre lo siguiente:

\begin{enumerate}[label=\alph*)]
 \item El límite de una sucesión convergente es único.
 \item Toda sucesión convergente está acotada.
\end{enumerate}

\subsubsection*{Demostración}

\begin{enumerate}[label=\alph*)]
 \item Sean $\ell$ y $\ell'$ límites de la sucesión $(x_n)$. Tenemos que $\forall \varepsilon>0$, existen $n',n'' \in \N$ tales que $|x_{n\geq n'}-\ell|<\varepsilon$ y $|x_{n\geq n''}-\ell'|<\varepsilon$. Sin pérdida de generalidad, si $n'<n''$, los términos $x_n$ con $n\geq n''>n'$ satisfacen que \begin{align*}
  |x_n-\ell| &<\varepsilon && \text{(1)}\\
  |x_n-\ell'| &<\varepsilon && \text{(2)}
 \end{align*}
 Por (c) de LE4, se cumple que $|x_n-\ell'|=|\ell'-x_n|$ y por esto, \begin{align*}
  |\ell'-x_n|<\varepsilon && \text{(3)}
 \end{align*}
 Tomando (1) y (3), por (d) de LE3, se verifica que \begin{align*}
  |\ell'-x_n| + |x_n-\ell| &< 2\varepsilon
 \end{align*}
 Y, por la desigualdad del triángulo, tenemos que \begin{align*}
  \big|(\ell'-x_n)+(x_n-\ell)\big| &\leq |\ell'-x_n| + |x_n-\ell|\\
  |\ell'-\ell| &\leq |\ell'-x_n| + |x_n-\ell|
 \end{align*}
 De este modo, $|\ell'-\ell| < 2\varepsilon$. Como esta desigualdad se cumple para todo $\varepsilon>0$, en particular se verifica para $\varepsilon=\nicefrac{\varepsilon_0}{2}$ con $\varepsilon_0>0$ arbitrario pero fijo, así obtenemos que \begin{align*}
  |\ell'-\ell| &< 2 \left(\frac{\varepsilon_0}{2}\right)\\
  |\ell'-\ell| &< \varepsilon_0
 \end{align*}
 Finalmente, como $\varepsilon_0$ es arbitrario, por (a) de LE5, sigue que $\ell'=\ell$. Por tanto, el límite de cada sucesión convergente es único. \qed
%
 \item Sea $(x_n)$ una sucesión convergente. Por definición, $\forall \epsilon>0, \exists n_\varepsilon \in \N$ tal que los términos $x_n$ con $n\geq n_\varepsilon$ satisfacen que \begin{align*}
  |x_n - \ell| &< \epsilon \\
  |x_n - \ell| + |\ell| &< \epsilon + |\ell|
 \end{align*}
 Luego, por la desigualdad del triángulo, \begin{align*}
  \big|(x_n-\ell)+\ell\big| &\leq |x_n-\ell| + |\ell|\\
  |x_n| &\leq |x_n-\ell| + |\ell|
 \end{align*}
 Por transitividad, $|x_n|< \epsilon + |\ell|$, lo que implica que $\{x_{n\geq n_\varepsilon}\}$ está cotado superiormente.
 
 Por otra parte, el conjunto de índices $n<n_\varepsilon$ está acotado, y por esto, $\{x_{n<n_\varepsilon}\}$ es finito, por lo que tiene cota superior. %proof https://math.stackexchange.com/questions/548806/a-finite-set-always-has-a-maximum-and-a-minimum
 
 Finalmente, el conjunto $\{x_{n<n_\varepsilon}\} \cup \{x_{n\geq n_\varepsilon}\}$ está acotado superiormente, y por tanto, $(x_n)$ está acotada. \qed
\end{enumerate}

\textbf{\textit{Teorema.}} Todo conjunto finito no vacío tiene elemento mínimo y elemento máximo, es decir, para todo conjunto finito $A\neq \emptyset$, $\exists m,M\in A$ tales que $m\leq a\leq M, \forall a\in A$.

\textbf{Demostración:} Sea $n\in \N$ y $A \coloneqq \{a_1, \dots, a_n\}$ no vacío.

Procedemos por inducción sobre el número de elementos de $A$. \begin{enumerate}[label=\roman*)]
 \item Si $n=1$, tenemos $A\coloneqq\{a_1\}$, por lo que $m=a_1$ y $M=a_1$ cumplen la condición requerida.
 \item Supongamos que la proposición se cumple para $n=k$.
 \item Si $n=k+1$, tenemos $A\coloneqq \{a_1, \dots, a_k, a_{k+1}\}$. Luego, por hipótesis de inducción, el conjunto \[A' \coloneqq A \setminus \{a_{k+1}\} = \{a_1, \dots, a_k\}\] tiene elemento mínimo y máximo, es decir, $\exists m',M'\in A'$ tales que $\forall a'\in A', m'\leq a' \leq M'$.
 
 Notemos que para cada $a\in A$ tenemos $a=a_{k+1}$ o $a\in A'$. Por tricotomía, $a_{k+1}$ cumple con alguno de los siguientes casos:
 \begin{enumerate}[label=\alph*)]
  \item Si $a_{k+1}<m'$, tenemos que $m=a_{k+1}<m'\leq a' \leq M'=M$.
  \item Si $m' \leq a_{k+1}\leq M'$, entonces $m=m'\leq a_{k+1} \leq M'=M$.
  \item Si $m'<a_{k+1}$, tenemos que $m=m'\leq a' \leq M'<a_{k+1}=M$.
 \end{enumerate}
 En cualquier caso $\exists m,M\in A$ tales que $m\leq a\leq M, \forall a\in A$. \qed
\end{enumerate}

\end{document}
