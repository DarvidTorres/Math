\documentclass[11pt]{article}

\usepackage[top=0.6in,bottom=0.6in,right=0.6in,left=0.6in]{geometry} % Margins of the page
\usepackage{amsfonts, amssymb, amsmath, amsthm} % math symbols
\usepackage{enumitem} % enumerate lists
\usepackage{mathtools}
%\usepackage[spanish]{babel}
\usepackage{braket} % symbols for sets
\usepackage{nicefrac} % one-line fractions with nice format

%Conjuntos de números
\newcommand{\N}{\mathbb{N}}
\newcommand{\Z}{\mathbb{Z}}
\newcommand{\Q}{\mathbb{Q}}
\newcommand{\R}{\mathbb{R}}

%Shorter comands
\let\epsilon\varepsilon
\let\oldemptyset\emptyset
\let\emptyset\varnothing
\let\set\Set

%bold all lists$
\setlist[enumerate]{font=\bfseries}

\setlength{\parindent}{0pt} %no indent for the document
\setlength{\parskip}{1em} %add space between paragraphs
\pagestyle{empty}

\begin{document}

\title{\vspace{-2cm}Cálculo diferencial e Integral I \\ Semestre 2023-1 \\ Grupo 4031}
\author{Problemas de: funciones \\ \text{Torres Brito David Israel}}
\date{\today}
\maketitle
\thispagestyle{empty}

1. Encuentre el dominio de las siguientes funciones: \begin{enumerate}[label=\roman*)]
 \item $f(x)=\sqrt{1-x^2}$
 
 \begin{align*}
  0 &\leq 1-x^2 \\
  x^2 &\leq 1\\
  \sqrt{x^2} = |x| &\leq 1= \sqrt{1}\\
 \end{align*} Tenemos dos casos: \begin{enumerate}[label=\alph*)]
  \item Si $x\geq 0$, entonces $|x|=x\leq 1$. Por lo que $dom(f)=[0, 1]$.
  \item Si $x<0$, entonces $|x|=-x \leq 1$, osea, $-1\leq x$. Por lo que $dom(f)=[-1,0]$.
 \end{enumerate} Por tanto, $dom(f)=[0, 1]\cup[-1, 0]$.

 \item $f(x)=\sqrt[3]{1+x}$
 
 $dom(f)=\R$.

 \item $f(x)=\sqrt{|1-x^2|}$
 
 como $|x|\geq 0, \forall x\in \R$, tenemos que $dom(f)=\R$.

 \item $f(x)=\frac{1}{1-x}+\frac{1}{x-2}$
 
 Tenemos que $1-x\neq 0 $ y $x-2\neq 0$. Entonces, $x\neq 1$ y $x\neq 2$. Por lo que $dom(f)=\R\backslash \set{1,2}$.

 \item $f(x)=\sqrt{1-\sqrt{x^2-1}}$
 
 Tenemos que $x^2-1\geq 0$ (a) y $\sqrt{x^2-1} \geq 1$ (b). \begin{enumerate}[label=\arabic*)]
  \item \begin{align*}
   0 &\leq x^2-1\\
   1 &\leq x^2\\
   \sqrt{1} = 1 &\leq |x|=\sqrt{x^2}
  \end{align*} Tenemos dos casos: \begin{enumerate}[label=\alph*)]
   \item Si $x\geq 0$, entonces $|x|=x\leq 1$. 
   \item Si $x<0$, entonces $|x|=-x \leq 1$, osea, $-1\leq x$.
  \end{enumerate}

  \item \begin{align*}
   1 &\leq \sqrt{x^2-1}\\
   1^2=1 &\leq |x^2-1|=\Bigl(\sqrt{x^2-1}\Bigr)^2
  \end{align*}Tenemos dos casos: \begin{enumerate}[label=\alph*)]
   \item Si $x^2-1\geq 0$, obtenemos los casos de (1). 
   \item Si $x^2-1<0$, entonces $|x^2-1|=-x^2+1 \geq 1$, osea, $0\geq x^2$, lo que es solo es válido cuando $x=0$.
  \end{enumerate}
 \end{enumerate}
 Por tanto, $dom(f)=[-1,1]$.

 \item $f(x)=\frac{x^2-1}{x+1}$
 
 Tenemos que $x+1\neq 0$, por lo que $x\neq -1$. Entonces $dom(f)=\R\backslash\set{-1}$.

\end{enumerate}

2. Si $f(x)=\nicefrac{1}{(1+x)}$, calcule las siguientes expresiones: \begin{enumerate}[label=\roman*)]
 \item $f\bigl(f(x)\bigr)$. \begin{align*}
  f\bigl(f(x)\bigr)&=f\biggl(\frac{1}{1+x}\biggr)\\
  &=\frac{1}{1+\frac{1}{1+x}}\\
  &=\frac{1}{\frac{1+x}{1+x}+\frac{1}{1+x}}\\
  &= \frac{1}{\frac{1+x+1}{1+x}}\\
  &= \frac{1}{\frac{2+x}{1+x}}\\
  &= \frac{1+x}{2+x}
 \end{align*}

 \item $f(\nicefrac{1}{x})$ \begin{align*}
  f\biggl(\frac{1}{x}\biggr) &= \frac{1}{1+\frac{1}{x}}\\
  &= \frac{1}{\frac{x}{x}+\frac{1}{x}}\\
  &= \frac{1}{\frac{1+x}{x}}\\
  &= \frac{x}{1+x}
 \end{align*}

 \item $\nicefrac{1}{f(x)}$.\begin{align*}
  \frac{1}{f(x)} &= \frac{1}{\frac{1}{1+x}}\\
  &=\frac{1+x}{1}\\
  &=1+x
 \end{align*}

 \item $f(cx)$. \begin{align*}
  \frac{1}{1+cx}
 \end{align*}

 \item $f(x+y)$ \begin{align*}
  \frac{1}{1+x+y}
 \end{align*}

 \item $f(x)+f(y)$ \begin{align*}
  \frac{1}{1+x} + \frac{1}{1+y} &= \frac{(1+y)+(1+x)}{(1+x)(1+y)}\\
  &= \frac{2+x+y}{1+x+y+xy}
 \end{align*}

\end{enumerate}

3. Sean $f,g$ y $h$ tres funciones. Demuestre o de un contraejemplo para determinar si las siguientes afirmaciones son verdaderas o falsas.
\begin{enumerate}[label=\roman*)]
  \item $f\circ (g+h)=f\circ g+ f\circ h$. La afirmación es falsa.
  
  Contraejemplo: sean $f(x)=\sqrt{x}, g(x)=x, h(x)=\frac{1}{x}$. Tenemos que 
  \begin{center}
  \noindent\begin{minipage}[r]{.5\linewidth}
  \begin{align*}
  f\circ (g+h) &= f(g+h)\\
  &=f\biggl(x+\frac{1}{x}\biggr)\\
  &=f\biggl(\frac{2x+1}{x}\biggr)\\
  &=\sqrt{\frac{2x+1}{x}}
  \end{align*}
  \end{minipage}%
  \begin{minipage}[l]{.5\linewidth}
  \begin{align*}
  f\circ g + f\circ h &= f(g(x)) + f(h(x)) \\
  &= f(x) + f\biggl(\frac{1}{x}\biggr)\\
  &= \sqrt{x}+\sqrt{\frac{1}{x}}\\
  \\
  \end{align*}
  \end{minipage}
  \end{center}

  Evaluando $x=1$ en ambas funciones tenemos que $f\circ (g+h) = \sqrt{3} \neq 2 = f\circ g + f\circ h$.

  \item $(g+h)\circ f = g \circ f + h \circ f$. La afirmación es verdadera, como consecuencia inmediata de la definición de composición de funciones.
  
  Por definición, $(g+h)\circ f$ implica evaluar elementos de la imágen de $f$ en la suma de $g+h$, lo que a su vez implica evaluar $g$ y $h$ en elementos de la imágen de $f$ simultáneamente y luego sumar $g(f(x))$ y $h(f(x))$, es decir $g\circ f + h\circ f$.
  
  \item $\frac{1}{f\circ g}=\frac{1}{f} \circ g$. La afirmación es verdadera, como consecuencia inmediata de la definición de composición de funciones.
  
  Por definición, $\frac{1}{f} \circ g$ implica evaluar elementos de la imágen de $g$ en $f$, conservando la forma $\frac{1}{f}$, es decir, $\frac{1}{f\circ g}$.

\pagebreak

  \item $\frac{1}{f\circ g}=f\circ \left(\frac{1}{g}\right)$. La afirmación es falsa.
  
  Contraejemplo: sean $f(x)=-x$ y $g(x)=\sqrt{x}$. Tenemos que 
  \begin{center}
  \noindent\begin{minipage}[r]{.5\linewidth}
  \begin{align*}
  \frac{1}{f\circ g} &= \frac{1}{f(g(x))}\\
  &= \frac{1}{f(-x)}\\
  &= \frac{1}{\sqrt{-x}}
  \end{align*}
  \end{minipage}%
  \begin{minipage}[l]{.5\linewidth}
  \begin{align*}
  f \circ \frac{1}{g} &= f \left(\frac{1}{g}\right)\\
  &= f\left(\frac{1}{\sqrt{x}}\right)\\
  &=-\frac{1}{\sqrt{x}}
  \end{align*}
  \end{minipage}
  \end{center}

  Evaluando $x=-1$ en ambas funciones tenemos que $\frac{1}{f\circ g} = 1 \neq -1 = f \circ \frac{1}{g}$.

\end{enumerate}

7. Pruebe que si $f$ es una función tal que para toda función $g$ se satisface que $(f\circ g)(x)=(g\circ f)(x)$ para toda $x\in \R$, entonces $f(x)=x$ para toda $x\in \R$.

\textbf{Demostración:} Por hipótesis, \begin{align*}
  (f\circ g)(x) &=(g\circ f)(x)\\
  f\bigl(g(x)\bigr) &= g\bigl(f(x)\bigr)
\end{align*} para toda función $g$, por lo que —en particular, debe ser cierto para una función constante $g(x)=c$, es decir, $f\bigl(g(x)\bigr) = f(c)$. De la hipótesis	sigue que $f(c)=g\bigl(f(x)\bigr)$, y como $g$ es constante, tenemos que $g\bigl(f(x)\bigr)=c$, por lo que $f(c)=c$. Como $g(x)$ puede ser cualquier constante, en particular $c=x$, para cualquier $x\in \R$. Por tanto, $f(x)=x, \forall x\in \R$. \qed

\end{document}