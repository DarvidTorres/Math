\part*{Notación sigma}

\textbf{Definición:} Sea $a,b\in \R$ y $f: \R\to \R$, definimos a la sumatoria como sigue:

\[
    \sum_{{\color{teal}n}=a}^{b} f({\color{teal}n}) \defined \left\{
    \begin{array}{@{}l@{\thinspace}l}
    f(a) + \sum_{{\color{violet}n}=a+1}^{b} f({\color{violet}n}) &,  \ \text{si}  \ b\geq a\\
    0 &,  \ \text{si}  \ b<a.
    \end{array} \right. \]

Decimos que
\begin{itemize}
  \item $n$ es el índice,
  \item $a$ el límite inferior,
  \item $b$ el límite superior,
  \item $f(n)$ el elemento típico (o genérico)
\end{itemize}
de la sumatoria. También decimos que $n$ \textit{itera} desde $a$ hasta $b$.

\textbf{Ejemplos:}

\begin{enumerate}
  \item \begin{align*}
    \sum_{i=2}^{5} i^2 &= (2)^2 + \sum_{i=3}^{5} i^2\\
    &= 4 + (3)^2 + \sum_{i=4}^{5} i^2\\
    &= 4 + 9 + (4)^2 + \sum_{i={\color{magenta}5}}^{{\color{magenta}5}} i^2 && (*)\\
    &= 4 + 9 + 16 + ({\color{magenta}5})^2 + \sum_{i=6}^{5} i^2\\
    &= 4 + 9 + 16 + 25 + 0\\
    &= 54
  \end{align*}

  \item \begin{align*}
    \sum_{m=-3}^{-1} 2m  &= 2(-3) + \sum_{m=-2}^{-1} 2m\\
    &= -6 + 2(-2) + \sum_{m={\color{magenta}-1}}^{{\color{magenta}-1}} 2m  && (\dag)\\
    &= -6 + -4 + 2({\color{magenta}-1}) + \sum_{m=0}^{-1} 2m\\
    &= -6 + -4 + -2 + 0\\
    &= -12
  \end{align*}

  \item \begin{align*}
    \sum_{n=0}^{3} 2^n &= 2^0 + \sum_{n=1}^{3} 2^n\\
    &= 1 + 2^1 + \sum_{n=2}^{3} 2^n\\
    &= 1 + 2 + \sum_{n={\color{magenta}3}}^{{\color{magenta}3}} 2^n  && (\ddag)\\
    &= 1 + 2 + 2^{\color{magenta}3} + \sum_{n=4}^{3} 2^n\\
    &= 1 + 2 + 8 + 0\\
    &= 11
  \end{align*}

  \item \begin{align*}
    \sum_{j=0}^{-1} j &= 0
  \end{align*}

  \item \begin{align*}
    \sum_{n=1}^{3} \frac{k}{n+1} &= \frac{k}{(1)+1} + \sum_{n=2}^{3} \frac{k}{n+1}\\
    &= \frac{k}{2} + \frac{k}{(2)+1} + \sum_{n=3}^{3} \frac{k}{n+1}\\
    &= \frac{k}{2} + \frac{k}{3} + \frac{k}{(3)+1} + \sum_{n=4}^{3} \frac{k}{n+1}\\
    &= \frac{k}{2} + \frac{k}{3} + \frac{k}{4} + 0\\
    &= \frac{k}{2} + \frac{k}{3} + \frac{k}{4}\\
  \end{align*}
\end{enumerate}


  El lector notará que, en el caso en que los límites inferior y superior son iguales ($*, \dag, \ddag$), la imagen de la sumatoria es el elemento típico \textit{evaluado} en el índice, es decir,

  \textbf{Observación:} Si $a=b$, entonces
  \begin{align*}
    \sum_{n=a}^{b} f(n) &= f(n) && \text{(Índices iguales de la sumatoria)}
  \end{align*}

  \begin{proof}\leavevmode
  \begin{align*}
    \sum_{n=a}^{b} f(n) &= \sum_{n=a}^{a} f(n) && \text{Hipótesis}\\
    &= f(a) + \sum_{n=a+1}^{a} f(n) && \text{Definición}\\
    &= f(a) + 0 && \text{Definición}\\
    &= f(n) && \qedhere
  \end{align*}
  \end{proof}

  \textbf{Nota:} En este caso, el índice \textit{itera} en un único valor.

  A partir de esto tenemos que:
  \[
    \sum_{n=a}^{b} f(n) = \left\{
    \begin{array}{@{}l@{\thinspace}l}
    f(a) + \sum_{n=a+1}^{b} f(n) &,  \ \text{si}  \ a<b.\\
    f(n) &,  \ \text{si}  \ a=b\\
    0 &,  \ \text{si}  \ a>b
    \end{array} \right. \]

    El lector notará también que la suma del primer termino hasta el ultimo es
    igual a la suma del ultimo hasta el primero, es decir,

    \textbf{Proposición:} Sea $(b-a)\in \N$ arbitrario pero fijo, entonces
    \begin{align*}
      \sum_{n=a}^{b} f(n) = f(b) + \sum_{n=a}^{b-1} f(n) && \text{(Sumatoria inversa)}
    \end{align*}

    \begin{proof}\leavevmode
      \begin{enumerate}[label=\Roman*)]
%        \item Si $b-a\in \set{0}$,
%        \begin{align*}
%          \sum_{n=a}^{b} f(n) &= f(b) && \text{Definición}\\
%          &= f(b) + 0\\
%          &= f(b) + \sum_{n=a}^{a-1} f(n) && \text{$a-1<a$}\\
%          &= f(b) + \sum_{n=a}^{b-1} f(n)
%        \end{align*}
          \item Se verifica para $(b-a)=1$,
          \begin{align*}
            \sum_{n=a}^{a+1} f(n) &= f(a) + \sum_{n=a+1}^{a+1} f(n) && \text{Definición}\\
            &= f(a) + \sum_{n=b}^{b} f(n) && \text{Hipótesis}\\
            &= f(a) + f(b)\\
            &= f(b) + \sum_{n=a}^{a} f(n)\\
            &= f(b) + \sum_{n=a}^{b-1} f(n) && \text{$b-a=1 \Rightarrow b-1=a$}
          \end{align*}

          \item Supongamos que si $b-a=k$, entonces
          \begin{align*}
            \sum_{n=a}^{b} f(n) &= f(b) + \sum_{n=a}^{b-1} f(n)
          \end{align*}

          \item Notemos que si $b-a=k+1$, se tiene que
          \begin{align*}
            \sum_{n=a}^{a+k+1} f(n) &= f(a) + \sum_{n=a+1}^{a+k+1} f(n) && \text{Definición}\\
            &= f(a) + f(a+k+1) + \sum_{n=a+1}^{a+k} f(n) && \text{Hip. Ind.}\\
            &= f(a+k+1) + f(a) + \sum_{n=a+1}^{a+k} f(n)\\
            &= f(a+k+1) + \sum_{n=a}^{a+k} f(n) && \text{Definición}\\
            &= f(b) + \sum_{n=a}^{b-1} f(n)
          \end{align*}
        \end{enumerate}
    \end{proof}

    \textbf{Observación:} Sea $a,b\in \R$.
    \begin{itemize}
      \item Si $b=a$, entonces $(b-a)=0$, y por índices iguales de la sumatoria se tiene que $\sum_{n=a}^{b} f(n) = f(n)$.
      \item Por definición, si $b<a$ se tiene que $\sum_{n=a}^{b} f(n) =0$, que en particular se verifica si $b-a\in \set{-n:n\in \N}$.
    \end{itemize}
    A partir de esta observación y de la Sumatoria Inversa se tiene que

    \textbf{Definición:} Si $(b-a)\in \Z$, entonces

    \[\sum_{n=a}^{b} f(n) \defined \left\{
    \begin{array}{@{}l@{\thinspace}l}
      0 &,  \ \text{si}  \ a>b\\
      f(n) &,  \ \text{si}  \ a=b\\
      f(b) + \sum_{n=a}^{b-1} f(n) &,  \ \text{si}  \ a<b.
    \end{array} \right. \]

    De este modo, siempre que la \textit{distancia} entre los límites de la sumatoria sea un número entero, contaremos con una definición alternativa para la sumatoria. Dado que contamos con una definición que puede ser planteada de dos maneras, podemos utilizar cualquiera (de las dos) a conveniencia; por ejemplo:
\begin{center}
  \begin{minipage}[c]{.5\linewidth}
    \begin{align*}
      \sum_{n=-1}^{1} n^3 &= (-1)^3 + \sum_{n=0}^{1} n^3\\
      &= -1 + 0^3 + \sum_{n=1}^1 n^3\\
      &= -1 + 0 + 1^3\\
      &= 0
    \end{align*}
   \end{minipage}%
  \begin{minipage}[c]{.5\linewidth}
    \begin{align*}
      \sum_{n=-1}^{1} n^3 &= 1^3 + \sum_{n=-1}^{0} n^3\\
      &= 1 + 0^3 + \sum_{n=-1}^{-1} n^3\\
      &= 1 + 0 + (-1)^3\\
      &= 0
    \end{align*}
  \end{minipage}
  \end{center}

\subsection*{Lista de Ejercicios 11 (LE11)}

Sea $a,b,p,q,s,t\in \Z$, demuestre lo siguiente:

\begin{enumerate}[label=\alph*)]
  \item \[\sum_{n=p}^{q} g(n) + \sum_{n=s}^{t} h(n) = \sum_{n=s}^{t} h(n) + \sum_{n=p}^{q} g(n)\qquad \text{(Conmutatividad de la sumatoria)}\]
  \begin{proof}\leavevmode
    \begin{enumerate}[label=\Roman*.]
      \item Primero probaremos que $\Bigl(\sum_{n=a}^{b} f(n)\Bigr)\in \R,\forall n\in \Z$, es decir, que la imagen de la sumatoria siempre es un número real; la motivación es que, al estar definida \textit{recursivamente}, la función podría parecer asignar números reales a funciones, pero este no es el caso.
      
      Por definición, si $a>b$, entonces, $\Bigl(\sum_{n=a}^{b} f(n)\Bigr)=0\in \R$; si $a=b$, entonces $\Bigl(\sum_{n=a}^{b} f(n)\Bigr)=f(n)\in \R$. Para el caso $a<b$ procedemos por inducción:
      \begin{enumerate}[label=\roman*)]
        \item Se verifica para $b=a+1$,
        \begin{align*}
          \sum_{n=a}^{b} f(n) &= \sum_{n=a}^{a+1} f(n)\\
          &= f(n+1) + \sum_{n=a}^{a} f(n)\\
          &= f(n+1) + f(n)
        \end{align*}
        Como $f(n+1)\in \R$ y $f(n)\in \R$ y la suma es cerrada en $\R$ se tiene que $\Bigl(f(n+1) + f(n)\Bigr)\in \R$, osea, $\Bigl(\sum_{n=a}^{b} f(n)\Bigr)\in \R$.
        \item Supongamos que $\Bigl(\sum_{n=a}^{b} f(n)\Bigr)\in \R$, con $b=a+k$, para algún $k\in \N$.
      \item Si $b=a+k+1$,
      \begin{align*}
        \sum_{n=a}^{b} f(n) &= \sum_{n=a}^{a+k+1} f(n)\\
        &= f(n+k+1) + \sum_{n=a}^{a+k} f(n)\\
        &= f(n+k+1) + \sum_{n=a}^{a+k} f(n)
      \end{align*}
      Como $f(n+k+1)\in \R$ y $\sum_{n=a}^{a+k} f(n)\in \R$ (hip. ind.), se tiene que $\Bigl(f(n+k+1) + \sum_{n=a}^{a+k} f(n)\Bigr)\in \R$, es decir, $\Bigl(\sum_{n=a}^{b} f(n)\Bigr)\in \R$.
      \end{enumerate}
      En cualquier caso $\Bigl(\sum_{n=a}^{b} f(n)\Bigr)\in \R$.
    

    \item Finalmente demostramos la Conmutatividad de la sumatoria.
    
    Como $\Bigl(\sum_{n=p}^{q} g(n)\Bigr)\in \R$ y $\Bigl(\sum_{n=s}^{t} h(n)\Bigr)\in \R$, por conmutatividad de la suma en $\R$, sigue que \[\sum_{n=p}^{q} g(n) + \sum_{n=s}^{t} h(n) = \sum_{n=s}^{t} h(n) + \sum_{n=p}^{q} g(n)\]
    \end{enumerate}
  \end{proof}

  \bfit{Corolario:} \[\sum_{n=p}^{q} g(n) \cdot \sum_{n=s}^{t} h(n) = \sum_{n=s}^{t} h(n) \cdot \sum_{n=p}^{q} g(n)\]
  \begin{proof}\leavevmode
    Como $\Bigl(\sum_{n=p}^{q} g(n)\Bigr)\in \R$ y $\Bigl(\sum_{n=s}^{t} h(n)\Bigr)\in \R$, la igualdad se verifica por la conmutatividad de la multiplicación en $\R$.
  \end{proof}

  \item \[\sum_{n=a}^{b} f(n) + \sum_{n=a}^{b} g(n) = \sum_{n=a}^{b} \Bigl(f(n) + g(n)\Bigr) \qquad \text{(Asociatividad de la sumatoria)}\]
  \begin{proof}\leavevmode
    \begin{enumerate}[label=\Roman*)]
      \item Si $a>b$,
      \begin{align*}
        \sum_{n=a}^{b} f(n) + \sum_{n=a}^{b} g(n) = 0 = \sum_{n=a}^{b} \Bigl(f(n) + g(n)\Bigr)
      \end{align*}
      \item Si $a=b$,
      \begin{align*}
        \sum_{n=a}^{b} f(n) + \sum_{n=a}^{b} g(n) &= f(n) + g(n) = \sum_{n=a}^{b} \Bigl(f(n) + g(n)\Bigr)
      \end{align*}
      \item Si $b>a$,
      \begin{enumerate}[label=\roman*)]
        \item Se comprueba para $b=a+1$,
        \begin{align*}
          \sum_{n=a}^{a+1} \Bigl(f(n) + g(n)\Bigr) &= \Bigl(f(a) + g(a)\Bigr) + \sum_{n=a+1}^{a+1} \Bigl(f(n) + g(n)\Bigr)\\
          &= f(a) + g(a) + \Bigl(f(a+1) + g(a+1)\Bigr)\\
          &= \Bigl(f(a) + f(a+1)\Bigr) + \Bigl(g(a) + g(a+1)\Bigr) && \text{Asociatividad (de la suma)}\\
          &= \Biggl(f(a) + \sum_{n=a+1}^{a+1} f(n)\Biggr) + \Biggl(g(a) + \sum_{n=a+1}^{a+1} g(n)\Biggr)\\
          &= \sum_{n=a}^{a+1} f(n) + \sum_{n=a}^{a+1} g(n)
        \end{align*}
        \item Supongamos que se verifica para $b=a+k$, con $k\in \N$, es decir, suponemos que
        \begin{align*}
          \sum_{n=a}^{a+k} \Bigl(f(n) + g(n)\Bigr) &= \sum_{n=a}^{a+k} f(n) + \sum_{n=a}^{a+k} g(n)
        \end{align*}
        \item Notemos que
        \begin{align*}
          \sum_{n=a}^{a+k+1} \Bigl(f(n) + g(n)\Bigr) &= \Bigl(f(a+k+1) + g(a+k+1)\Bigr) + \sum_{n=a}^{a+k} \Bigl(f(n) + g(n)\Bigr)\\
          &= f(a+k+1) + g(a+k+1) + \sum_{n=a}^{a+k} f(n) + \sum_{n=a}^{a+k} g(n) && \text{Hip. Inducción}\\
          &= \Biggl(f(a+k+1) + \sum_{n=a}^{a+k} f(n)\Biggr) + \Biggl(g(a+k+1) + \sum_{n=a}^{a+k} g(n)\Biggr)\\
          &= \sum_{n=a}^{a+k+1} f(n) + \sum_{n=a}^{a+k+1} g(n)
        \end{align*}
      \end{enumerate}
    \end{enumerate}
  \end{proof}


  \item Sea $c\in \R$, \[c\cdot \sum_{n=a}^{b} f(n) = \sum_{n=a}^{b} \Bigl(c \cdot f(n)\Bigr) \qquad \text{(Distributividad de la sumatoria)}\]
    \begin{proof}\leavevmode
      \begin{enumerate}[label=\Roman*)]
        \item Si $a>b$, \begin{align*}
          c\cdot \sum_{n=a}^{b} f(n) = c\cdot 0 = \sum_{n=a}^{b} \Bigl(c \cdot f(n)\Bigr)
        \end{align*}
        \item Si $a=b$, \begin{align*}
          c\cdot \sum_{n=a}^{b} f(n) = c\cdot f(n) = \sum_{n=a}^{b} \Bigl(c \cdot f(n)\Bigr) 
        \end{align*}
        \item Si $b>a$,
          \begin{enumerate}[label=\roman*)]
          \item Se comprueba para $b=a+1$, \begin{align*}
            \sum_{n=a}^{a+1} \Bigl(c \cdot f(n)\Bigr) &= c\cdot f(a) + \sum_{n=a+1}^{a+1} c\cdot f(n)\\
            &= c\cdot f(a) + c\cdot f(a+1)\\
            &= c \cdot \Bigl(f(a) + f(a+1)\Bigr)\\
            &= c\cdot \Biggl(f(a) + \sum_{n=a+1}^{a+1} f(n)\Biggr)\\
            &= c\cdot \sum_{n=a}^{a+1} f(n)
          \end{align*}
          \item Supongamos que se verifica para $b=a+k$, con $k\in \N$, es decir, suponemos que \begin{align*}
            \sum_{n=a}^{a+k} \Bigl(c \cdot f(n)\Bigr) = c\cdot \sum_{n=a}^{a+k} f(n) 
          \end{align*}
          \item Notemos que \begin{align*}
            \sum_{n=a}^{a+k+1} \Bigl(c \cdot f(n)\Bigr) &= c\cdot f(a+k+1) + \sum_{n=a}^{a+k} \Bigl(c \cdot f(n)\Bigr)\\
            &= c\cdot f(a+k+1) + c\cdot \sum_{n=a}^{a+k} f(n) && \text{Hip. Inducción}\\
            &= c \cdot \Biggl(f(a+k+1)+ \sum_{n=a}^{a+k} f(n)\Biggr)\\
            &= c\cdot \sum_{n=a}^{a+k+1} f(n)
          \end{align*}
          \end{enumerate}
        \end{enumerate}
    \end{proof}
  
    \textbf{Corolario:} Sea $s,t\in \R$ \begin{enumerate}[label=\roman*)]  
      \item \[s\cdot \sum_{n=a}^{b} f(n) + t\cdot \sum_{n=a}^{b} g(n) = \sum_{n=a}^{b} \Bigl(s\cdot f(n) + t\cdot g(n)\Bigr)\]
      \begin{proof}
        \begin{align*}
          s\cdot \sum_{n=a}^{b} f(n) + t\cdot \sum_{n=a}^{b} g(n) &= \sum_{n=a}^{b} \Bigl(s\cdot f(n)\Bigr) + \sum_{n=a}^{b} \Bigl(t\cdot g(n)\Bigr) && \text{Distributividad de la sumatoria}\\
          &= \sum_{n=a}^{b} \Bigl(s\cdot f(n) + t\cdot g(n)\Bigr) && \text{Asociatividad} \qedhere
        \end{align*}
      \end{proof}
      \item \[\sum_{n=a}^{b} f(n) - \sum_{n=a}^{b} g(n) = \sum_{n=a}^{b} \Bigl(f(n) - g(n)\Bigr)\]
      \begin{proof}
        \begin{align*}
          \sum_{n=a}^{b} f(n) - \sum_{n=a}^{b} g(n) &= \sum_{n=a}^{b} f(n) + (-1)\sum_{n=a}^{b} g(n)\\
          &= \sum_{n=a}^{b} \Bigl(f(n) + (-1)\cdot g(n)\Bigr) && \text{Por (i) de este corolario}\\
          &= \sum_{n=a}^{b} \Bigl(f(n) - g(n)\Bigr)
        \end{align*}
      \end{proof}
    \end{enumerate}

  % \item \[\sum_{n=a}^{b} \left(\sum_{m=s}^{t} f(n,m)\right) = \sum_{m=s}^{t} \left(\sum_{n=a}^{b} f(n,m)\right)\]
  
  % \begin{proof}\leavevmode
  %   \begin{enumerate}[label=\Roman*)]
  %     \item Si $b=a$ y $s=t$
      
  %     \begin{align*}
  %       \sum_{n=a}^{b} \left(\sum_{m=s}^{t} f(n,m)\right) &= \sum_{n=a}^{a} \left(\sum_{m=s}^{s} f(n,m)\right)\\
  %       &= \sum_{n=a}^{a} f(n,s)\\
  %       &= f(a,s)\\
  %       &= \sum_{m=s}^{s} f(a,m)\\
  %       &= \sum_{m=s}^{s} \left(\sum_{n=a}^{a} f(n,m)\right)\\
  %       &= \sum_{m=s}^{t} \left(\sum_{n=a}^{b} f(n,m)\right)
  %     \end{align*}
      
  %     \item Si $b>a$ y $s=t$
      
  %     \begin{align*}
  %       \sum_{n=a}^{b} \left(\sum_{m=s}^{t} f(n,m)\right) &= \sum_{n=a}^{b} \left(\sum_{m=s}^{s} f(n,m)\right)\\
  %       &= \sum_{n=a}^{b} f(n,s)\\
  %       &= \\
  %       &= \sum_{m=s}^{t} \left(\sum_{n=a}^{b} f(n,m)\right)
  %     \end{align*}
      
  %     \item Si $b=a$ y $s>t$
  %     \item Si $b>a$ y $s>t$    
  %     \item Si $b=a$ y $s<t$
  %     \item Si $b<a$ y $s=t$
  %     \item Si $b<a$ y $s<t$
  %   \end{enumerate}
  % \end{proof}
  
  \item \[\sum_{n=a}^{b} \left(\sum_{m=s}^{t} \Bigl(f(n) \cdot  g(m)\Bigr)\right) = \left(\sum_{n=a}^{b} f(n)\right)  \cdot \left(\sum_{m=s}^{t} g(m)\right)\]
  
  \begin{proof}\leavevmode
    Sea $n\in D(f)$ arbitrario pero fijo. Notemos que en la sumatoria $\sum_{m=s}^{t} \Bigl(f(n) \cdot  g(m)\Bigr)$, $f(n)$ es constante, por lo que
    \begin{align*}
      \sum_{n=a}^{b} \left(\sum_{m=s}^{t} \Bigl(f(n) \cdot  g(m)\Bigr)\right) &= \sum_{n=a}^{b} \left( f(n) \sum_{m=s}^{t} g(m)\right)
    \end{align*}

    De la misma manera, en la sumatoria (de índice $n$) $\sum_{n=a}^{b} \Bigl( f(n) \sum_{m=s}^{t} g(m)\Bigr)$, se tiene que $\sum_{m=s}^{t} g(m)$ es constante, por lo que
    \begin{align*}
      \sum_{n=a}^{b} \left( f(n) \sum_{m=s}^{t} g(m)\right) &= \left(\sum_{m=s}^{t} g(m)\right) \cdot \left(\sum_{n=a}^{b} f(n)\right)\\
      &= \left(\sum_{n=a}^{b} f(n)\right) \cdot \left(\sum_{m=s}^{t} g(m)\right)
    \end{align*}
  \end{proof}

  \item Si $a<b$, entonces \[\sum_{n=a}^{b} \Bigl(f(n) \cdot  g(n)\Bigr) \neq \left(\sum_{n=a}^{b} f(n)\right)  \cdot \left(\sum_{n=a}^{b} g(n)\right)\]
  
  \begin{proof}\leavevmode
    Sea $a$ y $b=a+1$, tenemos que
    \begin{align*}
      \sum_{n=a}^{b} \big(f(n)\cdot g(n)\big) &= f(a)\cdot g(a) + \sum_{a+1}^{a+1} \big(f(n)\cdot g(n)\big)\\
      &= f(a)\cdot g(a) + f(a+1)\cdot g(a+1)\\
      &\neq f(a)\cdot g(a) + f(a+1) \cdot g(a+1) + f(a+1) \cdot g(a) + f(a) \cdot g(a+1)\\
      &= g(a) \cdot \big(f(a)+f(a+1)\big) + g(a+1) \cdot \big(f(a)+f(a+1)\big)\\
      &= \big(f(a)+f(a+1)\big) \cdot \big(g(a)+g(a+1)\big)\\
      &= \left(f(a) + \sum_{n=a+1}^{a+1} f(n)\right) \cdot \left(g(a) + \sum_{n=a+1}^{a+1} g(n)\right)\\
      &= \left(\sum_{n=a}^{b} f(n)\right)  \cdot \left(\sum_{n=a}^{b} g(n)\right)
    \end{align*}
  \end{proof}
  
  \item \[\sum_{n=a}^{b} \frac{f(n)}{g(n)} \neq \frac{\sum_{n=a}^{b} f(n)}{\sum_{n=a}^{b}g(n)}\]
  
  

  \item Sea $c\in \R$, si $a\leq b$, entonces \begin{align*}
    \sum_{n=a}^b c &= (b-a+1)c
  \end{align*}
  \begin{proof}\leavevmode
    \begin{enumerate}[label=\Roman*)]
      \item Se comprueba para $a=b$, \begin{align*}
        \sum_{n=a}^b c = c = 1\cdot c = (b-a+1)\cdot c
      \end{align*}
      \item Si $a<b$ se tiene que \begin{enumerate}[label=\roman*)]
        \item Se verifica para $b=a+1$, \begin{align*}
          \sum_{n=a}^{a+1} c &= c + \sum_{n=a+1}^{a+1} c = c + c = 2c = (2+a-a)c = (1+1+a-a)c = \Bigl((a+1)-a+1\Bigr) c
        \end{align*}
        \item Supongamos que se cumple para $b=a+k$, con $k\in \N$; es decir, suponemos que \begin{align*}
          \sum_{n=a}^{a+k} c &= \Bigl((a+k)-a+1\Bigr) c
        \end{align*}
        \item Notemos que \begin{align*}
          \sum_{n=a}^{a+k+1} c &= c + \sum_{n=a}^{a+k} c\\
          &= c + \Bigl((a+k)-a+1\Bigr) c && \text{Hip. Inducción}\\
          &= \Bigl(1 + \bigl((a+k)-a+1\bigr) \Bigr)c\\
          &= \Bigl((a+k+1)-a+1\Bigr)c
        \end{align*}
      \end{enumerate}
    \end{enumerate}
  \end{proof}

  \textbf{Nota:} En esta proposición se restringe que $a\leq b$, pues si $a>b$, se tiene que $\sum_{n=a}^{b} c = 0 \neq (b-a+1)c$; únicamente en el caso que $c=0$, se cumpliría la igualdad con $a>b$.

  \textbf{Definición:} $(b-a+1)$ es el número de \textit{iteraciones}, \textit{cíclos}, o \textit{sumandos} de la sumatoria $\sum_{n=a}^{b} f(n)$.

  \bfit{Corolario:} Si $c\in \R$ y $n\in \N$, entonces $\sum_{i=1}^n c = nc$.
  \begin{proof} $\sum_{i=1}^n c = \bigl((n-1)+1\bigr) c = nc$.
  \end{proof}

  \item Sea $\ell, m\in \R$ y $c\in \Z$, encuentre las condiciones que deben cumplirse para que
  
  \[\sum_{n=a}^b f(\ell+m\cdot n) = \sum_{n=a+c}^{b+c} f(\ell+m\cdot n-c)\]

  \begin{enumerate}[label=\Roman*)]
    \item Notemos que si $c=0$, la proposición es \textit{tautológica}; por lo que, en adelante, suponemos que $c\neq 0$.
    \item Si $a>b$, no importa qué valores tome $\ell$ o $m$, la proposición se verifica por definición:
    \begin{align*}
      \sum_{n=a}^b f(\ell+m\cdot n) = 0 = \sum_{n=a+c}^{b+c} f(\ell+m\cdot n-c)
    \end{align*}
    \item Si $a=b$ y $m=0$,
    \begin{align*}
      \sum_{n=a+c}^{b+c} f(\ell+m\cdot n-c) &= \sum_{n=a+c}^{a+c} f(\ell+m\cdot n-c)\\
      &= f\bigl(\ell+0\cdot (a+c)-c\bigr)\\
      &=f(\ell+0-c)\\
      &= f(\ell-c)\\
      &\neq f(\ell)\\
      &= f(\ell+0)\\
      &= f(\ell+0\cdot a)\\
      &= \sum_{n=a}^a f(\ell+m\cdot n)\\
      &= \sum_{n=a}^b f(\ell+m\cdot n)
    \end{align*}
    Por lo que descartamos este caso.

    \item Si $a=b$, $m\neq 0$ y $m\neq 1$,
    \begin{align*}
      \sum_{n=a+c}^{b+c} f(\ell+m\cdot n-c) &= \sum_{n=a+c}^{a+c} f(\ell+m\cdot n-c)\\
      &= f\bigl(\ell+m(a+c)-c\bigr)\\
      &= f\bigl(\ell+ma+mc-c\bigr)\\
      &\neq f\bigl(\ell+ma\bigr)\\
      &= \sum_{n=a}^a f(\ell+m\cdot n)\\
      &= \sum_{n=a}^b f(\ell+m\cdot n)
    \end{align*}
    Por lo que descartamos este caso.

    \item Si $a=b$ y $m=1$,
    \begin{align*}
      \sum_{n=a+c}^{b+c} f(\ell+m\cdot n-c) &= \sum_{n=a+c}^{a+c} f(\ell+m\cdot n-c)\\
      &= f\bigl(\ell+1\cdot (a+c)-c\bigr)\\
      &= f(\ell + a)\\
      &= f(\ell+1\cdot a)\\
      &= \sum_{n=a}^a f(\ell+m\cdot n)\\
      &= \sum_{n=a}^b f(\ell+m\cdot n)
    \end{align*}

    \item Si $a<b$ y $m=0$.
    
    Para $b=a+1$ se tiene
    \begin{align*}
      \sum_{n=a}^b f(\ell+m\cdot n) &=\sum_{n=a}^{a+1} f(\ell+m\cdot n)\\
      &= f\bigl(\ell+0\cdot (a+1)\bigr) + \sum_{n=a}^{a} f(\ell+0\cdot n)\\
      &= f\bigl(\ell+0\cdot (a+1)\bigr) + f(\ell+0\cdot a)\\
      &= f(\ell) + f(\ell)\\
      &= 2 f(\ell)\\
      &\neq 2 f(\ell - c) \\
      &= f(\ell-c) + f(\ell-c)\\
      &= f\bigl(\ell+0\cdot (a+c)-c\bigr) + f\Bigl(\ell +0\cdot (a+c+1)-c\Bigr)\\
      &= f\bigl(\ell+0\cdot (a+c)-c\bigr) + \sum_{n=a+c+1}^{a+c+1} f(\ell+m\cdot n-c)\\
      &= \sum_{n=a+c}^{(a+1)+c} f(\ell+m\cdot n-c)\\
      &= \sum_{n=a+c}^{b+c} f(\ell+m\cdot n-c)
    \end{align*}
    Por lo que descartamos este caso.

    \item Si $a<b$, $m\neq 0$ y $m\neq 1$,
    Para $b=a+1$ se tiene
    \begin{align*}
      \sum_{n=a}^b f(\ell+m\cdot n) &=\sum_{n=a}^{a+1} f(\ell+m\cdot n)\\
      &= f\bigl(\ell+m\cdot (a+1)\bigr) + \sum_{n=a}^{a} f(\ell+m\cdot n)\\
      &= f\bigl(\ell+m\cdot (a+1)\bigr) + f(\ell+m\cdot a)\\
      &= f(\ell+ma+m) + f(\ell+ma)\\
      &\neq f(\ell+ma+mc-c) + f(\ell+ma+mc+m-c)\\
      &= f\bigl(\ell+m\cdot (a+c)-c\bigr) + f\Bigl(\ell +m\cdot (a+c+1)-c\Bigr)\\
      &= f\bigl(\ell+m\cdot (a+c)-c\bigr) + \sum_{n=a+c+1}^{a+c+1} f(\ell+m\cdot n-c)\\
      &= \sum_{n=a+c}^{(a+1)+c} f(\ell+m\cdot n-c)\\
      &= \sum_{n=a+c}^{b+c} f(\ell+m\cdot n-c)
    \end{align*}
    Por lo que descartamos este caso.

    \item Si $a<b$, $m=1$,
    \begin{enumerate}[label=\roman*)]
      \item Si $b=a+1$,
      \begin{align*}
        \sum_{n=a}^b f(\ell+m\cdot n) &=\sum_{n=a}^{a+1} f(\ell+n)\\
        &= f\bigl(\ell+ (a+1)\bigr) + \sum_{n=a}^{a} f(\ell+ n)\\
        &= f(\ell+a+1) + f(\ell+a)\\
        &= f(\ell+a+1) + f(\ell +(a+c)-c)\\
        &= f\bigl(\ell+(a+1+c)-c\bigr) + \sum_{n=a+c}^{a+c} f(\ell+ n-c)\\
        &= \sum_{n=a+c}^{(a+1)+c} f(\ell+n-c)\\
        &= \sum_{n=a+c}^{b+c} f(\ell+m\cdot n-c)
      \end{align*}

      \item Supongamos que se verifica para $b=a+k$, con $k\in \N$; es decir, supenmos que
      \begin{align*}
        \sum_{n=a}^{a+k} f(\ell+n) &= \sum_{n=a+c}^{(a+k)+c} f(\ell+n-c)
      \end{align*}

      \item Notemos que
      \begin{align*}
        \sum_{n=a}^{a+k+1} f(\ell+n) &= f\Bigl(\ell+ (a+k+1)\Bigr) + \sum_{n=a}^{a} f(\ell+ n)\\
        &= f(\ell+a+k+1) + f(\ell+a)\\
        &= f(\ell+a+k+1) + f(\ell +(a+c)-c)\\
        &= f\bigl(\ell+(a+k+1+c)-c\bigr) + \sum_{n=a+c}^{a+c} f(\ell+ n-c)\\
        &= \sum_{n=a+c}^{(a+k+1)+c} f(\ell+n-c)
      \end{align*}
    \end{enumerate}


    Por lo que en general, planteamos la proposición como sigue:

    Si $c\in \Z$, $\ell \in \R$, entonces \[\sum_{n=a}^b f(\ell+n) = \sum_{n=a+c}^{b+c} f(\ell+n-c) \qquad \text{(Cambio de límites 1)}\]
    
  \end{enumerate}

  \item Sea $c\in \Z$, $m\in \R$, \[\sum_{n=a}^b f(m-n) = \sum_{n=a+c}^{b+c} f\bigl(m-(n-c)\bigr) \qquad \text{(Cambio de límites 2)}\]
  \begin{proof}\leavevmode
    \begin{enumerate}[label=\roman*)]
    \item Si $a>b$,
    \begin{align*}
      \sum_{n=a}^b f(m-n) = 0 = \sum_{n=a+c}^{b+c} f\bigl(m-(n-c)\bigr)
    \end{align*}

    \item Si $a=b$,
    \begin{align*}
      \sum_{n=a+c}^{b+c} f\bigl(m-(n-c)\bigr) &= \sum_{n=a+c}^{a+c} f\bigl(m-(n-c)\bigr)\\
      &= f\Bigl(m-\bigl((a+c)-c\bigr)\Bigr)\\
      &= f(m-a)\\
      &= \sum_{n=a}^a f(m-n)\\
      &= \sum_{n=a}^b f(m-n)
    \end{align*}

    \item Si $a<b$,
    \begin{enumerate}[label=\roman*)]
      \item Se verifica para $b=a+1$,
      \begin{align*}
        \sum_{n=a+c}^{(a+1)+c} f\bigl(m-(n-c)\bigr) &= f\Bigl(m-\bigl((a+c)-c\bigr)\Bigr) + \sum_{n=a+c+1}^{a+c+1} f\bigl(m-(n-c)\bigr)\\
        &= f(m-a) + f\Bigl(m-\bigl((a+c+1)-c\bigr)\Bigr)\\
        &= f(m-a) + f\bigl(m-(a+1)\bigr)\\
        &= f(m-a) + f(m-a-1)\\
        &= f(m-a) + f\bigl(m-(a+1)\bigr)\\
        &= f(m-a) + \sum_{n=a+1}^{a+1} f(m-n)\\
        &= \sum_{n=a}^{a+1} f(m-n)
      \end{align*}

      \item Supongamos qu ese verifica para $b=a+k$, con $k\in \N$; es decir, suponemos que
      \begin{align*}
        \sum_{n=a}^{a+k} f(m-n) = \sum_{n=a+c}^{(a+k)+c} f\bigl(m-(n-c)\bigr)
      \end{align*}

      \item Notemos que
      \begin{align*}
        \sum_{n=a+c}^{(a+k+1)+c} f\bigl(m-(n-c)\bigr) &= f\Bigl(m-\bigl((a+k+1+c)-c\bigr)\Bigr) + \sum_{n=a+c}^{a+k+c} f\bigl(m-(n-c)\bigr)\\
        &= f\bigl(m-(a+k+1)\bigr) + \sum_{n=a}^{a+k} f(m-n) && \text{Hip. Ind.}\\
        &= \sum_{n=a}^{a+k+1} f(m-n)
      \end{align*}
    \end{enumerate}
  \end{enumerate}
  \end{proof}

  \item Sea $m\in \R$ y $c\in \Z$, \[\sum_{n=a}^{b}f\bigl(m\pm n\bigr) = \sum_{n=a+c}^{b+c}f\bigl(m\pm (n- c)\bigr) \qquad \text{(Cambio de índice)}\]
  \begin{proof}\leavevmode
    \begin{enumerate}[label=\roman*)]
      \item Por el cambio de límites 1 se tiene que 
      \begin{align*}
        \sum_{n=a}^{b}f\bigl(m+ n\bigr) = \sum_{n=a+c}^{b+c}f\bigl(m+ (n- c)\bigr)
      \end{align*}
      \item Por el cambio de límites 2 se tiene que
      \begin{align*}
        \sum_{n=a}^{b}f\bigl(m-n\bigr) = \sum_{n=a+c}^{b+c}f\bigl(m-(n- c)\bigr)
      \end{align*}
    \end{enumerate}
  \end{proof}

  \textbf{Nota:} El lector encontrará que en ocasiones, en lugar de utilizar este teorema simplemente se trabaja con susbsituciones sobre el índice, especialmente para funciones identidad. Por ejemplo, tomando la sumatoria de $f(n)=n$, que \textit{itera} de $1$ hasta $b$,
  
  \[\sum_{n=1}^{b} n\]
  
  Sea $m=n-1$, entonces $m+1=n$. Cuando el límite inferior $n=1$, se tiene que $m=1-1=0$. Luego, al considerar la \textit{distancia} entre los límites, tenemos que
  \begin{align*}
    b - n &= b - (m+1)\\
    &= (b-1) - m
  \end{align*}
  
  Al sustituir $n$ por $m$ en la sumatoria, tenemos
  
  \[\sum_{n=1}^{b} n = \sum_{m=0}^{b-1} (m+1)\]

  \item Si $s\leq j\leq t$, \[\sum_{n=s}^{t}f(n) = \sum_{n=s}^{j}f(n) + \sum_{n=j+1}^{t}f(n) \qquad \text{(Partir la suma)}\]
  
  \textbf{Nota:} Alternativamente podemos escribir esta igualdad como sigue: Si $s\leq j\leq t$, entonces $\sum_{n=s}^{t}f(n) = \sum_{n=s}^{j-1}f(n) + \sum_{n=j}^{t}f(n)$. El lector debería verificar esta equivalencia

  \begin{proof}
    Consideremos los casos:
    \begin{enumerate}[label=\Roman*)]
      \item Si $s=j=t$,
      \begin{align*}
        \sum_{n=s}^{t}f(n) &= f(s)\\
        &= f(s) + 0\\
        &= \sum_{n=s}^{j}f(n) + \sum_{n=j+1}^{t}f(n)
      \end{align*}
      \item Si $s<j=t$,
      \begin{align*}
        \sum_{n=s}^{t}f(n) &= \sum_{n=s}^{j}f(n)\\
        &= \sum_{n=s}^{j}f(n) + 0\\
        &= \sum_{n=s}^{j}f(n) + \sum_{n=j+1}^{t}f(n)
      \end{align*}
      \item Si $s=j<t$,
      \begin{align*}
        \sum_{n=s}^{t}f(n) &= f(s) + \sum_{n=s+1}^{t}f(n)\\
        &= \sum_{n=s}^{j}f(n) + \sum_{n=j+1}^{t}f(n)
      \end{align*}
      \item Si $s<j<t$.
        Sea $j\in \Z$ arbitrario pero fijo,
        \begin{enumerate}[label=\roman*)]
          \item Si $t=j+1$,
          \begin{align*}
            \sum_{n=s}^{t}f(n) &=  \sum_{n=s}^{j+1}f(n)\\
            &= f(j+1) + \sum_{n=s}^{j}f(n)\\
            &= \sum_{n=s}^{j}f(n) + f(j+1) && \text{Conmutatividad}\\
            &= \sum_{n=s}^{j}f(n) + \sum_{n=j+1}^{t}f(n)
          \end{align*}
          \item Supongamos que $\sum_{n=s}^{j+k}f(n) = \sum_{n=s}^{j}f(n) + \sum_{n=j+1}^{j+k}f(n)$ para algún $k\in \N$.
          \item Si $t=j+k+1$, notemos que
          \begin{align*}
            \sum_{n=s}^{t}f(n) &= \sum_{n=s}^{j+k+1}f(n)\\
            &= f(j+k+1) + \sum_{n=s}^{j+k}f(n)\\
            &= f(j+k+1) + \sum_{n=s}^{j}f(n) + \sum_{n=j+1}^{j+k}f(n) && \text{Hip. Ind.}\\
            &= \sum_{n=s}^{j}f(n) + \sum_{n=j+1}^{j+k}f(n) +f(j+k+1)\\
            &= \sum_{n=s}^{j}f(n) + \sum_{n=j+1}^{j+k+1}f(n)\\
            &= \sum_{n=s}^{j}f(n) + \sum_{n=j+1}^{t}f(n)
          \end{align*}
        \end{enumerate}
    \end{enumerate}
  \end{proof}

  \textbf{Nota:} En esta proposición se restringe que $s\leq j\leq t$, pues la proposición no es válida para todo $s\geq j\geq t$:
  \begin{enumerate}[label=\Roman*)]
  \item Si $s>j>t$,
  \begin{align*}
    \sum_{n=s}^{t} f(n) = 0 = \sum_{n=s}^{j} f(n) + \sum_{n=j+1}^{t} f(n)
  \end{align*}

  \item Si $s>j=t$,
  \begin{align*}
    \sum_{n=s}^{t} f(n)= 0 = \sum_{n=s}^{j} f(n) + \sum_{n=j+1}^{t}
  \end{align*}

  \item Si $s=j>t$,
  \begin{enumerate}[label=\roman*)]
    \item $\sum_{n=s}^{j} f(n) + \sum_{n=j+1}^{t} f(n) = f(n) + 0 = f(n)$.
    \item $\sum_{n=s}^{t} f(n) = 0$.
    
    El lector notará que para partir la suma en este caso, debe cumplirse que $f(n)=0$, pero esto dependerá de cada función y de los íncides, por lo que en general, $f(n)\neq 0$, por ejemplo para cualquier sumatoria $\sum_{n=a}^{b} c$, donde $c\neq 0$.

  \end{enumerate}

  \bfit{Corolario:} \[\sum_{n=a}^{b}f(n) = \sum_{n=0}^{b} f(n) - \sum_{n=0}^{a-1} f(n)\]
  \begin{proof}\leavevmode
    \begin{align*}
      \sum_{n=0}^{b} f(n) - \sum_{n=0}^{a-1} f(n) &= \sum_{n=0}^{a} f(n) + \sum_{a+1}^{b} f(n) - \sum_{n=0}^{a-1}f(n) && \text{Partir la suma}\\
      &= \sum_{n=0}^{a-1} f(n) + \sum_{n=a}^{a} f(n) + \sum_{a+1}^{b} f(n) - \sum_{n=0}^{a-1}f(n) && \text{Partir la suma}\\
      &= \sum_{n=a}^{a} f(n) + \sum_{a+1}^{b} f(n)\\
      &= \sum_{n=a}^{b}f(n)
    \end{align*}
  \end{proof}

  \end{enumerate}

  \item \[\sum_{n=a}^{b}f(n) = \sum_{n=0}^{b-a} f(b-n)\]
  \begin{proof}\leavevmode
    \begin{enumerate}[label=\Roman*)]
    \item Si $b<a$, entonces $b-a<0$, por lo que
    \begin{align*}
      \sum_{n=0}^{b-a} f(b-n) = 0 = \sum_{n=a}^{b}f(n)
    \end{align*}
    
    \item Si $b=a$,
    \begin{align*}
      \sum_{n=0}^{b-a} f(b-n) &= \sum_{n=0}^{0} f(b-n)\\
      &= f(b-0)\\
      &= f(b)\\
      &= \sum_{n=a}^{b} f(n)
    \end{align*}
    

    \item Si $b>a$,
    \begin{enumerate}[label=\roman*)]
      \item Se verifica para $b=a+1$,
      \begin{align*}
        \sum_{n=0}^{b-a} f(b-n) &= \sum_{n=0}^{(a+1)-a} f(a+1-n)\\
        &= \sum_{n=0}^{1} f(a+1-n)\\
        &= f(a+1-0) + \sum_{n=1}^{1} f(a+1-n)\\
        &= f(a+1) + f(a+1-1)\\
        &= f(a+1) + f(a)\\
        &= f(a) + f(a+1)\\
        &= \sum_{n=a}^{a+1} f(n)\\
        &= \sum_{n=a}^{b} f(n)
      \end{align*}

      \item Supongamos que se verifica para $b=a+k$, con $k\in \N$, es decir, suponemos que
      \begin{align*}
        \sum_{n=a}^{a+k}f(n) = \sum_{n=0}^{(a+k)-a} f(a+k-n) = \sum_{n=0}^{k} f(a+k-n)
      \end{align*}

      \item Si $b=a+k+1$,
      \begin{align*}
        \sum_{n=a}^{a+k+1} f(n) &= f(a+k+1) + \sum_{n=a}^{a+k} f(n)\\
        &= f(a+k+1) + \sum_{n=0}^{k} f(a+k-n) && \text{Hip. Ind.}\\
        &= f\bigl(a+k-(-1)\bigr) + \sum_{n=0}^{k} f(a+k-n)\\
        &= \sum_{n=-1}^k f(a+k-n) && \text{Definición (de sumatoria)}\\
        &= \sum_{n=-1+(1)}^{k+(1)} f\bigl(a+k-(n-1)\bigr) && \text{Cambio de índice}\\
        &= \sum_{n=0}^{k+1} f(a+k+1-n)
      \end{align*}
      
    \end{enumerate}

    \end{enumerate}
  \end{proof}

  \bfit{Corolario:} \[\sum_{n=0}^{b}f(n) = \sum_{n=0}^{b} f(b-n)\]

  \begin{proof}\leavevmode
    La porposición se verifica por el teorema para $a=0$.
  \end{proof}

    
  \end{enumerate}

\subsection*{Una nota sobre la notación sigma}

% \textbf{Abuso de la notación}

% Considere el siguiente argumento planteado por un estudiante, y decida si está o no de acuerdo:

% Para números reales $a$ y $b$ tales que $a<b$, hemos definido $\sum_{\underline{n=a}}^{b} f(n) = f(a) + \sum_{\underline{n=a+1}}^{b}f(n)$; el lector notará que del lado izquierdo de esta igualdad $n=a$ pero del lado derecho $n=a+1$, es decir, se emplea el mismo símbolo $n$ para números enteros distintos, lo que resulta útil en este caso ya que la definición de sumatoria está dada \textit{recursivamente}; sin embargo, de manera minuciosa (y pedante), se podría reescribir como $\sum_{n=a}^{b}f(n) = f(a) + \sum_{n+1=a+1}^{b}f(n+1)$, pero en cada iteración tendríamos que cambiar el límite inferior y, con ello, la variable a evaluar, lo que llegaría a ser laborioso, especialmente si se requiere de expresar múltiples iteraciones; por ello, se acude al abuso mencionado.

\textbf{Extensión}

Usualmente, el alcance de una suma se extiende hasta el primer símbolo de suma o resta que no está entre paréntesis o que no es parte de algún término más amplio (por ejemplo, en el numerador de una fracción), de manera que:

\begin{align*}
  \sum_{i=1}^{n} i^2 + 1 = \left(\sum_{i=1}^{n}i^2\right) + 1 = 1 + \sum_{i=1}^{n} i^2 \neq \sum_{i=1}^{n} \bigl(i^2+1\bigr)
\end{align*}

dado que esto puede resultar confuso, generalmente es más seguro encerrar el argumento de la sumatoria entre paréntesis (como en la segunda forma arriba) o mover los términos finales al principio (como en la tercera forma arriba). Una excepción (a la confusión) es cuando se suman dos sumas, como en

\begin{align*}
  \sum_{i=1}^{n} i^2 + \sum_{i=1}^{n^2} i = \left(\sum_{i=1}^{n}i^2\right) + \left(\sum_{i=1}^{n^2}i\right)
\end{align*}

\textbf{Límites no enteros}

En principio, la sumatoria está bien definida para casos en los que los límites no son enteros, por ejemplo:

\begin{align*}
  \sum_{i=1/2}^{9/4} i &= 1/2 + \sum_{i=3/2}^{9/4} i \\
  &= 1/2 + 3/2 + \sum_{i=5/2}^{9/4} i\\
  &= 1/2 + 3/2 + 0\\
  &= 2
\end{align*}

Sin embargo, este uso no es común. El ejemplo también sirve para ilustrar que, para la segunda definición de sumatoria, es necesario que la \textit{distancia} entre el límite superior y el inferior sea un entero. Por ejemplo, tratar lo siguiente sería un error

\begin{align*}
  \sum_{i=1/2}^{9/4} i &\neq 9/4 + \sum_{i=1/2}^{5/4} i\\
  &\neq 9/4 + 5/4 + \sum_{i=1/2}^{1/4} i\\
  &\neq 9/4 + 5/4 + 0\\
  &= 14/4\\
  &= 7/2\\
  &= 3 + 1/2
\end{align*}

\textbf{Sumatoria sobre conjuntos indexados}

El índice de la sumatoria puede iterar sobre los elementos de un conjunto indexado finito, por ejemplo, \[\sum_{i\in \set{2, 3, 5}} i^2 = (2)^2 + (3)^2 + (5)^2 = 38\]

\begin{align*}
  \sum_{i=0}^{n} \sum_{j=0}^{i} (i+1)(j+1) &= \sum_{i=0}^{1} \sum_{j=0}^{i} (i+1)(j+1)\\
  &= \sum_{i=0}^{1} \left(\sum_{j=0}^{i} (i+1)(j+1)\right)\\
  &= \left(\sum_{i=0}^{1} (i+1)\right) \cdot \left(\sum_{j=0}^{i} (j+1)\right)\\
  &= \left((0+1) + \sum_{i=1}^{1} (i+1)\right) \cdot \left(\sum_{j=0}^{0} (j+1)\right)\\
\end{align*}

$\set{n| a \leq n \leq b}$.

Sea $A,B\subseteq \R$, y $f:A\to B$. Sea $a, b \in A$.


