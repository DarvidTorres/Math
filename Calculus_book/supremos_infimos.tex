\part*{Supremos e Ínfimos}

\section*{Entorno e intervalos}

\bfit{Definición:}  Sea $a, b\in \R$, definimos
\begin{itemize}
 \item $(a,b)\defined \set{x\in \R| a<x<b}$. Decimos que $(a,b)$ es el intervalo abierto (de $a$ a $b$).
 \item $[a,b]\defined \set{x\in \R|a\leq x\leq b}$. Decimos que $[a, b]$ es el intervalo cerrado (de $a$ a $b$).
 \item $[a,b)\defined \set{x\in \R|a\leq x <b}$.
 \item $(a,b]\defined \set{x\in \R|a<x\leq b}$.
 \item $(a, \infty)\defined \set{x\in \R | a<x}$.
 \item $[a, \infty) \defined \set{x\in \R| a\leq x}$.
 \item $(-\infty, a) \defined \set{x\in \R|a<x}$.
 \item $(-\infty, a] \defined \set{x\in \R|a\leq x}$.
\end{itemize}

\subsection*{Una nota sobre la notación de algunos intervalos}

Introducimos los símbolos $\infty$ y $-\infty$ de manera sintáctica, es decir, únicamente como parte de la notación empleada para algunos intervalos, como descrito arriba. En otras palabas, su uso es \textit{formal}.

\textbf{Observación:} Respectivamente, los conjuntos $\R^+$ y $\R^-$ son intervalos de la forma $(0, \infty)$ y $(-\infty, 0)$.

\bfit{Definición.} Sea $a \in \R$ y $\varepsilon>0$. Definimos al entorno de centro $a$ y radio $\varepsilon$, como el conjunto: \[E_\varepsilon(a)\defined \{ x\in \R: |x-a|<\varepsilon\}\]
\textbf{Notación:} También denotamos al entorno de centro $a$ y radio $\varepsilon$ como $E_{(a, \varepsilon)}$, o si el radio es claro, $E_{(a)}$. También decimos que $E_\varepsilon(a)$ es el entorno-$\varepsilon$ (epsilon) de $a$.

\subsection*{Lista de Ejercicios 5 (LE5)}

Sean $a,b \in \R$. Demuestre lo siguiente:

\begin{enumerate}[label=\alph*)]
  
  \item $E_\varepsilon(a) = (a-\varepsilon, \ a+\varepsilon)$.
  
  \begin{proof}\leavevmode
    Por definición, $E_\varepsilon(a) = \set{x\in \R: |x-a|<\varepsilon}$. Notemos que
    \begin{align*}
      |x - a | &< \varepsilon\\
      -\varepsilon &< x-a < \varepsilon && \text{Teorema para eliminar valores absolutos}\\
      a -\varepsilon &< x < a+ \varepsilon
    \end{align*}
    Es decir que $\forall x\in \R$ y $\forall \varepsilon>0$, tales que $|x-a|<\varepsilon$ se tiene que $x\in (a-\varepsilon, \ a+ \varepsilon)$. Osea que $E_\varepsilon(a) = (a-\varepsilon, \ a+\varepsilon)$.
  \end{proof}
  \textbf{Observación:} El centro del entorno $E_\varepsilon(a)$ es el punto medio de los extremos del intervalo $(a-\varepsilon, \ a+\varepsilon)$:
  \begin{align*}
    a - \varepsilon &< a + \varepsilon\\
    a- \varepsilon &< \frac{(a - \varepsilon)+(a + \varepsilon)}{2} < a + \varepsilon && \text{Punto medio}\\
    a- \varepsilon &< a < a + \varepsilon
  \end{align*}
  
  \item Sea $a<b$, entonces $\exists! \varepsilon>0$ tal que $a+\varepsilon=b$.
 \begin{proof}\leavevmode
  \begin{enumerate}[label=\roman*)]
    \item Primero probaremos su existencia. Sea $a<b$. Entonces $b-a>0$. Tomando $\varepsilon=b-a$, tenemos que $a+\varepsilon=a+b-a=b$.
    \item Ahora probaremos la unicidad. Sean $\varepsilon>0$ y $\delta>0$ tales que $a+\varepsilon=b$ y $a+\delta=b$, se tiene que $\varepsilon=b-a$ y $\delta=b-a$, es decir, $\varepsilon=\delta$. \qedhere
  \end{enumerate}
 \end{proof}
 
 \textbf{Nota:} Con esta prueba verificamos que todo número $b$, mayor que $a$, puede escribirse como la suma de $a$ y algún número positivo.
 
\clearpage\pagebreak
  
  \item Para toda $a$ y $b$ tales que $a<b$, $\exists x\in \R$ y $\varepsilon>0$ tales que $(a, b)= (x- \varepsilon, \ x+\varepsilon)$.
  \begin{proof}\leavevmode
    Notemos que
    \begin{align*}
      a &< b\\
      a &< \frac{a+b}{2} < b && \text{Punto medio}\\
    \end{align*}
    Tomando $x=\frac{a+b}{2}$, se tiene que $x<b$. Sabemos que $\exists \varepsilon>0$ tal que $b=x+\varepsilon$. Luego,
    \begin{align*}
      b-\frac{a+b}{2} &= \frac{a+b}{2}-a\\
      (x+\varepsilon) - x &= x - a\\
      \varepsilon &= x - a\\
      a &= x - \varepsilon
    \end{align*}
    Es decir, $\exists x\in \R$ y $\varepsilon>0$ tales que $(a, b)= (x- \varepsilon, x+\varepsilon)$.
  \end{proof}
    
  \item Si $a \leq b + \varepsilon$ para toda $\varepsilon > 0$, entonces $a \leq b$.
 
  \begin{proof} 
   Sean $a$ y $b$ números reales tales que $a \leq b + \varepsilon$, $\forall \varepsilon > 0$. Supongamos que $a > b$. Luego, $a-b>0$. Notemos que $(a-b) \cdot \frac{1}{2} > 0 \cdot \frac{1}{2}$, es decir $\frac{(a-b)}{2} > 0$. Sea $\varepsilon = \frac{(a-b)}{2}$, sigue que $a=2\varepsilon+b$. Además, $2\varepsilon > \varepsilon$, de donde obtenemos $2 \varepsilon + b > \varepsilon + b$. De este modo, $a > b+\varepsilon$, pero esto contradice nuestra hipótesis. Por tanto, $a \leq b$. 
  \end{proof}
  
 \item Si $0 \leq a < \varepsilon$ para toda $\varepsilon > 0$, entonces $a=0$.
 
 \begin{proof} 
  Supongamos que $0<a$, sigue que $0<\frac{a}{2}<a$. En particular, $\varepsilon=\frac{a}{2}$, entonces $\varepsilon<a$, pero esto contradice nuestra hipótesis de que $a< \varepsilon$ para toda $\varepsilon>0$. Por tanto, $a=0$. 
 \end{proof}

 \item Si $x\in V_\varepsilon(a)$ para toda $\varepsilon>0$, entonces $x=a$.
 \begin{proof} 
  Si $x\in V_\varepsilon(a)$ tenemos que $|x-a|<\varepsilon$. Además, $0\leq |x-a|$, por definición. Así, $0\leq |x-a|<\varepsilon$. Como esta desigualdad se cumple para toda $\varepsilon>0$, sigue que $|x-a|=0$. De este modo, $|x-a|=x-a=0$. Por tanto, $x=a$. 
 \end{proof}
 
 \item Sea $U:=\{x\in \R: 0<x<1\}$. Si $a\in U$, sea $\varepsilon$ el menor de los números $a$ y $1-a$. Demuestre que $V_\varepsilon(a) \subseteq U$.
 \begin{proof} \leavevmode
  \begin{enumerate}[label=\roman*)]
   \item Si $a>1-a$, tenememos $\varepsilon=1-a$. Sea $y\in V_\varepsilon(a)$, entonces $|y-a|<1-a$. Por el teorema para eliminar valores absolutos, sigue que $a-1<y-a<1-a$ (*). Tomando el lado derecho de (*) obtenemos $y<1$. Luego, de la hipótesis sigue que $2a>1$, osea $2a-1>0$. Del lado izquierdo de la desigualdad (*), tenemos $2a-1<y$, por lo que $0<y$.
   \item Si $1-a>a$, tenemos $\varepsilon=a$. Sea $y\in V_\varepsilon(a)$, entonces $|y-a|<a$. Por el teorema para eliminar valores absolutos, sigue que $-a<y-a<a$. Sumando $a$ en esta desigualdad obtenemos $0<y<2a$. Luego, de la hipótesis sigue que $1>2a$, entonces $0<y<1$.\end{enumerate}
   En cualquier caso, $0<y<1$, lo que implica que $V_\varepsilon(a) \subseteq U$. 
 \end{proof}

 \item Demuestre que si $a\neq b$, entonces existen $U_\varepsilon(a)$ y $V_\varepsilon(b)$ tales que $U\cap V =\emptyset$.
 \begin{proof} 
  Supongamos que para toda $U_\varepsilon(a)$ y $V_\varepsilon(b)$ se cumple que $U_\varepsilon(a) \cap V_\varepsilon(b) \neq \emptyset$. Entonces, existe $x$ tal que $x\in U_\varepsilon(a)$ y $x\in V_\varepsilon(b)$. Como en ambos entornos tenemos $\varepsilon>0$ arbitraria, sigue que $x=a$ y $x=b$, pero esto contradice el supuesto de que $a\neq b$. Por tanto, deben existir $U_\varepsilon(a)$ y $V_\varepsilon(b)$ tales que $U\cap V =\emptyset$. 
 \end{proof}
\end{enumerate}

\section*{Conjuntos acotados}

\bfit{Definición:} Sea $A\subset \R$ con $A\neq \emptyset$, decimos que $A$:
\begin{itemize}
 \item está acotado superiormente, si $\exists K\in \R$ tal que $a \leq K, \forall a\in A$. En este caso decimos que $K$ es cota superior de $A$.

 \item está acotado inferiormente, si $\exists k\in \R$ tal que $k \leq a, \forall a\in A$. En este caso decimos que $k$ es cota inferior de $A$.

 \item está acotado, si $\exists M\in \R$ tal que $|a|\leq M,\forall a \in A$. En este caso decimos que $M$ es una cota de $A$.
\end{itemize}

\textbf{Observación:}
\begin{itemize}
  \item Si $K$ es una cota superior de $A$, entonces cualquier número real mayor a $K$ también es cota superior de $A$. 
  \item Si $k$ es una cota inferior de $A$, entonces cualquier número real menor a $k$ también es cota inferior de $A$.
  \item Si $M$ es una cota de $A$, entonces cualquier número real mayor que $M$ también es cota de $A$.
\end{itemize}

\bfit{Definición:} Sea $A\subset \R$ con $A\neq \emptyset$, decimos que:
\begin{itemize}
  \item Si $M$ es una cota superior de $A$ y $M\in A$, decimos que $M$ es el elemento máximo de $A$, y lo denotamos como $\max(A)$.
  \item Si $m$ es una cota inferior de $A$ y $m\in A$, decimos que $m$ es el elemento mínimo de $A$, y lo denotamos como $\min(A)$.
\end{itemize}

% \textbf{Observación:} Por definición, si $A$ tiene elemento máximo, entonces está acotado superiormente, y si $A$ tiene elemento mínimo, entonces está acotado inferiormente.

\subsection*{Lista de ejercicios}

Demuestre lo siguiente:

\begin{enumerate}[label=\alph*)]
  \item Sea $A\subset \R$ tal que $A\neq \emptyset$. $A$ está acotado si y solo si $A$ está acotado superior e inferiormente.
 
  \begin{proof} \leavevmode
   \begin{itemize}
    \item[$\Rightarrow)$] Supongamos que $A$ está acotado. Sea $a\in A$, por definición, $\exists M\in \R$ tal que $|a|\leq M$. Por el teorema para eliminar el valor absoluto en algunas desigualdades, sigue que $-M\leq a \leq M$, por lo que $A$ está acotado superior e inferiormente.
    \item[$\Leftarrow)$] Supongamos que $A$ está acotado superior e inferiormente. Sea $a\in A$, entonces $\exists k, K\in \R$ tales que $k\leq a \leq K$. Notemos que
    \begin{align*}
     -k &\leq |k|\\
     -k &\leq |k| + |K|\\
     -|K|-|k| &\leq k\\
     - \big(|K|+|k|\big) &\leq k
    \end{align*}
    Como $k\leq a$, por transitividad sigue que, $-\big(|K|+|k|\big) \leq a$. Similarmente,
    \begin{align*}
     K &\leq |K|\\
     K &\leq |K| + |k|
    \end{align*}
    Como $a\leq K$, por transitividad sigue que, $a\leq |K|+|k|$. Es decir, se verifica que \[-\big(|K|+|k|\big) \leq a \leq |K|+|k|\]
    y, por el teorema para eliminar el valor absoluto en algunas desigualdades, sigue que $|a| \leq |K| + |k|$. Por tanto, $A$ está acotado. \qedhere
   \end{itemize}
  \end{proof}
  
  \item Sea $A\subseteq\R$, $A\neq \emptyset$. $A$ está acotado superiormente si y solo si $-A\defined \set{-a: a\in A}$ está acotado inferiormente.
  
  \begin{proof}\leavevmode
  \begin{itemize}
  \item[$\Rightarrow)$] Supongamos que $A$ está acotado superiormente. Entonces, $\exists K\in \R$ tal que $a\leq K, \forall a\in A$, de donde $-K\leq -a, \forall -a\in -A$. Por lo que $-A$ está acotado inferiormente.
  \item[$\Leftarrow)$] Supongamos que $-A$ está acotado inferiormente. Entonces, $\exists k\in \R$ tal que $k\leq -a, \forall -a\in -A$, de donde $-(-a)=a\leq -k, \forall a\in A$, por lo que $A$ está acotado superiormente. \qedhere
  \end{itemize}
  \end{proof}
  
  \item Si $A$ tiene elemento máximo, este es único.
  
  \begin{proof}\leavevmode
    Sea $M$ y $M'$ elementos máximos de $A$. Como $M'\in A$, y $M$ es una cota superior de $A$, entonces $M'\leq M$. Análogamente, $M\leq M'$. Por tanto, $M=M'$.
  \end{proof}
  
  \item Si $A$ tiene elemento mínimo, este es único.
  \begin{proof}\leavevmode
    Sea $m$ y $m'$ elementos máximos de $A$. Como $m'\in A$, y $m$ es una cota inferior de $A$, entonces $m'\leq m$. Análogamente, $m\leq m'$. Por tanto, $m=m'$.
  \end{proof}
  
  \item Sea $A\subseteq \R$. Se verifica que $\min(A)\leq \max(A)$.
  (El elemento mínimo es menor o igual que el elemento máximo).
  \begin{proof}\leavevmode
    Sea $a\in A$. Se tiene que $\min(A)\leq a$ y $a\leq \max(A)$. Por lo que $\min(A)\leq \max(A)$.
  \end{proof}
  
  \textbf{Observación:} $\min(A)=\max(A)$ si y solo si $A$ es un conjunto unitario. (El lector debería verificar este hecho).
  
  \item Sea $A\subseteq B \subseteq \R$, se verifica que
  \begin{enumerate}[label=\roman*)]
    \item $\max(A) \leq \max(B)$, y
    \item $\min(B)\leq \min(A)$.
  \end{enumerate}
  
  \item Sea $A\subseteq \R$ y $-A\defined \set{-a| a\in A}$. Se verifica que $\max(A)=-\min(-A)$.
  
  \item Sean $a,b\in \R$. Se verifica que $\max\set{a, b} = \frac{a+b+|a-b|}{2}$ y $\min\set{a, b} = \frac{a+b-|a-b|}{2}$.
  
  \item Sea $K\in \R$ y $A=\set{x\in \R|x<K}$. $A$ no tiene elemento máximo.
  \begin{proof}\leavevmode
    Supongamos que $M=\max(A)$. Por definición, $M\in A$, por lo que $M<K$. Luego, $M<\frac{M+K}{2}<K$, por lo que $\frac{M+K}{2}\in A$, pero esto es una contradicción, pues $M$ es cota superior de $A$. Por tanto, $A$ no tiene elemento máximo.
  \end{proof}
  
  \item Sea $k\in \R$ y $A=\set{x\in \R|k<x}$. $A$ no tiene elemento mínimo.
  
  \begin{proof}
    Supongamos que $m=\min(A)$. Por definición, $m\in A$, por lo que $k<m$. Luego, $k<\frac{k+m}{2}<m$, por lo que $\frac{k+m}{2}\in A$, pero esto es una contradicción, pues $m$ es cota inferior de $A$. Por tanto, $A$ no tiene elemento mínimo.
  \end{proof}
  
\end{enumerate}

\section*{Propiedad de completez de \(\mathbb{R}\)}

\subsection*{Supremos e ínfimos}

\bfit{Definición:} Sea $A\subset \R$ tal que $A\neq \emptyset$ y $A$ está acotado superiormente, decimos que un número real $S$ es supremo de $A$ si:
\begin{itemize}
 \item $S$ es cota superior de $A$, y
 \item Si $K$ es cota superior de $A$, entonces $S\leq K$. ($S$ es la cota superior más pequeña de $A$).
\end{itemize}

\textbf{Notación}: Denotamos al supremo de $A$ como $\sup(A)$.

\textbf{Nota:} La definición de supremo no implica que este pertenezca al conjunto.

\bfit{Definición:} Sea $A\subset \R$ tal que $A\neq \emptyset$ y $A$ está acotado inferiormente, decimos que un número real $L$ es ínfimo de $A$ si: \begin{itemize}
 \item $L$ es cota inferior de $A$, y
 \item Si $K$ es cota inferior de $A$, entonces $K\leq L$. ($L$ es la cota inferior más grande de $A$).
\end{itemize}

\textbf{Notación}: Denotamos al ínfimo de $A$ como $\inf(A)$.

\textbf{Nota:} La definición de ínfimo no implica que este pertenezca al conjunto.

\textbf{Una nota sobre el supremo y el ínfimo}

A pesar de que hemos definido al supremo e ínfimo para subconjuntos no vacíos de los números reales, en realidad no es posible ---utilizando únicamente las propiedades demostradas hasta este punto, probar que, en efecto, el supremo e ínfimo existen para todo subconjunto no vacío de los reales que esté acotado.

Consideremos el siguiente argumento: Sea $A\subseteq \R$ tal que $A\neq \emptyset$ y $A$ está acotado superiormente por $S$. Sea $a\in A$, por definición, $a\leq S$, y por la densidad de los números reales, $\exists x\in \R$ tal que $a\leq x \leq S$, por lo que $x$ es una cota superior de $A$. Es decir, para cualquier cota superior de un conjunto, es posible encontrar un número real que sea menor o igual que este, y mayor o igual a los elementos del conjunto. Si la relación entre las cotas se cumple con igualdad, es decir que $a\leq x = S$, podríamos conjeturar que $S$ es el supremo de $A$, sin embargo, si se tiene que $a\leq x < S$, sería natural pensar que $x$ es el supremo de $A$, pues es menor que la cota superior $S$. No obstante, $\exists y\in \R$ tal que $a\leq y \leq x<S$, y nos encontramos en la misma situación que antes, si $a\leq y = x$, entonces $x$ es candidato para ser el supremo de $A$, pero si $a\leq y < x$, podríamos inferir que $y$ es el supremo de $A$, recursivamente sin encontrar un número real que satisfaga la propiedad de ser la cota inferior más pequeña de $A$, para cualquier subconjunto no vacío de $\R$. Análogamente, la existencia del ínfimo no está garantizada ni puede probarse. Por tanto, introducimos la existencia del supremo como un axioma:



\textbf{Axioma del supremo:} Todo subconjunto no vacío del conjunto de los números reales que sea acotado superiormente tiene supremo.

\subsection*{Lista de ejercicios}

Demuestre lo siguiente:

\begin{enumerate}[label=\alph*)]
  \item Si $A\subseteq \R, A\neq \emptyset$ y $A$ está acotado inferiormente, entonces $A$ tiene ínfimo.
  \begin{proof}\leavevmode
    Sea $A\subseteq \R, A\neq \emptyset$ y $A$ está acotado inferiormente. El conjunto $-A \coloneqq \set{-a: a\in A}$ está acotado superiormente y, por el axioma del supremo, $-A$ tiene supremo. Sea $M\coloneqq \sup{(A)}$ y $a\in A$, entonces $M\geq -a$, de donde sigue que $-M\leq a$, esto es $-M$ es el ínfimo de $A$.
  \end{proof}
  
  \item Sea $A$ un subconjunto no vacío de $\R$. Si $A$ tiene supremo, este es único.
 
  \begin{proof} 
   Supongamos que $s_1$ y $s_2$ son supremos de $A$. Por definición, $s_1$ es una cota superior de $A$ y $s_2$ es elemento supremo, entonces $s_2\leq s_1$. Análogamente, $s_1\leq s_2$. Por tanto, $s_1=s_2$.
  \end{proof}
 
  \item Sea $A$ un subconjunto no vacío de $\R$. Si $A$ tiene ínfimo, este es único.
  
  \begin{proof} 
   Supongamos que $m_1$ y $m_2$ son ínfimos de $A$. Por definición, $m_1$ es una cota inferior de $A$ y $m_2$ es elemento ínfimo, entonces $m_1\leq m_2$. Análogamente, $m_2\leq m_1$. Por tanto, $m_1=m_2$. 
  \end{proof}
  
  \item El conjunto de los números naturales no está acotado superiormente.
  \begin{proof}\leavevmode
    Supongamos que el conjunto de los números naturales está acotado superiormente. Entonces existe un número real $M$ tal que $n\leq M, \forall n\in \N$. Como el conjunto de los números naturales es no vacío, entonces, por el axioma del supremo, $\N$ tiene supremo.

    Sea $L\coloneqq \sup{(\N)}$. Como $L-1$ no es cota superior de $\N$, ya que $L>L-1$ y $L$ es la cota superior más pequeña, existe un núero natural $n_0$ tal que $n_0>L-1$, lo cual implica que $n_0+1<L$, pero esto contradice la hipótesis	de que $L$ es supremo de $\N$. Por tanto, el conjunto de los números naturales no está acotado superiormente.
  \end{proof}
  
 \item Sea $A\subset \R$ y $A\neq \emptyset$. Una cota superior $M$ de $A$, es el supremo de $A$ si y solo si $\forall b\in \R$ tal que $b<M$, entonces $\exists a\in A$ tal que $b<a$.
 \begin{proof}\leavevmode
 \begin{itemize}
 \item[$\Rightarrow)$] Supongamos que $M$ es el supremo de $A$ y sea $b\in \R$ tal que $b<M$. Se tiene que $b$ no es cota superior de $A$, es decir que $\exists a\in A$ tal que $b<a$.
 \item[$\Leftarrow)$] Supongamos que $\forall b\in \R$ tal que $b<M$, $\exists a\in A$ tal que $b<a$. Supongamos que $M$ no es el supremo de $A$, es decir, existe una cota superior $c$ de $A$ tal que $c<M$, y por hipótesis, $\exists a_0\in A$ tal que $c<a_0$, pero esto es una contradicción. \qedhere
 \end{itemize}
 \end{proof}

 \item Sea $A\subset \R$ y $A\neq \emptyset$. Una cota superior $M$ de $A$, es el supremo de $A$ si y solo si para toda $\varepsilon>0$ existe $a_\varepsilon \in A$ tal que $M-\varepsilon<a_\varepsilon$.

  \begin{proof}\leavevmode
  \begin{itemize}
  \item[$\Rightarrow)$] Sea $M$ el supremo de $A$ y $\varepsilon>0$. Como $M<M+\varepsilon$ implica que $M-\varepsilon<M$, entonces $M-\varepsilon$ no es una cota superior de $A$, por lo que $\exists a_\varepsilon$ tal que $a_\varepsilon>M-\varepsilon$.
  \item[$\Leftarrow)$] Sea $M$ una cota superior de $A$ tal que $\forall \varepsilon>0, \exists a_{\varepsilon}$ tal que $M-\varepsilon<a_{\varepsilon}$. Supongamos que $M$ no es el supremo de $A$, entonces $b\in \R$, el cual es una cota superior de $A$, tal que $a_\varepsilon \leq b < M$. Elegimos $\varepsilon = M-b$, con lo que $M-(M-b)<a_\varepsilon$, es decir, $b<a_{\varepsilon}$, pero esto es una contradicción. Por tanto, $M$ es el supremo de $A$. \qedhere
  \end{itemize}
  \end{proof}
  
\item Sea $A\subseteq B\subseteq \R$ no vacíos y $B$ es acotado; se verifica que \[\inf (B) \leq \inf (A) \leq \sup (A) \leq \sup (B)\] (El supremo preserva el orden y el ínfimo lo invierte).

  \begin{proof}\leavevmode
    \begin{enumerate}[label=\roman*)]
      \item Sea $x\in A$. Se tiene que $x\in B$, y por definición, $\inf (B) \leq x$, por lo que $\inf (B)$ es cota inferior de $A$. Por definición, el ínfimo de $A$ es mayor o igual que todas sus cotas inferiores, es decir, $\inf (B) \leq \inf (A)$.
      \item Sea $x\in A$. Por definición, $\inf (A) \leq x \leq \sup (A)$, por lo que $\inf (A) \leq \sup (A)$. (El ínfimo es menor o igual que el supremo).
      \item Sea $x\in A$. Se tiene que $x\in B$, y por definición, $x\leq \sup (B)$, por lo que $\sup (B)$ es cota superior de $A$, y por definición, el supremo de $A$ es menor o igual que todas sus cotas superiores, es decir, $\sup (A) \leq \sup (B)$. \qedhere
    \end{enumerate}
  \end{proof}
  
  \bfit{Corolario:}\begin{enumerate}[label=\roman*)]
  \item $\inf (B) \leq \sup (A)$.
  \item $\inf (A) \leq \sup (B)$.
  \end{enumerate}
  \item Sea $\emptyset \neq B\subseteq \R$, se verifica que $\forall \epsilon > 0$, $\exists b_\epsilon\in B$ tal que $b_\epsilon < \inf (B) + \epsilon$.
  
  \begin{proof}\leavevmode
    Sea $\epsilon >0$. Notemos que
    \begin{align*}
      0 &< \epsilon\\
      \inf (B) &< \inf (B) + \epsilon
    \end{align*}
    Por lo que $\inf (B) +\epsilon$ no es una cota inferior de $B$, entonces $\exists b_\epsilon\in B$ tal que $b_\epsilon < \inf (B) + \epsilon$.
  \end{proof}
  
  \item Sea $\emptyset \neq A\subseteq \R$, se verifica que $\forall \epsilon > 0$, $\exists a_\epsilon\in A$ tal que $\sup (A) - \epsilon < a_\epsilon$.
  \begin{proof}\leavevmode
    Sea $\epsilon>0$. Notemos que 
    \begin{align*}
      0 &< \epsilon\\
      \sup (A) &< \epsilon + \sup (A)\\
      \sup (A) - \epsilon &< \sup (A)
    \end{align*}
    Por lo que $\sup (A)-\epsilon$ no es una cota superior de $A$, por lo que $\exists a_\epsilon\in A$ tal que $\sup (A) - \epsilon < a_\epsilon$.
  \end{proof}
  
  \item Sea $A, B\subseteq \R$ no vacíos, tales que $a\leq b$, $\forall a\in A$ y $\forall b\in B$, entonces $\sup(A)\leq \inf(B)$.
  
  \begin{proof}\leavevmode
    
  Pr definición, $\sup (A) \leq b, \forall b\in B$.
  
  Supongamos que $\inf(B) < \sup(A)$, entonces $\sup(A) - \inf(B) > 0$. Sea $\epsilon = \sup(A) - \inf(B)$, entonces $\exists b_\epsilon\in B$ tal que
    \begin{align*}
      b_\epsilon &< \inf(B) + \epsilon\\
      b_\epsilon &< \inf(B) + \sup(A) - \inf(B)\\
      b_\epsilon &< \sup(A) && \contradiction
    \end{align*}
  \end{proof}
  
  \item Sea $A+B \defined \set{a+b: a\in A, b\in B}$, entonces:
  \begin{itemize}
    \item $\sup(A + B) = \sup(A) + \sup(B)$.
    \item $\inf(A + B) = \inf(A) + \inf(B)$.
  \end{itemize}
  
  \item Sea $A, B\subseteq \R^+$ no vacíos, $A\cdot B \defined \set{a\cdot b: a\in A, b\in B}$, entonces
  \begin{itemize}
    \item $\sup(A\cdot B) = \sup(A) \cdot \sup(B)$.
    \item $\inf(A\cdot B) = \inf(A) \cdot \inf(B)$.
  \end{itemize}
  
  \begin{proof}\leavevmode
    Sea $a\in A$ y $b\in B$. Tenemos que $a\leq \alpha$ y $b\leq \beta$, de donde sigue que $ab \leq \alpha\beta$ (*), por lo que $AB$ está acotado superiormente por $\alpha\beta$.
    
    Ahora, sea $\epsilon_1>0$ y $\epsilon_2 = \min \left(\frac{\alpha}{2}, \frac{\beta}{2}, \frac{\epsilon_1}{\alpha+\beta+1}\right)$, es claro que $\epsilon_2>0$. Luego, por definición, $\exists a\in A$ tal que $a > \alpha - \epsilon_2 > 0$ y $\exists b\in B$ tal que $b>\beta - \epsilon_2 >0$. Por lo que
    \begin{align*}
      ab &> (\alpha - \epsilon_2) (\beta - \epsilon_2) \\
      &= \alpha\beta - (\alpha + \beta)\epsilon_2 + \epsilon^2_2\\
      &> \alpha\beta - (\alpha+\beta)\epsilon_2\\
      &> \alpha\beta - \epsilon_1 && \text{(**)}
    \end{align*}
    Como esta desigualdad se verifica para cualquier $\epsilon_1>0$, se tiene que $\alpha\beta = \sup(AB)$.
  \end{proof}
  
  (*) Por Cálculo 1; se tiene que $a \leq \alpha \Rightarrow ab \leq \alpha b$, pues $b>0$ y $b \leq \beta \Rightarrow \alpha b \leq \alpha \beta$, pues $\alpha>0$. Por transitividad, $ab\leq \alpha \beta$.
  
  (**) Si $\epsilon_2 = \frac{\epsilon_1}{\alpha+\beta+1}$ se tiene que
  \begin{align*}
    (\alpha+\beta)\epsilon_2 = \frac{(\alpha+\beta)\epsilon_1}{\alpha+\beta+1} < \epsilon_1
  \end{align*}
  Por lo que $-(\alpha+\beta)\epsilon_2 > - \epsilon_1$.
  
  \item Sea $A\subseteq \R^+$ no vacío, $\frac{1}{A} \defined \set{\frac{1}{a}: a\in A}$, entonces
  \begin{itemize}
    \item $\frac{1}{\sup(A)} = \inf(\frac{1}{A})$.
  \end{itemize}
  
  \item Sea $\gamma=\text{ínf}(A)$ y suponga que $\gamma>0$. Defínase ahora al siguiente conjunto: \[\frac{B}{A} \defined \set{\frac{b}{a}| a\in A, b\in B}\]
  
  Demuestre que $\frac{B}{A}$ está acotado superiormente; más aún, diga quién es el supremo de $\frac{B}{A}$ y justifique su afirmación.
  
  \begin{proof}\leavevmode
    Definimos el conjunto $A^{-1} \defined \set{\frac{1}{a}| a\in A}$. Sea $a\in A$, se tiene que $\gamma \leq a$. Como $\gamma$ es positivo, de esta desigualdad se sigue que $\frac{1}{a} \leq \frac{1}{\gamma}$, por lo que $\frac{1}{\gamma}$ es una cota superior de $A^{-1}$. Ahora, consideremos que $a'$ es una cota superior de $A^{-1}$, entonces $\frac{1}{a} \leq a'$, y se tiene que $0<\frac{1}{a}$, de donde obtenemos que $\frac{1}{a'} \leq a$, por lo que $\frac{1}{a'}$ es cota inferior de $A$. Por definición de ínfimo, $\frac{1}{a'}\leq \gamma$, lo que implica que $\frac{1}{\gamma} \leq a'$, es decir, $\frac{1}{\gamma}$ es la menor de las cotas superiores de $A^{-1}$, osea $\sup(A^{-1})=\frac{1}{\gamma}$.
    
    Notemos que $\frac{b}{a} = b\frac{1}{a}$, así que definimos el conjunto $BA^{-1}\defined \set{b\frac{1}{a}|b\in B, \frac{1}{a}\in A^{-1}}$, y por el inciso (ii) de este problema, se tiene que $\sup(BA^{-1})=\beta \frac{1}{\gamma}$, es decir, $\sup(\frac{B}{A})=\frac{\beta}{\gamma}$.
  \end{proof}
  
  \item Sea $A\subseteq \R$ no vacío, $rA \defined \set{r a| a\in A}$,
  \begin{enumerate}[label=\Roman*)]
    \item Si $r\geq 0$,
    \begin{enumerate}[label=\roman*)]
      \item $\inf(r A) = r\inf(A)$.
      \item $\sup(r A) = r\sup(A)$
    \end{enumerate}
    \item Si $r\leq 0$,
    \begin{enumerate}[label=\roman*)]
      \item $\inf(rA)=r\sup(A)$.
      \item $\sup(rA)=r\sup(A)$.
    \end{enumerate}
  \end{enumerate}
  \bfit{Corolario:} Sea $r=-1$ y la notación $-A\defined (-1)A\set{-a:a\in A}$,
  \begin{enumerate}[label=\roman*)]
  \item $\inf(-A) = -\sup(A)$.
  \item $\sup(-A) = -\inf(A)$.
  \end{enumerate}
  
  \item $\inf(A)=-\sup(-A)$.
  \item $\sup(A)=-\inf(-A)$.
  
\end{enumerate}

\subsection*{Densidad}

\bfit{Definición:} Sea $A\subseteq \R$. Decimos que $A$ es denso (en $\R$), o que cumple la propiedad de densidad, si $\forall x,y \in \R$ con $x<y$, $\exists a\in A$ tal que $a\in (x, y)$.

\textbf{Observación:} $\R$ es denso, pues $\R\subseteq \R$ y para cada $x<y$ se tiene que $x<\frac{x+y}{2}<y$, es decir, $\frac{x+y}{2}\in (x, y)$, y también $\frac{x+y}{2}\in \R$.