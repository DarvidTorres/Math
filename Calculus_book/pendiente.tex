\part*{Pendiente}

\subsection*{Axioma del supremo}



\subsection*{Propiedad Arquimediana del conjunto de los números reales}

Para cada número real $x$ existe un número natural $n$ tal que $x<n$.

\textbf{Demostración:}

Supongamos que existe $x\in \R$ tal que $n\leq x, \forall n\in \N$. Notemos que $x$ es una cota superior de $\N$, pero esto contradice el teorema que establece que el conjunto de los números naturales no está acotado superiormente. Por tanto, se satisface la propiedad arquimediana del conjunto de los números reales. \qed

\subsection*{Lista de Ejercicios 9 (LE9)}

\begin{enumerate}[label=\alph*)]
 \item Si $S \coloneqq \set{\frac{1}{n}: n\in \N}$, entonces $\inf{S=0}$.
 \item Si $t>0$, entonces $\exists n\in \N$ tal que $0<\frac{1}{n}<t$.
 \item Si $y>0$, entonces $\exists n\in \N$ tal que $n-1\leq y< n$.
 \item Sea $x\in \R$, demuestre que $\exists! \, n\in \Z$ tal que $n\leq x<n+1$.
\end{enumerate}

\subsubsection*{Demostración}

\begin{enumerate}[label=\alph*)]
 \item Sabemos que $0<n^{-1}\leq 1, \forall n\in \N$, por lo que $S$ está acotado inferiormente por $0$; de esto sigue que $S$ tiene ínfimo. Sea $w\coloneqq \inf{S}$. Por definición, $\frac{1}{n}\geq w\geq 0, n\in \N$. Supongamos que $w>0$. Por la propiedad arquimediana $\exists n_0$ tal que $\frac{1}{w} < n_0$, de donde sigue que $w<\frac{1}{n_0}$ con $\frac{1}{n_0} \in S$, lo cual es una contradicción. Por tanto, $w=0$. \qed
 
 \item Por la propiedad arquimediana $\exists n$ tal que $\frac{1}{t}<n$. Como $n$ y $t$ son mayores que $0$, sigue que $0<\frac{1}{n}<t$. \qed
 
 \item Por la propiedad arquimediana, el conjunto $E\coloneqq \set{m\in \N: y<m}$ es no vacío. Además, por el principio del buen orden, $\exists n\in E$ tal que $n\leq m, \forall m\in E$. Notemos que $n-1<n$, por lo que $n-1\notin E$, lo que implica que $n-1\leq y<n$. \qed
 
 \item Definimos el conjunto $A\coloneqq \set{n\in \Z: x<n}$. Por la propiedad arquimediana $\exists n_0 \in \N$ tal que $x<n_0$, así $n_0\in A$, por lo que $A\neq \emptyset$. Sabemos también que $A$ está acotado inferiormente, de manera que $A$ tiene elemento mínimo. Sea $n$ el elemento mínimo de $A$. Notemos que $n-1<n$, de donde sigue que $n-1\leq x<n$. Luego, $n-1\in \Z$, al que definimos como $m=n-1$, por lo que $m\leq x<m+1$.
 
 Finalmente, supongamos que $\exists m, n\in \Z$ tales que $m\leq x<m+1$ y $n\leq x<n+1$. Si $m\neq n$, sin pérdida de generalidad, $m>n$. Por ello, \begin{align*}
  n < m &\leq x<n+1 \\
  n < m &<n+1 \\
  0 < m-n &<1
 \end{align*}  
 Lo que contradice la cerradura de la suma en $\Z$. Por tanto, $m=n$, es decir, el número entero que satisface $n\leq x<n+1$ es único. \qed
% 
 %\textbf{Demostración alternativa:}
%
 %Definimos el conjunto $A\coloneqq \set{n\in \Z: n\leq x}$. Por la propiedad arquimediana $\exists n_0\in \N$ tal que $n_0>x$. Observemos que \begin{enumerate}[label=\roman*)]
 %\item Si $x\geq 0$, $-x\leq 0$ y $-x<n_0$, de donde sigue que $-n_0<x$, por lo que $-n_0\in A$.
 %\item Si $x<0$, $x<n_0$, de donde sigue que $-n_0\leq x$, por lo que $-n_0\in A$.
 %\end{enumerate} Consecuentemente, $A$ es no vacío. También sabemos que $A$ está acotado superiormente, por el axioma del supremo, $A$ tiene supremo. Sea $m\coloneqq \sup(A)$. Por definición, $m\leq x$. Notemos que $m+1>m$. Luego, si $m+1\leq x$ tendríamos que $m+1\in A$ pero como $m$ es el supremo de $A$ seguiría que $m \geq m+1$, lo cual es una contradicción, entonces debe ser el caso que $m\leq x<m+1$.\qed
\end{enumerate}

\section*{Funciones}

\bfit{Definición:}  Sean $a$ y $b$ objetos cualesquiera, definimos la pareja ordenada $(a,b)$ como sigue: \[
 (a,b)\coloneqq \set{\set{a}, \set{a,b}}\]
Al objeto $a$ lo llamaremos primer componente de la pareja ordenada $(a,b)$ y al objeto $b$ lo llamaremos segundo componente de la pareja ordenada $(a,b)$.

\textbf{Teorema:} $(a,b)=(c,d)$ si y solo si $a=c$ y $b=d$.

\textbf{Demostración:} Pendiente





\section*{Sucesiones}

\bfit{Definición:}  Una sucesión es una función %$X: n\in \N \mapsto x_n \in \R$
\begin{align*}
 X: \ & \N \to \R \\
 \ &  n \mapsto x_n 
\end{align*}
%
Llamamos a $x_n$ el n-ésimo término. Otras etiquetas para la sucesión son $(x_n)$, $(x_n:n\in \N)$, que denotan orden y se diferencian del rango de la función $\{x_n:n\in \N\}\subseteq \R$.

\bfit{Definición:}  Una sucesión $(x_n)$ es convergente si $\exists \ell \in \R$ tal que para cada $\varepsilon>0$ existe un número natural $n_\varepsilon$ (que depende de $\varepsilon$) de modo que los términos $x_n$ con $n\geq n_\varepsilon$ satisfacen que $|x_n-\ell|<\varepsilon$.

Decimos que $(x_n)$ converge a $\ell \in \R$ y llamamos a $\ell$ el límite de la sucesión y escribimos $\lim (x_n) = \ell$.

%Notemos que por LE5(a) se cumple que $x_n=x$ con $n\geq m$. Esto es falso.
\bfit{Definición:}  Una sucesión es divergente si no es convergente.

\bfit{Definición:}  Una sucesión $(x_n)$ está acotada si $\exists M\in \R^+$ tal que $|x_n|\leq M, \forall n\in \N$.

\subsection*{Lista de Ejercicios 10 (LE10)}

Demuestre lo siguiente:

\begin{enumerate}[label=\alph*)]
 \item El límite de una sucesión convergente es único.
 \item Toda sucesión convergente está acotada.
\end{enumerate}

\subsubsection*{Demostración}

\begin{enumerate}[label=\alph*)]
 \item Sean $\ell$ y $\ell'$ límites de la sucesión $(x_n)$. Tenemos que $\forall \varepsilon>0$, existen $n',n'' \in \N$ tales que $|x_{n\geq n'}-\ell|<\varepsilon$ y $|x_{n\geq n''}-\ell'|<\varepsilon$. Sin pérdida de generalidad, si $n'<n''$, los términos $x_n$ con $n\geq n''>n'$ satisfacen que \begin{align*}
  |x_n-\ell| &<\varepsilon && \text{(1)}\\
  |x_n-\ell'| &<\varepsilon && \text{(2)}
 \end{align*}
 Por (c) de LE4, se cumple que $|x_n-\ell'|=|\ell'-x_n|$ y por esto, \begin{align*}
  |\ell'-x_n|<\varepsilon && \text{(3)}
 \end{align*}
 Tomando (1) y (3), por (d) de LE3, se verifica que \begin{align*}
  |\ell'-x_n| + |x_n-\ell| &< 2\varepsilon
 \end{align*}
 Y, por la desigualdad del triángulo, tenemos que \begin{align*}
  \big|(\ell'-x_n)+(x_n-\ell)\big| &\leq |\ell'-x_n| + |x_n-\ell|\\
  |\ell'-\ell| &\leq |\ell'-x_n| + |x_n-\ell|
 \end{align*}
 De este modo, $|\ell'-\ell| < 2\varepsilon$. Como esta desigualdad se cumple para todo $\varepsilon>0$, en particular se verifica para $\varepsilon=\nicefrac{\varepsilon_0}{2}$ con $\varepsilon_0>0$ arbitrario pero fijo, así obtenemos que \begin{align*}
  |\ell'-\ell| &< 2 \left(\frac{\varepsilon_0}{2}\right)\\
  |\ell'-\ell| &< \varepsilon_0
 \end{align*}
 Finalmente, como $\varepsilon_0$ es arbitrario, por (a) de LE5, sigue que $\ell'=\ell$. Por tanto, el límite de cada sucesión convergente es único. \qed
%
 \item Sea $(x_n)$ una sucesión convergente. Por definición, $\forall \epsilon>0, \exists n_\varepsilon \in \N$ tal que los términos $x_n$ con $n\geq n_\varepsilon$ satisfacen que \begin{align*}
  |x_n - \ell| &< \epsilon \\
  |x_n - \ell| + |\ell| &< \epsilon + |\ell|
 \end{align*}
 Luego, por la desigualdad del triángulo, \begin{align*}
  \big|(x_n-\ell)+\ell\big| &\leq |x_n-\ell| + |\ell|\\
  |x_n| &\leq |x_n-\ell| + |\ell|
 \end{align*}
 Por transitividad, $|x_n|< \epsilon + |\ell|$, lo que implica que $\{x_{n\geq n_\varepsilon}\}$ está cotado superiormente.
 
 Por otra parte, el conjunto de índices $n<n_\varepsilon$ está acotado, y por esto, $\{x_{n<n_\varepsilon}\}$ es finito, por lo que tiene cota superior. %proof https://math.stackexchange.com/questions/548806/a-finite-set-always-has-a-maximum-and-a-minimum
 
 Finalmente, el conjunto $\{x_{n<n_\varepsilon}\} \cup \{x_{n\geq n_\varepsilon}\}$ está acotado superiormente, y por tanto, $(x_n)$ está acotada. \qed
\end{enumerate}

\textbf{\textit{Teorema.}} Todo conjunto finito no vacío tiene elemento mínimo y elemento máximo, es decir, para todo conjunto finito $A\neq \emptyset$, $\exists m,M\in A$ tales que $m\leq a\leq M, \forall a\in A$.

\textbf{Demostración:} Sea $n\in \N$ y $A \coloneqq \{a_1, \dots, a_n\}$ no vacío.

Procedemos por inducción sobre el número de elementos de $A$. \begin{enumerate}[label=\roman*)]
 \item Si $n=1$, tenemos $A\coloneqq\{a_1\}$, por lo que $m=a_1$ y $M=a_1$ cumplen la condición requerida.
 \item Supongamos que la proposición se cumple para $n=k$.
 \item Si $n=k+1$, tenemos $A\coloneqq \{a_1, \dots, a_k, a_{k+1}\}$. Luego, por hipótesis de inducción, el conjunto \[A' \coloneqq A \setminus \{a_{k+1}\} = \{a_1, \dots, a_k\}\] tiene elemento mínimo y máximo, es decir, $\exists m',M'\in A'$ tales que $\forall a'\in A', m'\leq a' \leq M'$.
 
 Notemos que para cada $a\in A$ tenemos $a=a_{k+1}$ o $a\in A'$. Por tricotomía, $a_{k+1}$ cumple con alguno de los siguientes casos:
 \begin{enumerate}[label=\alph*)]
  \item Si $a_{k+1}<m'$, tenemos que $m=a_{k+1}<m'\leq a' \leq M'=M$.
  \item Si $m' \leq a_{k+1}\leq M'$, entonces $m=m'\leq a_{k+1} \leq M'=M$.
  \item Si $m'<a_{k+1}$, tenemos que $m=m'\leq a' \leq M'<a_{k+1}=M$.
 \end{enumerate}
 En cualquier caso $\exists m,M\in A$ tales que $m\leq a\leq M, \forall a\in A$. \qed
\end{enumerate}
