\part*{Pendiente}


\subsection*{Lista de Ejercicios 8 (LE8)}

Sean $a$ y $b$ números reales, demuestre lo siguiente:

\begin{enumerate}[label=\alph*)]
 \item $0 \leq a^{2n} \, \forall n\in \N$.
 \item Si $0\leq a$, entonces $ 0 \leq a^n, \, \forall n\in \N$.
 \item Si $0 \leq a <b$, entonces $a^n < b^n, \, \forall n\in \N$.
 \item Si $0 \leq a <b$, entonces $a^n \leq ab^n < b^n \, \forall n\in \N$.
 \item Si $0<a<1$, entonces $a^n<a \, \forall n\in \N$.
 \item Si $1<a$, entonces $a<a^n \, \forall n\in \N$.
\end{enumerate}

\subsubsection*{Demostración}

\begin{enumerate}[label=\alph*)]

 %A
 \item Pendiente

 %B
 \item Por inducción matemática. 

 \begin{enumerate}[label=\roman*)]
  \item Verificamos que se cumple para $n=1$. \begin{align*}
  0 &\leq a^1 \\
  0 &\leq a
  \end{align*}
  \item Suponemos que se cumple para $n=k$, para algún $k \in \N$. Es decir,  suponemos que \[0 \leq a^k\]
  \item Probaremos a partir de (ii) que $0 \leq a^{k+1}$. En efecto, por hipótesis de  inducción \begin{align*}
  0 &\leq a^k \\
  0 \cdot a &\leq a^k \cdot a \\
  0 &\leq a^{k+1}
  \end{align*}
 \end{enumerate}

 %C
 \item Por inducción matemática.
 \begin{enumerate}[label=\roman*)]
  \item Verificamos que se cumple para $n=1$. \begin{align*}
  a^1 &< b^1 \\
  a &< b
  \end{align*}
  \item Suponemos que se cumple para $n=k$, para algún $k\in \N$. Es decir, suponemos que \[a^k < b^k\]
  \item Probaremos, a partir de (ii) que $a^{k+1} < b^{k+1}$. En efecto, por (c) de LE5, garantizamos que $0 \leq a^k$, lo que nos permite, por (a) de LE5, afirmar que
  \begin{align*}
  a^k \cdot a &< b^k \cdot b \\
  a^{k+1} &< b^{k+1}
  \end{align*}
 \end{enumerate}

 %D
 \item Tenemos que $a<b$, como $0\leq a<b$, sigue que $0<b$, entonces $a\cdot b < b\cdot b$, osea $ab<b^2$. Luego, $a \cdot a \leq ab$. Finalmente, $a^2\leq ab < b^2$.
 
 %E
 \item Pendiente
 
 %F
 \item Pendiente

\end{enumerate}

\bfit{Definición:}  Sea $A$ un subconjunto no vacío de $\R$, decimos que $A$ está acotado:
\begin{itemize}
 \item superiormente si $\exists M\in \R$ tal que $a \leq M, \forall a\in A$. En este caso decimos que $M$ es cota superior de $A$.

 \item inferiormente si $\exists m\in \R$ tal que $m \leq a, \forall a\in A$. En este caso decimos que $m$ es cota inferior de $A$.

 \item si está acotado superior e inferiormente.
 %\item si $\exists n\in \R$ tal que $|a|\leq n,\forall a \in A$. En este caso decimos que $n$ es una cota de $A$.
\end{itemize}

\bfit{Definición:}  Sea $A\subset \R$ tal que $A$ es no vacío y está acotado superiormente, decimos que un número real $S$ es supremo de $A$ si $S$ satisface las siguientes condiciones:
\begin{itemize}
 \item $S$ es cota superior de $A$.
 \item Si $K$ es cota superior de $A$, entonces $S\leq K$.% ($S$ es la cota superior más pequeña de $A$).
\end{itemize}

En este caso escribimos $S=\sup(A)$.

\bfit{Definición:} . Sea $A$ un subconjunto no vacío del conjunto de los números reales, acotado inferiormente, decimos que un número real $L$ es ínfimo de $A$ si $L$ satisface las siguientes condiciones: \begin{itemize}
 \item $L$ es cota inferior de $A$.
 \item Si $K$ es cota inferior de $A$, entonces $K\leq L$, es decir, $L$ es la cota inferior más grande de $A$.
\end{itemize}

En este caso escribimos $M=\inf(A)$

\subsection*{Lista de ejercicios 8 (LE8)}

%Falso o verdadero: \begin{enumerate}[label=\arabic*.]
% \item Si $E$ es un subconjunto de $\R$ acotado superiormente, entonces $E$ es un conjunto acotado.
% Falso. Consideremos el conjunto $\R\backslash \R^+$, el cual es un subconjunto de $\R$, y es no vacío, pues $-1\in \R\backslash \R^+$. Además, $b\leq 0, \forall b\in \R\backslash\R^+$, por lo que el conjunto está acotado superiormente. Supongamos que el conjunto propuesto está acotado. Es decir, suponemos que $\exists m$ tal que $|b|\leq m, \forall b\in \R\backslash \R^+$. Por (f) de LE4, $-m \leq b$ y, por transitividad, $-m\leq 0$, de donde sigue que $-m-1\leq -1$, pero $-1<0$, entonces $-m-1<0$, lo que implica que $-m-1\in \R\backslash \R^+$, por lo que $|-m-1|\leq m$. Luego, notemos que $|-m-1|=-(-m-1)$, es decir, tenemos que $m+1\leq m$, pero de esto se concluye que $1\leq 0$, lo cual es una contradicción. Por tanto, aunque $\R\backslash \R^+$ está acotado superiormente, no está acotado.
% \item Si $E$ es un subconjunto acotado de $\R$, entonces $E$ está acotado superiormente e Inferiormente.
% Verdadero. Sea $E$ un subconjunto no vacío de $\R$. Si $E$ está acotado, entonces $\exists m$ tal que $|b|\leq m,\forall b \in E$. Por (f) de LE4, $-m\leq b \leq m$, por lo que el conjunto está acotado superiormente e inferiormente.
%\end{enumerate}
Demuestre lo siguiente:

\begin{enumerate}
 \item Sea $A$ un subconjunto no vacío de $\R$. $A$ está acotado si y solo si $A$ está acotado superior e inferiormente.
 
 \begin{proof} \leavevmode
  \begin{itemize}
   \item[$\Rightarrow)$] ads
   \item[$\Leftarrow)$] asdf
  \end{itemize}
 \end{proof}

 \item Sea $A$ un subconjunto no vacío de $\R$, si $A$ tiene supremo, este es único.
 
 \begin{proof} 
  Supongamos que $s_1$ y $s_2$ son supremos de $A$. Como $s_1$ es una cota superior de $A$ y $s_2$ es elemento supremo, entonces $s_2\leq s_1$. Similarmente, $s_1\leq s_2$. Por tanto, $s_1=s_2$. 
 \end{proof}

 \item Sea $A$ un subconjunto no vacío de $\R$, si $A$ tiene ínfimo, este es único.
 
 \begin{proof} 
  Supongamos que $m_1$ y $m_2$ son ínfimos de $A$. Como $m_1$ es una cota superior de $A$ y $m_2$ es elemento ínfimo, entonces $m_1\leq m_2$. Similarmente, $m_2\leq m_1$. Por tanto, $m_1=m_2$. 
 \end{proof}

 \item Una cota superior $M$ de un conjunto no vacío $S$ de $\R$ es el supremo de $S$ si y solo si para toda $\varepsilon>0$ existe $s_\varepsilon \in S$ tal que $M-\varepsilon<s_\varepsilon$.

 \begin{proof} 
  \begin{enumerate}[label=\roman*)]
   \item Sea $M$ una cota superior de $S$ tal que $\forall \epsilon>0, \exists s_{\epsilon}$ tal que $M-\epsilon<s_{\epsilon}$. Si $M$ no es el supremo de $S$, tendríamos que $\exists V$ tal que $s_\epsilon \leq V < M$. Elegimos $\epsilon = M-V$, con lo que $V<s_{\epsilon}$, lo que contradice nuestra hipótesis. Por tanto, $M$ es el supremo de $S$.
   \item Sea $M$ el supremo de $S$ y $\epsilon>0$. Como $M<M+\epsilon$, entonces $M-\epsilon$ no es una cota superior de $S$, por lo que $\exists s_\epsilon$ tal que $s_\epsilon>M-\epsilon$. \qedhere
   \end{enumerate} 
 \end{proof}
\end{enumerate}

\subsection*{Axioma del supremo}

Todo subconjunto no vacío del conjunto de los números reales que sea acotado superiormente tiene supremo.

\textbf{\textit{Teorema.}} El conjunto de los números naturales no está acotado superiormente.

\textbf{Demostración:}

Supongamos que el conjunto de los números naturales está acotado superiormente. Entonces existe un número real $M$ tal que $n\leq M, \forall n\in \N$. Como el conjunto de los números naturales es no vacío, entonces, por el axioma del supremo, $\N$ tiene supremo.

Sea $L\coloneqq \sup{(\N)}$. Como $L-1$ no es cota superior de $\N$, ya que $L>L-1$ y $L$ es la cota superior más pequeña, existe un núero natural $n_0$ tal que $n_0>L-1$, lo cual implica que $n_0+1<L$, pero esto contradice la hipótesis	de que $L$ es supremo de $\N$. Por tanto, el conjunto de los números naturales no está acotado superiormente. \qed

\textbf{\textit{Teorema.}} Si $A\subseteq \R, A\neq \emptyset$ y $A$ está acotado inferiormente, entonces $A$ tiene ínfimo.

\textbf{Demostración:}

Sea $A\subseteq \R, A\neq \emptyset$ y $A$ está acotado inferiormente. El conjunto $-A \coloneqq \set{-a: a\in A}$ está acotado superiormente y, por el axioma del supremo, $-A$ tiene supremo. Sea $M\coloneqq \sup{(A)}$, entonces $M\geq -a, \forall -a\in -A$. Notemos que $-M\leq a, \forall a\in A$, esto es $-M$ es el ínfimo de $A$. \qed

\subsection*{Lista de Ejercicios}

\begin{enumerate}[label=\alph*)]
  \item Sea $A\subseteq B\subseteq \R$ no vacíos y $B$ es acotado; se verifica que \[\inf (B) \leq \inf (A) \leq \sup (A) \leq \sup (B)\]

  \begin{proof}\leavevmode
    \begin{enumerate}[label=\roman*)]
      \item Sea $x\in A$. Se tiene que $x\in B$, y por definición, $\inf (B) \leq x$, por lo que $\inf (B)$ es cota inferior de $A$. Luego, $\inf (B) \leq \inf (A)$.
      \item Sea $x\in A$. Por definición, $\inf (A) \leq x \leq \sup (A)$, por lo que $\inf (A) \leq \sup (A)$.
      \item Sea $x\in A$. Se tiene que $x\in B$, y por definición, $x\leq \sup (B)$, por lo que $\sup (B)$ es cota superior de $A$, y por definición, $\sup (A) \leq \sup (B)$.
    \end{enumerate}
  \end{proof}
  
  \bfit{Corolario:}\begin{enumerate}[label=\roman*)]
  \item $\inf (B) \leq \sup (A)$.
  \item $\inf (A) \leq \sup (B)$.
  \end{enumerate}
  \item Sea $\emptyset \neq B\subseteq \R$, se verifica que $\forall \epsilon > 0$, $\exists b\in B$ tal que $\inf (B) \leq b < \inf (B) + \epsilon$
  
  \begin{proof}\leavevmode
    Sea $\epsilon >0$. Notemos que
    \begin{align*}
      0 &< \epsilon\\
      \inf (B) &< \inf (B) + \epsilon
    \end{align*}
    Por lo que $\inf (B) +\epsilon$ no es una cota inferior de $B$, entonces $\exists b\in B$ tal que $b < \inf (B) + \epsilon$.
  \end{proof}
  
  \item Sea $\emptyset \neq A\subseteq \R$, se verifica que $\forall \epsilon > 0$, $\exists a\in A$ tal que $\sup (A) - \epsilon < a \leq \sup (A)$.
  \begin{proof}\leavevmode
    Sea $\epsilon>0$. Notemos que 
    \begin{align*}
      0 &< \epsilon\\
      \sup (A) &< \epsilon + \sup (A)\\
      \sup (A) - \epsilon &< \sup (A)
    \end{align*}
    Por lo que $\sup (A)-\epsilon$ no es una cota superior de $A$, por lo que $\exists a\in A$ tal que $\sup (A) - \epsilon < a$.
  \end{proof}
  
  \item Sea $A, B\subseteq \R$ no vacíos, tales que $a\leq b$, $\forall a\in A$ y $\forall b\in B$, entonces $\sup(A)\leq \inf(B)$.
  
  \begin{proof}\leavevmode
  Supongamos que $\inf(B) < \sup(A)$, entonces $\sup(A) - \inf(B) > 0$. Sea $\epsilon = \sup(A) - \inf(B)$, entonces $\exists b_\epsilon\in B$ tal que
    \begin{align*}
      \inf(B) &\leq b_\epsilon < \inf(B) + \epsilon\\
      \inf(B) &\leq b_\epsilon < \inf(B) + \sup(A) - \inf(B)\\
      \inf(B) &\leq b_\epsilon < \sup(A)
    \end{align*}
    Por lo que $b_\epsilon$ no es una cota superior de $A$, pero eso contradice la hipótesis de que $a\leq b$, $\forall a\in A$ y $\forall b\in B$.
  \end{proof}

\end{enumerate}

\subsection*{Propiedad Arquimediana del conjunto de los números reales}

Para cada número real $x$ existe un número natural $n$ tal que $x<n$.

\textbf{Demostración:}

Supongamos que existe $x\in \R$ tal que $n\leq x, \forall n\in \N$. Notemos que $x$ es una cota superior de $\N$, pero esto contradice el teorema que establece que el conjunto de los números naturales no está acotado superiormente. Por tanto, se satisface la propiedad arquimediana del conjunto de los números reales. \qed

\subsection*{Lista de Ejercicios 9 (LE9)}

\begin{enumerate}[label=\alph*)]
 \item Si $S \coloneqq \set{\frac{1}{n}: n\in \N}$, entonces $\inf{S=0}$.
 \item Si $t>0$, entonces $\exists n\in \N$ tal que $0<\frac{1}{n}<t$.
 \item Si $y>0$, entonces $\exists n\in \N$ tal que $n-1\leq y< n$.
 \item Sea $x\in \R$, demuestre que $\exists! \, n\in \Z$ tal que $n\leq x<n+1$.
\end{enumerate}

\subsubsection*{Demostración}

\begin{enumerate}[label=\alph*)]
 \item Sabemos que $0<n^{-1}\leq 1, \forall n\in \N$, por lo que $S$ está acotado inferiormente por $0$; de esto sigue que $S$ tiene ínfimo. Sea $w\coloneqq \inf{S}$. Por definición, $\frac{1}{n}\geq w\geq 0, n\in \N$. Supongamos que $w>0$. Por la propiedad arquimediana $\exists n_0$ tal que $\frac{1}{w} < n_0$, de donde sigue que $w<\frac{1}{n_0}$ con $\frac{1}{n_0} \in S$, lo cual es una contradicción. Por tanto, $w=0$. \qed
 
 \item Por la propiedad arquimediana $\exists n$ tal que $\frac{1}{t}<n$. Como $n$ y $t$ son mayores que $0$, sigue que $0<\frac{1}{n}<t$. \qed
 
 \item Por la propiedad arquimediana, el conjunto $E\coloneqq \set{m\in \N: y<m}$ es no vacío. Además, por el principio del buen orden, $\exists n\in E$ tal que $n\leq m, \forall m\in E$. Notemos que $n-1<n$, por lo que $n-1\notin E$, lo que implica que $n-1\leq y<n$. \qed
 
 \item Definimos el conjunto $A\coloneqq \set{n\in \Z: x<n}$. Por la propiedad arquimediana $\exists n_0 \in \N$ tal que $x<n_0$, así $n_0\in A$, por lo que $A\neq \emptyset$. Sabemos también que $A$ está acotado inferiormente, de manera que $A$ tiene elemento mínimo. Sea $n$ el elemento mínimo de $A$. Notemos que $n-1<n$, de donde sigue que $n-1\leq x<n$. Luego, $n-1\in \Z$, al que definimos como $m=n-1$, por lo que $m\leq x<m+1$.
 
 Finalmente, supongamos que $\exists m, n\in \Z$ tales que $m\leq x<m+1$ y $n\leq x<n+1$. Si $m\neq n$, sin pérdida de generalidad, $m>n$. Por ello, \begin{align*}
  n < m &\leq x<n+1 \\
  n < m &<n+1 \\
  0 < m-n &<1
 \end{align*}  
 Lo que contradice la cerradura de la suma en $\Z$. Por tanto, $m=n$, es decir, el número entero que satisface $n\leq x<n+1$ es único. \qed
% 
 %\textbf{Demostración alternativa:}
%
 %Definimos el conjunto $A\coloneqq \set{n\in \Z: n\leq x}$. Por la propiedad arquimediana $\exists n_0\in \N$ tal que $n_0>x$. Observemos que \begin{enumerate}[label=\roman*)]
 %\item Si $x\geq 0$, $-x\leq 0$ y $-x<n_0$, de donde sigue que $-n_0<x$, por lo que $-n_0\in A$.
 %\item Si $x<0$, $x<n_0$, de donde sigue que $-n_0\leq x$, por lo que $-n_0\in A$.
 %\end{enumerate} Consecuentemente, $A$ es no vacío. También sabemos que $A$ está acotado superiormente, por el axioma del supremo, $A$ tiene supremo. Sea $m\coloneqq \sup(A)$. Por definición, $m\leq x$. Notemos que $m+1>m$. Luego, si $m+1\leq x$ tendríamos que $m+1\in A$ pero como $m$ es el supremo de $A$ seguiría que $m \geq m+1$, lo cual es una contradicción, entonces debe ser el caso que $m\leq x<m+1$.\qed
\end{enumerate}

\section*{Funciones}

\bfit{Definición:}  Sean $a$ y $b$ objetos cualesquiera, definimos la pareja ordenada $(a,b)$ como sigue: \[
 (a,b)\coloneqq \set{\set{a}, \set{a,b}}\]
Al objeto $a$ lo llamaremos primer componente de la pareja ordenada $(a,b)$ y al objeto $b$ lo llamaremos segundo componente de la pareja ordenada $(a,b)$.

\textbf{Teorema:} $(a,b)=(c,d)$ si y solo si $a=c$ y $b=d$.

\textbf{Demostración:} Pendiente



\section*{Entorno}

\bfit{Definición:}  Sea $a$, $b$ números reales, definimos el intervalo \begin{itemize}
 \item abierto, como $(a,b)\defined \set{x\in \R| a<x<b}$
 \item semicerrado-abierto, como $[a,b)\defined \set{x\in \R|a\leq x <b}$
 \item semiabierto-cerrado, como $(a,b]\defined \set{x\in \R|a<x\leq b}$
 \item cerrado, como $[a,b]\defined \set{x\in \R|a\leq x\leq b}$.
\end{itemize}

\textbf{Definición.} Sea $\ell \in \R$ y $\varepsilon>0$. El vecindario-$\varepsilon$ de $\ell$ es el conjunto $V_\varepsilon(\ell)\defined \{ x\in \R: |x-\ell|<\varepsilon\}$.

Notemos que por el teorema para eliminar valores absolutos en algunas desigualdades, \[|x-\ell| < \epsilon= -\epsilon < x-\ell < \epsilon= \ell-\epsilon < x < \ell+\epsilon\]
Por lo que el vecindario-$\epsilon$ de $\ell$ es equivalente al intervalo abierto: $(\ell-\epsilon, \ \ell+\epsilon)$.

\subsection*{Lista de Ejercicios 5 (LE5)}

Sean $a,b \in \R$. Demuestre lo siguiente:

\begin{enumerate}[label=\alph*)]
 \item Si $0 \leq a < \varepsilon$ para toda $\varepsilon > 0$, entonces $a=0$.
 
 \begin{proof} 
  Supongamos que $0<a$, sigue que $0<\frac{a}{2}<a$. En particular, $\epsilon=\frac{a}{2}$, entonces $\varepsilon<a$, pero esto contradice nuestra hipótesis de que $a< \varepsilon$ para toda $\varepsilon>0$. Por tanto, $a=0$. 
 \end{proof}

 \item Si $a \leq b + \varepsilon$ para toda $\varepsilon > 0$, entonces $a \leq b$.
 
 \begin{proof} 
  Sean $a$ y $b$ números reales tales que $a \leq b + \varepsilon$, $\forall \varepsilon > 0$. Supongamos que $a > b$. Luego, $a-b>0$. Notemos que $(a-b) \cdot \frac{1}{2} > 0 \cdot \frac{1}{2}$, es decir $\frac{(a-b)}{2} > 0$. Sea $\varepsilon = \frac{(a-b)}{2}$, sigue que $a=2\varepsilon+b$. Además, $2\varepsilon > \varepsilon$, de donde obtenemos $2 \varepsilon + b > \varepsilon + b$. De este modo, $a > b+\varepsilon$, pero esto contradice nuestra hipótesis. Por tanto, $a \leq b$. 
 \end{proof}

 \item Si $x\in V_\varepsilon(a)$ para toda $\varepsilon>0$, entonces $x=a$.
 \begin{proof} 
  Si $x\in V_\varepsilon(a)$ tenemos que $|x-a|<\varepsilon$. Además, $0\leq |x-a|$, por definición. Así, $0\leq |x-a|<\varepsilon$. Como esta desigualdad se cumple para toda $\varepsilon>0$, por (p) de LE3, sigue que $|x-a|=0$. De este modo, $|x-a|=x-a$ con $x-a=0$. Por tanto, $x=a$. 
 \end{proof}

 \item Sea $U:=\{x: 0<x<1\}$. Si $a\in U$, sea $\varepsilon$ el menor de los números $a$ y $1-a$. Demuestre que $V_\varepsilon(a) \subseteq U$.
 \begin{proof} \leavevmode
  \begin{enumerate}[label=\roman*)]
   \item Si $a>1-a$, tenememos $\varepsilon=1-a$. Sea $y\in V_\varepsilon(a)$, entonces $|y-a|<1-a$. De (f) de LE4 sigue que $a-1<y-a<1-a$ (*). Tomando el lado derecho de (*) obtenemos $y<1$. Luego, de la hipótesis sigue que $2a>1$, osea $2a-1>0$. Del lado izquierdo de la desigualdad (*), tenemos $2a-1<y$, por lo que $0<y$.
   \item Si $1-a>a$, tenemos $\varepsilon=a$. Sea $y\in V_\varepsilon(a)$, entonces $|y-a|<a$. De (f) de LE4 sigue que $-a<y-a<a$. Sumando $a$ en esta desigualdad obtenemos $0<y<2a$. Luego, de la hipótesis sigue que $1>2a$, entonces $0<y<1$.\end{enumerate}
   En cualquier caso, $0<y<1$, lo que implica que $V_\varepsilon(a) \subseteq U$. 
 \end{proof}

 \item Demuestre que si $a\neq b$, entonces existen $U_\varepsilon(a)$ y $V_\varepsilon(b)$ tales que $U\cap V =\emptyset$.
 \begin{proof} 
  Supongamos que para toda $U_\varepsilon(a)$ y $V_\varepsilon(b)$ se cumple que $U_\varepsilon(a) \cap V_\varepsilon(b) \neq \emptyset$. Entonces, existe $x$ tal que $x\in U_\varepsilon(a)$ y $x\in V_\varepsilon(b)$. Como en ambas vecindades tenemos $\epsilon>0$ arbitraria, por (a) de LE5, sigue que $x=a$ y $x=b$, pero esto contradice el supuesto de que $a\neq b$. Por tanto, deben existir $U_\varepsilon(a)$ y $V_\varepsilon(b)$ tales que $U\cap V =\emptyset$. 
 \end{proof}
\end{enumerate}

\section*{Sucesiones}

\bfit{Definición:}  Una sucesión es una función %$X: n\in \N \mapsto x_n \in \R$
\begin{align*}
 X: \ & \N \to \R \\
 \ &  n \mapsto x_n 
\end{align*}
%
Llamamos a $x_n$ el n-ésimo término. Otras etiquetas para la sucesión son $(x_n)$, $(x_n:n\in \N)$, que denotan orden y se diferencian del rango de la función $\{x_n:n\in \N\}\subseteq \R$.

\bfit{Definición:}  Una sucesión $(x_n)$ es convergente si $\exists \ell \in \R$ tal que para cada $\varepsilon>0$ existe un número natural $n_\varepsilon$ (que depende de $\varepsilon$) de modo que los términos $x_n$ con $n\geq n_\varepsilon$ satisfacen que $|x_n-\ell|<\varepsilon$.

Decimos que $(x_n)$ converge a $\ell \in \R$ y llamamos a $\ell$ el límite de la sucesión y escribimos $\lim (x_n) = \ell$.

%Notemos que por LE5(a) se cumple que $x_n=x$ con $n\geq m$. Esto es falso.
\bfit{Definición:}  Una sucesión es divergente si no es convergente.

\bfit{Definición:}  Una sucesión $(x_n)$ está acotada si $\exists M\in \R^+$ tal que $|x_n|\leq M, \forall n\in \N$.

\subsection*{Lista de Ejercicios 10 (LE10)}

Demuestre lo siguiente:

\begin{enumerate}[label=\alph*)]
 \item El límite de una sucesión convergente es único.
 \item Toda sucesión convergente está acotada.
\end{enumerate}

\subsubsection*{Demostración}

\begin{enumerate}[label=\alph*)]
 \item Sean $\ell$ y $\ell'$ límites de la sucesión $(x_n)$. Tenemos que $\forall \varepsilon>0$, existen $n',n'' \in \N$ tales que $|x_{n\geq n'}-\ell|<\varepsilon$ y $|x_{n\geq n''}-\ell'|<\varepsilon$. Sin pérdida de generalidad, si $n'<n''$, los términos $x_n$ con $n\geq n''>n'$ satisfacen que \begin{align*}
  |x_n-\ell| &<\varepsilon && \text{(1)}\\
  |x_n-\ell'| &<\varepsilon && \text{(2)}
 \end{align*}
 Por (c) de LE4, se cumple que $|x_n-\ell'|=|\ell'-x_n|$ y por esto, \begin{align*}
  |\ell'-x_n|<\varepsilon && \text{(3)}
 \end{align*}
 Tomando (1) y (3), por (d) de LE3, se verifica que \begin{align*}
  |\ell'-x_n| + |x_n-\ell| &< 2\varepsilon
 \end{align*}
 Y, por la desigualdad del triángulo, tenemos que \begin{align*}
  \big|(\ell'-x_n)+(x_n-\ell)\big| &\leq |\ell'-x_n| + |x_n-\ell|\\
  |\ell'-\ell| &\leq |\ell'-x_n| + |x_n-\ell|
 \end{align*}
 De este modo, $|\ell'-\ell| < 2\varepsilon$. Como esta desigualdad se cumple para todo $\varepsilon>0$, en particular se verifica para $\varepsilon=\nicefrac{\varepsilon_0}{2}$ con $\varepsilon_0>0$ arbitrario pero fijo, así obtenemos que \begin{align*}
  |\ell'-\ell| &< 2 \left(\frac{\varepsilon_0}{2}\right)\\
  |\ell'-\ell| &< \varepsilon_0
 \end{align*}
 Finalmente, como $\varepsilon_0$ es arbitrario, por (a) de LE5, sigue que $\ell'=\ell$. Por tanto, el límite de cada sucesión convergente es único. \qed
%
 \item Sea $(x_n)$ una sucesión convergente. Por definición, $\forall \epsilon>0, \exists n_\varepsilon \in \N$ tal que los términos $x_n$ con $n\geq n_\varepsilon$ satisfacen que \begin{align*}
  |x_n - \ell| &< \epsilon \\
  |x_n - \ell| + |\ell| &< \epsilon + |\ell|
 \end{align*}
 Luego, por la desigualdad del triángulo, \begin{align*}
  \big|(x_n-\ell)+\ell\big| &\leq |x_n-\ell| + |\ell|\\
  |x_n| &\leq |x_n-\ell| + |\ell|
 \end{align*}
 Por transitividad, $|x_n|< \epsilon + |\ell|$, lo que implica que $\{x_{n\geq n_\varepsilon}\}$ está cotado superiormente.
 
 Por otra parte, el conjunto de índices $n<n_\varepsilon$ está acotado, y por esto, $\{x_{n<n_\varepsilon}\}$ es finito, por lo que tiene cota superior. %proof https://math.stackexchange.com/questions/548806/a-finite-set-always-has-a-maximum-and-a-minimum
 
 Finalmente, el conjunto $\{x_{n<n_\varepsilon}\} \cup \{x_{n\geq n_\varepsilon}\}$ está acotado superiormente, y por tanto, $(x_n)$ está acotada. \qed
\end{enumerate}

\textbf{\textit{Teorema.}} Todo conjunto finito no vacío tiene elemento mínimo y elemento máximo, es decir, para todo conjunto finito $A\neq \emptyset$, $\exists m,M\in A$ tales que $m\leq a\leq M, \forall a\in A$.

\textbf{Demostración:} Sea $n\in \N$ y $A \coloneqq \{a_1, \dots, a_n\}$ no vacío.

Procedemos por inducción sobre el número de elementos de $A$. \begin{enumerate}[label=\roman*)]
 \item Si $n=1$, tenemos $A\coloneqq\{a_1\}$, por lo que $m=a_1$ y $M=a_1$ cumplen la condición requerida.
 \item Supongamos que la proposición se cumple para $n=k$.
 \item Si $n=k+1$, tenemos $A\coloneqq \{a_1, \dots, a_k, a_{k+1}\}$. Luego, por hipótesis de inducción, el conjunto \[A' \coloneqq A \setminus \{a_{k+1}\} = \{a_1, \dots, a_k\}\] tiene elemento mínimo y máximo, es decir, $\exists m',M'\in A'$ tales que $\forall a'\in A', m'\leq a' \leq M'$.
 
 Notemos que para cada $a\in A$ tenemos $a=a_{k+1}$ o $a\in A'$. Por tricotomía, $a_{k+1}$ cumple con alguno de los siguientes casos:
 \begin{enumerate}[label=\alph*)]
  \item Si $a_{k+1}<m'$, tenemos que $m=a_{k+1}<m'\leq a' \leq M'=M$.
  \item Si $m' \leq a_{k+1}\leq M'$, entonces $m=m'\leq a_{k+1} \leq M'=M$.
  \item Si $m'<a_{k+1}$, tenemos que $m=m'\leq a' \leq M'<a_{k+1}=M$.
 \end{enumerate}
 En cualquier caso $\exists m,M\in A$ tales que $m\leq a\leq M, \forall a\in A$. \qed
\end{enumerate}
