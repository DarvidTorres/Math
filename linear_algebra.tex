\documentclass[11pt]{article}

\usepackage[top=0.5in,bottom=0.5in,right=0.5in,left=0.5in]{geometry}
%\usepackage[T1]{fontenc}
%\usepackage[spanish]{babel}
\usepackage{amsfonts, amssymb, amsmath, amsthm, enumitem}
\usepackage{mathtools} %\coloneqq command
\usepackage{braket}
%\usepackage[spanish]{babel}
\usepackage{adjustbox}
\usepackage{nicefrac} % for Elegant fractions in one line https://tex.stackexchange.com/questions/128496/elegant-fractions-in-one-line/128498
\usepackage{physics} % for nice brackets \qty(...) \qty[...] \qty{...} https://www.reddit.com/r/LaTeX/comments/7qkbfv/redefine_left_right/

\usepackage{array,multirow}

\renewcommand*{\proofname}{\textbf{Demostración:}}

%Conjuntos de números
\newcommand{\N}{\mathbb{N}}
\newcommand{\Z}{\mathbb{Z}}
\newcommand{\Q}{\mathbb{Q}}
\newcommand{\I}{\mathbb{I}}
\newcommand{\R}{\mathbb{R}}

%Shorter comands
\newcommand{\defined}{\coloneqq}
\newcommand{\bfit}[1]{\textbf{\textit{#1}}}
\let\epsilon\varepsilon
\let\oldemptyset\emptyset
\let\emptyset\varnothing
\let\set\Set
\let\union\cup
\let\propersubset\subset
\let\subset\subseteq
\let\intersection\cap

%the code below manipulates space around \align environment https://tex.stackexchange.com/questions/47400/remove-vertical-space-around-align
%\usepackage{etoolbox}
%\newcommand{\zerodisplayskips}{%
%\setlength{\abovedisplayskip}{0em}}%
%%\setlength{\belowdisplayskip}{-0em}%
%%\setlength{\abovedisplayshortskip}{0em}}%
%%\setlength{\belowdisplayshortskip}{-0em}}
%\appto{\normalsize}{\zerodisplayskips}
%\appto{\small}{\zerodisplayskips}
%\appto{\footnotesize}{\zerodisplayskips}

%bold all lists$
\setlist[enumerate]{font=\bfseries}
%\setlist{leftmargin=0em}
\setlist[itemize]{topsep=0pt}%{noitemsep, topsep=0pt}

\setlength{\parindent}{0pt} %no indent for the document
\setlength{\parskip}{1em} %add space between paragraphs
\pagestyle{empty}

\begin{document}

\title{Álgebra Lineal I}
\author{Darvid \\ \texttt{darvid.torres@gmail.com}}
\date{\today}
\maketitle
\thispagestyle{empty}

\part*{Espacios vectoriales}

\bfit{Definición:} Una operación binaria $(*)$ sobre un cojunto $V$ es una función:
\begin{align*}
 * &: V \times V \to V\\
 &*(u,v) = u*v\in V.
\end{align*}
\bfit{Definición:}  Sea $K$ un campo. Un espacio vectorial sobre $K$, es un conjunto $V$ no vacío, dotado con dos operaciones binarias, suma: $+$ y multiplicación por escalares $\cdot$, las cuales satisfacen los siguientes:

\section*{Axiomas}

\begin{enumerate}
    \item Cerradura (de la suma): Si $u,v\in V$, entonces $u+v \in V$.
    \item Conmutatividad (de la suma): Si $u,v\in V$, entonces $u+v=v+u$.
    \item Asociatividad (de la suma): Si $u,v,w\in V$, entonces $u+(v+w)=(u+v)+w$.
    \item Neutro aditivo: $\exists 0\in V$ tal que si $v\in V$, entonces $0+v=v$.
    \item Inverso aditivo: Si $v\in V$, entonces $\exists (-v)\in V$ tal que $v+(-v)=0$.
    \item Multiplicación por escalares: Si $\alpha \in K$, entonces $\alpha \cdot v\in V$.
    \item Asociatividad (de la multiplicación por escalares): Si $\alpha, \beta\in K$ y $v\in V$, entonces $(\alpha\cdot \beta)\cdot v = \alpha \cdot (\beta \cdot v)$.
    \item Neutro multiplicativo: Sea $1$ es el elemento identidad en $K$ y $v\in V$, entonces $1\cdot v=v$.
    \item P. Distributiva: Si $\alpha, \beta \in K$ y $u,v\in V$, entonces \begin{itemize}
        \item $\alpha \cdot (u+v) = \alpha \cdot u + \alpha \cdot v$.
        \item $(\alpha+\beta) \cdot v = \alpha \cdot v + \beta \cdot v$.
    \end{itemize}
\end{enumerate}

\bfit{Definición:} A los elementos de $K$ los llamaremos escalares y a los de $V$, vectores.

\textbf{Nota:} El uso de los símbolos $+$ y $\cdot$ no debe confundirse con las operaciones definidas sobre $K$, sin embargo, abusando de la notación, utilizaremos los mismos. Es decir, deberíamos utilizar símbolos distintos para denotar la suma y multiplicación en $V$, respecto de los de $K$, pero para simplificar la escritura, prescindiremos de ello.

\subsection*{Lista de ejercicios 1 (LE1)}
\begin{enumerate}
    \item Sea $F$ un campo y un espacio vectorial sobre sí mismo
\end{enumerate}

\end{document}