\documentclass[11pt]{article}

\usepackage[top=0.6in,bottom=0.6in,right=1in,left=1in]{geometry}
\usepackage{amsfonts, amssymb, amsmath, amsthm, enumitem}
\usepackage{mathtools}
\usepackage{braket}
\usepackage{nicefrac} % for Elegant fractions in one line https://tex.stackexchange.com/questions/128496/elegant-fractions-in-one-line/128498

%Conjuntos de números
\newcommand{\N}{\mathbb{N}}
\newcommand{\Z}{\mathbb{Z}}
\newcommand{\Q}{\mathbb{Q}}
\newcommand{\R}{\mathbb{R}}

%Shorter comands
\let\epsilon\varepsilon
\let\oldemptyset\emptyset
\let\emptyset\varnothing
\let\set\Set

%bold all lists$
\setlist[enumerate]{font=\bfseries}

\setlength{\parindent}{0pt} %no indent for the document
\setlength{\parskip}{1em} %add space between paragraphs
\pagestyle{empty}

\begin{document}

\title{\vspace{-2cm}Cálculo diferencial e Integral I \\ Semestre 2023-1 \\ Grupo 4031}
\author{Problemas de: números reales \\ \text{Torres Brito David Israel}}
\date{\today}
\maketitle
\thispagestyle{empty}

\begin{enumerate}
 \item Encuentre todos los números reales que satisfagan las siguientes desigualdades: \begin{enumerate}[label=(\alph*)]
  \item $4-x<3-2x$
  \begin{align*}
   (4-x)+2x-4&< (3-2x)+2x-4 && \text{Ley de la cancelación}\\
   4+(-x+2x)-4 &< 3+(-2x+2x)-4 && \text{Asociatividad}\\
   (-x+2x)+4-4 &< (-2x+2x)+3-4 && \text{Conmutatividad}\\
   (-x+2x)+0 &< 0+3-4 && \text{Inverso aditivo}\\
   -x+2x &< 3-4 && \text{Neutro aditivo}\\
   x &< -1 && \text{Definición}
  \end{align*}

 \item $5-x^2<-2$
 \begin{align*}
  (5-x^2)+x^2+2 &< (-2)+x^2+2 && \text{Ley de la cancelación}\\
  5+(-x^2+x^2)+2 &< (-2)+x^2+2 && \text{Asociatividad}\\
  5+0+2 &< (-2)+x^2+2 && \text{Inverso aditivo}\\
  5+2 &< (-2)+x^2+2 && \text{Neutro aditivo}\\
  5+2 &< x^2+(-2)+2 && \text{Conmutatividad}\\
  5+2 &< x^2 + (-2+2) &&\text{Asociatividad}\\
  5+2 &< x^2&&\text{Inverso aditivo}\\
  7 &< x^2 && \text{Definición}
 \end{align*}
Sabemos que $0<7$, y por el ejercicio 4, $\sqrt{7}<\sqrt{x^2}$. Luego, \begin{enumerate}[label=\roman*)]
 \item Si $0\leq x$, entonces $\sqrt{x^2}=x$, por definición.
 \item Si $x<0$, entonces $\sqrt{x^2}=-x$, por definición.
\end{enumerate}
 \end{enumerate}

 \item Pruebe que si $a,b\in \R$ son mayores o iguales a $0$, entonces $a^2\leq b^2$ si y solo si $a\leq b$.
 \begin{enumerate}[label=\roman*)]
  \item Si $a^2\leq b^2$, \begin{align*}
   a^2 + (-a^2)&\leq b^2 +(-a^2)&& \text{Ley de la cancelación}\\
   0 &\leq b^2-a^2 && \text{Inverso aditivo}\\
   0 &\leq bb - aa && \text{Definición}\\
   0 &\leq bb - aa + 0 && \text{Neutro aditivo}\\
   0 &\leq bb - aa + (ab-ab) && \text{Inverso aditivo}\\
   0 &\leq (bb + ab) + (- ab - aa) && \text{Asociatividad}\\
   0 &\leq (bb + ab)-(ab+aa) && \text{Demostrado previamente}\\
   0 &\leq b(b+a)-a(b+a) && \text{P. Distributiva}\\
   0 &\leq (b+a)(b-a) && \text{P. Distributiva}
  \end{align*} Por hipótesis, $a\geq 0$ y $b\geq 0$, y por propiedad de los positivos, $a+b\geq 0$. Sigue que:\begin{align*}
   0 \cdot (b+a)^{-1}&\leq (b+a)^{-1}\cdot (b+a)(b-a) && \text{Ley de la cancelación}\\
   0 &\leq (b+a)^{-1}\cdot (b+a)(b-a) && \text{Demostrado anteriormente}\\
   0 &\leq 1 \cdot (b-a) && \text{Inverso multiplicativo}\\
   0 &\leq b-a && \text{Neutro multiplicativo}\\
   0 + a &\leq b-a+a && \text{Ley de la cancelación}\\
   0+a &\leq b+0 && \text{Inverso aditivo}\\
   a &\leq b && \text{Neutro aditivo}
  \end{align*}
  \item Si $a\leq b$. \begin{align*}
   a -a &\leq b-a && \text{Ley de la cancelación}\\
   0 &\leq b-a && \text{Inverso aditivo}
  \end{align*} Debido a que $b\geq a$, tenemos que $b-a\geq 0$. Sigue que: \begin{align*}
   0 \cdot (b+a) &\leq (b-a)\cdot (b+a) && \text{Ley de la multiplicación}\\
   0 &\leq (b-a)\cdot (b+a) && \text{Ley de la multiplicación}\\
   0 &\leq b(b-a)+a(b-a) && \text{P. Distributiva}\\
   0 &\leq bb-ab+ab-aa &&\text{P. Distributiva}\\
   0 &\leq bb + 0 -aa && \text{Inverso aditivo}\\
   0 &\leq bb -aa && \text{Neutro aditivo}\\
   0 + aa &\leq bb -aa + aa && \text{Ley de la cancelación}\\
   aa &\leq bb + 0 && \text{Inverso aditivo}\\
   aa &\leq bb &&\text{Neutro aditivo}\\
   a^2 &\leq b^2 && \text{Definición}
  \end{align*}
 \end{enumerate} \qed
\end{enumerate}

\end{document}