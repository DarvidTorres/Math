\documentclass[11pt]{article}

\usepackage[top=0.6in,bottom=0.6in,right=1in,left=1in]{geometry}
\usepackage{amsfonts, amssymb, amsmath, amsthm, enumitem}
\usepackage{mathtools}
\usepackage{braket}
\usepackage{nicefrac} % for Elegant fractions in one line https://tex.stackexchange.com/questions/128496/elegant-fractions-in-one-line/128498

%Conjuntos de números
\newcommand{\N}{\mathbb{N}}
\newcommand{\Z}{\mathbb{Z}}
\newcommand{\Q}{\mathbb{Q}}
\newcommand{\R}{\mathbb{R}}

%Shorter comands
\let\epsilon\varepsilon
\let\oldemptyset\emptyset
\let\emptyset\varnothing
\let\set\Set

%bold all lists$
\setlist[enumerate]{font=\bfseries}

\setlength{\parindent}{0pt} %no indent for the document
\setlength{\parskip}{1em} %add space between paragraphs
\pagestyle{empty}

\begin{document}

\title{\vspace{-2cm}Cálculo diferencial e Integral I \\ Semestre 2023-1 \\ Grupo 4031}
\author{Problemas de: números reales \\ \text{Torres Brito David Israel}}
\date{\today}
\maketitle
\thispagestyle{empty}

\begin{enumerate}
 \item Encuentre todos los números reales que satisfagan las siguientes desigualdades: \begin{enumerate}[label=(\alph*)]
  \item $4-x<3-2x$
  \begin{align*}
   (4-x)+2x-4&< (3-2x)+2x-4 && \text{Ley de la cancelación}\\
   4+(-x+2x)-4 &< 3+(-2x+2x)-4 && \text{Asociatividad}\\
   (-x+2x)+4-4 &< (-2x+2x)+3-4 && \text{Conmutatividad}\\
   (-x+2x)+0 &< 0+3-4 && \text{Inverso aditivo}\\
   -x+2x &< 3-4 && \text{Neutro aditivo}\\
   x &< -1 && \text{Definición}
  \end{align*}

 \item $5-x^2<-2$
 \begin{align*}
  (5-x^2)+x^2+2 &< (-2)+x^2+2 && \text{Ley de la cancelación}\\
  5+(-x^2+x^2)+2 &< (-2)+x^2+2 && \text{Asociatividad}\\
  5+0+2 &< (-2)+x^2+2 && \text{Inverso aditivo}\\
  5+2 &< (-2)+x^2+2 && \text{Neutro aditivo}\\
  5+2 &< x^2+(-2)+2 && \text{Conmutatividad}\\
  5+2 &< x^2 + (-2+2) &&\text{Asociatividad}\\
  5+2 &< x^2&&\text{Inverso aditivo}\\
  7 &< x^2 && \text{Definición}
 \end{align*}
Sabemos que $0<7$, y por el ejercicio 4, $\sqrt{7}<\sqrt{x^2}$. Luego, \begin{enumerate}[label=\roman*)]
 \item Si $0\leq x$, entonces $\sqrt{x^2}=x$, por definición. Así, $\sqrt{7} < x$.
 \item Si $x<0$, entonces $\sqrt{x^2}=-x$, por definición. Así, $\sqrt{7} < -x$. Luego, \begin{align*}
  \sqrt{7} +(x -\sqrt{7})&< -x +(x - \sqrt{7}) && \text{Ley de la cancelación}\\
   x+(\sqrt{7} -\sqrt{7})&< (-x +x) - \sqrt{7} && \text{Asociando}\\
  x + 0 &< 0 - \sqrt{7} && \text{Inverso aditivo}\\
  x &< -\sqrt{7} && \text{Neutro aditivo}
 \end{align*}
\end{enumerate} De este modo, $x<-\sqrt{7}$ o $\sqrt{7} < x$.

\pagebreak

\item $(x-1)(3+x)<0$

Caso (1): si $(x-1)<0$ y $(3+x)<0$,

\noindent\begin{minipage}[l]{.5\linewidth}
 \begin{align*}
  (x-1) +1 &< 0 +1 && \text{Cancelación}\\
  x+(-1+1) &< 0 + 1 && \text{Asociando}\\
  x + 0 &< 0 +1 && \text{Inverso aditivo}\\
  x &< 1 && \text{Neutro aditivo}
 \end{align*} \end{minipage}%
 \begin{minipage}[l]{.5\linewidth}
  \begin{align*}
   (3+x) -3 &< 0 - 3 && \text{Cancelación}\\
   x + (3-3) &< 0 - 3 && \text{Asociando}\\
   x + 0 &< 0 - 3 && \text{Inverso aditivo}\\
   x &< - 3 && \text{Neutro aditivo}
  \end{align*}
 \end{minipage}

 Por lo que $x<-3$.

 Caso (2): si $0<(x-1)$ y $0<(3+x)$,

 \noindent\begin{minipage}[c]{.5\linewidth}
  \begin{align*}
   0 +1&< (x-1)+1&& \text{Cancelación}\\
   0 +1&< x+(-1+1)&& \text{Asociando}\\
   0 +1&< x+0&& \text{Inverso aditivo}\\
   1&< x&& \text{Neutro aditivo}
  \end{align*} \end{minipage}%
  \begin{minipage}[c]{.5\linewidth}
   \begin{align*}
    0 - 3 &< (3+x) - 3 && \text{Cancelación}\\
    0 - 3 &< x + (3-3) && \text{Asociando}\\
    0 - 3 &< x + 0 && \text{Inverso aditivo}\\
    -3 &< x && \text{Neutro aditivo}
   \end{align*}
  \end{minipage}

  Por lo que $1<x$.

  Así $x<-3$ o $1<x$.

  \item \begin{align*}
   \frac{x-1}{x+1} &> 0 && \text{$x\neq -1$}\\
   (x-1)(x+1)^{-1} &> 0&& \text{Notación}\\
   (x-1)(x+1)^{-1} \cdot (x+1)&> 0\cdot (x+1)&& \text{Ley de la cancelación}\\
   (x-1)(x+1)^{-1} \cdot (x+1)&> 0&& \text{Demostrado anteriormente}\\
   (x-1) \cdot 1 &> 0&& \text{Inverso multiplicativo}\\
   (x-1) &> 0&& \text{Neutro multiplicativo}\\
   (x-1) +1&> 0+1&& \text{Ley de la cancelación}\\
   x + 0 &> 0 + 1 && \text{Inverso aditivo}\\
   x &>1 && \text{Neutro aditivo}
 \end{align*}

  \item \[\frac{1}{x}-\frac{1}{1-x}<0\] Por definición, $x\neq 0$ y $1-x\neq 0$, por lo que $x\neq 1$. Luego, \begin{align*}
  x^{-1}-(1-x)^{-1} &< 0 && \text{Notación}\\
  x^{-1}-(1-x)^{-1} +(1-x)^{-1}&< 0 +(1-x)^{-1} && \text{Ley de la cancelación}\\
  x^{-1}+0&< 0 +(1-x)^{-1} && \text{Inverso aditivo}\\
  x^{-1}&<(1-x)^{-1} && \text{Neutro aditivo}
  \end{align*}

 \textbf{ Caso (1)}: si $x>0$, preserva el orden al multiplicar, esto es:\begin{align*}
   x^{-1} \cdot x &<(1-x)^{-1} \cdot x && \text{Demostrado anteriormente}\\
   1 &<(1-x)^{-1} \cdot x && \text{Inverso multiplicativo}
  \end{align*}
  \begin{enumerate}[label=\roman*)]
   \item Si $0<(1-x)$, sigue que $x<1$, pero esto contradice el supuesto initial.
   
 \pagebreak
 
   \item Si $(1-x)<0$, cambia el orden al multiplicar, esto es: \begin{align*}
    (1-x)\cdot (1-x)^{-1} \cdot x &< (1-x)\cdot1  && \text{Demostrado anteriormente}\\
    1 \cdot x &< (1-x)\cdot1  && \text{Inverso multiplicativo}\\
   x &< 1-x  && \text{Neutro multiplicativo}\\
   x +x&< 1-x +x && \text{Ley de la cancelación}\\
   x +x&< 1 && \text{Inverso aditivo}\\
   2x&< 1 && \text{Definición}\\
   x&< \frac{1}{2} && \text{Ley de la cancelación}
   \end{align*}
  \end{enumerate}

  \textbf{Caso (2)}: si $x<0$, cambia el orden al multiplicar, esto es: \begin{align*}
   (1-x)^{-1} \cdot x &< x^{-1} \cdot x && \text{Demostrado anteriormente}\\
   (1-x)^{-1} \cdot x &< 1 && \text{Inverso multiplicativo}
  \end{align*}
  \begin{enumerate}[label=\roman*)]
   \item Si $(1-x)<0$, sigue que $1<x$, pero esto contradice el supuesto initial.
   \item Si $0<(1-x)$, preserva el orden al multiplicar, esto es: \begin{align*}
   (1-x)\cdot (1-x)^{-1}\cdot x &< (1-x)\cdot x^{-1} \cdot x && \text{Demostrado anteriormente}\\
   1\cdot x &< (1-x)\cdot 1 && \text{Inverso multiplicativo}\\
   x &< (1-x) && \text{Neutro multiplicativo}\\
   x+x &< (1-x) +x&& \text{Ley de la cancelación}\\
   x+x &< 1&& \text{Inverso aditivo}\\
   2x &< 1&& \text{Definición}\\
   x &< \frac{1}{2} && \text{Ley de la cancelación}
   \end{align*}
  \end{enumerate}
  En cualquier caso, $x<\frac{1}{2}$.

 \end{enumerate}

 \item Pruebe que si $x,y\in \R$, entonces $x^3-y^3=(x-y)(x^2+xy+y^2)$. \begin{align*}
  (x-y)(x^2+xy+y^2) &= x^2(x-y)+xy(x-y)+y^2(x-y) && \text{P. Distributiva}\\
  &= x(x^2)-y(x^2)+x(xy)-y(xy)+x(y^2)-y(y^2) && \text{P. Distributiva}\\
  &= x^3-yx^2+x^2y-xy^2+xy^2-y^3 && \text{Definición}\\
  &= x^3-+0+0-y^3 && \text{Inverso aditivo}\\
  &= x^3-y^3 && \text{Neutro aditivo}
 \end{align*}

 \item Pruebe que si $x,y\in\R$ son distintos de $0$, entonces $x^2+xy+y^2>0$.
 
 Supongamos que $x^2+xy+y^2\leq 0$.

 Por el ejercicio anterior, sabemos que $x^3-y^3=(x-y)(x^2+xy+y^2)$. Luego,\begin{enumerate}[label=\roman*)]
  \item Si $x^2+xy+y^2 =0$, entonces \begin{align*}
   x^3-y^3&=0 && \text{Por ejercicio anterior}\\
   x^3 - y^3 + y^3&=0 + y^3 && \text{Ley de la cancelación}\\
   x^3 + 0 &= 0 + y^3 && \text{Inverso aditivo}\\
   x^3 &= y^3 && \text{Neutro aditivo}\\
   x &= y &&\text{}
  \end{align*}
  
  De esto sigue que $x^2+xy+y^2=x^2+x^2+x^2=3x^2$, y por hipótesis	$3x^2=0$, lo que es una contradicción.
  \item Si $x^2+xy+y^2 <0$, entonces \begin{align*}
   x^3-y^3&<0 && \text{Por ejercicio anterior}\\
   x^3 - y^3 + y^3&<0 + y^3 && \text{Ley de la cancelación}\\
   x^3 + 0 &< 0 + y^3 && \text{Inverso aditivo}\\
   x^3 &< y^3 && \text{Neutro aditivo}\\
   x &< y &&\text{}
  \end{align*}
  De esto sigue que $(x-y)<0$, por lo que $(x-y)(x^2+xy+y^2)>0=x^3-y^3$, lo que es una contradicción.
 \end{enumerate}
 Por tanto, $x^2+xy+y^2>0$.

 \item Pruebe que si $a,b\in \R$ son mayores o iguales a $0$, entonces $a^2\leq b^2$ si y solo si $a\leq b$.
 \begin{enumerate}[label=\roman*)]
  \item Si $a^2\leq b^2$, \begin{align*}
   a^2 + (-a^2)&\leq b^2 +(-a^2)&& \text{Ley de la cancelación}\\
   0 &\leq b^2-a^2 && \text{Inverso aditivo}\\
   0 &\leq bb - aa && \text{Definición}\\
   0 &\leq bb - aa + 0 && \text{Neutro aditivo}\\
   0 &\leq bb - aa + (ab-ab) && \text{Inverso aditivo}\\
   0 &\leq (bb + ab) + (- ab - aa) && \text{Asociatividad}\\
   0 &\leq (bb + ab)-(ab+aa) && \text{Demostrado previamente}\\
   0 &\leq b(b+a)-a(b+a) && \text{P. Distributiva}\\
   0 &\leq (b+a)(b-a) && \text{P. Distributiva}
  \end{align*} Por hipótesis, $a\geq 0$ y $b\geq 0$, y por propiedad de los positivos, $a+b\geq 0$. Sigue que:\begin{align*}
   0 \cdot (b+a)^{-1}&\leq (b+a)^{-1}\cdot (b+a)(b-a) && \text{Ley de la cancelación}\\
   0 &\leq (b+a)^{-1}\cdot (b+a)(b-a) && \text{Demostrado anteriormente}\\
   0 &\leq 1 \cdot (b-a) && \text{Inverso multiplicativo}\\
   0 &\leq b-a && \text{Neutro multiplicativo}\\
   0 + a &\leq b-a+a && \text{Ley de la cancelación}\\
   0+a &\leq b+0 && \text{Inverso aditivo}\\
   a &\leq b && \text{Neutro aditivo}
  \end{align*}
  \item Si $a\leq b$. \begin{align*}
   a -a &\leq b-a && \text{Ley de la cancelación}\\
   0 &\leq b-a && \text{Inverso aditivo}
  \end{align*} Debido a que $a$ y $b$ son mayores o iguales que $0$, por axioma de orden $a+b\geq 0$, de la desigualdad anterior obtenemos:
  
  \pagebreak

  \begin{align*}
   0 \cdot (b+a) &\leq (b-a)\cdot (b+a) && \text{Ley de la multiplicación}\\
   0 &\leq (b-a)\cdot (b+a) && \text{Ley de la multiplicación}\\
   0 &\leq b(b-a)+a(b-a) && \text{P. Distributiva}\\
   0 &\leq bb-ab+ab-aa &&\text{P. Distributiva}\\
   0 &\leq bb + 0 -aa && \text{Inverso aditivo}\\
   0 &\leq bb -aa && \text{Neutro aditivo}\\
   0 + aa &\leq bb -aa + aa && \text{Ley de la cancelación}\\
   aa &\leq bb + 0 && \text{Inverso aditivo}\\
   aa &\leq bb &&\text{Neutro aditivo}\\
   a^2 &\leq b^2 && \text{Definición}
  \end{align*}
 \end{enumerate} \qed

 \item Pruebe que si $a,b\in \R$, entonces $a^2\leq b^2$ si y sólo si $|a|\leq |b|$.
 
\textbf{Demostración:}

Sea $a^2\leq b^2$. \begin{enumerate}[label=\roman*)]
 \item Si $0\leq a$ y $0\leq b$, por el ejercicio 4, de la hipótesis sigue que $a\leq b$, y por definición, $a=|a|$ y $b=|b|$, es decir, $|a|\leq |b|$.
 \item Si $a<0$ y $0 \leq b$, por definciión $|a|=-a$ y $|b|=b$. Notemos que $(-a)(-a)=a^2=|a||a|$. Similarmente, $b^2=|b||b|$. Luego, por hipótesis tenemos que $|a||a|=a^2\leq b^2 = |b||b|$, es decir, $|a|^2\leq |b|^2$. Como el valor absoluto siempre es mayor o igual a $0$, por el ejercicio 4 sigue que $|a|\leq |b|$.
 \item Si $0\leq a$ y $b<0$, por definciión $|a|=a$ y $|b|=-b$. Notemos que $(-b)(-b)=b^2=|b||b|$. Similarmente, $a^2=|a||a|$. Luego, por hipótesis tenemos que $|a||a|=a^2\leq b^2 = |b||b|$, es decir, $|a|^2\leq |b|^2$. Como el valor absoluto siempre es mayor o igual a $0$, por el ejercicio 4 sigue que $|a|\leq |b|$.
 \item Si $a<0$ y $b<0$, por definición $|a|=-a$ y $|b|=-b$. Notemos que $(-a)(-a)=a^2=|a||a|$ y $(-b)(-b)=b^2=|b||b|$. Luego, por hipótesis, tenemos que $|a||a|=a^2\leq b^2 = |b||b|$, es decir $|a|^2\leq |b|^2$. Como el valor absoluto siempre es mayor o igual a $0$, por el ejercicio 4 sigue que $|a|\leq |b|$.
\end{enumerate}
Por otra parte, supongamos que $|a|\leq |b|$. \begin{enumerate}[label=\roman*)]
 \item Si $0\leq a$ y $0\leq b$, por definición $|a|=a$ y $|b|=b$. Por hipótesis tenemos que $a=|a|\leq |b|=b$, es decir $a\leq b$, y por el ejercicio 4, de la hipótesis	sigue que $a^2\leq b^2$.
 \item Si $a<0$ y $0 \leq b$, por definciión $|a|=-a$ y $|b|=b$. Por hipótesis tenemos que $-a=|a|\leq |b|=b$, es decir $-a\leq b$. Notemos que $(-a)(-a)=a^2 \leq b (-a)$. Similarmente, $(-a)b \leq b^2=bb$. Por transitividad, $a^2\leq b^2$.
 \item Si $0\leq a$ y $b<0$, por definciión $|a|=a$ y $|b|=-b$. Por hipótesis tenemos que $a=|a|\leq |b|=-b$, es decir $a\leq -b$. Notemos que $aa=a^2 \leq (-b)a$. Similarmente, $a(-b) \leq b^2=(-b)(-b)$. Por transitividad, $a^2\leq b^2$.
 \item Si $a<0$ y $b<0$, por definición $|a|=-a$ y $|b|=-b$. Por hipótesis tenemos que $-a=|a|\leq |b|=-b$, es decir $-a\leq -b$. Notemos que $(-a)(-a)=a^2 \leq (-b)(-a)$. Similarmente, $(-a)(-b) \leq b^2=(-b)(-b)$. Por transitividad, $a^2\leq b^2$.
\end{enumerate} \qed

 \item En los siguientes incisos, escriba el mismo número quitando (al menos) un signo de valor absoluto. \begin{enumerate}[label=(\alph*)]
  \item $|\sqrt{2}+\sqrt{3}-\sqrt{5}+\sqrt{6}|$ (no use calculadora)
  \item $||\sqrt{2}+\sqrt{3}|-|\sqrt{5}+\sqrt{7}||$ (no use calculadora)
  \item $||a-b|-|a|-|b||$
  \item $||a+b|+|c|-|a+b+c||$
  \item $|x^2-2xy+y^2|$
  \item $x-|x-|x||$
 \end{enumerate}
 \item Calcule todos los números reales que satisfacen las siguientes condiciones:\begin{enumerate}[label=(\alph*)]
 \item $|x-3|=8$
 \item $|x+4|<2$
 \item $|x-||+|x+1|<2$
 \item $|x-1|+|x+1|>2$
 \item $|x-1||x+1|=0$
 \item $|x-1||x+2|=3$
 \end{enumerate}
 \item Pruebe las siguientes identidades: \begin{enumerate}[label=(\alph*)]
  \item Si $x\neq 0$, entonces $\big|x^{-1}\big|=|x|^{-1}$
  \begin{align*}
   \big|x^{-1}\big| &= \bigg|\frac{1}{x}\bigg| && \text{Notación}\\
   &= \frac{|1|}{|x|} && \text{Ejercicio siguiente}\\
   &= \frac{1}{|x|} && \text{$0<1$}\\
   &= |x|^{-1} && \text{Notación}
  \end{align*} \qed

  \item Si $x\neq 0$, entonces $|\nicefrac{y}{x}|=|y|/|x|$
   
  \begin{enumerate}[label=\roman*)]
      \item Si $x \geq 0$ y $y>0$, entonces $|x|=x$ y $|y|=y$. Además, $\frac{1}{y} >0$, de donde sigue que $\frac{x}{y} \geq 0$ por lo que $\big| \frac{x}{y} \big| = \frac{x}{y}$. De este modo, $ \big| \frac{x}{y} \big| = \frac{|x|}{|y|}$.
      \item Si $x \geq 0$ y $b<0$, entonces $|x|=x$ y $|y|=-b$. Además, $\frac{1}{y} <0$, de donde sigue que $\frac{x}{y} \leq 0$, por lo que $\big| \frac{x}{y} \big| =- \frac{x}{y}$. De este modo, $ \big| \frac{x}{y} \big| = \frac{|x|}{|y|}$.
      \item Si $x<0$ y $y>0$, entonces $|x|=-a$ y $|y|=y$. Además, $\frac{1}{y} >0$, de donde sigue que $\frac{x}{y} < 0$, por lo que $\big| \frac{x}{y} \big| =- \frac{x}{y}$. De este modo, $ \big| \frac{x}{y} \big| = \frac{|x|}{|y|}$.
      \item Si $x<0$ y $b<0$, entonces $|x|=-a$ y $|y|=-b$. Además, $\frac{1}{y} <0$, de donde sigue que $\frac{x}{y} > 0$ por lo que $\big| \frac{x}{y} \big| = \frac{x}{y}$. De este modo, $ \big| \frac{x}{y} \big| = \frac{|x|}{|y|}$.
  \end{enumerate}
 \end{enumerate}

\item Pruebe que si $a,b\in \R$, entonces $|a|-|b|\leq |a-b|\leq |a|+|b|$.

\item Pruebe que si $a,b\in \R$, entonces $\big| |a|-|b| \big|\leq |a-b|$.

Por la desigualdad del triángulo tenemos que
        \begin{align*}
            |(a-b)+b| &\leq |a-b|+|b| \\
            |a| &\leq |a-b|+|b| \\
            |a|-|b| &\leq |a-b| && \text{(1)}
        \end{align*}
        Similarmente, 
        \begin{align*}
            |(b-a)+a| &\leq |b-a|+|a| \\
            |b| &\leq |b-a|+|a| \\
            |b|-|a| &\leq |b-a| \\
            -|b-a| &\leq |a|-|b|\\
            -|a-b| &\leq |a|-|b| && \text{(2)}
        \end{align*}
        Finalmente, sabemos que $|a|\leq b$ si y solo si $-a\leq b\leq a$, y tomando (1) y (2), $\big| |a| - |b| \big| \leq |a-b|$.
\end{enumerate}
\end{document}