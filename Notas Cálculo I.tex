\documentclass[11pt]{article}

\usepackage[top=0.5in,bottom=0.5in,right=0.5in,left=0.5in]{geometry}
%\usepackage[T1]{fontenc}
%\usepackage[spanish]{babel}
\usepackage{amsfonts, amssymb, amsmath, amsthm, enumitem}
\renewcommand*{\proofname}{\textbf{Demostración:}}
\usepackage{mathtools} %coloneqq command
\usepackage{braket}
%\usepackage[spanish]{babel}
\usepackage{adjustbox}
\usepackage{nicefrac} % for Elegant fractions in one line https://tex.stackexchange.com/questions/128496/elegant-fractions-in-one-line/128498

%Conjuntos de números
\newcommand{\N}{\mathbb{N}}
\newcommand{\Z}{\mathbb{Z}}
\newcommand{\Q}{\mathbb{Q}}
\newcommand{\I}{\mathbb{I}}
\newcommand{\R}{\mathbb{R}}

%Shorter comands
\newcommand{\defined}{\coloneqq}
\let\epsilon\varepsilon
\let\oldemptyset\emptyset
\let\emptyset\varnothing
\let\set\Set
\let\union\cup
\let\propersubset\subset
\let\subset\subseteq
\let\intersection\cap

%the code below manipulates space around \align environment https://tex.stackexchange.com/questions/47400/remove-vertical-space-around-align
%\usepackage{etoolbox}
%\newcommand{\zerodisplayskips}{%
%\setlength{\abovedisplayskip}{0em}%
%\setlength{\belowdisplayskip}{-0em}%
%\setlength{\abovedisplayshortskip}{0em}%
%\setlength{\belowdisplayshortskip}{-0em}}
%\appto{\normalsize}{\zerodisplayskips}
%\appto{\small}{\zerodisplayskips}
%\appto{\footnotesize}{\zerodisplayskips}

%bold all lists$
\setlist[enumerate]{font=\bfseries}

\setlength{\parindent}{0pt} %no indent for the document
\setlength{\parskip}{1em} %add space between paragraphs
\pagestyle{empty}

\begin{document}

\title{\vspace{-2cm}Cálculo I}
\author{Darvid \\ \texttt{darvid.torres@gmail.com}}
\date{\today}
\maketitle
\thispagestyle{empty}

\section*{Números reales}

Existe un conjunto llamado conjunto de los números reales, denotado por $\R$. A los elementos de este conjunto los llamaremos números reales. Este conjunto está dotado con dos operaciones binarias:\vspace{-0.5cm}
\begin{center}
\noindent\begin{minipage}[r]{5.5cm}
\begin{align*}
    \text{Suma} \ + : \R \times \R &\to \R\\
    (m,n) &\mapsto m+n
\end{align*}
\end{minipage}%
\begin{minipage}[l]{6.5cm}
\begin{align*}
    \text{y} \qquad \text{Multiplicación} \ \cdot : \R \times \R &\to \R\\
    (m,n) &\mapsto m \cdot n
\end{align*}
\end{minipage}
\end{center}

La notación anterior denota la cerradura de estas operaciones, es decir, que para cuales quiera dos números reales $(m,n)$, la suma y multiplicación son números reales, $(m+n)\in \R$ y $(m\cdot n)\in \R$.

Las suma y multiplicación de números reales satisfacen los siguientes \textbf{axiomas de campo}:

\begin{enumerate}[label=S\arabic*.]
    \item Conmutatividad (de la suma).
    
    La suma es conmutativa. Esto significa que para cualesquiera números reales $m$ y $n$ se verifica que:\[ m+n = n+m
    \]
    \item Asociatividad (de la suma).
    
    La suma es asociativa. Esto significa que para cualesquiera números reales $m$, $n$ y $l$ se verifica que: \[(m+ n)+l = m+(n+ l)
    \]
\end{enumerate}
\vspace{-1em}
El lector observará que el axioma de \textbf{asociatividad} (de la suma) no incluye todas las formas en que podríamos sumar tres números reales (en general diferentes) $a$, $b$ y $c$:

\begin{adjustbox}{minipage=0.5\linewidth,margin=0pt \smallskipamount,center}
    \noindent\begin{minipage}[c]{.5\linewidth}
    \begin{enumerate}[label=\roman*.]
        \item $(a+b)+c$
        \item $a+(b+c)$
        \item $a+(c+b)$
        \item $(a+c)+b$
        \item $(c+a)+b$
        \item $c+(a+b)$
    \end{enumerate}
    \end{minipage}%
\begin{minipage}[c]{.5\linewidth}
    \begin{enumerate}[start=7,label=\roman*.]
        \item $c+(b+a)$
        \item $(c+b)+a$
        \item $(b+c)+a$
        \item $b+(c+a)$
        \item $b+(a+c)$
        \item $(b+a)+c$
    \end{enumerate}
\end{minipage}
\end{adjustbox}
La razón es que las anteriores pueden obtenerse utilizando los axiomas de \textbf{conmutatividad} (de la suma) y \textbf{asociatividad} (de la suma), con lo que garantizamos la igualdad de todas ellas, y descartamos la necesidad de enumerarlas todas como axiomas.

%Notemos que para sumar tres números diferentes $(a,b,c)$, siempre requerimos ejecutar, primero, la suma de dos de ellos $(a+b)$, y luego tomar este resultado para sumarlo al tercer número $(a+b)+c$, a esto alude el lado izquierdo de la igualdad de la asociatividad de la suma (\textbf{S2}). También podemos cambiar el orden, realizando la suma de $a$ y $b$, luego tomar el número $c$ y sumarle a este último el resultado que habíamos obtenido con anterioridad: $c+(a+b)$. Estos resultados (\textbf{i} y \textbf{vii}) satisfacen igualdad debido a la conmutatividad de la suma (\textbf{S1}), $(a+b)+c=c+(a+b)$.
Notemos que las formas (\textbf{i} y \textbf{vii}) satisfacen igualdad debido a la conmutatividad de la suma (\textbf{S1}), $(a+b)+c=c+(a+b)$. Asimismo, por asociatividad de la suma (\textbf{S2}), las formas \textbf{(i)} y \textbf{(ii)} satisfacen igualdad, por lo que tenemos $(a+b)+c=a+(b+c)$. De esto el lector puede inferir cuál es el uso de estos axiomas y cómo demostrar la igualdad de todas las formas de sumar tres números reales.

\vspace{-1em}\begin{proof} \begin{align*}
    (a+b)+c &= a+(b+c) && \text{Asociatividad} && \text{Formas (i) y (ii)}\\
    &= a+(c+b) && \text{Conmutatividad} && \text{Forma (iii)}\\
    &= (a+c)+b && \text{Asociatividad} && \text{Forma (iv)}\\
    &= (c+a)+b && \text{Conmutatividad}&& \text{Forma (v)}\\
    &= c+(a+b) && \text{Asociatividad}&& \text{Forma (vi)}\\
    &= c+(b+a) && \text{Conmutatividad}&& \text{Forma (vii)}\\
    &= (c+b)+a && \text{Asociatividad}&& \text{Forma (viii)}\\
    &= (b+c)+a && \text{Conmutatividad}&& \text{Forma (ix)}\\
    &= b+(c+a) && \text{Asociatividad}&& \text{Forma (x)}\\
    &= b+(a+c) && \text{Conmutatividad}&& \text{Forma (xi)}\\
    &= (b+a)+c && \text{Asociatividad}&& \text{Forma (xii)} \qedhere
\end{align*} \end{proof} \vspace{-1em}

Continuamos con el listado de axiomas:

\begin{enumerate}[start=3,label=S\arabic*.]
    \item Neutro aditivo.
    
    Existe un número real llamado elemento neutro para la suma o cero, denotado por $0$, el cual satisface la siguiente condición: $ m+0=m,\forall m \in \R$
    \item Inverso aditivo.
    
    Para cada número real $m$ existe un número real llamado inverso aditivo de $m$, denotado por $-m$ (a menos $m$); la propiedad que caracteriza a este elemento es: $m + (-m) = 0$
\end{enumerate}
\vspace{-1em}
Utilizando los axiomas enunciados hasta ahora \textbf{S[1-4]}, podemos obtener resultados útiles, por ejemplo, si tenemos números reales $a,b,c$ tales que $a+c=b+c$, encontraremos que $a=b$. El lector acostumbrado a matemáticas de bachillerato, podría intentar demostrar este hecho de la siguiente manera: \begin{align*}
    a+c &= b+c\\
    a &= b+c-c\\
    %a+0 &= b + 0 \\
    a &= b
\end{align*}
No obstante, aunque en principio el resultado anterior no es incorrecto, debemos justificar cada \textit{paso} mediante las propiedades conocidas hasta este punto. Por lo anterior, una forma más precisa de demostrar la proposición es la siguiente:

Sean $a,b$ y $c$ números reales, tales que $a+c=b+c$, entonces $a=b$. (Ley de la cancelación —de la suma).

\vspace{-1em}\begin{proof} 
    \begin{align*}
        a &= a+0 && \text{Neutro aditivo}\\
        &= a+ \bigl(c+(-c)\bigr) && \text{Inverso aditivo}\\
        &= (a+c) + (-c) && \text{Asociatividad}\\
        &= (b+c) + (-c) && \text{Hipótesis}\\
        &= b + \bigl(c+(-c)\bigr) && \text{Asociatividad}\\
        &= b + 0 && \text{Inverso aditivo}\\
        &= b &&\text{Neutro aditivo} \qedhere
    \end{align*}    
\end{proof} \vspace{-1em}

\textbf{Observación:} En el segundo paso de la demostración, podíamos —en virtud del axioma \textbf{S4}, sustituir $0$ por $b+(-b)$ o por $c+(-c)$ (o cualquier otra suma igual a $0$), sin embargo, no en todos los casos resultaría útil. El lector debería comprobar qué ocurre si sustituimos $0$ por $b+(-b)$ en la demostración anterior.

\textbf{Nota:} Si el contexto es claro, enunciaremos esta proposición como ley de la cancelación.

\subsection*{Lista de Ejercicios 1 (LE1)}

\begin{enumerate}[label=\alph*)]
    \item Demuestre que el elemento neutro para la suma es único. (Unicidad del neutro aditivo).
    
\vspace{-1em}\begin{proof} 
    Supongamos que existen 0 y $\tilde{0}$ números reales tales que $a+0 = a$ y $a+\tilde{0} = a$. Luego, \begin{align*}
        0 &=a+(-a) && \text{Inverso aditivo}\\
        &=\left( a+\tilde{0} \right)+\left(-a\right) && \text{Hipótesis}\\
        &=\left( \tilde{0}+a \right)+\left(-a\right) && \text{Conmutatividad}\\
        &=\tilde{0} + \bigl( a + \left(-a \right)\bigr) && \text{Asociatividad}\\
        &=\tilde{0} + 0 && \text{Inverso aditivo}\\
        &=\tilde{0} && \text{Neutro aditivo}\qedhere
        \end{align*}    
\end{proof} \vspace{-1em}
    \item Demuestre que el inverso aditivo de cada número real es único. (Unicidad del inverso aditivo).
    
    \vspace{-1em}\begin{proof} 
        Sea $a\in \R$ arbitrario pero fijo. Supongamos que existen $-a$ y $-\tilde{a}$ números reales tales que $a + \left(-a\right) = 0$ y $a + \left(- \tilde{a}\right) = 0$. Notemos que:
    \begin{align*}
    -a &= -a+0 && \text{Neutro aditivo}\\
    &= 0+\left(-a\right) && \text{Conmutatividad}\\
    &= \bigl(a+\left(-\tilde{a} \right)\bigr)+\left(-a\right) && \text{Hipótesis}\\
    &= \bigl(\left(-\tilde{a} \right)+a\bigr)+\left(-a\right) && \text{Conmutatividad}\\
    &= \left(-\tilde{a} \right)+\bigl(a+\left(-a\right)\bigr) && \text{Asociatividad}\\
    &= \left(-\tilde{a} \right)+0 && \text{Inverso aditivo}\\
    &= -\tilde{a} && \text{Neutro aditivo} \qedhere
    \end{align*}    
    \end{proof} \vspace{-1em}

    \item Demuestre que $-0 = 0$. (Cero es igual a su inverso aditivo).
    \vspace{-1em}\begin{proof} 
        Por la propiedad del neutro aditivo tenemos que $0+0=0$. Además, el inverso aditivo de $0$ satisface que $0 + (-0) = 0$. Debido a que el inverso aditivo de cada número real es único, de la igualdad anterior sigue que $-0 = 0$. \qedhere    
    \end{proof} \vspace{-1em}

    \textbf{\textit{Corolario:}} Para todo número real $a$ distinto de cero, se tiene que $-a\neq 0$.

    Sea $a$ un número real distinto de cero. Supongamos que $-a=0$. Por este teorema tenemos que $-a=-0$; por la unicidad del inverso aditivo sigue que $-a$ es inverso aditivo de $0$, por lo que $a=0$, lo que contradice nuestro supuesto. Por tanto $-a\neq 0$.
    %Otra forma:
    %Sea $a$ un número real distinto de cero tal que $-a=0$. El inverso aditivo satisface que $a+(-a)=0$, y de la hipótesis, $a+0=0$. De esta igualdad sigue que $a=0$, lo que contradice nuestro supuesto inicial, por lo que que $-a\neq 0$.

    \item Sea $a$ un número real arbitrario pero fijo, demuestre que: $-(-a)=a$.
        
    \vspace{-1em}\begin{proof} 
        El inverso aditivo de $a$ satisface que $a + (-a) = 0$, y por conmutatividad tenemos que $(-a) + a = 0$, de esta igualdad se sigue que $a$ es inverso aditivo de $(-a)$. Similarmente, el inverso aditivo de $(-a)$ satisface que $(-a) + \bigl(-(-a)\bigr) = 0$, y por la unicidad del inverso aditivo, sigue que $-(-a) = a$. \qedhere    
    \end{proof} \vspace{-1em}

    \item Pruebe que si $a,b\in \R$, entonces $-(a+b)=(-a)+(-b)$. (Distribución del signo).

    \vspace{-1em}\begin{proof} 
        \begin{align*}
            0 &= 0 + 0 && \text{Neutro aditivo} \\
            &= \bigl(a+(-a)\bigr) + \bigl(b + (-b)\bigr) && \text{Inverso aditivo} \\
            &= a + \Bigl((-a)+ \bigl(b + (-b)\bigr)\Bigr) && \text{Asociatividad} \\
            &= a + \Bigl( \bigl((-a)+b\bigr) +(-b)\Bigr) && \text{Asociatividad} \\
            &= a + \Bigl( \bigl(b+(-a)\bigr) +(-b)\Bigr) && \text{Conmutatividad} \\
            &= a + \Bigl( b + \bigl( (-a)+(-b) \bigr) \Bigr) && \text{Asociatividad} \\
            &= (a+b) + \bigl((-a)+ (-b)\bigr) && \text{Asociatividad}
            \end{align*}
            Por la unicidad del inverso aditivo, tenemos que $(-a)+ (-b)=-(a+b)$.\qedhere    
    \end{proof} \vspace{-1em}

    \textbf{Nota:} Cada demostración que realizamos, al ser probada para números reales arbitrarios, esto es, no para elementos de $\R$ en particular, nos permite reutilizar las formas como esquema para otras proposiciones. Por ejemplo, el resultado $-(m+n)=(-m)+(-n)$, nos permite sustituir $m$ y $n$ por cuales quiera números reales, como en el ejemplo que sigue:

    \item Pruebe que si $a,b\in \R$, entonces $-\bigl(a+(-b)\bigr)=b+(-a)$.

    \vspace{-1em}\begin{proof} 
        \begin{align*}
            -\bigl(a+(-b)\bigr)&= (-a) + \bigl(-(-b)\bigr) &&\text{Distribución del signo} \\
            &= (-a) + b &&\text{Unicidad del inverso aditivo} \\
            &= b +(-a) &&\text{Conmutatividad} \qedhere
            \end{align*}    
    \end{proof} \vspace{-1em}
\end{enumerate}

Continuemos enunciando los \textbf{axiomas} de campo:

\begin{enumerate}[label=M\arabic*.]
    \item Conmutatividad (de la multiplicación).
    
    La multiplicación es conmutativa. Esto significa que para cualesquiera números reales $m$ y $n$ se verifica que: $ m \cdot n = n \cdot m $.

    \item Asociatividad (de la multiplicación).
    
    La multiplicación es asociativa. Esto significa que para cualesquiera números reales $m$, $n$ y $l$ se verifica que: $ m \cdot (n \cdot l) = (m \cdot n) \cdot l $.
\end{enumerate}

Tal como lo notamos en el caso de la suma, el axioma de \textbf{asociatividad} (de la multiplicación) no incluye todas las formas en que podríamos multiplicar tres números reales (en general diferentes) $a$, $b$ y $c$:

\begin{adjustbox}{minipage=0.5\linewidth,margin=0pt \smallskipamount,center}
    \noindent\begin{minipage}[c]{.5\linewidth}
    \begin{enumerate}[label=\roman*.]
        \item $(a\cdot b)\cdot c$
        \item $a\cdot (b\cdot c)$
        \item $a\cdot (c\cdot b)$
        \item $(a\cdot c)\cdot b$
        \item $(c\cdot a)\cdot b$
        \item $c\cdot (a\cdot b)$
    \end{enumerate}
    \end{minipage}%
\begin{minipage}[c]{.5\linewidth}
    \begin{enumerate}[start=7,label=\roman*.]
        \item $c\cdot (b\cdot a)$
        \item $(c\cdot b)\cdot a$
        \item $(b\cdot c)\cdot a$
        \item $b\cdot (c\cdot a)$
        \item $b\cdot (a\cdot c)$
        \item $(b\cdot a)\cdot c$
    \end{enumerate}
\end{minipage}
\end{adjustbox}

Podemos proceder análogamente al caso de la suma para probar la equivalencia de estas formas.

\vspace{-1em}\begin{proof} \begin{align*}
    (a\cdot b)\cdot c &= a\cdot (b\cdot c) && \text{Asociatividad} && \text{Formas (i) y (ii)}\\
    &= a\cdot (c\cdot b) && \text{Conmutatividad} && \text{Forma (iii)}\\
    &= (a\cdot c)\cdot b && \text{Asociatividad} && \text{Forma (iv)}\\
    &= (c\cdot a)\cdot b && \text{Conmutatividad}&& \text{Forma (v)}\\
    &= c\cdot (a\cdot b) && \text{Asociatividad}&& \text{Forma (vi)}\\
    &= c\cdot (b\cdot a) && \text{Conmutatividad}&& \text{Forma (vii)}\\
    &= (c\cdot b)\cdot a && \text{Asociatividad}&& \text{Forma (viii)}\\
    &= (b\cdot c)\cdot a && \text{Conmutatividad}&& \text{Forma (ix)}\\
    &= b\cdot (c\cdot a) && \text{Asociatividad}&& \text{Forma (x)}\\
    &= b\cdot (a\cdot c) && \text{Conmutatividad}&& \text{Forma (xi)}\\
    &= (b\cdot a)\cdot c && \text{Asociatividad}&& \text{Forma (xii)} \qedhere
\end{align*} \end{proof} \vspace{-1em}

Continuamos con el listado de axiomas:

\begin{enumerate}[label=M\arabic*.]

    \item Neutro multiplicativo.
    
    Existe un número real distinto de cero, llamado elemento identidad para la multiplicación o uno, denotado por $1$, que satisface la siguiente condición: $ m \cdot 1 = m,\forall m \in \R $.

    \item Inverso multiplicativo.
    
    Para cada número real $m$ distinto de cero existe un número real llamado inverso multiplicativo de $m$, denotado por $m^{-1}$, este elemento tiene la siguiente propiedad: $m \cdot m^{-1} = 1$.
\end{enumerate}

\subsection*{Lista de Ejercicios 2 (LE2)}
    \begin{enumerate}[label=\alph*)]
    \item Demuestre que el elemento identidad para la multiplicación es único. (Unicidad del neutro multiplicativo).
    
    \vspace{-1em}\begin{proof} 
        Supongamos que existen $1$ y $\tilde{1}$ números reales tales que $a\cdot 1=a$ y $a\cdot\tilde{1}=a$. Luego,
    \begin{align*}
    1 &= a \cdot a^{-1} && \text{Inverso multiplicativo}\\
    &= \left( a \cdot \tilde{1} \right) \cdot a^{-1} && \text{Por hipótesis}\\
    &= \left( \tilde{1} \cdot a \right) \cdot a^{-1} && \text{Conmutatividad}\\
    &= \tilde{1} \cdot \left( a \cdot a^{-1} \right) && \text{Asociatividad}\\
    &= \tilde{1} \cdot 1 && \text{Inverso multiplicativo}\\
    &= \tilde{1} && \text{Neutro multiplicativo}\qedhere
    \end{align*}     
    \end{proof} \vspace{-1em}
    
    \item Demuestre que el inverso multiplicativo de cada número real distinto de cero es único. (Unicidad del inverso multiplicativo).
    
    \vspace{-1em}\begin{proof} Supongamos que existen $a^{-1}$ y $\tilde{a}^{-1}$ números reales, distintos de cero, tales que $a \cdot a^{-1} = 1$ y $a \cdot \tilde{a}^{-1} = 1$. Luego,
        \begin{align*}
            a^{-1} &= a^{-1} \cdot 1 && \text{Neutro multiplicativo} \\
            &= a^{-1} \cdot \left(a \cdot \tilde{a}^{-1} \right) && \text{Por hipótesis} \\
            &= \left( a^{-1} \cdot a \right) \cdot \tilde{a} ^{-1} && \text{Asociatividad} \\
            &= \left(a \cdot a^{-1} \right) \cdot \tilde{a}^{-1} && \text{Conmutatividad} \\
            &= 1 \cdot \tilde{a}^{-1} && \text{Inverso multiplicativo} \\
            &= \tilde{a}^{-1} \cdot 1 && \text{Conmutatividad} \\
            &= \tilde{a}^{-1} && \text{Neutro multiplicativo}\qedhere
            \end{align*}     
    \end{proof} \vspace{-1em}
    
    \item Demuestre que $1=1^{-1}$.
    \vspace{-1em}\begin{proof} 
        Por axioma del neutro multiplicativo tenemos que $1\cdot 1 = 1$ y $1\neq 0$, por lo que existe $1^{-1}$ tal que $1 \cdot 1^{-1}=1$. Por unicidad del inverso multiplicativo, de la igualdad anterior sigue que $1=1^{-1}$.
    \end{proof} \vspace{-1em}

    \item Sea $a$ un número real distinto de cero; demuestre que: $\left( a^{-1} \right)^{-1}=a$.
    
    \vspace{-1em}\begin{proof} 
        El inverso multiplicativo de $a$ satisface que $a\cdot a^{-1}=1$, y por conmutatividad, $a^{-1} \cdot a=1$, de esta igualdad se sigue que $a$ es inverso multiplicativo de $a^{-1}$. Similarmente, el inverso multiplicativo de $a^{-1}$ satisface que $ a^{-1} \cdot \left( a^{-1} \right)^{-1} =1$, y por la unicidad del inverso multiplicativo, sigue que $\left( a^{-1} \right)^{-1}=a$.  
    \end{proof} \vspace{-1em}

\pagebreak
    
    \item Sean $a$ y $b$ números reales distintos de cero, demuestre que: $(a \cdot b)^{-1}=a^{-1} \cdot b^{-1}$.
    \vspace{-1em}\begin{proof} 
        \begin{align*}
        && \quad \left(a \cdot b \right) \cdot  \left(a^{-1} \cdot b^{-1}  \right)	&=	 \left( \left(a \cdot b\right) \cdot a^{-1}  \right) \cdot b^{-1}  	&& \text{Asociatividad}\\
        && \quad &=	 \left( \left(b\cdot a \right) \cdot a^{-1}  \right) \cdot b^{-1}  	&& \text{Conmutatividad}\\
        && \quad &=	 \Bigl(b\cdot  \left(a \cdot a^{-1}\right) \Bigr) \cdot b^{-1}	&& \text{Asociatividad}\\
        && \quad &=	 \left(b\cdot 1 \right) \cdot b^{-1}	&& \text{Inverso multiplicativo}\\
        && \quad &=	b\cdot b^{-1}	&& \text{Neutro multiplicativo}\\
        && \quad &=	1	&& \text{Inverso multiplicativo}
        \end{align*}
        Sigue que $\left(a^{-1} \cdot b^{-1} \right)$ es inverso multiplicativo de $\left( a \cdot b\right)$, y por la unicidad del inverso multiplicativo, sigue que $\left(a^{-1} \cdot b^{-1} \right) = \left( a \cdot b\right)^{-1}$.    
    \end{proof} \vspace{-1em}
    
    \item Sean $a,b,c\in \R$ con $c\neq 0$. Demuestre que si $a\cdot c=b\cdot c$, entonces $a=b$.
    \vspace{-1em}\begin{proof} 
        \begin{align*}
            a &= a\cdot 1 && \text{Neutro multiplicativo}\\
            &= a \cdot (c\cdot c^{-1}) && \text{Inverso multiplicativo}\\
            &= (a \cdot c) \cdot c^{-1} && \text{Asociatividad}\\
            &= (b \cdot c) \cdot c^{-1} && \text{Hipótesis}\\
            &= b \cdot (c\cdot c^{-1}) && \text{Asociatividad}\\
            &= b \cdot 1 && \text{Inverso multiplicativo}\\
            &= b && \text{Neutro multiplicativo}\qedhere
            \end{align*}    
    \end{proof} \vspace{-1em}
    \textbf{Observación:} Para esta proposición requerimos que $c\neq 0$, ya que de caber la posibilidad de que $c=0$, no tendríamos garantía de que $a=b$. El lector debería verificar este hecho. (Ej. $2\cdot 0 = 1 \cdot 0$).

    \textbf{Nota:} A esta proposición la llamaremos \textbf{ley de la cancelación} (de la multiplicación). Si el contexto es claro, omitiremos el paréntesis y simplemente la enunciaremos como ley de la cancelación.
    \end{enumerate}

    Ahora, introducimos el \textbf{axioma} que nos permite relacionar las operaciones de suma y multiplicación:

\begin{enumerate}[label=P.D.]
    \item Propiedad distributiva.
    
    Distribución de la multiplicación sobre la suma. Para cualesquiera números reales $m$, $n$ y $l$ se verifica que: $ m \cdot (n+l)=m \cdot n+m \cdot l $.
\end{enumerate}

\pagebreak

\subsection*{Una nota sobre rigurosidad}

En este apartado utilizaremos un ejemplo para puntualizar la falta de rigor en la que pueden caer los estudiantes; dichas puntualizaciones pueden parecer exageradas y el lector podría considerar que el autor está siendo \textit{pedante} en el uso sintáctico de los axiomas, pero la idea es proveer al estudiante de un uso riguroso de la formalización matemática.

Si $a$ es un número real, entonces $a\cdot 0 = 0$. El siguiente es un esbozo de la prueba propuesta por un estudiante:
\begin{align*}
    0 &= ab+ (-ab) && \text{Inverso aditivo}\\
    &= b \cdot \bigl(a + (-a)\bigr) && \text{P. Distributiva}\\
    &= b \cdot 0 && \text{Inverso aditivo}
\end{align*}

De inmediato podemos señalar la \textbf{ambigüedad} en el uso de la notación en el primer paso. ¿Qué debemos entender por $-ab$? Por como fue enunciada en el esbozo anterior, se pretendía que $-ab$ fuera el inverso aditivo de $ab$, es decir, se intuye que $-ab=-(ab)$, pero al \textit{invocar} la propiedad distributiva en el segundo paso, se asumió que $-ab=(-a)\cdot b$, es decir, se suponen dos significados para $-ab$, que \textit{a priori} no son iguales (requiere demostración). Asimismo, al utilizar la propiedad distributiva se estaría utilizando conmutatividad no enunciada, pues la síntaxis del axioma (\textbf{P.D.}) indica que el número a la izquierda de la suma, debe ser \textit{distribuido} a la izquierda de los componentes de la suma, es decir, $b\cdot (a+(-a))=b\cdot a + b\cdot (-a)$, y a partir de esto, al enunciar conmutatividad, tenemos $b\cdot a + b\cdot (-a)=a\cdot b + (-a)\cdot b$.

Al considerar el comentario anterior, el estudiante reescribe el esbozo como sigue: \begin{align*}
    0 &= a\cdot b + \bigl(-(a\cdot b)\bigr) && \text{Inverso aditivo}\\
    &= a\cdot b + \bigl((-a)\cdot b\bigr) && \text{¿?}\\
    &= b\cdot a + \bigl(b \cdot (-a)\bigr) && \text{Conmutatividad}\\
    &= b\cdot \bigl(a+(-a)\bigr) && \text{P. Distributiva}\\
    &= b\cdot 0 && \text{Inverso aditivo}
\end{align*}

El estudiante nota que para demostrar que $a\cdot 0 = 0$, requiere probar que el inverso aditivo de $a\cdot b$ es igual a $(-a)\cdot b$, es decir, se busca probar que $a\cdot b+(-a)\cdot b=0$. Observemos ahora el siguiente esbozo para esta prueba:
\begin{align*}
    0 &= b\cdot 0 && \text{¿?}\\
    &= b\cdot \bigl(a+(-a)\bigr) && \text{Inverso aditivo}\\
    &= b\cdot a + b\cdot (-a) && \text{P. Distributiva}\\
    &= a\cdot b + (-a) \cdot b && \text{Conmutatividad}
\end{align*}
No obstante, la proposición que debía servir para probar la demostración objetivo, pareciera necesitar de la proposición misma. En otras palabras, se ha propuesto un argumento circular, por lo que no es posible verificar la proposición original de este modo. Requerimos pues, depender únicamente de axiomas o proposiciones previamente probadas para continuar.

\pagebreak

\subsection*{Lista de Ejercicios 3 (LE3)}
    \begin{enumerate}[label=\alph*)]

    \item Demuestre que $\forall a\in \R$, $a \cdot 0 = 0$. (Multiplicación por $0$).
    \vspace{-1em}\begin{proof} 
        \begin{align*}
            a\cdot0&=a\cdot0+0 && \text{Neutro aditivo}\\
            &=a\cdot0+\bigl(a+\left(-a\right)\bigr) && \text{Inverso aditivo}\\
            &=a\cdot0+\bigl(a\cdot1+\left(-a\right)\bigr) && \text{Neutro multiplicativo}\\
            &=\left(a\cdot0+a\cdot1\right)+\left(-a\right) && \text{Asociatividad}\\
            &=\bigl(a\cdot\left(0+1\right)\bigr)+\left(-a\right) && \text{P. Distributiva}\\
            &=a\cdot1+\left(-a\right) && \text{Neutro aditivo}\\
            &=a+\left(-a\right) && \text{Neutro multiplicativo}\\
            &=0 && \text{Inverso aditivo} \qedhere
        \end{align*}    
    \end{proof} \vspace{-1em}

    \item Si $a$ y $b$ son números reales tales que $ a \cdot b = 0 $, demuestre que $a=0$ o $b=0$.
    
    \vspace{-1em}\begin{proof} 
    Supongamos que $a$ es distinto de $0$.
    \begin{align*}
    b &= b \cdot 1	&& \text{Neutro multiplicativo} \\
    &= b \cdot  \left(a \cdot a^{-1}  \right) 	&& \text{Inverso multiplicativo} \\
    &= \left(b\cdot a\right)  \cdot a^{-1}	&& \text{Asociatividad} \\
    &= \left(a\cdot b\right)  \cdot a^{-1}	&& \text{Conmutatividad} \\
    &= 0 \cdot a^{-1}	&& \text{Por hipótesis}\\
    &= a^{-1} \cdot 0	&& \text{Conmutatividad}\\
    &= 0 && \text{Multiplicación por $0$} \qedhere
    \end{align*}    
    \end{proof} \vspace{-1em}

    \textbf{Nota:} Esta proposición es verdadera si al menos uno de los números $a$ o $b$ resultan ser igual a $0$. Aunque también podríamos negar la igualdad para ambos y llegar a una contradicción; para ello, al procedimiento anterior añadimos el supuesto de que a su vez $b\neq0$, alcanzando la contradicción a partir de este hecho.

    \item $(-1)=(-1)^{-1}$.
    
    \vspace{-1em}\begin{proof} 
        \begin{align*}
            1 + (-1) &= 0 && \text{Inverso aditivo}\\
            (-1)\cdot (-1)^{-1} + (-1) &= 0 && \text{Inverso multiplicativo}\\
            (-1)\cdot (-1)^{-1} + (-1) \cdot 1 &= 0 && \text{Neutro multiplicativo}\\
            (-1) \cdot \bigl((-1)^{-1} + 1\bigr) &= 0 && \text{P. Distributiva}
        \end{align*} De la igualdad anterior, sigue que $(-1)=0$ o $(-1)^{-1} + 1=0$. Veamos los casos: \begin{enumerate}[label=\roman*)]
            \item Sup. $(-1)=0$. El inverso aditivo satisface que $1+(-1)=0$, pero de la hipótesis sigue que $1+0=0$, y por neutro aditivo, $1=0$, lo que contradice el axioma del neutro multiplicativo. Por lo que descartamos este caso.
            \item Sup. $(-1)^{-1} + 1=0$; por conmutatividad $1+(-1)^{-1}=0$, lo que implica que $(-1)^{-1}$ es inverso aditivo de $1$, y por unicidad, $(-1)^{-1}=(-1)$. \qedhere
        \end{enumerate}
    \end{proof} \vspace{-1em}

    \item Sean $a$ y $b$ números reales, demuestre que: $ (-a) \cdot b = -(a \cdot b) = a \cdot (-b)$. (Multiplicación por inverso aditivo).% (Ley de los signos: menos por más/más por menos es menos).
    
    \vspace{-1em}\begin{proof} 
        \begin{align*}
            0 &= b\cdot 0 && \text{Multiplicación por $0$}\\
            &= b \cdot \bigl(a+(-a)\bigr) && \text{Inverso aditivo}\\
            &= b\cdot a + b\cdot (-a) && \text{P. Distributiva}\\
            &= a\cdot b + (-a) \cdot b && \text{Conmutatividad}
        \end{align*} De la igualdad anterior sigue que $(-a)\cdot b$ es inverso aditivo de $a\cdot b$, y por la unicidad del inverso aditivo, se verifica que $(-a)\cdot b = -(a\cdot b)$.
        
        Análogamente,
        \begin{align*}
            0 &= a\cdot 0 && \text{Multiplicación por $0$}\\
            &= a \cdot \bigl(b+(-b)\bigr) && \text{Inverso aditivo}\\
            &= a\cdot b + a\cdot (-b) && \text{P. Distributiva}
        \end{align*} De la igualdad anterior sigue que $a\cdot (-b)$ es inverso aditivo de $a\cdot b$, y por la unicidad del inverso aditivo, se verifica que $a\cdot (-b) = -(a\cdot b)$.   
    \end{proof} \vspace{-1em}
    %Otra forma:
    %\vspace{-1em}\begin{proof} 
    %\begin{align*}
    %    (-a) \cdot b &= \bigl( \left(-1 \right) \cdot a \bigr) \cdot b && \text{Multiplicación por ($-1$)}\\
    %    &= (-1) \cdot (a \cdot b) && \text{Asociatividad}\\
    %    &= -(a \cdot b) && \text{Multiplicación por ($-1$)} && \text{(*)}\\
    %    &= -(b \cdot a) && \text{Conmutatividad}\\
    %    &= (-1) \cdot (b\cdot a) && \text{Multiplicación por ($-1$)}\\
    %    &= \bigl((-1)\cdot b\bigr) \cdot a && \text{Asociatividad}\\
    %    &= a\cdot \bigl((-1)\cdot b\bigr) && \text{Conmutatividad}\\
    %    &= a \cdot (-b) && \text{Multiplicación por ($-1$)} && \text{(**)}
    %    \end{align*} Por (*) y (**) tenemos que $ (-a) \cdot b = -(a \cdot b) = a \cdot (-b)$.
    %\end{proof} \vspace{-1em}
    \textbf{Nota:} A partir de esta demostración evitamos la ambigüedad que se mencionó en el apartado \textit{Una nota sobre rigurosidad}.

    \textbf{\textit{Corolario:}}
    \begin{enumerate}[label=\roman*)]
        \item Si $a$ y $b$ son números reales, entonces, $ (-a) \cdot (-b) = a \cdot b $.% (Ley de los signos: menos por menos es más).
    
        Por el teorema, $(-a)\cdot (-b) = a\cdot \bigl(-(-b)\bigr)$, y sabemos que $-(-b)=b$, por lo que $(-a)\cdot (-b) =a\cdot b$.
        %Otra demostración
        %\vspace{-1em}\begin{proof} 
        %    \begin{align*}
        %        (-a) \cdot (-b) &= (-a) \cdot \bigl( (-1) \cdot b \bigr) && \text{Multiplicación por ($-1$)}\\
        %        &= \bigl( (-a) \cdot (-1) \bigr) \cdot b && \text{Asociatividad}\\
        %        &= \bigl( (-1) \cdot (-a) \bigr) \cdot b && \text{Conmutatividad}\\
        %        &= -(-a) \cdot b && \text{Multiplicación por ($-1$)}\\
        %        &= a \cdot b && \text{Unicidad del inverso aditivo} \qedhere
        %    \end{align*}    
        %\end{proof} \vspace{-1em}
        %\textbf{Nota:} A las proposiciones $ (-m) \cdot n = -(m \cdot n) $ y $ (-m) \cdot (-n) = m \cdot n $, las enunciaremos como \textbf{leyes de los signos}.

        \item Si $a\in \R$, entonces, $(-1) \cdot a =-a $. (Multiplicación por -1).

        Por el teorema, $(-1)\cdot a = -(1\cdot a) = -a$.
        
        %\vspace{-1em}\begin{proof} 
        %    \begin{align*}
        %    0 &= a \cdot 0 && \text{Multiplicación por $0$}\\
        %    &= a \cdot \bigl(1+(-1)\bigr) && \text{Inverso aditivo}\\
        %    &= a \cdot 1 + a \cdot (-1)  && \text{P. Distributiva}\\
        %    &= a + a \cdot (-1)  && \text{Neutro multiplicativo}\\
        %    &= a + (-1) \cdot a  && \text{Conmutatividad}
        %    \end{align*} Sigue que $(-1) \cdot a$ es inverso aditivo de $a$, el cual es único, por lo que $(-1) \cdot a = -a$.
        %\end{proof} \vspace{-1em}
        %\textbf{Observación:} Al multiplicar cualquier número real por ($-1$) obtenemos el inverso multiplicativo de ese número real.
        %
        %Otra forma de demostrar este hecho es la siguiente: \begin{align*}
        %    -a&=-a+0 && \text{Neutro aditivo} \\
        %    &=-a+a \cdot 0 && \text{Multiplicación por $0$} \\
        %    &=-a+a \cdot  \bigl( 1+ \left( -1 \right)  \bigr) && \text{Inverso aditivo} \\
        %    &=-a+ \bigl( a \cdot 1+a \cdot  \left( -1 \right)  \bigr) && \text{P. Distributiva} \\
        %    &=-a+ \bigl( a+a \cdot  \left( -1 \right)  \bigr) && \text{Inverso aultiplicativo} \\
        %    &= \left( -a+a \right) +a \cdot  \left( -1 \right) && \text{Asociatividad} \\
        %    &=a+ \left( -a \right) +a \cdot  \left( -1 \right) && \text{Conmutatividad} \\
        %    &=0+a \cdot  \left( -1 \right) && \text{Inverso aditivo} \\
        %    &=a \cdot  \left( -1 \right) + 0 && \text{Conmutatividad} \\
        %    &=a \cdot  \left( -1 \right) && \text{Neutro aditivo} \\
        %    &= \left( -1 \right)  \cdot a && \text{Conmutatividad}
        %\end{align*} \qed

        \item $-(a^{-1})=(-a)^{-1}$. \vspace{-1em} \begin{align*}
            (-a)^{-1} &= \bigl((-1)\cdot a\bigr)^{-1} && \text{Multiplicación por -1}\\
            &= (-1)^{-1} \cdot a^{-1} && \text{(e) de LE2}\\
            &= -1 \cdot a^{-1} && \text{(c) de LE3}\\
            &= -a^{-1} && \text{Multiplicación por -1}
        \end{align*}

    \end{enumerate}% casos especiales (-a)b=-ab

    \item Demuestre que para todo número real $a$ distinto de cero, $a^{-1}\neq 0$. (Cero no es inverso multiplicativo).
    
    \vspace{-1em}\begin{proof} 
        Sea $a$ un número real distinto de cero tal que $a^{-1}=0$. Al multiplicar por $0$ tenemos que $a\cdot a^{-1}=0$. Además, el inverso multiplicativo satisface que $a\cdot a^{-1}=1$, por lo que $1=0$, pero esto contradice el axioma del neutro multiplicativo. Entonces, $a^{-1}\neq 0, \forall a\in \R\backslash \set{0}$.    
    \end{proof} \vspace{-1em}

    \textbf{Nota:} El axioma del neutro multiplicativo no implica directamente que $0$ no pueda ser inverso multiplicativo de algún número, únicamente indica que si el número es diferente de cero existe su inverso multiplicativo. Por otra parte, el axioma no especifica que para 0 el inverso multiplicativo no existe, sin embargo, si suponemos su existencia, podemos probar que implica la misma contradicción a la que llegamos en esta demostración (el lector debería verificar este hecho).

    Por otra parte, apesar de que no hemos definido la división, esta puede ser entendida como la multiplicación por algún inverso multiplicativo. De esta demostración podemos concluir que no es posible dividir por cero.

\end{enumerate}

\pagebreak

    \subsection*{Una nota sobre notación}

    Reutilizar \textit{esquemas} de proposiciones probadas nos permite agilizar la escritura de demostraciones; establecer notación tiene el mismo propósito. No obstante, cada vez que acordemos el uso de notación esta debe ser justificada; en ocasiones la justificación es tan simple como utilizar diferentes \textit{etiquetas} para los mismos \textit{objetos}, pero en otras, la notación requiere de explicación para evitar ambigüedad.

    \textbf{Notación:}

    \begin{enumerate}[label=\roman*)]
    \item Si $m$ y $n$ son números reales, representaremos con el símbolo $m-n$ a la suma $m+ (-n)$.
    
    Podemos usar esta notación sin ambigüedad ya que hemos probado que $(-a)\cdot b = -(a \cdot b)$.

    \item Si $m_1$, $m_2$ y $m_3$ son números reales, representaremos con el símbolo $m_1+m_2+ m_3$ a la suma de estos.
    
    Podemos usar esta notación sin ambigüedad ya que hemos probado que todas las formas de sumar tres números reales son equivalentes.

    \item Si $m$ y $n$ son números reales, representaremos con el símbolo $mn$ a la multiplicación $m\cdot n$; a esta multiplicación la llamaremos el producto de $m$ y $n$.
    
    A partir de esta notación, tenemos que $a\cdot (b\cdot c)= a\cdot bc=abc$; ya que hemos verificado que todas las formas de multiplicar tres números reales son equivalentes, tenemos que $abc$ representa el producto de cualesquiera números reales $a$, $b$ y $c$ sin importar la asociatividad presente.

    Observemos que sería un error reescribir la multiplicación $a\cdot (-b)$ como $a-b$, pues dicha notación se ha reservado para la suma en (i), por lo que $a\cdot (-b)$ debería reescribirse como $a(-b)$.

    \item Si $m$ es un número real, representaremos con el símbolo $m^2$ a la multiplicación $m\cdot m$, y lo llamaremos el cuadrado de $m$.
    
    \item Si $m$ es un número real, representaremos con el símbolo $-m^{-1}$ al inverso multiplicativo de $-m$ o al inverso aditivo de $m^{-1}$.
        
    Podemos usar esta notación sin ambigüedad ya que hemos probado que $-(a^{-1})=(-a)^{-1}$.
    
    \item Si $m$ y $n$ son números reales y $n$ es distinto de cero, representaremos con el símbolo $ \frac{m}{n}$ al número $m \cdot n^{-1} $.
    
    \item Al número $1+1$ lo denotaremos con el símbolo $2$ y lo llamaremos número dos. Al número $2+1$ lo denotaremos con el símbolo $3$ y lo llamaremos número tres...
    \end{enumerate}

\pagebreak

\subsection*{Lista de ejercicios 4 (LE4)}

Sean $a$, $b$, $c$ y $d$ números reales, demuestre lo siguiente:

\begin{enumerate}[label=\alph*)]
    \item Si $ab^{-1}=ba^{-1}$, entonces $a^2=b^2$.
    \vspace{-1em}\begin{proof} 
        \begin{align*}
            a^2 &= a\cdot a && \text{Notación}\\
            &= a\cdot a \cdot 1 && \text{Neutro multiplicativo}\\
            &= a\cdot a\cdot b\cdot b^{-1} && \text{Inverso multiplicativo}\\\
            &= a\cdot a\cdot b^{-1} \cdot b && \text{Conmutatividad}\\
            &= a\cdot b\cdot a^{-1} \cdot b && \text{Hipótesis}\\
            &= a\cdot a^{-1} \cdot b \cdot b&& \text{Conmutatividad}\\
            &= 1 \cdot b\cdot b && \text{Neutro multiplicativo}\\
            &= b^2 && \text{Notación} \qedhere
        \end{align*}
    \end{proof} \vspace{-1em}

    \item Si $a^2=b^2$, entonces $a=b$ o $a=-b$.
    \vspace{-1em}\begin{proof} 
        \begin{align*}
            0 &= b^2 - b^2 && \text{Inverso aditivo}\\
            &= a^2 - b^2 && \text{Hipótesis}\\
            &= a^2 - b^2 + 0&& \text{Neutro aditivo}\\
            &= a^2 - b^2 + ab-ab&& \text{Inverso aditivo}\\
            &= a^2 + a b - b^2 -ab && \text{Conmutatividad}\\
            &= (a^2 + a b) + (- b^2)+(-a b)&& \text{Asociatividad}\\
            &= (a a + a b) + (- b b)+(-a b)&& \text{Notación}\\
            &= (a a + a b) -(b b + a b)&& \text{Distribución del signo}\\
            &= a (a+b) - b(b+a) && \text{P. Distributiva}\\
            &= (a+b)  (a-b) && \text{P. Distributiva}
            \end{align*} Por el ejercicio (b) de LE3, de la igualdad anterior tenemos que $a+b=0$ o $a-b=0$. Sumando inverso aditivo de $b$ en ambas igualdades tenemos que $a=-b$ o $a=b$.
    \end{proof} \vspace{-1em}

    \item $(a+b)(a-b)=a^2-b^2$. (Diferencia de cuadrados).
    \vspace{-1em}\begin{proof} 
        \begin{align*}
        (a+b)(a-b) &= (a+b)\bigl(a+(-b)\bigr) && \text{Notación}\\
        &= (a+b)\cdot a + (a+b)\cdot (-b) && \text{P. Distributiva}\\
        &= a(a+b) + (-b)(a+b) && \text{Conmutatividad}\\
        &= a\cdot a + a\cdot b + (-b)\cdot a + (-b) \cdot b && \text{P. Distributiva}\\
        &= a^2+ ab -ba -b^2 && \text{Notación}\\
        &=a^2-b^2 && \text{Inverso aditivo} \qedhere
        \end{align*}    
    \end{proof} \vspace{-1em}

    \item $a=\frac{a}{1}$. (División por $1$).
    \vspace{-1em}\begin{proof} 
        El neutro multiplicativo satisface que $a=a\cdot 1=a\cdot 1^{-1}$, y por notación $a=\frac{a}{1}$.
    \end{proof} \vspace{-1em}

\pagebreak

    \item $\frac{a}{b} \cdot \frac{c}{d} = \frac{ac}{bd}$, si $b, d \neq 0$.
    \vspace{-1em}\begin{proof} 
        \begin{align*}
            \frac{a}{b} \cdot \frac{c}{d} &= \left( a \cdot b^{-1} \right) \cdot \left( c \cdot d^{-1} \right) && \text{Notación}\\
                &= a \cdot \Bigl( b^{-1} \cdot \left( c \cdot d^{-1} \right) \Bigr) && \text{Asociatividad}\\
                    &= a \cdot \Bigl( c \cdot \left( b^{-1} \cdot d^{-1} \right) \Bigr) && \text{Conmutatividad}\\
            &= \left( a \cdot c \right) \cdot \left( b^{-1} \cdot d^{-1} \right) && \text{Asociatividad}\\
            &= \left( a \cdot c \right) \cdot \left( b \cdot d \right)^{-1} && \text{(e) de LE2}\\
            &= \frac{ac}{bd} && \text{Notación} \qedhere
        \end{align*}    
    \end{proof} \vspace{-1em}

    \textbf{\textit{Corolario:}}\begin{enumerate}[label=\roman*)]
        \item $\frac{a}{b}=a\cdot \frac{1}{b}$, si $b\neq 0$. (Definición de la división).
        \begin{align*}
            a \cdot \frac{1}{b} &= \frac{a}{1}\cdot \frac{1}{b} && \text{División por $1$}\\
            &= \frac{a\cdot 1}{1\cdot b} && \text{Teorema}\\
            &= \frac{a}{b} && \text{Neutro multiplicativo} \qedhere
        \end{align*}

        \item $a \cdot \frac{c}{b} = \frac{ac}{b}$, si $b \neq 0$.
        
        De la división por $1$, tenemos que $a\cdot \frac{c}{b}=\frac{a}{1}\cdot \frac{c}{b}$, y por este teorema $\frac{a}{1}\cdot \frac{c}{b}=\frac{ac}{b\cdot 1}$, osea $a \cdot \frac{c}{b} = \frac{ac}{b}$.
        %Otra forma de demostrar esta proposición es la siguiente:
        %\begin{align*}
        %    a \cdot \frac{c}{b} &= a \cdot \left( c \cdot b^{-1} \right) && \text{Notación}\\
        %    &= \left( a \cdot c \right) \cdot b^{-1} && \text{Asociatividad}\\
        %    &= \frac{ac}{b} && \text{Notación}
        %\end{align*}   
        
        \item $\frac{a}{b} = \frac{ac}{bc}$, si $b,c \neq 0$.\begin{align*}
            \frac{a}{b}&=\frac{a}{b}\cdot 1 && \text{Neutro multiplicativo}\\
            &= \frac{a}{b}\cdot \bigl(c \cdot c^{-1}\bigr) && \text{Inverso multiplicativo}\\
            &= \frac{a}{b} \cdot \frac{c}{c} && \text{Notación}\\
            &= \frac{ac}{bc} && \text{Teorema} \qedhere
        \end{align*}
        %Otra forma de demostrar esta proposición es la siguiente:
        %\begin{align*}
        %    \frac{ac}{bc} &= a \cdot c \cdot \left( bc \right)^{-1} && \text{Notación}\\
        %    &= a \cdot c \cdot b^{-1} \cdot c^{-1} && \text{(e) de LE2}\\
        %    &= a \cdot b^{-1} \cdot c \cdot c^{-1} && \text{Conmutatividad}\\
        %    &= a \cdot b^{-1} \cdot 1 && \text{Inverso multiplicativo}\\
        %    &= a \cdot b^{-1} && \text{Neutro multiplicativo}\\
        %    &= \frac{a}{b} && \text{Notación}
        %\end{align*}    

        \item $\frac{-a}{-b}=\frac{a}{b}$, si $b\neq 0$.
        \begin{align*}
            \frac{-a}{-b} &= \frac{(-1)\cdot a}{(-1)\cdot b} && \text{Multiplicación por ($-1$)}\\
            &=\frac{-1}{-1} \cdot \frac{a}{b} && \text{Teorema}\\
            &= (-1)\cdot (-1)^{-1}\cdot \frac{a}{b} && \text{Notación}\\
            &= 1\cdot \frac{a}{b} && \text{Inverso multiplicativo}\\
            &= \frac{a}{b} && \text{Neutro multiplicativo}
            \end{align*}
        \textbf{Nota:} En esta prueba está implícito que $-b$ tiene inverso multiplicativo, lo cual es válido ya que de la hipótesis y (f) de LE3, sigue que $-b\neq 0$.
    \end{enumerate} %corolario

\pagebreak

    \item $\frac{a}{b} \pm \frac{c}{d} = \frac{ad \pm bc}{bd} $, si $b, d \neq 0$. (Suma de fracciones).
    
    \textbf{Demostración:}\begin{align*}
    \frac{a}{b} \pm \frac{c}{d}  &=	a \cdot b^{-1} \pm c \cdot d^{-1} && \text{Notación}\\
    &=	\left( a \cdot 1 \right)   \cdot b^{-1} \pm \left( c \cdot 1 \right) \cdot d^{-1} && \text{Neutro multiplicativo}\\
    &=	\Bigl( a \cdot  \left( d \cdot d^{-1} \right) \Bigr) \cdot b^{-1} \pm \Bigl( c \cdot  \left( b \cdot b^{-1}  \right)  \Bigr)  \cdot d^{-1} && \text{Inverso multiplicativo}\\
    &=	\Bigl(  \left( a \cdot d \right) \cdot d^{-1} \Bigr) \cdot b^{-1} \pm \Bigl(  \left( c \cdot b \right) \cdot b^{-1} \Bigr) \cdot d^{-1} && \text{Asociatividad}\\
    &=	\left( a \cdot d \right)  \cdot  \left( d^{-1} \cdot b^{-1}  \right) \pm \left( c \cdot b \right)  \cdot  \left( b^{-1} \cdot d^{-1}  \right) && \text{Asociatividad}\\
    &=	\left( a \cdot d \right)  \cdot  \left( b^{-1} \cdot d^{-1}  \right) \pm \left( c \cdot b \right)  \cdot  \left( b^{-1} \cdot d^{-1}  \right) && \text{Conmutatividad}\\
    &=	\left( b^{-1} \cdot d^{-1}  \right) \cdot \left( a \cdot d \pm c \cdot b \right) && \text{P. Distributiva}\\
    &=	\left( a \cdot d \pm c \cdot b \right) \cdot  \left( b^{-1} \cdot d^{-1} \right) && \text{Conmutatividad}\\
    &=	\left( a \cdot d \pm c \cdot b \right) \cdot  \left( b \cdot d \right)^{-1} && \text{(e) de LE2}\\
    &=	\left( a \cdot d \pm b \cdot c \right) \cdot \left( b \cdot d \right)^{-1} && \text{Conmutatividad}\\
    &=	\frac{ad \pm bc}{bd} && \text{Notación}
    \end{align*}

    \item $\frac{a}{-b} = -\frac{a}{b}=\frac{-a}{b}$, si $b\neq 0$.
    
    \textbf{Demostración:} \begin{align*}
        \frac{a}{-b} &= \frac{-(-a)}{-b} && \text{Unicidad del inverso aditivo}\\
        &= \bigl(-(-a)\bigr) \cdot (-b)^{-1} && \text{Notación}\\
        &=(-1)\cdot (-a) \cdot (-b)^{-1} && \text{Multiplicación por ($-1$)}\\
        &=(-1) \cdot \frac{-a}{-b} && \text{Notación}\\
        &= (-1) \cdot \frac{a}{b} && \text{(d) de LE4}\\
        &= -\frac{a}{b} && \text{Multiplicación por ($-1$)} && \text{(*)}\\
        &= (-1) \cdot \frac{a}{b} && \text{Multiplicación por ($-1$)}\\
        &= \frac{(-1)\cdot a}{b} && \text{(a) de LE4}\\
        &= \frac{-a}{b} && \text{Multiplicación por ($-1$)} && \text{(**)}
    \end{align*} De las igualdades (*) y (**) tenemos que $\frac{a}{-b} = -\frac{a}{b}=\frac{-a}{b}$.

\pagebreak

    \item $\frac{\frac{a}{b}}{\frac{c}{d}} = \frac{ad}{bc}$, si $b, c, d \neq 0$. (Regla del sandwich).
    
    \textbf{Demostración:}\begin{align*}
        \frac{\frac{a}{b}}{\frac{c}{d}} &= \frac{\left( a \cdot b^{-1} \right)}{\left( c \cdot d^{-1} \right)} && \text{Notación}\\
        &= \left( a \cdot b^{-1} \right) \cdot \left( c \cdot d^{-1} \right)^{-1} && \text{Notación}\\
        &= \left( a \cdot b^{-1} \right) \cdot \left( c^{-1} \cdot \left( d^{-1} \right) ^{-1} \right) && \text{(e) de LE2}\\
        &= \left( a \cdot b^{-1} \right) \cdot \left( c^{-1} \cdot d \right) && \text{Unicidad del inverso multiplicativo}\\
        &= \left( a \cdot b^{-1} \right) \cdot \left( d \cdot c^{-1} \right) && \text{Conmutatividad}\\
        &= \frac{a}{b} \cdot \frac{d}{c} && \text{Notación}\\
        &= \frac{ad}{bc} && \text{(d) de LE4}
    \end{align*} 

\end{enumerate}

\textbf{Ejercicio:} Pruebe que si $a,b\in \R$ son tales que $a+(-b)=b+(-a)$, entonces $a=b$.
    
Un intento por demostrar la proposición anterior puede lucir como sigue: \begin{align*}
a+(-b)&=b+(-a) && \text{Hipótesis} \\
\bigl(a+(-b)\bigr)+b&=\bigl(b+(-a)\bigr)+b&&\text{Ley de la cancelación} \\
a + \bigl((-b)+b\bigr)&= b+\bigl((-a)+b\bigr)&&\text{Asociatividad} \\
a + \bigl(b+(-b)\bigr)&= b+\bigl(b+(-a)\bigr)&&\text{Conmutatividad} \\
a + 0&= b+\bigl(b+(-a)\bigr)&&\text{Inverso aditivo} \\
a + 0&= (b+b)+(-a)&&\text{Asociatividad} \\
(a + 0) + a&= \bigl((b+b)+(-a)\bigr)+a&&\text{Ley de la cancelación} \\
(a + 0) + a&= (b+b)+\bigl((-a)+a\bigr)&&\text{Asociatividad} \\
(a + 0) + a&= (b+b)+\bigl(a+(-a)\bigr)&&\text{Conmutatividad} \\
(a + 0) + a&= (b+b)+0&&\text{Inverso aditivo} \\
a+ a&= b+b&&\text{Neutro aditivo} \\
a\cdot 1+ a\cdot 1&= b\cdot 1+ b\cdot 1&&\text{Neutro multiplicativo} \\
a\cdot (1+1)&= b\cdot (1+1)&&\text{P. Distributiva} \\
a\cdot (2) &= b\cdot (2) &&\text{Notación} \\
a &= b && \text{¿Ley de la cancelación?}
\end{align*}

El lector cuidadoso notará que la ley de la cancelación (de la multiplicación) exige que el número a \textit{ser cancelado} sea diferente de $0$. Sin embargo, hasta este punto no hemos demostrado que $2\neq 0$, por lo que la proposición no puede ser demostrada utilizando este hecho. Sin embargo, con los axiomas que hemos listado y los resultados que hemos derivado de ellos no es suficiente para probar tal proposición (el lector debería verificar este hecho).

Por lo anterior, resulta necesario añadir elementos a nuestro conjunto de axiomas.

\pagebreak
\section*{Axiomas de orden}

Existe un subconjunto del conjunto de los números reales llamado conjunto de los números reales positivos, denotado con el símbolo $\R^+$. A los elementos de este conjunto los llamaremos números reales positivos. El conjunto $\R^+$ satisface los siguientes \textbf{axiomas (de orden)}:
%
\begin{enumerate}[label=O\arabic*)]
\item Si $m, n \in \R^+$, entonces $m + n \in \R^+$. (Cerradura de la suma en $\R^+$)
\item Si $m, n \in \R^+$, entonces $m \cdot n \in \R^+$. (Cerradura de la multiplicación en $\R^+$)
\item Para cada número real $m$ se cumple una y sólo una de las siguientes condiciones (Tricotomía):
    \begin{enumerate}[label=\roman*)]
    \item $m \in \R^+$.
    \item $m = 0$.
    \item $-m \in \R^+$.\
    \end{enumerate}
\end{enumerate}

\textbf{Definición:} Sean $a$ y $b$ números reales, decimos que:%
%
\begin{itemize}
    \item $a$ es menor que $b$ o que $b$ es mayor que $a$ y escribimos $a<b$ o $b>a$, si $b-a \in \R^+$.
    \item $a$ es menor que o igual que $b$ o que $b$ es mayor o igual que $a$, y escribimos $a \leq b$ o $b \geq a$, si $b - a \in \R^+$ o $a = b$.
\end{itemize}

\textbf{Notación:} Sean $a$, $b$ y $c$ números reales, utilizaremos la notación $a<b<c$ para indicar que $a<b$ y $b<c$.

\subsection*{Lista de Ejercicios 5 (LE5)}

Sean $a$, $b$, $c$ y $d$ números reales, demuestre lo siguiente:
\begin{enumerate}[label=\alph*)]
    \item $a \in \R^+$ si y solo si $a>0$. (Definición de número real positivo).
    \vspace{-1em}\begin{proof} \leavevmode
        \begin{itemize}
            \item[$\Rightarrow)$] Supongamos que $a \in \R^+$. El cero satisface que $a=a+0=a-0$, y por la hipótesis $a-0 \in \R^+$, lo que por definición implica que $a>0$.
            \item[$\Leftarrow)$] Supongamos que $a>0$. Por definición, $a-0 \in \R^+$, y sabemos que el cero satisface que $a-0=a+0=a$, por lo que $a \in \R^+$. \qedhere
        \end{itemize}
    \end{proof} \vspace{-1em}

    \textbf{Observación:} Todo número real positivo es mayor a cero, y todo número real mayor a cero es un número real positivo.
    
    \textbf{Nota:} Esta demostración, cuya forma es $m\in \R^+ \iff m>0, \forall m\in \R$, nos permite reparar en el hecho de que la definición de un número real positivo no está asociada al signo que acompaña al número, es decir, la proposición es válida para $-a>0$, es decir, $-a\in \R^+ \iff -a>0, \forall -a\in \R$. El lector debería verificar este hecho.

    \item $1 \in \R^+$. (Uno es positivo).
    \vspace{-1em}\begin{proof} 
        Supongamos que $1 \notin \R^+$. Por axioma del neutro multiplicativo, sabemos que $1\neq 0$. Luego, por tricotomía tenemos que $-1 \in \R^+$. Por la cerradura de la multiplicación en $\R^+$ se verifica que $(-1) \cdot (-1) \in \R^+$, y por la ley de los signos tenemos que $(-1) \cdot (-1) = 1 \in \R^+$, pero esto contradice el supuesto inicial. Por tanto, $1$ es un número real positivo.    
    \end{proof} \vspace{-1em}

\pagebreak

    \item $a<b$ si y solo si $a+c<b+c$. (Ley de la cancelación de la suma en desigualdades).
    
    \vspace{-1em}\begin{proof} \leavevmode
    \begin{itemize}
        \item[$\Rightarrow)$] Si $a<b$, por definición, $b-a \in \R^+$. Luego, \begin{align*}
            b - a &= b -a + 0 && \text{Neutro aditivo}\\
            &= b-a+c-c && \text{Inverso aditivo}\\
            &= b+c-a-c && \text{Conmutatividad}\\
            &= b+c+(-a)+(-c) && \text{Notación}\\
            &= b+c+(-1)a+(-1)c && \text{Multiplicación por ($-1$)}\\
            &= b+c+(-1)(a+c) && \text{P. Distributiva}\\
            &= b+c+\big(-(a+c)\big) && \text{Multiplicación por ($-1$)}\\
            &= b+c-(a+c) && \text{Notación}
            \end{align*} De este modo, $b+c-(a+c)\in \R^+$, es decir, $a+c<b+c$.
        \item[$\Leftarrow)$] Si $a+c<b+c$, por definición $b+a-(a+c)\in \R^+$. Luego, \begin{align*}
            b+c-(a+c) &= b+c-a-c && \text{Distribución del signo}\\
            &= b-a+c-c && \text{Conmutatividad}\\
            &= b-a+0 && \text{Inverso aditivo}\\
            &= b-a && \text{Neutro aditivo}
        \end{align*} De este modo, $b-a\in \R^+$, es decir, $a<b$. \qedhere
        \end{itemize}    
    \end{proof} \vspace{-1em}

    \textbf{Nota:} Si el contexto es claro, enunciaremos esta proposición como ley de la cancelación.

    \item Si $a<b$ y $c < d$, entonces $a+c<b+d$. (La suma de desigualdades preserva el orden).
    
    \vspace{-1em}\begin{proof} 
        Por definición $b-a\in \R^+$ y $d-c\in \R^+$. Por la cerradura de la suma en $\R^+$ se verifica que $(b-a)+(d-c) \in \R^+$. Luego, \begin{align*}
            (b-a)+(d-c) &= b-a+d-c && \text{Notación}\\
            &= b+d-a-c && \text{Conmutatividad}\\
            &= b+d+(-a)+(-c) && \text{Notación}\\
            &= b+d+(-1)a+(-1)c && \text{Multiplicación por ($-1$)}\\
            &= b+d+ (-1) (a + c) && \text{P. Distributiva}\\
            &= b+d +\big(-(a+c) \big) && \text{Multiplicación por ($-1$)}\\
            &= b+d - (a+c) && \text{Notación}
            \end{align*}
        De este modo, $b+d-(a+c)\in \R^+$, es decir, $a+c<b+d$.    
    \end{proof} \vspace{-1em}

    \textbf{Observación:} En la suma de desigualdades se preserva el orden.

    \textbf{Nota:} Esta proposición difiere de la ley de la cancelación (de la suma en desigualdades) ya que no se satisface una doble implicación, es decir, si $a+c<b+d$, no es posible demostrar —a partir de esta hipótesis únicamente, que $a<b$. El lector debería verificar este hecho. (Ej. $a=4, c=0, b=3, d=4, a+c=4+0=4<7=3+4=b+d$, pero $a=4<3=b$ es una contradicción).

\pagebreak

    \item $-a<b$ si y solo si $-b<a$.
    \vspace{-1em}\begin{proof} \leavevmode
        \begin{center}
            \begin{minipage}[l]{.5\linewidth}
                \begin{itemize}
                    \item[$\Rightarrow)$] \begin{align*}
                        -a &< b && \text{Hipótesis}\\
                        -a+a-b &< b +a-b && \text{Ley de cancelación}\\
                        a-a-b &< b-b+a && \text{Conmutatividad}\\
                        0-b &< 0 + a && \text{Inverso aditivo}\\
                        -b &< a && \text{Neutro aditivo}
                    \end{align*}
                \end{itemize}
            \end{minipage}%
            \begin{minipage}[r]{.5\linewidth}
                \begin{itemize}
                    \item[$\Leftarrow)$] \begin{align*}
                        -b &< a && \text{Hipótesis}\\
                        -b + b-a &< a + b-a && \text{Ley de cancleación}\\
                        b-b-a &< b+a-a && \text{Conmutatividad}\\
                        0 -a &< b + 0 && \text{Inverso aditivo}\\
                        -a &< b && \text{Neutro aditivo} \qedhere
                    \end{align*}
                \end{itemize}
            \end{minipage}
            \end{center}    
    \end{proof} \vspace{-1em}

    \textbf{\textit{Corolario:}}
    \begin{enumerate}[label=\roman*)]
        \item $a<-b$ si y solo si $b<-a$.
        \begin{itemize}
            \item[$\Rightarrow)$] Si $a<-b$, por la unicidad del inverso aditivo, $-(-a) < -b$, y por este teorema, $-(-b)=b<-a$.
            \item[$\Leftarrow)$] Si $b<-a$, por la unicidad del inverso aditivo, $-(-b)<-a$, y por este teorema, $-(-a)=a<-b$.
        \end{itemize}

        \item $a<b$ si y solo si $-b<-a$.
        \begin{itemize}
            \item[$\Rightarrow)$] Si $a<b$, por la unicidad del inverso aditivo, $-(-a) < b$, y por este teorema, $-b<-a$.
            \item[$\Leftarrow)$] Si $-b<-a$, por este teorema $-(-a)=a<b$. \qedhere
        \end{itemize}
    
        \item $0<a$ si y solo si $-a<0$. (Definición de número real negativo).
        \begin{itemize}
            \item[$\Rightarrow)$] Si $0<a$, sabemos que $0=-0<a$, y por este teorema, $-a<0$.
            \item[$\Leftarrow)$] Si $-a<0$, por este teorema, $-0=0<a$. \qedhere
        \end{itemize}
        %Otra forma de demostrar este hecho es la siguiente:
        %\vspace{-1em}\begin{proof}\leavevmode
        %    \begin{enumerate}[label=\roman*)]
        %        \item Supongamos que $0<a$. Notemos que: \begin{align*}
        %            0 + (-a) &< a + (-a) && \text{Ley de la cancelación}\\
        %            0 + (-a) &< 0 && \text{Inverso aditivo}\\
        %            -a &< 0 && \text{Neutro aditivo}
        %        \end{align*}
        %        \item Supongamos que $-a<0$. Notemos que:
        %        \begin{align*}
        %            -a + a &< 0 + a && \text{Ley de la cancelación}\\
        %            0 &< 0 + a && \text{Inverso aditivo}\\
        %            0 &< a && \text{Neutro aditivo} \qedhere
        %        \end{align*}
        %    \end{enumerate}    
        %\end{proof} \vspace{-1em}
        \textbf{Observación:} El inverso aditivo de cualquier número real positivo es menor a cero.

        \textbf{Definición:} Ai $a$ es un número real tal que $a<0$, diremos que $a$ es un número real negativo. El conjunto de los números reales negativos se representa con el símbolo $\R^-$.
        %\textbf{Definición:} El conjunto de los números reales negativos se representa con el símbolo $\R^-$ y se define como los elementos en los números reales cuyos inversos aditivos pertenecen al conjunto de los números reales positivos:
    %\[
    %    \R^-\coloneqq \set{m\in \R: -a\in \R^+}
    % \]
    %  
    \end{enumerate} % Corolario

    \item Si $a<b$ y $0<c$, entonces $ac<bc$. (Multiplicación por positivo).
    
    \vspace{-1em}\begin{proof} 
        Por definición $b-a \in \R^+$ y $c \in \R^+$. Por la cerradura de la multiplicación en $\R^+$ se verifica que $c(b-a) \in \R^+$. Por la propiedad distributiva sigue que $c(b-a)=cb-ca$ y por conmutatividad tenemos que $cb-ca=bc-ac$. De este modo, $bc-ac \in \R^+$, es decir, $ac<bc$.    
    \end{proof} \vspace{-1em}

    \textbf{Observación:} La multiplicación por números reales positivos preserva el orden de la desigualdad.
    %\textbf{Nota:} De este resultado sigue que si $m<0$ y $0<n$, entonces $mn<0=0\cdot n$ —multiplicación por cero.

    \item Si $a<b$ y $c<0$, entonces $bc<ac$. (Multiplicación por negativo).
    
    \vspace{-1em}\begin{proof} 
        Por definición $b-a \in \R^+$ y $0 - c \in \R^+$, por A3 sigue que $ -c \in \R^+$. Luego, por O2 $-c(b-a) \in \R^+$. Notemos que:
    \begin{align*}
    -c(b-a) &= -c \Bigl( b + (-a) \Bigr) && \text{Notación}\\
    &= (-c) \cdot b + (-c) \cdot (-a) && \text{P. Distributiva}\\
    &= (-c) \cdot b + c \cdot a && \text{Por (k) de LE1}\\
    &= -(c \cdot b) + c \cdot a && \text{Por (i) de LE1}\\
    &= ca -(cb) && \text{Conmutatividad}\\
    &= ac - (bc) && \text{Conmutatividad}
    \end{align*}
    Entonces $ac - bc \in \R^+$, es decir, $ac>bc$.    
    \end{proof} \vspace{-1em}

    \textbf{Observación:} La multiplicación por números reales negativos cambia el orden de la desigualdad.
    %\textbf{Nota:} De esta demostración sigue que: \begin{enumerate}[label=\roman*)]
    %    \item Si $0<m$ y $n<0$, entonces $mn<0$. (Positivo por negativo/negativo por positivo es negativo).
    %    \item Si $m<0$ y $n<0$, entonces $0<mn$. (Negativo por negativo es positivo).
    %\end{enumerate}

    \item Si $0<a$, entonces $0<a^{-1}$. (Inverso multiplicativo positivo).
    
    \textbf{Demostración:} Supongamos que $a^{-1}<0$. Como $0<a$, al multiplicar en desigualdades preserva el orden, por lo que $a^{-1} \cdot a<0 \cdot a$. Por un lado, tenemos el inverso multiplicativo $a^{-1} \cdot a = 1$, y por el otro, tenemos una multiplicación por cero, $0\cdot a=0$, con lo que tenemos que $1<0$, pero sabemos —por (b) de LE3, que $1>0$, por lo que $a^{-1}$ no puede ser menor a cero. Además, sabemos que $0$ no es inverso multiplicativo —por (f) de LE3. Finalmente, por tricotomía, $a^{-1}>0$. \qed

    \item Si $a<0$, entonces $a^{-1}<0$. (Inverso multiplicativo negativo).
    
    \textbf{Demostración:} Supongamos que $0<a^{-1}$. Como $a<0$, al multiplicar en desigualdades cambia el orden, por lo que $a^{-1} \cdot a < 0 \cdot a$. Por un lado, tenemos el inverso multiplicativo $a^{-1}\cdot a =1$, y por el otro, tenemos una multiplicación por cero, $0\cdot a =0$, con lo que tenemos que $1<0$, pero sabemos —por (b) de LE3, que $1>0$, por lo que $a^{-1}$ no puede ser mayor a cero. Además, sabemos que $0$ no es inverso multiplicativo —por (f) de LE3. Finalmente, por tricotomía, $a^{-1}<0$. \qed

    \item Sea $a,b\in \R$. Encuentre las condiciones que deben cumplirse para que $ab<0$ o $0<ab$. (Ley de los signos).
    
    Si $a$ o $b$ son cero, tenemos que $ab=0$, por lo que descartamos esta posiblidad. Por tricotomía, $0<a$ o $a<0$ y $0<b$ o $b<0$, entonces observemos los casos:
    \begin{enumerate}[label=\roman*)]
        \item Si $0<a$ y $0<b$, por la cerradura de la multiplicacón en $\R^+$, tenemos que $0<ab$.
        \item Sin pérdida de generalidad, si $0<a$ y $b<0$, tenemos que $0<-b$, y por la cerradura de la multiplicación en $\R^+$, $0<-ab$, por lo que $ab<0$.
        \item Si $a<0$ y $b<0$, entonces $0<-a$ y $0<-b$, por lo que $0<(-a)(-b)=ab$.
    \end{enumerate}
    Conclusión:
    \begin{enumerate}[label=\arabic*)]
        \item Por (i) y (ii) sabemos que para verificar $ab<0$, un componente del producto debe ser positivo y el otro negativo.
        \item Por (iii) sabemos que para verificar $0<ab$, ambos componentes del producto deben ser positivos o ambos negativos.
    \end{enumerate}

    \textbf{Nota:} Nos referiremos a (1) y (2) como ley de los signos.    
    %\item $0<ab$ si y solo si $0<a$ y $0<b$ o $a<0$ y $b<0$. (Producto positivo).
    %\textbf{Demostración:}
    %\begin{itemize}
    %    \item[$\Rightarrow$)] Supongamos que $0<ab$. Sin pérdida de generalidad, observemos la tricotomía de $a$. Si $a=0$, tenemos que $ab=0$, lo que contradice la hipótesis, por lo que tenemos dos casos:
    %    \begin{enumerate}[label=\roman*)]
    %        \item si $a>0$, el inverso multiplicativo es positivo, $a^{-1}>0$. Entonces, \begin{align*}
    %            0 &< ab && \text{Hipótesis}\\
    %            0 \cdot a^{1} &< ab \cdot a^{-1} && \text{Multiplicación por positivo}\\
    %            0 &< ab \cdot a^{1} && \text{Multiplicación por cero}\\
    %            0 &< ba \cdot a^{1} && \text{Conmutatividad}\\
    %            0 &< b && \text{Inverso multiplicativo}
    %        \end{align*}
    %        \item si $a<0$, el inverso multiplicativo es negativo, $a^{-1}<0$. Entonces, \begin{align*}
    %            0 &< ab && \text{Hipótesis}\\
    %            ab \cdot a^{-1} &< 0 \cdot ab && \text{Multiplicación por negativo}\\
    %            ab \cdot a^{1} &< 0&& \text{Multiplicación por cero}\\
    %            ba \cdot a^{1} &< 0&& \text{Conmutatividad}\\
    %            b &< 0&& \text{Inverso multiplicativo}
    %        \end{align*}
    %    \end{enumerate}
    %    \item[$\Leftarrow$)] Por tricotomía, de la disyunción tenemos casos excluyentes:
    %    \begin{enumerate}[label=\roman*)]
    %        \item Si $0<a$ y $0<b$, por la cerradura de la multiplicación en $\R^+$, $0<ab$.
    %        \item Si $a<0$ y $b<0$, \begin{align*}
    %            a &< 0 && \text{Hipótesis}\\
    %            0 \cdot b &< a\cdot b && \text{Multiplicación por negativo}\\
    %            0 &< ab && \text{Multiplicación por cero}
    %        \end{align*}
    %    \end{enumerate}
    %\end{itemize} \qed

    \item Si $a<b$ y $b<c$, entonces $a<c$. (Transitividad).
    \vspace{-1em}\begin{proof} 
        Por definición $b-a \in \R^+$ y $c-b \in \R^+$. Por la cerradura de la suma, $(b-a) + (c-b) \in \R^+$. Luego, \begin{align*}
        (b-a) + (c-b) &= b - a + c -b && \text{Notación}\\
        &= b-a -b+c && \text{Conmutatividad}\\
        &= b-b -a+c && \text{Conmutatividad}\\
        &= 0 - a +c && \text{Inverso aditivo}\\
        &= -a +c && \text{Neutro aditivo}\\
        &= c-a && \text{Conmutatividad}
    \end{align*} Entonces $c-a \in \R^+$, es decir, $a<c$.    
    \end{proof} \vspace{-1em}

    \item Sea $a\in \R$. Encuentre las condiciones que deben cumplirse para que $a^{-1}<a$ o $a<a^{-1}$.
    
    Para que $\exists a^{-1}$, requerimos $a\neq 0$. También, sabemos que $a\neq 1$ y $a\neq -1$ pues $1=1^{-1}$ y $-1=(-1)^{-1}$, pero buscamos desigualdad. Entonces, observemos los casos: \begin{enumerate}[label=\roman*)]
        \item Si $a<-1$, entonces $1<-a$, y por transitividad $0<-a$, por lo que $0<-a^{-1}$, luego \begin{align*}
            1 &< -a \\
            1\cdot \bigl(-a^{-1}\bigr) &< (-a) \cdot \bigl(-a^{-1}\bigr)\\
            -a^{-1} &< 1\\
            -1 &< a^{-1}
        \end{align*} Por transitividad, $a < a^{-1}$.
        \item Si $-1<a<0$, por notación $-1<a$ y $a<0$, de donde sigue que $-a<1$ y $0<-a$, por lo que $0<-a^{-1}$. \begin{align*}
            -a &< 1 \\
            (-a) \cdot (-a^{-1}) &< 1\cdot (-a^{-1})\\
            1 &< -a^{-1}\\
            a^{-1} &< -1
        \end{align*} Por transitividad, $a^{-1}<a$.
        \item Si $0<a<1$, por notación $0<a$ y $a<1$, de donde sigue que $0<a^{-1}$. Luego \begin{align*}
            a &< 1 \\
            a\cdot a^{-1} &< 1 \cdot a^{-1}\\
            1 &< a^{-1}
        \end{align*} Por transitividad $a<a^{-1}$.
        \item Si $1<a$, por transitividad $0<a$, por lo que $0<a^{-1}$. Luego, \begin{align*}
            1 <& a\\
            1 \cdot a^{-1} &< a \cdot a^{-1}\\
            a^{-1} &< 1
        \end{align*} Por transitividad $a^{-1}<a$.
    \end{enumerate}

    Conclusión: \begin{itemize}
        \item Por (i) y (iii), $a<a^{-1}$, si $a<-1$ o $0<a<1$.
        \item Por (ii) y (iv), $a^{-1}<a$, si $-1<a<0$ o $1<a$.
    \end{itemize}

    \item Sea $a<b$. Encuentre las condiciones que deben cumplirse para que $\frac{1}{a}<\frac{1}{b}$ o $\frac{1}{b}<\frac{1}{a}$.
    
    Sabemos que $a\neq 0$ y $b\neq 0$, pues requerimos la existencia de su inverso multiplicativo. Luego, por tricotomía, $0<a$ o $a<0$ y $0<b$ o $b<0$, entonces observemos los casos: \begin{enumerate}[label=\roman*)]
        \item Si $0<a$ y $0<b$, tenemos que $0<ab$, por lo que $0<\frac{1}{ab}$. Entonces, $a\cdot \frac{1}{ab} = \frac{1}{b} < \frac{1}{a} = b \cdot \frac{1}{ab}$.
        \item Si $a<0$ y $0<b$, entonces $ab<0$, por lo que $\frac{1}{ab}<0$. De donde sigue que $b \cdot \frac{1}{ab} = \frac{1}{a} < \frac{1}{b} = a\cdot \frac{1}{ab}$.
        \item Si $b<0$ y $0<a$, por transitividad, $b<a$, lo que contradice la hipótesis, por lo que descartamos este caso.
        \item Si $a<0$ y $b<0$, entonces $0<ab$, por lo que $\frac{1}{b} < \frac{1}{a}$, como se probó en (i).
    \end{enumerate}
    Conclusión: \begin{itemize}
        \item Por (i) y (iv), si ambos componentes son positivos o ambos negativos, entonces los inversos multiplicativos invierten el orden.
        \item Por (iii), si el componente menor es negativo y el mayor es positivo, entonces los inversos multiplicativos conservan el orden.
    \end{itemize}
    %
    %\item Si $0 \leq a<b$ y $0 \leq c<d$, entonces $ac<bd$.
    %
    %\begin{enumerate}[label=\roman*)]
    %    \item Si $a=0$ o $c=0$, por (g) de LE1 se verifica que $ac=0$. Luego, por (j) de LE3, se verifica que $0<b$ y $0<d$. Así, $ac<bd$.
    %    \item Si $a>0$ y $c>0$. Por hipótesis, $a<b$, y por (e) de LE3, sigue que $ac<bc$. También, tenemos que $c<d$, y por (e) de LE3, sigue que $bc<db$. Finalmente, por (j) de LE3, se verifica que $ac<bd$. \qed
    %\end{enumerate}
    %
    %\item Si $a<b$ y $0<ab$, entonces $b^{-1}<a^{-1}$.
    %
    %Notemos que:
    %\begin{align*}
    %a &< b && \text{Por hipótesis} \\
    %a-a &< b-a && \text{Por (d) de LE3} \\
    %0 &< b-a && \text{Inverso aditivo} \\
    %0 \cdot \frac{1}{ab} &< (b-a) \cdot \frac{1}{ab} && \text{Por (h) y (e) de LE3}\\
    %0 &< \frac{b-a}{ab} && \text{Multiplicación por $0$ y (a) de LE2}\\
    %0 &< \frac{1}{a} - \frac{1}{b} && \text{Por (c) de LE2}\\
    %\frac{1}{b} &< \frac{1}{a} && \text{Por (d) de LE3}
    %\end{align*} \qed

    \item Si $a<b$ demuestre que $a<\frac{a+b}{2}<b$. (Punto medio).
    
    Por (a) de LE3 sabemos que $1 \in \R^+$, y por O1 se cumple que $1+1 \in \R^+$, es decir $2 \in \R^+$. Por (b) de LE3 se verifica que $0<2$ y por (i) de LE3 tenemos que $0<\frac{1}{2}$. Notemos que:
    \vspace{-1em}\begin{proof} \leavevmode
        \begin{center}
        \begin{minipage}[l]{.5\linewidth}
        \begin{align*}
        a &< b && \text{Por hipótesis} \\
        a + a &< b+a && \text{Ley de cancelación} \\
        2a &< b+a && \text{Por definición} \\
        2a \cdot \frac{1}{2} &< (b+a) \cdot \frac{1}{2} && \text{Por (e) de LE3} \\
        \frac{2a}{2} &< \frac{b+a}{2} && \text{Por (a) de LE2} \\
        a &< \frac{b+a}{2} && \text{Inverso multiplicativo}
        \end{align*}
        \end{minipage}%
        \begin{minipage}[r]{.5\linewidth}
        \begin{align*}
        a &< b && \text{Por hipótesis} \\
        a + b &< b+b && \text{Ley de cancelación} \\
        a +b &< 2b && \text{Por definición} \\
        (a+b) \cdot \frac{1}{2} &< 2b \cdot \frac{1}{2} && \text{Por (e) de LE3} \\
        \frac{a+b}{2} &< \frac{2b}{2} && \text{Por (a) de LE2} \\
        \frac{a+b}{2} &< b && \text{Por A8}
        \end{align*}
        \end{minipage}
        \end{center} Finalmente, por notación, $a < \frac{a+b}{2} < b$.
    \end{proof} \vspace{-1em}

    \textbf{Observación:} $b-\frac{a+b}{2} = \frac{a+b}{2}-a$ (el lector debería verificar este hecho).

    \textbf{Definición:} Al número $\frac{a+b}{2}$ lo llamaremos el punto medio entre $a$ y $b$.

    \item $a^2\geq 0$.
    
    Si $0 \leq a$, $ 0\cdot a \leq a \cdot a$, osea, $0 \leq a^2$. Si $a<0$, $0\cdot a < a \cdot a$, osea, $0 \leq a^2$. En cualquier caso $a\geq0$.\qed

    \item Sea $\frac{a}{b}<\frac{c}{d}$. Encuentre las condiciones que deben cumplirse para que $\frac{a}{b}<\frac{a+c}{b+d}<\frac{c}{d}$. (Mediante).

    Sabemos que $b\neq 0$ y $d\neq 0$. También, por definición, $\frac{c}{d}-\frac{a}{b}\in \R^+$, es decir, $(bc-ad)/(bd)>0$. Como $b$ y $d$ son distintos de cero, tenemos que $bd\neq 0$. Asimismo, $bc-ad\neq 0$.

    Buscamos que $\frac{a}{b}<\frac{a+c}{b+d}<\frac{c}{d}$, para lo que es necesario que

    \begin{center}
    \begin{minipage}[l]{.5\linewidth}
    \begin{align*}
        \frac{a}{b} &< \frac{a+c}{b+d}\\
        0 &< \frac{a+c}{b+d} - \frac{a}{b}\\
        &= \frac{b(a+c)-\bigl(a(b+d)\bigr)}{b(b+d)}\\
        &= \frac{ab+bc-ab-ad}{b(b+d)}\\
        &= \frac{bc-ad}{b(b+d)}\\
        &= \frac{bc-ad}{bd} \cdot \frac{d}{b+d}
    \end{align*}
    \end{minipage}%
    \begin{minipage}[r]{.5\linewidth}
    \begin{align*}
        \frac{a+c}{b+d} &< \frac{c}{d}\\
        0 &< \frac{c}{d} -\frac{a+c}{b+d}\\
        &= \frac{c(b+d)-\bigl(d(a+c)\bigr)}{d(b+d)}\\
        &= \frac{bc+cd-ad-cd}{d(b+d)}\\
        &= \frac{bc-ad}{d(b+d)}\\
        &= \frac{bc-ad}{bd} \cdot \frac{b}{b+d}
    \end{align*}
    \end{minipage}
    \end{center}
    Como $(bc-ad)/(bd)>0$, entonces $\frac{d}{b+d}>0$ y $\frac{b}{b+d}>0$. Por esto, $b+d\neq 0$. Finalmente, tenemos dos casos: \begin{enumerate}[label=\roman*)]
        \item Si $b>0$, entonces $b+d>0$ y $d>0$.
        \item Si $b<0$, entonces $b+d<0$ y $d<0$.
    \end{enumerate}

    Por tanto, debe cumplirse que $b$ y $d$ deben ser ambos positivos o ambos negativos.

    \item Si $0 \leq a < \varepsilon$ para toda $\varepsilon > 0$, entonces $a=0$.
    
    Supongamos que $0<a$, sigue que $0<\frac{a}{2}<a$. En particular, $\epsilon=\frac{a}{2}$, entonces $\varepsilon<a$, pero esto contradice nuestra hipótesis de que $a< \varepsilon$ para toda $\varepsilon>0$. Por tanto, $a=0$.\qed

    \item Si $a \leq b + \varepsilon$ para toda $\varepsilon > 0$, entonces $a \leq b$.
    
    Sean $a$ y $b$ números reales tales que $a \leq b + \varepsilon$, $\forall \varepsilon > 0$. Supongamos que $a > b$. Luego, $a-b>0$. Notemos que $(a-b) \cdot \frac{1}{2} > 0 \cdot \frac{1}{2}$, es decir $\frac{(a-b)}{2} > 0$. Sea $\varepsilon = \frac{(a-b)}{2}$, sigue que $a=2\varepsilon+b$. Además, $2\varepsilon > \varepsilon$, de donde obtenemos $2 \varepsilon + b > \varepsilon + b$. De este modo, $a > b+\varepsilon$, pero esto contradice nuestra hipótesis. Por tanto, $a \leq b$.\qed

\end{enumerate}

%\textbf{Definición:} El conjunto de los números reales no negativos es el conjunto: \[
%        \R^+\cup \set{0}
%        \]
%

\pagebreak

\subsection*{Valor absoluto}

\textbf{Definición:} Sea $a$ un número real, definimos el valor absoluto de $a$, denotado por $|a|$ como sigue:

    \[
        |a| = \left\{
    \begin{array}{@{}r@{\thinspace}l}
        a, &  \ \text{si}  \ a>0\\
        0, &  \ \text{si}  \ a=0\\
        -a, & \  \text{si} \  a<0
    \end{array} \right. \]

\textbf{Observación:} $|a|\geq 0, \ \forall a\in \R$.

Notemos que la definición es equivalente a las siguientes:

\begin{center}
\begin{minipage}[c]{.3\linewidth}
    \[|a| = \left\{
        \begin{array}{@{}r@{\thinspace}l}
            a, & \ \text{si} \ a\geq 0\\
            -a, & \ \text{si} \ a<0
        \end{array} \right.\]
    \end{minipage}%
\begin{minipage}[c]{.3\linewidth}
    \[|a| = \left\{
        \begin{array}{@{}r@{\thinspace}l}
            a, & \ \text{si} \ a>0\\
            -a, & \ \text{si} \ a\leq 0
        \end{array} \right.\]
\end{minipage}
\end{center}

El lector de vería verificar este hecho. (\textit{Hint}: $0=-0$).

\subsection*{Lista de Ejercicios 4 (LE4)}

Sean $a$, $b$, $c$ números reales, demuestre lo siguiente:

\begin{enumerate}[label=\alph*)]
\item $|a| \geq \pm a$.

\vspace{-1em}\begin{proof}
    Por casos.
    \begin{enumerate}[label=\roman*)]
        \item Si $0 \leq a$, por definición, $|a|=a$, por lo que $a\leq |a|$. Luego, por la hipótesis tenemos que $-a \leq 0$, y por transitividad, $-a\leq |a|$.
        \item Si $a<0$, por definición, $|a|=-a$, por lo que $-a\leq |a|$. Luego, por la hipótesis tenemos que $0<-a$, y por transitividad, $a<|a|$.
    \end{enumerate}
    En cualquier caso, $|a| \geq \pm a$.
\end{proof} \vspace{-1em}    

\item $|a|=|-a|$.
\vspace{-1em}\begin{proof}
    Por casos.
    \begin{enumerate}[label=\roman*)]
        \item Si $0 \leq a$, por definición, $|a|=a$. Luego, por la hipótesis tenemos que $-a \leq 0$. Si $-a<0$, $|-a|=a$ y si $-a=0$, $|-a|=a$. De este modo, $|a|=|-a|$.
        \item Si $a<0$, por definición, $|a|=-a$. Luego, por la hipótesis tenemos que $0<-a$, por lo que $|-a|=-a$. De este modo, $|a|=|-a|$.
    \end{enumerate}
    En cualquier caso, $|a|=|-a|$.
\end{proof} \vspace{-1em}

\item $|ab|=|a||b|$.

\vspace{-1em}\begin{proof}
    Por casos.
    \begin{enumerate}[label=\roman*)]
        \item Si $a>0$ y $b>0$, por definición, $|a|=a$ y $|b|=b$. Luego, $ab>0$ por lo que $|ab|=ab$. Por tanto, $|ab| =|a||b|$.
        \item Si $a>0$ y $b<0$, por definición, $|a|=a$ y $|b|=-b$. Luego, $ab<0$ por lo que $|ab|=-ab$. Por tanto, $|  ab|=|a||b|$.
        \item Si $a<0$ y $b<0$, por definición, $|a|=-a$ y $|b|=-b$. Luego, $ab>0$ por lo que $|ab|=ab$. Por tanto, $|  ab|=|a||b|$.
    \end{enumerate}
    En cualquier caso, $|ab|=|a||b|$.
\end{proof} \vspace{-1em}

\pagebreak

\item $|a+b|\leq |a|+|b|$. (Desigualdad del triángulo).

\vspace{-1em}\begin{proof} 
    Por casos.
    \begin{enumerate}[label=\roman*)]
        \item Si $0 \leq a+b$, entonces $|a+b|=a+b$. Como, $a \leq |a|$ y $b \leq |b|$, entonces, $a+b \leq |a|+|b|$. Por tanto, $|a+b| \leq |a|+|b|$.
        \item Si $a+b<0$, entonces $|a+b|=-a-b$. Como, $-a \leq |a|$ y $-b \leq |b|$, entonces, $-a-b \leq |a|+|b|$. Por tanto, $|a+b| \leq |a|+|b|$. \qedhere
    \end{enumerate}    
\end{proof} \vspace{-1em}

\item $ \big| |a|-|b| \big| \leq |a-b|$.
    \begin{proof} 
    Por la desigualdad del triángulo,
    \begin{center}\vspace{-2.5em}
    \begin{minipage}[t]{.5\linewidth}
    \begin{align*}
    |(a-b)+b| &\leq |a-b|+|b| \\
    |a| &\leq |a-b|+|b| \\
    |a|-|b| &\leq |a-b| && \text{(1)}
    \end{align*}
    \end{minipage}%
    \begin{minipage}[t]{.5\linewidth}
    \begin{align*}
    |(b-a)+a| &\leq |b-a|+|a| \\
    |b| &\leq |b-a|+|a| \\
    |b|-|a| &\leq |b-a| \\
    -|b-a| &\leq |a|-|b| && \text{(2)}
    \end{align*}
    \end{minipage}
    \end{center}
    Luego, aplicando (f) de LE4 en (1) y (2), $\big| |a| - |b| \big| \leq |a-b|$.    
    \end{proof} \vspace{-1em}

\item Si $b\neq 0$, entonces $\left| \frac{a}{b} \right| = \frac{|a|}{|b|}$.

\begin{enumerate}[label=\roman*)]
    \item Si $a \geq 0$ y $b>0$, entonces $|a|=a$ y $|b|=b$. Además, $\frac{1}{b} >0$, de donde sigue que $\frac{a}{b} \geq 0$ por lo que $\big| \frac{a}{b} \big| = \frac{a}{b}$. Por tanto, $ \big| \frac{a}{b} \big| = \frac{|a|}{|b|}$.
    \item Si $a \geq 0$ y $b<0$, entonces $|a|=a$ y $|b|=-b$. Además, $\frac{1}{b} <0$, de donde sigue que $\frac{a}{b} \leq 0$, por lo que $\big| \frac{a}{b} \big| =- \frac{a}{b}$. Por tanto, $ \big| \frac{a}{b} \big| = \frac{|a|}{|b|}$.
    \item Si $a<0$ y $b>0$, entonces $|a|=-a$ y $|b|=b$. Además, $\frac{1}{b} >0$, de donde sigue que $\frac{a}{b} < 0$, por lo que $\big| \frac{a}{b} \big| =- \frac{a}{b}$. Por tanto, $ \big| \frac{a}{b} \big| = \frac{|a|}{|b|}$.
    \item Si $a<0$ y $b<0$, entonces $|a|=-a$ y $|b|=-b$. Además, $\frac{1}{b} <0$, de donde sigue que $\frac{a}{b} > 0$ por lo que $\big| \frac{a}{b} \big| = \frac{a}{b}$. Por tanto, $ \big| \frac{a}{b} \big| = \frac{|a|}{|b|}$.
\end{enumerate}
\item $|a|<b$ si y solo si $-c<b<c$.

\begin{enumerate}[label=\roman*)]
    \item Supongamos que $|b|<c$. Por (a) de LE4, $ \pm b \leq |b|$, de donde sigue que $-b<c$ y $b<c$. Luego, $-c<b$. De este modo, $-c<b<c$.
    \item Supongamos que $-c<b<c$. Luego,
        \begin{enumerate}[label=\arabic*)]
            \item Si $b \geq 0$, entonces $|b|=b$. Por lo que $|b|<c$.
            \item Si $b < 0$, entonces $|b|=-b$. Por hipótesis, $-c<b$, por lo que $-b<c$. Así $|b|<c$.
        \end{enumerate}
\end{enumerate}

\item $|a|^2=a^2$.

Por (o) de LE3, $a^2\geq 0$, por lo que \begin{align*}
    a^2 &= |a^2|\\
    &= |a\cdot a|\\
    &= |a| \cdot |a| && \text{Por (b) de LE4}\\
    &= |a|^2
\end{align*} \qed
\end{enumerate}

\pagebreak

\textbf{Definición.} Sea $a \in \R$ y $\varepsilon>0$. El vecindario-$\varepsilon$ de $a$ es el conjunto $V_\varepsilon(a):=\{ x\in \R: |x-a|<\varepsilon\}$.

\subsection*{Lista de Ejercicios 5 (LE5)}

Sean $a,b \in \R$. Demuestre lo siguiente:

\begin{enumerate}[label=\alph*)]
    \item Si $x\in V_\varepsilon(a)$ para toda $\varepsilon>0$, entonces $x=a$.
    \item Sea $U:=\{x: 0<x<1\}$. Si $a\in U$, sea $\varepsilon$ el menor de los números $a$ y $1-a$. Demuestre que $V_\varepsilon(a) \subseteq U$.
    \item Demuestre que si $a\neq b$, entonces existen $U_\varepsilon(a)$ y $V_\varepsilon(b)$ tales que $U\cap V =\emptyset$.
\end{enumerate}

\textbf{Demostración.}

\begin{enumerate}[label=\alph*)]
    \item Si $x\in V_\varepsilon(a)$ tenemos que $|x-a|<\varepsilon$. Además, $0\leq |x-a|$, por definición. Así, $0\leq |x-a|<\varepsilon$. Como esta desigualdad se cumple para toda $\varepsilon>0$, por (p) de LE3, sigue que $|x-a|=0$. De este modo, $|x-a|=x-a$ con $x-a=0$. Por tanto, $x=a$. \qed
    
    \item \begin{enumerate}[label=\roman*)]
        \item Si $a>1-a$, tenememos $\varepsilon=1-a$. Sea $y\in V_\varepsilon(a)$, entonces $|y-a|<1-a$. De (f) de LE4 sigue que $a-1<y-a<1-a$ (*). Tomando el lado derecho de (*) obtenemos $y<1$. Luego, de la hipótesis sigue que $2a>1$, osea $2a-1>0$. Del lado izquierdo de la desigualdad (*), tenemos $2a-1<y$, por lo que $0<y$.
        \item Si $1-a>a$, tenemos $\varepsilon=a$. Sea $y\in V_\varepsilon(a)$, entonces $|y-a|<a$. De (f) de LE4 sigue que $-a<y-a<a$. Sumando $a$ en esta desigualdad obtenemos $0<y<2a$. Luego, de la hipótesis sigue que $1>2a$, entonces $0<y<1$.\end{enumerate}
        En cualquier caso, $0<y<1$, lo que implica que $V_\varepsilon(a) \subseteq U$.
        \qed

    \item Supongamos que para toda $U_\varepsilon(a)$ y $V_\varepsilon(b)$ se cumple que $U_\varepsilon(a) \cap V_\varepsilon(b) \neq \emptyset$. Entonces, existe $x$ tal que $x\in U_\varepsilon(a)$ y $x\in V_\varepsilon(b)$. Como en ambas vecindades tenemos $\epsilon>0$ arbitraria, por (a) de LE5, sigue que $x=a$ y $x=b$, pero esto contradice el supuesto de que $a\neq b$. Por tanto, deben existir $U_\varepsilon(a)$ y $V_\varepsilon(b)$ tales que $U\cap V =\emptyset$. \qed
    \end{enumerate}

\textbf{Definición:} Sea $A$ un subconjunto del conjunto de los números reales, decimos que $A$ es un conjunto inductivo si se cumplen las siguientes condiciones:
    \begin{enumerate}
        \item $1 \in A$.
        \item Si $n \in A$ entonces se verifica que $n+1 \in A$.
    \end{enumerate}

\subsection*{Lista de Ejercicios 6 (LE6)}

\begin{enumerate}[label=\arabic*)]
    \item ¿El conjunto de los números reales es un conjunto inductivo?
    \item ¿$\R^+$ es un conjunto inductivo?
    \item Sea $A\coloneqq \{B \subseteq B: \text{B es un conjunto inductivo}\}$. Demuestre que $A\neq \emptyset$ y que $C=\bigcap B$ es un conjunto inductivo.
\end{enumerate}

\pagebreak

\textbf{Respuesta}

\begin{enumerate}[label=\arabic*)]
    \item Sí, ya que $1 \in \R$, y si $n$ es un número real, $n+1 \in \R$ por la cerradura de la suma en $\R$.
    \item Sí, pues $1\in \R^+$ y y si $n$ es un número real positivo, $n+1 \in \R^+$ por el axioma de orden 1.
    \item Claramente $A \neq \emptyset$, pues $\R, \R^+ \subseteq A$. \\[5pt]Luego, por hipótesis, $\forall B \in A$ tenemos que $B\subseteq \R $ por lo que $C\subseteq \R$. Además, $\forall B\in A$, se verifica que $1\in B$. Consecuentemente, $1\in C$. Por otra parte, si $n\in B$ para todo $B\in A$, tendremos que $n+1\in B$, por lo que $n+1 \in C$. Por tanto, $C$ es un conjunto inductivo.
\end{enumerate}

\textbf{Definición.} Al conjunto $C$ de (3) de LE6 lo llamaremos conjunto de los números naturales y lo denotaremos con el símbolo $\N$.

\subsection*{Lista de ejercicios 7 (LE7)}

Demuestre lo siguiente:

\begin{enumerate}[label=\alph*)]
    \item La suma de números naturales es un número natural.
    \item La multiplicación de números naturales es un número natural.
    \item $n\geq 1, \forall n\in \N$.
    \item $0<b^{-1}\leq 1, \forall n\in \N$.
    \item $\forall n\in \N$ con $n>1$ se verifica que $n-1\in \N$.
    \item Sean $m$ y $n$ números naturales tales que $m>n$, demuestre que $m-n\in\N$.
    \item Sea $x\in \R^+$. Si $n\in \N$ y $x+n\in \N$, desmuestre que $x\in \N$.
    \item Sea $x\in \R$, si $n\in \N$ y $n-1<x<n$, demuestre que $x$ no es un número natural.
\end{enumerate}

\textbf{Demostración.}

\begin{enumerate}[label=\alph*)]
% A    
    \item Sea $m\in \N$ arbitrario pero fijo. Definimos $A=\{ n\in \N : m+n \in \N \}$. Por definición, $1\in \N$ y $m+1\in \N$, entonces $1\in A$, es decir, $A\neq \emptyset$. \\[5pt] Por otra parte, si $n\in A$ debe ser el caso que $n\in \N$ y $m+n\in \N$. Como $\N$ es un conjunto inductivo, $n+1 \in \N$ y $(a+n)+1 \in \N$, luego, por la asociatividad de la suma, $m+(n+1)\in \N$. Por la condición de $A$, se cumple que $n+1\in A$, por lo que $A$ es un conjunto inductivo. De esto se concluye que $\N\subseteq A$ y como $A\subseteq \N$, $A=\N$. En otras palabras, la suma de números naturales es un número natural. \qed
% B    
    \item Sea $m\in \N$ arbitrario pero fijo. Definimos $A=\{n\in \N: m\cdot n \in \N\}$. Por definición, $1 \in \N$. Adenás, $m\cdot 1 \in \N$, entonces $1 \in A$, es decir $A \neq \emptyset$.\\[5pt] 
    Luego, si $n \in A$ debe ser el caso que $n\in \N$ y $m \cdot n \in \N$. Por (a) de LE7 se verifica que $(a \cdot n) + m \in \N$. Notemos que $(a \cdot n) + m=m \cdot (n+1)$, osea, $m \cdot (n+1) \in \N$. Como $\N$ es un conjunto inductivo, tenemos que $n+1\in \N$. De este modo, $n+1\in A$. Lo que implica que $A$ es un conjunto inductivo. De esto se concluye que $\N \subseteq A$ y como $A\subseteq \N$, $A=\N$. En otras palabras, la multiplicación de números naturales es un número natural. \qed
% C
    \item Sea $A\coloneqq \{n\in \N: n\geq 1\}$. Como $1\in \N$ y $1\geq 1$, tenemos que $1\in A$.\\[5pt]
    Si $n\in A$ debe ser el caso que $n\in \N$ y $1\leq n$. Además, por (a) de LE7, $n+1\in \N$. Luego, notemos que $0 \leq 1$ de donde sigue que $n \leq n+1$. Por transitividad, $1\leq n+1$, por lo que $n+1\in A$, lo que implica que $A$ es un conjunto inductivo, es decir, $\N\subseteq A$ y como $A\subseteq \N$, $A=N$. En otras palabras, $n\geq 1, \forall n\in\N$. \qed
% D
    \item Por (d) de LE7, $n\geq 1, \forall n\in \N$. Si $n=1$, tenemos que $b^{-1}=1>0$. Si $n>1$, tenemos que $n>0$, por lo que $b^{-1}>0$. En cualquier caso, $1\geq b^{-1}>0, \forall n\in \N$.
% E
    \item Sea $A \coloneqq \set{n\in \N | n>1, n-1\in \N}$. Si $n\in A$ debe ser porque $n>1$ y $n-1\in \N$. Como $n\in \N$ y $\N$ es un conjunto inductivo, se verifica $n+1\in\N$. Notemos que \begin{align*}
        (n+1)-1 &= n+(1-1) \\
        &= n+ 0\\
        &= n
    \end{align*}
    Entonces, $(n+1)-1\in \N$. También, $n>1$ implica que $n>0$ y $n+1>1$, por lo que $n+1\in A$. De este modo, $A$ es un conjunto inductivo, con lo que $\N \subseteq A$, y como $A\subseteq \N$, $A=\N$. Por tanto $\forall n\in \N$ con $n>1$ se verifica que $n-1\in \N$. \qed
% F
    \item Sea $A \coloneqq \set{n\in \N| n<m, m-n\in\N \ \text{con} \ m\in\N}$. Por definición, $1\in \N$ y $1+1\in \N$. Por (a) y (b) de LE3, $1>0$, de donde sigue que $1+1>1$. Por (e) de LE7, se verifica que $(1+1)-1\in \N$, por lo que $1\in A$. \\[5pt] Si $n \in A$ debe ser porque $m-n\in \N$ y $m>n$, de donde obtenemos $m+1>n+1$. Como $m,n\in \N$ y $\N$ es un conjunto inductivo, $n+1\in \N$ y $m+1 \in \N$. Notemos que $m+1-(n+1)=m-n$, por lo que $n+1\in A$. De este modo, $A$ es un conjunto inductivo, con lo que $\N \subseteq A$, y como $A\subseteq \N$, $A=\N$. \qed 
%
    %Para algún $m\in \N$ tal que $m>1$, por (e) de LE7, se verifica que $m-1\in \N$, por lo que $1\in A$. \\[5pt] \textbf{Nota}: Se abusa de la notación y no debe entenderse que $m$ es fija, sino que depende del número natural con que se relaciona.   
% G    
    \item Por (b) de LE3, $x>0$, por lo que $x+n>n$. Por hipótesis, $x+n, n\in \N$, y por (f) de LE7 $(x+n)-n \in \N$, osea, $x\in \N$. \qed
%H    
    \item Supongamos que $x\in \N$. Por hipótesis tenemos que $x<n$ y $x>n-1$. Notemos que \begin{align*}
        x &< n\\
        x -n &< n-n\\
        x-n &< 0\\
        x-n +1 &< 1
    \end{align*}
    Del mismo modo, \begin{align*}
        n-1 &< x\\
        n-1-(n-1) &< x - (n-1)\\
        0 &< x-n+1\\
        n &< x+1
    \end{align*}
    Como $\N$ es un conjunto inductivo, $x+1\in \N$, y como $x+1>n$, con $n\in \N$, por (e) de LE7, $x+1-n \in \N$, y por (c) de LE7, $x+1-n\geq 1$. Pero tenemos que $x-n+1<1$, osea $1\leq x+1-n<1$, lo cual es una contradicción. Por tanto, $x$ no es un número natural. \qed
\end{enumerate}

\pagebreak

\textbf{Definición.} Sea $E$ un subconjunto no vacío de $\R$, decimos que $E$ está acotado: \begin{itemize}
    \item Superiormente si existe un número real $m$ tal que $b \leq m, \forall b\in E$. En este caso decimos que $E$ es cota superior de $E$.
    \item Inferiormente si existe un número real $l$ tal que $l \leq b, \forall b\in E$. En este caso, decimos que $l$ es cota inferior de $E$.
    \item Si existe un número real $m$ tal que $|b|\leq m,\forall b \in E$. En este caso decimos que $m$ es una cota de $E$.
\end{itemize}

\textbf{Definición.} Sea $A$ un subconjunto no vacío del conjunto de los números reales, acotado superiormente, decimos que un número real $M$ es supremo de $A$ si $M$ satisface las siguientes condiciones: \begin{itemize}
    \item $M$ es cota superior de $A$.
    \item Si $K$ es una cota superior de $A$, entonces $M\leq K$, es decir, $M$ es la cota superior más pequeña de $A$.
\end{itemize}

En este caso escribimos $M=\sup{A}$.

\textbf{Definición}. Sea $A$ un subconjunto no vacío del conjunto de los números reales, acotado inferiormente, decimos que un número real $L$ es ínfimo de $A$ si $L$ satisface las siguientes condiciones: \begin{itemize}
    \item $L$ es cota inferior de $A$.
    \item Si $K$ es una cota inferior de $A$, entonces $K\leq L$, es decir, $L$ es la cota inferior más grande de $A$.
\end{itemize}

En este caso escribimos $M=\inf{A}$.onsideremos el P

\subsection*{Lista de ejercicios 8 (LE8)}

Falso o verdadero: \begin{enumerate}[label=\arabic*.]
    \item Si $E$ es un subconjunto de $\R$ acotado superiormente, entonces $E$ es un conjunto acotado.
    \item Si $E$ es un subconjunto acotado de $\R$, entonces $E$ está acotado superiormente e Inferiormente.
\end{enumerate}

Demuestre lo siguiente:

\begin{enumerate}[label=\arabic*., resume]
    \item Sea $A$ un subconjunto no vacío de $\R$, si $A$ tiene supremo, este es único.
    \item Sea $A$ un subconjunto no vacío de $\R$, si $A$ tiene ínfimo, este es único.
    \item Una cota superior $M$ de un conjunto no vacío $S$ de $\R$ es el supremo de $S$ si y solo si para toda $\varepsilon>0$ existe una $s_\varepsilon \in S$ tal que $M-\varepsilon<s_\varepsilon$.
\end{enumerate}

\subsubsection*{Respuesta}

\begin{enumerate}[label=\arabic*.]
    \item Falso. Consideremos el conjunto $\R\backslash \R^+$, el cual es un subconjunto de $\R$, y es no vacío, pues $-1\in \R\backslash \R^+$. Además, $b\leq 0, \forall b\in \R\backslash\R^+$, por lo que el conjunto está acotado superiormente. Supongamos que el conjunto propuesto está acotado. Es decir, suponemos que $\exists m$ tal que $|b|\leq m, \forall b\in \R\backslash \R^+$. Por (f) de LE4, $-m \leq b$ y, por transitividad, $-m\leq 0$, de donde sigue que $-m-1\leq -1$, pero $-1<0$, entonces $-m-1<0$, lo que implica que $-m-1\in \R\backslash \R^+$, por lo que $|-m-1|\leq m$. Luego, notemos que $|-m-1|=-(-m-1)$, es decir, tenemos que $m+1\leq m$, pero de esto se concluye que $1\leq 0$, lo cual es una contradicción. Por tanto, aunque $\R\backslash \R^+$ está acotado superiormente, no está acotado.
    \item Verdadero. Sea $E$ un subconjunto no vacío de $\R$. Si $E$ está acotado, entonces $\exists m$ tal que $|b|\leq m,\forall b \in E$. Por (f) de LE4, $-m\leq b \leq m$, por lo que el conjunto está acotado superiormente e inferiormente.
\end{enumerate}

\subsubsection*{Demostración}

\begin{enumerate}[label=\arabic*.]\setcounter{enumi}{2}
    \item Supongamos que $s_1$ y $s_2$ son supremos de $A$. Como $s_1$ es una cota superior de $A$ y $s_2$ es elemento supremo, entonces $s_2\leq s_1$. Similarmente, $s_1\leq s_2$. Por tanto, $s_1=s_2$. \qed
    \item Supongamos que $m_1$ y $m_2$ son ínfimos de $A$. Como $m_1$ es una cota superior de $A$ y $m_2$ es elemento ínfimo, entonces $m_1\leq m_2$. Similarmente, $m_2\leq m_1$. Por tanto, $m_1=m_2$. \qed
    \item \begin{enumerate}[label=\roman*)]
        \item Sea $M$ una cota superior de $S$ tal que $\forall \epsilon>0, \exists s_{\epsilon}$ tal que $M-\epsilon<s_{\epsilon}$. Si $M$ no es el supremo de $S$, tendríamos que $\exists V$ tal que $s_@a \leq V < M$. Elegimos $\epsilon = M-V$, con lo que $V<s_{\epsilon}$, lo que contradice nuestra hipótesis. Por tanto, $M$ es el supremo de $S$.
        \item Sea $M$ el supremo de $S$ y $\epsilon>0$. Como $M<M+\epsilon$, entonces $M-\epsilon$ no es una cota superior de $S$, por lo que $\exists s_\epsilon$ tal que $s_\epsilon>M-\epsilon$.
        \end{enumerate} \qed
\end{enumerate}

\subsection*{Principio del buen orden}

Todo subconjunto no vacío del conjunto de los números naturales tiene elemento mínimo. Esto significa que si $A\subseteq \N$ y $A \neq \emptyset$, entonces existe un elemento $c\in A$ tal que $c\leq a, \forall a\in A$.

\textbf{Observación:}

Sabemos —por (c) de LE7— que cualquier subconjunto no vacío de $\N$ está acotado inferiormente. El principio del buen orden nos garantiza que cualquier subconjunto no vacío de $\N$ contiene una de sus cotas inferiores, a la que llamamos elemento mínimo.

Notemos que si suponemos la existencia de un subconjunto no vacío de $\N$ tal que ninguna de sus cotas inferiores esté contenida en el conjunto, estaríamos negando el principio del buen orden. Es así cómo procedemos a probar el teorema.

\pagebreak

\textbf{Demostración:}

Sea $A\subseteq \N$ con $A\neq \emptyset$. Supongamos que $A$ no contiene ninguna de sus cotas inferiores, es decir, supongamos que si $c\leq a, \forall a\in A$, entonces $c\notin A$.

Definimos el conjunto $L\coloneqq \set{n\in \N: n\leq a, \forall a\in A}$. Es claro que $1\in L$. Veamos que si $n\in L$, tendríamos que $n\leq a, \forall a\in A$. Luego, si $n+1\notin L$, entonces $\exists a_0\in A$ tal que $n+1>a_0$, por lo que $n\leq a_0<n+1$, y —por (h) de LE7— no puede ser el caso que $n<a_0$, de donde sigue que $n=a_0$, pero esto contradice nuestro supuesto inicial, entonces, debe ser el caso que $n+1\in L$. Consecuentemente, $L$ es un conjunto inductivo, y —por definición— $\N\subseteq L$ y $L\subseteq \N$, lo que implica que $L=\N$.

Finalmente, notemos que $A$ y $L$ son disjuntos, y dado que $A\subseteq \N$ y $L=\N$, sigue que $A=\emptyset$, pero esto es una contradicción. Por tanto, si $A\subseteq \N$ con $A\neq \emptyset$, entonces $\exists c\in A$ tal que $c\leq a, \forall a\in A$. \qed
%
%\textbf{\textit{Teorema.}} Si $A\subseteq \N$ y $A\neq \emptyset$ y $A$ está acotado superiormente, entonces $A$ tiene elemento máximo, esto es existe un elemento $c\in A$ tal que $a\leq c, \forall a\in A$.
%
%\textbf{Demostración:}
%
%Sea $A\subseteq\N$ con $A\neq \emptyset$ y $A$ acotado superiormente. Supongamos que si $c\geq a, \forall a\in A$, entonces $c\notin A$.
%
%Definimos el conjunto $-A\coloneqq \set{-a: a\in A}$. Como $A$ está acotado superiormente, $\exists c\in \R$ tal que $c\geq a, \forall a\in A$. Notemos que $-a\geq -c, \forall -a\in -A$, lo que implica que $-A$ está acotado inferiormente, y por esto, $A$ tiene elemento mínimo. Sea $m$ el elemento mínimo de $-A$. Veamos que $m\leq -a, \forall -a\in -A$ de donde sigue que $a\leq -m, \forall a\in A$, con $-m\in A$ pero esto contradice nuestro supuesto inicial. Por tanto, $A$ tiene elemento máximo. \qed
%

\subsection*{Principio de inducción matemática}

Sea $S\subseteq \N$ tal que $S$ es un conjunto inductivo, entonces $S=\N$.

\textbf{Demostración:} Supongamos que $S\neq \N$, entonces el conjunto $\N\setminus S$ es no vacío (ya que de serlo, tendríamos $S=\N$), y —por el principio del buen orden— tiene elemento minimo. Sea $m$ el elemento mínimo de $\N\setminus S$. Por (c) de LE7, se verifica que $1 \leq m$. Por definición, $1\in S$ y por esto, $1\notin \N\setminus S$. Como $m\in \N\setminus S$ tenemos que $m\neq 1$, por lo que $m>1$, de donde sigue —por (e) de LE7— que $m-1\in \N$. Debido a que $m-1<m$ y $m$ es el elemento mínimo de $\N \setminus S$, $m-1\in S$. Luego, dado que $S$ es un conjunto inductivo, se verifica que $(a-1)+1=m\in S$ lo que es una contradicción. Por tanto, debe ser el caso que $S=\N$. \qed

\textbf{\textit{Teorema.}} Todo conjunto finito no vacío tiene elemento mínimo y elemento máximo, es decir, para todo conjunto finito $A\neq \emptyset$, $\exists m,M\in A$ tales que $m\leq a\leq M, \forall a\in A$.

\textbf{Demostración:} Sea $n\in \N$ y $A \coloneqq \{a_1, \dots, a_n\}$ no vacío.

Procedemos por inducción sobre el número de elementos de $A$. \begin{enumerate}[label=\roman*)]
    \item Si $n=1$, tenemos $A\coloneqq\{a_1\}$, por lo que $m=a_1$ y $M=a_1$ cumplen la condición requerida.
    \item Supongamos que la proposición se cumple para $n=k$.
    \item Si $n=k+1$, tenemos $A\coloneqq \{a_1, \dots, a_k, a_{k+1}\}$. Luego, por hipótesis de inducción, el conjunto \[A' \coloneqq A \setminus \{a_{k+1}\} = \{a_1, \dots, a_k\}\] tiene elemento mínimo y máximo, es decir, $\exists m',M'\in A'$ tales que $\forall a'\in A', m'\leq a' \leq M'$.
    
    Notemos que para cada $a\in A$ tenemos $a=a_{k+1}$ o $a\in A'$. Por tricotomía, $a_{k+1}$ cumple con alguno de los siguientes casos:
    \begin{enumerate}[label=\alph*)]
        \item Si $a_{k+1}<m'$, tenemos que $m=a_{k+1}<m'\leq a' \leq M'=M$.
        \item Si $m' \leq a_{k+1}\leq M'$, entonces $m=m'\leq a_{k+1} \leq M'=M$.
        \item Si $m'<a_{k+1}$, tenemos que $m=m'\leq a' \leq M'<a_{k+1}=M$.
    \end{enumerate}
    En cualquier caso $\exists m,M\in A$ tales que $m\leq a\leq M, \forall a\in A$. \qed
\end{enumerate}

\subsection*{Axioma del supremo}

Todo subconjunto no vacío del conjunto de los números reales que sea acotado superiormente tiene supremo.

\textbf{\textit{Teorema.}} El conjunto de los números naturales no está acotado superiormente.

\textbf{Demostración:}

Supongamos que el conjunto de los números naturales está acotado superiormente. Entonces existe un número real $M$ tal que $n\leq M, \forall n\in \N$. Como el conjunto de los números naturales es no vacío, entonces, por el axioma del supremo, $\N$ tiene supremo.

Sea $L\coloneqq \sup{(\N)}$. Como $L-1$ no es cota superior de $\N$, ya que $L>L-1$ y $L$ es la cota superior más pequeña, existe un núero natural $n_0$ tal que $n_0>L-1$, lo cual implica que $n_0+1<L$, pero esto contradice la hipótesis	de que $L$ es supremo de $\N$. Por tanto, el conjunto de los números naturales no está acotado superiormente. \qed

\textbf{\textit{Teorema.}} Si $A\subseteq \R, A\neq \emptyset$ y $A$ está acotado inferiormente, entonces $A$ tiene ínfimo.

\textbf{Demostración:}

Sea $A\subseteq \R, A\neq \emptyset$ y $A$ está acotado inferiormente. El conjunto $-A \coloneqq \set{-a: a\in A}$ está acotado superiormente y, por el axioma del supremo, $-A$ tiene supremo. Sea $M\coloneqq \sup{(A)}$, entonces $M\geq -a, \forall -a\in -A$. Notemos que $-M\leq a, \forall a\in A$, esto es $-M$ es el ínfimo de $A$. \qed

\subsection*{Propiedad Arquimediana del conjunto de los números reales}

Para cada número real $x$ existe un número natural $n$ tal que $x<n$.

\textbf{Demostración:}

Supongamos que existe $x\in \R$ tal que $n\leq x, \forall n\in \N$. Notemos que $x$ es una cota superior de $\N$, pero esto contradice el teorema que establece que el conjunto de los números naturales no está acotado superiormente. Por tanto, se satisface la propiedad arquimediana del conjunto de los números reales. \qed

\textbf{Definción.} \begin{itemize}
    \item Al conjunto $\N \cup {0} \cup {-n: n\in \N}$ lo llamaremos conjunto de los números enteros y lo representaremos con el símbolo $\Z$.
    \item Al conjunto ${-n: n\in \N}$ lo llamaremos conjunto de los números enteros negativos y lo representaremos con el símbolo $\Z^-$.
    \item Al conjunto $\N$ también lo llamaremos conjunto de los números enteros positivos y lo representaremos con el símbolo $\Z^+$.
\end{itemize}

\textbf{Observación.} Los conjuntos $\N$, ${0}$, ${-n: n\in \N}$ son disjuntos por pares.

\subsection*{Lista de Ejercicios 9 (LE9)}

\begin{enumerate}[label=\alph*)]
    \item Si $S \coloneqq \set{\frac{1}{n}: n\in \N}$, entonces $\inf{S=0}$.
    \item Si $t>0$, entonces $\exists n\in \N$ tal que $0<\frac{1}{n}<t$.
    \item Si $y>0$, entonces $\exists n\in \N$ tal que $n-1\leq y< n$.
    \item Sea $x\in \R$, demuestre que $\exists! \, n\in \Z$ tal que $n\leq x<n+1$.
\end{enumerate}

\subsubsection*{Demostración}

\begin{enumerate}[label=\alph*)]
    \item Por (d) de LE7, $S$ está acotado inferiormente por $0$; de esto sigue que $S$ tiene ínfimo. Sea $w\coloneqq \inf{S}$. Por definición, $\frac{1}{n}\geq w\geq 0, n\in \N$. Supongamos que $w>0$. Por la propiedad arquimediana $\exists n_0$ tal que $\frac{1}{w} < n_0$, de donde sigue que $w<\frac{1}{n_0}$ con $\frac{1}{n_0} \in S$, lo cual es una contradicción. Por tanto, $w=0$. \qed
    
    \item Por la propiedad arquimediana $\exists n$ tal que $\frac{1}{t}<n$. Como $n$ y $t$ son mayores que $0$, sigue que $0<\frac{1}{n}<t$. \qed
    
    \item Por la propiedad arquimediana, el conjunto $E\coloneqq \set{m\in \N: y<m}$ es no vacío. Además, por el principio del buen orden, $\exists n\in E$ tal que $n\leq m, \forall m\in E$. Notemos que $n-1<n$, por lo que $n-1\notin E$, lo que implica que $n-1\leq y<n$. \qed
    
    \item Definimos el conjunto $A\coloneqq \set{n\in \Z: x<n}$. Por la propiedad arquimediana $\exists n_0 \in \N$ tal que $x<n_0$, así $n_0\in A$, por lo que $A\neq \emptyset$. Sabemos también que $A$ está acotado inferiormente, de manera que $A$ tiene elemento mínimo. Sea $n$ el elemento mínimo de $A$. Notemos que $n-1<n$, de donde sigue que $n-1\leq x<n$. Luego, $n-1\in \Z$, al que definimos como $m=n-1$, por lo que $m\leq x<m+1$.
    
    Finalmente, supongamos que $\exists m, n\in \Z$ tales que $m\leq x<m+1$ y $n\leq x<n+1$. Si $m\neq n$, sin pérdida de generalidad, $m>n$. Por ello, \begin{align*}
        n < m &\leq x<n+1 \\
        n < m &<n+1 \\
        0 < m-n &<1
    \end{align*}  
    Lo que contradice la cerradura de la suma en $\Z$. Por tanto, $m=n$, es decir, el número entero que satisface $n\leq x<n+1$ es único. \qed
%    
    %\textbf{Demostración alternativa:}
%
    %Definimos el conjunto $A\coloneqq \set{n\in \Z: n\leq x}$. Por la propiedad arquimediana $\exists n_0\in \N$ tal que $n_0>x$. Observemos que \begin{enumerate}[label=\roman*)]
    %\item Si $x\geq 0$, $-x\leq 0$ y $-x<n_0$, de donde sigue que $-n_0<x$, por lo que $-n_0\in A$.
    %\item Si $x<0$, $x<n_0$, de donde sigue que $-n_0\leq x$, por lo que $-n_0\in A$.
    %\end{enumerate} Consecuentemente, $A$ es no vacío. También sabemos que $A$ está acotado superiormente, por el axioma del supremo, $A$ tiene supremo. Sea $m\coloneqq \sup{A}$. Por definición, $m\leq x$. Notemos que $m+1>m$. Luego, si $m+1\leq x$ tendríamos que $m+1\in A$ pero como $m$ es el supremo de $A$ seguiría que $m \geq m+1$, lo cual es una contradicción, entonces debe ser el caso que $m\leq x<m+1$.\qed
\end{enumerate}

\subsection*{Lista de Ejercicios \# (LE\#)}

Sean $a$ y $b$ números reales, demuestre lo siguiente:

\begin{enumerate}[label=\alph*)]
    \item $0 \leq a^{2n} \, \forall n\in \N$.
    \item Si $0\leq a$, entonces $ 0 \leq a^n, \, \forall n\in \N$.
    \item Si $0 \leq a <b$, entonces $a^n < b^n, \, \forall n\in \N$.
    \item Si $0 \leq a <b$, entonces $a^n \leq ab^n < b^n \, \forall n\in \N$.
    \item Si $0<a<1$, entonces $a^n<a \, \forall n\in \N$.
    \item Si $1<a$, entonces $a<a^n \, \forall n\in \N$.
\end{enumerate}

\subsubsection*{Demostración}

\begin{enumerate}[label=\alph*)]

    %A
    \item Pendiente

    %B
    \item Por inducción matemática. 

    \begin{enumerate}[label=\roman*)]
        \item Verificamos que se cumple para $n=1$. \begin{align*}
        0 &\leq a^1 \\
        0 &\leq a
        \end{align*}
        \item Suponemos que se cumple para $n=k$, para algún $k \in \N$. Es decir,  suponemos que \[0 \leq a^k\]
        \item Probaremos a partir de (ii) que $0 \leq a^{k+1}$. En efecto, por hipótesis de     inducción \begin{align*}
        0 &\leq a^k \\
        0 \cdot a &\leq a^k \cdot a \\
        0 &\leq a^{k+1}
        \end{align*}
    \end{enumerate}

    %C
    \item Por inducción matemática.
    \begin{enumerate}[label=\roman*)]
        \item Verificamos que se cumple para $n=1$. \begin{align*}
        a^1 &< b^1 \\
        a &< b
        \end{align*}
        \item Suponemos que se cumple para $n=k$, para algún $k\in \N$. Es decir, suponemos que \[a^k < b^k\]
        \item Probaremos, a partir de (ii) que $a^{k+1} < b^{k+1}$. En efecto, por (c) de LE5, garantizamos que $0 \leq a^k$, lo que nos permite, por (a) de LE5, afirmar que
        \begin{align*}
        a^k \cdot a &< b^k \cdot b \\
        a^{k+1} &< b^{k+1}
        \end{align*}
    \end{enumerate}

    %D
    \item Tenemos que $a<b$, como $0\leq a<b$, sigue que $0<b$, entonces $a\cdot b < b\cdot b$, osea $ab<b^2$. Luego, $a \cdot a \leq ab$. Finalmente, $a^2\leq ab < b^2$.
    
    %E
    \item Pendiente
    
    %F
    \item Pendiente

\end{enumerate}

\subsection*{Funciones}

\textbf{Definición:} Sean $a$ y $b$ objetos cualesquiera, definimos la pareja ordenada $(a,b)$ como sigue: \[
    (a,b)\coloneqq \set{\set{a}, \set{a,b}}\]
Al objeto $a$ lo llamaremos primer componente de la pareja ordenada $(a,b)$ y al objeto $b$ lo llamaremos segundo componente de la pareja ordenada $(a,b)$.

\textbf{Teorema:} $(a,b)=(c,d)$ si y solo si $a=c$ y $b=d$.

\textbf{Demostración:} Pendiente

\subsection*{Sucesiones}

\textbf{Definición:} Una sucesión es una función %$X: n\in \N \mapsto x_n \in \R$
\begin{align*}
    X: \ & \N \to \R \\
    \ &  n \mapsto x_n 
\end{align*}
%
Llamamos a $x_n$ el n-ésimo término. Otras etiquetas para la sucesión son $(x_n)$, $(x_n:n\in \N)$, que denotan orden y se diferencian del rango de la función $\{x_n:n\in \N\}\subseteq \R$.

\textbf{Definición:} Una sucesión $(x_n)$ es convergente si $\exists \ell \in \R$ tal que para cada $\varepsilon>0$ existe un número natural $n_\varepsilon$ (que depende de $\varepsilon$) de modo que los términos $x_n$ con $n\geq n_\varepsilon$ satisfacen que $|x_n-\ell|<\varepsilon$.

Decimos que $(x_n)$ converge a $\ell \in \R$ y llamamos a $\ell$ el límite de la sucesión y escribimos $\lim (x_n) = \ell$.

%Notemos que por LE5(a) se cumple que $x_n=x$ con $n\geq m$. Esto es falso.
\textbf{Definición:} Una sucesión es divergente si no es convergente.

\textbf{Definición:} Una sucesión $(x_n)$ está acotada si $\exists M\in \R^+$ tal que $|x_n|\leq M, \forall n\in \N$.

\subsection*{Lista de Ejercicios 10 (LE10)}

Demuestre lo siguiente:

\begin{enumerate}[label=\alph*)]
    \item El límite de una sucesión convergente es único.
    \item Toda sucesión convergente está acotada.
\end{enumerate}

\subsubsection*{Demostración}

\begin{enumerate}[label=\alph*)]
    \item Sean $\ell$ y $\ell'$ límites de la sucesión $(x_n)$. Tenemos que $\forall \varepsilon>0$, existen $n',n'' \in \N$ tales que $|x_{n\geq n'}-\ell|<\varepsilon$ y $|x_{n\geq n''}-\ell'|<\varepsilon$. Sin pérdida de generalidad, si $n'<n''$, los términos $x_n$ con $n\geq n''>n'$ satisfacen que \begin{align*}
        |x_n-\ell| &<\varepsilon && \text{(1)}\\
        |x_n-\ell'| &<\varepsilon && \text{(2)}
    \end{align*}
    Por (c) de LE4, se cumple que $|x_n-\ell'|=|\ell'-x_n|$ y por esto, \begin{align*}
        |\ell'-x_n|<\varepsilon && \text{(3)}
    \end{align*}
    Tomando (1) y (3), por (d) de LE3, se verifica que \begin{align*}
        |\ell'-x_n| + |x_n-\ell| &< 2\varepsilon
    \end{align*}
    Y, por la desigualdad del triángulo, tenemos que \begin{align*}
        \big|(\ell'-x_n)+(x_n-\ell)\big| &\leq |\ell'-x_n| + |x_n-\ell|\\
        |\ell'-\ell| &\leq |\ell'-x_n| + |x_n-\ell|
    \end{align*}
    De este modo, $|\ell'-\ell| < 2\varepsilon$. Como esta desigualdad se cumple para todo $\varepsilon>0$, en particular se verifica para $\varepsilon=\nicefrac{\varepsilon_0}{2}$ con $\varepsilon_0>0$ arbitrario pero fijo, así obtenemos que \begin{align*}
        |\ell'-\ell| &< 2 \left(\frac{\varepsilon_0}{2}\right)\\
        |\ell'-\ell| &< \varepsilon_0
    \end{align*}
    Finalmente, como $\varepsilon_0$ es arbitrario, por (a) de LE5, sigue que $\ell'=\ell$. Por tanto, el límite de cada sucesión convergente es único. \qed
%
    \item Sea $(x_n)$ una sucesión convergente. Por definición, $\forall \epsilon>0, \exists n_\varepsilon \in \N$ tal que los términos $x_n$ con $n\geq n_\varepsilon$ satisfacen que \begin{align*}
        |x_n - \ell| &< \epsilon \\
        |x_n - \ell| + |\ell| &< \epsilon + |\ell|
    \end{align*}
    Luego, por la desigualdad del triángulo, \begin{align*}
        \big|(x_n-\ell)+\ell\big| &\leq |x_n-\ell| + |\ell|\\
        |x_n| &\leq |x_n-\ell| + |\ell|
    \end{align*}
    Por transitividad, $|x_n|< \epsilon + |\ell|$, lo que implica que $\{x_{n\geq n_\varepsilon}\}$ está cotado superiormente.
    
    Por otra parte, el conjunto de índices $n<n_\varepsilon$ está acotado, y por esto, $\{x_{n<n_\varepsilon}\}$ es finito, por lo que tiene cota superior. %proof https://math.stackexchange.com/questions/548806/a-finite-set-always-has-a-maximum-and-a-minimum
    
    Finalmente, el conjunto $\{x_{n<n_\varepsilon}\} \cup \{x_{n\geq n_\varepsilon}\}$ está acotado superiormente, y por tanto, $(x_n)$ está acotada. \qed
\end{enumerate}

\end{document}
