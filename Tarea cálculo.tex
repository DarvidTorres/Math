\documentclass[11pt]{article}

\usepackage[top=0.6in,bottom=0.6in,right=1in,left=1in]{geometry}
\usepackage{amsfonts, amssymb, amsmath, amsthm, enumitem}
\usepackage{mathtools}
\usepackage{braket}
\usepackage{nicefrac} % for Elegant fractions in one line https://tex.stackexchange.com/questions/128496/elegant-fractions-in-one-line/128498

%Conjuntos de números
\newcommand{\N}{\mathbb{N}}
\newcommand{\Z}{\mathbb{Z}}
\newcommand{\Q}{\mathbb{Q}}
\newcommand{\R}{\mathbb{R}}

%Shorter comands
\let\epsilon\varepsilon
\let\oldemptyset\emptyset
\let\emptyset\varnothing
\let\set\Set

%bold all lists$
\setlist[enumerate]{font=\bfseries}

\setlength{\parindent}{0pt} %no indent for the document
\setlength{\parskip}{1em} %add space between paragraphs
\pagestyle{empty}

\begin{document}

\title{\vspace{-2cm}Cálculo diferencial e Integral I \\ Semestre 2023-1 \\ Grupo 4031}
\author{Problemas de: números reales \\ \text{Torres Brito David Israel}}
\date{\today}
\maketitle
\thispagestyle{empty}

\begin{enumerate}
\item Pruebe que si $a,b\in \R$, entonces $-(a+b)=(-a)+(-b)$.\\

\textbf{Demostración:} \begin{align*}
  0 &= 0 + 0 && \text{Neutro aditivo} \\
  &= \bigl(a+(-a)\bigr) + \bigl(b + (-b)\bigr) && \text{Neutro aditivo} \\
  &= (a+b) + \bigl((-a)+ (-b)\bigr) && \text{Asociatividad}
\end{align*}
Debido a que el inverso aditivo es único, tenemos que $(-a)+ (-b)=-(a+b)$.
\qed


\item Pruebe que si $a,b\in \R$ son tales que $a-b=b-a$, entonces $a=b$.

\textbf{Demostración:} \begin{align*}
  a-b &= b- a && \text{Hipótesis} \\
  (a-b)+b &= (b-a)+b &&\text{Ley de la cancelación} \\
  a &= (b+b)-a&&\text{Asociando} \\
  a + a &= b+b &&\text{Ley de la cancelación} \\
  2a &= 2b &&\text{Definición} \\
  a &= b && \text{Ley de la cancelación}
\end{align*} \qed

\item Pruebe que si $a,b\in \R$, entonces $-(a-b)=b-a$.

\textbf{Demostración:} \begin{align*}
  -(a-b) &= -(a+(-b)) &&\text{Notación} \\
  &= (-a) + \bigl(-(-b)\bigr) &&\text{Por ejercicio 1} \\
  &= (-a) + (b) &&\text{Unicidad del inverso aditivo} \\
  &= b -a &&\text{Conmutatividad} \\
\end{align*} \qed

\pagebreak

\item Pruebe que si $a,b\in \R$ son tales que $ab=0$, entonces $a=0$ o $b=0$.

\textbf{Demostración:} Primero demostraremos que si $a\in \R$, entonces $a\cdot 0=0$.
\begin{align*}
  a\cdot0&=a\cdot0+0 && \text{Neutro aditivo}\\
  &=a\cdot0+\bigl(a+\left(-a\right)\bigr) && \text{Neutro aditivo}\\
  &=a\cdot0+\bigl(a\cdot1+\left(-a\right)\bigr) && \text{Identidad de la multiplicación}\\
  &=\left(a\cdot0+a\cdot1\right)+\left(-a\right) && \text{Asociatividad}\\
  &=\bigl(a\cdot\left(0+1\right)\bigr)+\left(-a\right) && \text{P. Distributiva}\\
  &=a\cdot1+\left(-a\right) && \text{Neutro aditivo}\\
  &=a+\left(-a\right) && \text{Identidad de la multiplicación}\\
  &=0 && \text{Neutro aditivo}
\end{align*} \qed

Ahora, demostramos el ejercicio 4. Supongamos que $a\neq 0$. \begin{align*}
  b &= b \cdot 1	&& \text{Identidad de la multiplicación} \\
  &= b \cdot  \left(a \cdot a^{-1}  \right) 	&& \text{Inverso multiplicativo} \\
  &= \left(b \cdot a \right)  \cdot a^{-1}	&& \text{Asociatividad} \\
  &= \left(a \cdot b \right)  \cdot a^{-1}	&& \text{Conmutatividad} \\
  &= 0 \cdot a^{-1}	&& \text{Por hipótesis}\\
  &= a^{-1} \cdot 0	&& \text{Conmutatividad}\\
  &= 0 && \text{Probado arriba}
\end{align*} \qed

\item Pruebe que si $a,b\in \R$ son tales que $a^2=b^2$, entonces $a=b$ o $a=-b$.

\textbf{Demostración:}
\begin{align*}
  0 &= b^2 - b^2 && \text{Inverso aditivo}\\
  &= a^2 - b^2 && \text{Por hipótesis	}\\
  &= a \cdot a - b \cdot b && \text{Definición}\\
  &= (a \cdot a - b \cdot b) + 0&& \text{Neutro aditivo}\\
  &= (a \cdot a - b \cdot b) + (a\cdot b-a\cdot b)&& \text{Inverso aditivo}\\
  &= (a \cdot a - b \cdot b + a\cdot b) + (-a\cdot b)&& \text{Asociatividad}\\
  &= (a \cdot a + a\cdot b - b \cdot b) + (-a\cdot b)&& \text{Conmutatividad}\\
  &= (a \cdot a + a\cdot b) + (- b \cdot b-a\cdot b)&& \text{Asociatividad}\\
  &= (a \cdot a + a\cdot b) + (- b \cdot b)+(-a\cdot b)&& \text{Notación}\\
  &= (a \cdot a + a\cdot b) -(b \cdot b + a\cdot b)&& \text{Por ejercicio 1}\\
  &= a (a+b) - b(b+a) && \text{P. Distributiva}\\
  &= (a+b) \cdot (a-b) && \text{P. Distributiva}
\end{align*} Por el ejercicio 4, de la igualdad anterior tenemos que $a+b=0$ o $a-b=0$. Sumando inverso aditivo de $b$ tenemos $a=-b$ o $a=b$. \qed

\pagebreak

\item Pruebe que si $a,b\in \R$ son distintos de $0$ y tales que $ab^{-1}=ba^{-1}$, entonces $a=b$ o $a=-b$.

\textbf{Demostración:}
\begin{align*}
  ab^{-1}&=ba^{-1} && \text{Por hipótesis}\\
  ab^{-1} \cdot b &=ba^{-1} \cdot b&& \text{Ley de la cancelación}\\
  a(b^{-1} \cdot b) &=a^{-1}(b\cdot b) && \text{Asociando}\\
  a &=a^{-1}(b\cdot b) && \text{Identidad de la multiplicación}\\
  a \cdot a&=a^{-1}(b\cdot b) \cdot a&& \text{Ley de la cancelación}\\
  a \cdot a&= (b\cdot b) (a^{-1} \cdot a)&& \text{Asociando}\\
  a \cdot a&= b\cdot b && \text{Identidad de la multiplicación}\\
  a^2 &= b^2 && \text{Definición}
\end{align*}
Por el ejercicio 5, la igualdad anterior implica que $a=b$ o $a=-b$. \qed

\item Determine si las siguientes afirmaciones son falsas o verdaderas. Pruebe su respuesta.
  \begin{enumerate}[label=(\alph*)]
    \item Si $a,b\in \R$, entonces $a<a+b$.

    \textbf{Respuesta:} Falso.

    \textbf{Demostración:} Contraejemplo: $b=0$. \qed

    \item Si $a,b\in \R$, entonces $a<a+b$ o $b<a+b$.

    \textbf{Respuesta:} Falso.

    \textbf{Demostración:}  Sea $a=b=0$; tanto $a<a+b$ como $b<a+b$ fallan en cumplirse. \qed

    \item Si $a,b,c,d\in \R$ son tales que $a+c<b+d$, entonces $a<b$ y $c<d$.

    \textbf{Respuesta:} Falso.

    \textbf{Demostración:} Sea $a=0$, $b=2$ y $c=d=1$, tenemos que se cumple la hipótesis $0+1<2+1$, pero la proposición $c<d$ es falsa. \qed

    \item Si $a,b,c,d \in \R$ son tales que $ac<bd$, entonces $a<b$ y $c<d$.

    \textbf{Respuesta:} Falso.

    \textbf{Demostración:} Sea $a=1$, $b=d=-1$ y $c=0$, con lo que se cumple la hipótesis $(1)(0)=0<1=(-1)(-1)$, pero la proposición $a=1<-1=b$ y $c=0<-1=d$ es falsa. \qed

    \item Si $a,b\in \R$ son tales que $ab=a$, entonces $b=1$.

    \textbf{Respuesta:} Falso.

    \textbf{Demostración:} Sea $a=0$ y $b=-1$. En el ejercicio 4 demostramos que $ab=(0)(-1)=0$, con lo que se cumple la hipótesis $0=ab=a=0$, pero $b\neq 1$. \qed

    \item Si $a,b\in \R$ son tales que $a^2\leq b^2$, entonces $a\leq b$.

    \textbf{Respuesta:} Falso.

    \textbf{Demostración:} Sea $a=2$ y $b=-2$, tenemos que la hipótesis se cumple $a^2=4\leq 4=b^2$, pero la proposición $a=2\leq -2$ es falsa. \qed

    \end{enumerate}
  \end{enumerate}



\end{document}